\documentclass[10pt]{article}

% \usepackage{./arxiv}

\title{Covering Spaces}
\author{Swayam Chube}
\date{\today}

\usepackage[utf8]{inputenc} % allow utf-8 input
\usepackage[T1]{fontenc}    % use 8-bit T1 fonts
\usepackage{hyperref}       % hyperlinks
\usepackage{url}            % simple URL typesetting
\usepackage{booktabs}       % professional-quality tables
\usepackage{amsfonts}       % blackboard math symbols
\usepackage{nicefrac}       % compact symbols for 1/2, etc.
\usepackage{microtype}      % microtypography
\usepackage{graphicx}
\usepackage{natbib}
\usepackage{doi}
\usepackage{amssymb}
\usepackage{bbm}
\usepackage{amsthm}
\usepackage{amsmath}
\usepackage{xcolor}
\usepackage{theoremref}
\usepackage{enumitem}
\usepackage{mathpazo}
% \usepackage{euler}
\usepackage{mathrsfs}
\setlength{\marginparwidth}{2cm}
\usepackage{todonotes}
\usepackage{stmaryrd}
\usepackage[all,cmtip]{xy} % For diagrams, praise the Freyd–Mitchell theorem 
\usepackage{marvosym}
\usepackage{geometry}
\usepackage{titlesec}
\usepackage{tikz}
\usetikzlibrary{cd}

\renewcommand{\qedsymbol}{$\blacksquare$}

% Uncomment to override  the `A preprint' in the header
% \renewcommand{\headeright}{}
% \renewcommand{\undertitle}{}
% \renewcommand{\shorttitle}{}

\hypersetup{
    pdfauthor={Lots of People},
    colorlinks=true,
}

\newtheoremstyle{thmstyle}%               % Name
  {}%                                     % Space above
  {}%                                     % Space below
  {}%                             % Body font
  {}%                                     % Indent amount
  {\bfseries\scshape}%                            % Theorem head font
  {.}%                                    % Punctuation after theorem head
  { }%                                    % Space after theorem head, ' ', or \newline
  {\thmname{#1}\thmnumber{ #2}\thmnote{ (#3)}}%                                     % Theorem head spec (can be left empty, meaning `normal')

\newtheoremstyle{defstyle}%               % Name
  {}%                                     % Space above
  {}%                                     % Space below
  {}%                                     % Body font
  {}%                                     % Indent amount
  {\bfseries\scshape}%                            % Theorem head font
  {.}%                                    % Punctuation after theorem head
  { }%                                    % Space after theorem head, ' ', or \newline
  {\thmname{#1}\thmnumber{ #2}\thmnote{ (#3)}}%                                     % Theorem head spec (can be left empty, meaning `normal')

\theoremstyle{thmstyle}
\newtheorem{theorem}{Theorem}[section]
\newtheorem{lemma}[theorem]{Lemma}
\newtheorem{proposition}[theorem]{Proposition}

\theoremstyle{defstyle}
\newtheorem{definition}[theorem]{Definition}
\newtheorem{corollary}[theorem]{Corollary}
\newtheorem{porism}[theorem]{Porism}
\newtheorem{remark}[theorem]{Remark}
\newtheorem{example}[theorem]{Example}
\newtheorem*{notation}{Notation}

% Common Algebraic Structures
\newcommand{\R}{\mathbb{R}}
\newcommand{\Q}{\mathbb{Q}}
\newcommand{\Z}{\mathbb{Z}}
\newcommand{\N}{\mathbb{N}}
\newcommand{\bbC}{\mathbb{C}} 
\newcommand{\K}{\mathbb{K}} % Base field which is either \R or \bbC
\newcommand{\calA}{\mathcal{A}} % Banach Algebras
\newcommand{\calB}{\mathcal{B}} % Banach Algebras
\newcommand{\calI}{\mathcal{I}} % ideal in a Banach algebra
\newcommand{\calJ}{\mathcal{J}} % ideal in a Banach algebra
\newcommand{\frakM}{\mathfrak{M}} % sigma-algebra
\newcommand{\calO}{\mathcal{O}} % Ring of integers
\newcommand{\bbA}{\mathbb{A}} % Adele (or ring thereof)
\newcommand{\bbI}{\mathbb{I}} % Idele (or group thereof)

% Categories
\newcommand{\catTopp}{\mathbf{Top}_*}
\newcommand{\catGrp}{\mathbf{Grp}}
\newcommand{\catTopGrp}{\mathbf{TopGrp}}
\newcommand{\catSet}{\mathbf{Set}}
\newcommand{\catTop}{\mathbf{Top}}
\newcommand{\catRing}{\mathbf{Ring}}
\newcommand{\catCRing}{\mathbf{CRing}} % comm. rings
\newcommand{\catMod}{\mathbf{Mod}}
\newcommand{\catMon}{\mathbf{Mon}}
\newcommand{\catMan}{\mathbf{Man}} % manifolds
\newcommand{\catDiff}{\mathbf{Diff}} % smooth manifolds
\newcommand{\catAlg}{\mathbf{Alg}}
\newcommand{\catRep}{\mathbf{Rep}} % representations 
\newcommand{\catVec}{\mathbf{Vec}}

% Group and Representation Theory
\newcommand{\chr}{\operatorname{char}}
\newcommand{\Aut}{\operatorname{Aut}}
\newcommand{\GL}{\operatorname{GL}}
\newcommand{\im}{\operatorname{im}}
\newcommand{\tr}{\operatorname{tr}}
\newcommand{\id}{\mathbf{id}}
\newcommand{\cl}{\mathbf{cl}}
\newcommand{\Gal}{\operatorname{Gal}}
\newcommand{\Tr}{\operatorname{Tr}}
\newcommand{\sgn}{\operatorname{sgn}}
\newcommand{\Sym}{\operatorname{Sym}}
\newcommand{\Alt}{\operatorname{Alt}}

% Commutative and Homological Algebra
\newcommand{\spec}{\operatorname{spec}}
\newcommand{\mspec}{\operatorname{m-spec}}
\newcommand{\Tor}{\operatorname{Tor}}
\newcommand{\tor}{\operatorname{tor}}
\newcommand{\Ann}{\operatorname{Ann}}
\newcommand{\Supp}{\operatorname{Supp}}
\newcommand{\Hom}{\operatorname{Hom}}
\newcommand{\End}{\operatorname{End}}
\newcommand{\coker}{\operatorname{coker}}
\newcommand{\limit}{\varprojlim}
\newcommand{\colimit}{%
  \mathop{\mathpalette\colimit@{\rightarrowfill@\textstyle}}\nmlimits@
}
\makeatother


\newcommand{\fraka}{\mathfrak{a}} % ideal
\newcommand{\frakb}{\mathfrak{b}} % ideal
\newcommand{\frakc}{\mathfrak{c}} % ideal
\newcommand{\frakf}{\mathfrak{f}} % face map
\newcommand{\frakg}{\mathfrak{g}}
\newcommand{\frakh}{\mathfrak{h}}
\newcommand{\frakm}{\mathfrak{m}} % maximal ideal
\newcommand{\frakn}{\mathfrak{n}} % naximal ideal
\newcommand{\frakp}{\mathfrak{p}} % prime ideal
\newcommand{\frakq}{\mathfrak{q}} % qrime ideal
\newcommand{\fraks}{\mathfrak{s}}
\newcommand{\frakt}{\mathfrak{t}}
\newcommand{\frakz}{\mathfrak{z}}
\newcommand{\frakA}{\mathfrak{A}}
\newcommand{\frakI}{\mathfrak{I}}
\newcommand{\frakJ}{\mathfrak{J}}
\newcommand{\frakK}{\mathfrak{K}}
\newcommand{\frakL}{\mathfrak{L}}
\newcommand{\frakN}{\mathfrak{N}} % nilradical 
\newcommand{\frakO}{\mathfrak{O}} % dedekind domain
\newcommand{\frakP}{\mathfrak{P}} % Prime ideal above
\newcommand{\frakQ}{\mathfrak{Q}} % Qrime ideal above 
\newcommand{\frakR}{\mathfrak{R}} % jacobson radical
\newcommand{\frakU}{\mathfrak{U}}
\newcommand{\frakV}{\mathfrak{V}}
\newcommand{\frakW}{\mathfrak{W}}
\newcommand{\frakX}{\mathfrak{X}}

% General/Differential/Algebraic Topology 
\newcommand{\scrA}{\mathscr{A}}
\newcommand{\scrB}{\mathscr{B}}
\newcommand{\scrF}{\mathscr{F}}
\newcommand{\scrM}{\mathscr{M}}
\newcommand{\scrN}{\mathscr{N}}
\newcommand{\scrP}{\mathscr{P}}
\newcommand{\scrO}{\mathscr{O}} % sheaf
\newcommand{\scrR}{\mathscr{R}}
\newcommand{\scrS}{\mathscr{S}}
\newcommand{\bbH}{\mathbb H}
\newcommand{\Int}{\operatorname{Int}}
\newcommand{\psimeq}{\simeq_p}
\newcommand{\wt}[1]{\widetilde{#1}}
\newcommand{\RP}{\mathbb{R}\text{P}}
\newcommand{\CP}{\mathbb{C}\text{P}}

% Miscellaneous
\newcommand{\wh}[1]{\widehat{#1}}
\newcommand{\calM}{\mathcal{M}}
\newcommand{\calP}{\mathcal{P}}
\newcommand{\onto}{\twoheadrightarrow}
\newcommand{\into}{\hookrightarrow}
\newcommand{\Gr}{\operatorname{Gr}}
\newcommand{\Span}{\operatorname{Span}}
\newcommand{\ev}{\operatorname{ev}}
\newcommand{\weakto}{\stackrel{w}{\longrightarrow}}

\newcommand{\define}[1]{\textcolor{blue}{\textit{#1}}}
\newcommand{\caution}[1]{\textcolor{red}{\textit{#1}}}
\newcommand{\important}[1]{\textcolor{red}{\textit{#1}}}
\renewcommand{\mod}{~\mathrm{mod}~}
\renewcommand{\le}{\leqslant}
\renewcommand{\leq}{\leqslant}
\renewcommand{\ge}{\geqslant}
\renewcommand{\geq}{\geqslant}
\newcommand{\Res}{\operatorname{Res}}
\newcommand{\floor}[1]{\left\lfloor #1\right\rfloor}
\newcommand{\ceil}[1]{\left\lceil #1\right\rceil}
\newcommand{\gl}{\mathfrak{gl}}
\newcommand{\ad}{\operatorname{ad}}
\newcommand{\Stab}{\operatorname{Stab}}
\newcommand{\bfX}{\mathbf{X}}
\newcommand{\Ind}{\operatorname{Ind}}
\newcommand{\bfG}{\mathbf{G}}
\newcommand{\rank}{\operatorname{rank}}
\newcommand{\calo}{\mathcal{o}}
\newcommand{\frako}{\mathfrak{o}}
\newcommand{\Cl}{\operatorname{Cl}}

\newcommand{\idim}{\operatorname{idim}}
\newcommand{\pdim}{\operatorname{pdim}}
\newcommand{\Ext}{\operatorname{Ext}}
\newcommand{\co}{\operatorname{co}}
\newcommand{\bfO}{\mathbf{O}}
\newcommand{\bfF}{\mathbf{F}} % Fitting Subgroup
\newcommand{\Syl}{\operatorname{Syl}}
\newcommand{\nor}{\vartriangleleft}
\newcommand{\noreq}{\trianglelefteqslant}
\newcommand{\subnor}{\nor\!\nor}
\newcommand{\Soc}{\operatorname{Soc}}
\newcommand{\core}{\operatorname{core}}
\newcommand{\Sd}{\operatorname{Sd}}
\newcommand{\mesh}{\operatorname{mesh}}
\newcommand{\sminus}{\setminus}
\newcommand{\diam}{\operatorname{diam}}

\geometry {
    margin = 1in
}

\titleformat
{\section}
[block]
{\Large\bfseries\scshape}
{\S\thesection}
{0.5em}
{\centering}
[]


\titleformat
{\subsection}
[block]
{\normalfont\bfseries\sffamily}
{\S\S}
{0.5em}
{\centering}
[]


\begin{document}
\maketitle

\section{Covering Spaces}

\begin{lemma}[Uniqueness of Lifts]\thlabel{lem:uniqueness-of-lift}
    Let $p: E\to B$ be a covering map and $f: X\to B$ a continuous map from a connected topological space $X$. If $\wt f: X\to E$ is a lift of $f$, then it is unique.
\end{lemma}
\begin{proof}
    
\end{proof}

\begin{lemma}[Path Lifting]
    Let $p: E\to B$ be a covering map, let $p(e_0) = b_0$. Any path $f: [0, 1]\to B$ beginning at $b_0$ has a unique lifting to a path $\wt f: [0,1]\to E$ beginning at $e_0$.
\end{lemma}
\begin{proof}
    The uniqueness of the lift follows from \thref{lem:uniqueness-of-lift}. Begin with an open cover $\mathscr U$ of $B$ such that each $U\in\mathscr U$ is evenly covered by $p$. The collection $\{f^{-1}U\colon U\in\mathscr U\}$ is an open cover of $[0, 1]$. Using the Lebesgue Number Lemma, we can subdivide $[0, 1]$ as
    \begin{equation*}
        0 = s_0 < s_1 < \dots < s_n = 1
    \end{equation*}
    such that $[s_i, s_{i + 1}]\in f^{-1}U$ for some $U\in\mathscr U$. 
    
    We construct the lift $\wt f$ inductively. Suppose $\wt f$ has been defined on $[s_0, s_i]$ for $0\le i\le n - 1$. Choose $U\in\mathscr U$ such that $[s_i, s_{i + 1}]\subseteq f^{-1}U$. There is a unique open set $V\subseteq p^{-1}U$ such that $\wt f(s_i)\in V$ and $p|_V: V\to U$ is a homeomorphism. Define $\wt f$ to be $p|_V^{-1}\circ f$ on $[s_i, s_{i + 1}]$, on which it is obviously continuous. Due to the pasting lemma, we obtain a continuous function $\wt f$ on $[s_0, s_{i + 1}]$.
\end{proof}

\begin{lemma}[Homotopy Lifting]
    Let $p: E\to B$ be a covering map, $p(e_0) = b_0$, and $F: I\times I\to B$ be a homotopy with $F(0, 0) = b_0$. There is a unique lifting of $F$ to a continuous map $\wt F: I\times I\to E$ such that $\wt F(0, 0) = e_0$. If $F$ is a path homotopy, then $\wt F$ is a path homotopy.
\end{lemma}
\begin{proof}
    Again, the uniqueness of the lift follows from \thref{lem:uniqueness-of-lift}. Begin with an oepn cover $\mathscr U$ of $B$ such that each $U\in\mathscr U$ is evenly covered by $p$. The collection $\{F^{-1}U\colon U\in\mathscr U\}$ is an open cover of $I\times I$. Using the Lebesgue Number Lemma, we can find subdivisions 
    \begin{equation*}
        0 = s_0 < s_1 < \dots < s_n = 1\quad\text{ and }\quad 0 = t_0 < t_1 < \dots < t_n = 1,
    \end{equation*}
    such that the image of each rectangle $R_{i, j} = [s_i, s_{i + 1}]\times[t_i, t_{i + 1}]$ under $f$ is contained in some $U\in\mathscr U$. As in the preceding proof, we shall construct the lift $\wt F$ inductively in the following fashion: 
    \begin{equation*}
        R_{0, 0},\dots, R_{n - 1, 0}, R_{0, 1},\dots, R_{n - 1, 1},\dots R_{n - 1, n - 1}.
    \end{equation*}
    First, using path lifting, we can define $\wt F$ on $I\times\{0\}$ and $\{0\}\times I$. Due to the pasting lemma, $\wt F$ is continuous on $I\times\{0\}\cup\{0\}\times I$. 

    We shall now describe how to define $\wt F$ on $R_{i, j}$ given that $\wt F$ has been defined for all preceding rectangles in the sequence. Let $A$ denote the union of these rectangles and the left and bottom edges of $I\times I$. In particular, this means that $\wt F$ has been defined on the left and bottom edges of $R_{i, j}$; let their union be denoted by $L$. Let $U\in\mathscr U$ be such that $R_{i, j}\subseteq F^{-1}U$. Since $U$ is an evenly covered neighborhood, we can write $p^{-1}U = \bigsqcup V_\alpha$, where $p|_{V_\alpha}: V_\alpha\to U$ is a homeomorphism for each $\alpha$. The since $L$ is connected, its image under $\wt F$ is connected and contained in $\bigsqcup V_\alpha$. Let $V_\alpha$ be the unique slice over $U$ containing $\wt F(L)$. Finally, define $\wt F = p|_{V_\alpha}^{-1}\circ F$ on $R_{i, j}$. Due to the pasting lemma, it is clear that this extension of $\wt F$ is continuous on $A\cup R_{i, j}$. This constructs a lift $\wt F: I\times I\to E$, as required.

    Finally, if $F$ is a path homotopy, then $F$ is constant on $\{0\}\times I$ and $\{1\}\times I$, both of which are connected. Further, since $\wt F$ lifts $F$, the image $\wt F\left(\{0\}\times I\right)$ is contained in $p^{-1}(b_0)$, which is a discrete, and in particular, totally disconnected set. It follows that $\wt F$ is constant on $\{0\}\times I$. Similarly one can argue that $\wt F$ is constant on $\{1\}\times I$. Hence, $\wt F$ is a path homotopy too.
\end{proof}

\begin{corollary}
    Let $p: E\to B$ be a covering map and $p(e_0) = b_0$. Let $f, g: I\to B$ be two paths in $B$ from $b_0$ to $b_1$ and let $\wt f, \wt g: I\to B$ be their (unique) lifts to paths in $E$ beginning at $e_0$. If $f$ and $g$ are path homotopic, then $\wt f$ and $\wt g$ end at the same point and are path homotopic.
\end{corollary}
\begin{proof}
    
\end{proof}

\section{Classification of Covering Spaces}

\subsection{Existence of Covers}

\begin{theorem}
    Let $B$ be a path connected, locally path connected, semilocally simply connected topological space and $b_0\in B$ a basepoint. Given a subgroup $H\subseteq\pi_1(B, b_0)$, there is a path connected covering $p: E\to B$ and $e_0\in p^{-1}(b_0)$ such that 
    \begin{equation*}
        p_\ast\left(\pi_1(E_0, e_0)\right) = H.
    \end{equation*}
\end{theorem}
\begin{proof}
The construction proceeds in several bite-sized steps. 

\noindent\emph{Step 1. Construction of $E$.} Let $\scrP$ denote the set of all paths in $B$ that begin at $b_0$. Define $\alpha\sim\beta$ in $\scrP$ if and only if 
\begin{equation*}
    \alpha(1) = \beta(1)\quad\text{ and }\quad [\alpha\ast\overline\beta]\in H.
\end{equation*}
It is clear that the relation $\sim$ is an equivalence relation. Let $E = \scrP/\sim$ be the set of all equivalence classes under this relation. For $\alpha\in\scrP$, we use $\alpha^\sharp$ to denote its equivalence class in $E$. Define the map $p: E\to B$ by $p(\alpha^\sharp) = \alpha(1)$.

Before we proceed, we note two observations:
\begin{enumerate}[label=(\arabic*)]
    \item If $[\alpha] = [\beta]$ as elements of the fundamental groupoid, then $\alpha^\sharp = \beta^\sharp$. Indeed, since $[\alpha\ast\overline\beta] = e_{b_0}\in H$.
    \item If $\alpha^\sharp = \beta^\sharp$, then $(\alpha\ast\delta)^\sharp = (\beta\ast\delta)^\sharp$ for any path $\delta$ beginning at $\alpha(1) = \beta(1)$. Again, this is straightforward, since 
    \begin{equation*}
        \left[(\alpha\ast\delta)\ast\overline{(\beta\ast\delta)}\right] = [\alpha\ast\overline\beta]\in H.
    \end{equation*}
\end{enumerate}

\noindent\emph{Step 2. Topologizing $E$.} Let $\alpha\in\scrP$ and let $U$ be a path connected neighborhood of $\alpha(1)$. Define 
\begin{equation*}
    B(U,\alpha) = \left\{(\alpha\ast\delta)^\sharp\colon \delta\text{ is a path in $U$ beginning at }\alpha(1)\right\}.
\end{equation*}
Obviously, $\alpha^\sharp\in B(U,\alpha)$, which can be seen by taking $\delta$ to be the constant path at $\alpha(1)$. We contend that the sets $B(U,\alpha)$ form a basis for a topology on $E$.

First, if $\beta^\sharp\in B(U,\alpha)$, then $\beta^\sharp = (\alpha\ast\delta)^\sharp$ for some path $\delta$ in $U$ beginning at $\alpha(1)$. Now, using the aforementioned observations, 
\begin{equation*}
    (\beta\ast\overline\delta)^\sharp = \left(\alpha\ast\delta\ast\overline\delta\right)^\sharp = \alpha^\sharp,
\end{equation*}
where the first equality follows from (2) while the second equality follows from (1). Consequently, $\alpha^\sharp\in B(U,\beta)$. Next, we show that $B(U,\beta)\subseteq B(U,\alpha)$. Indeed, any element of $B(U,\beta)$ is of the form $(\beta\ast\gamma)^\sharp$ for some path $\gamma$ in $U$ beginning at $\beta(1)$. But since $\beta^\sharp = (\alpha\ast\delta)^\sharp$, using observation (2) above, we have 
\begin{equation*}
    (\beta\ast\gamma)^\sharp = (\alpha\ast\delta\ast\gamma)^\sharp\in B(U,\alpha).
\end{equation*}
Hence $B(U,\beta)\subseteq B(U,\alpha)$. But we argued that $\alpha^\sharp\in B(U,\beta)$, and hence, $B(U,\alpha)\subseteq B(U,\beta)$, whence $B(U,\beta) = B(U,\alpha)$, whenever $\beta^\sharp\in B(U,\alpha)$.

Finally, we show that the sets $B(U,\alpha)$ form a basis for a topology on $E$. Indeed, suppose $\beta^\sharp\in B(U_1,\alpha_1)\cap B(U_2,\alpha_2)$. Then, by definition, $\beta(1)\in U_1\cap U_2$. Since $B$ is locally path connected, there is a path connected neighborhood $V$ of $\beta(1)$ contained in $U_1\cap U_2$. It is clear that $B(V,\beta)\subseteq B(U_i,\beta) = B(U_i,\alpha_i)$ for $i\in\{1, 2\}$ due to the conclusion of the preceding paragraph. It follows hence that the $B(U,\alpha)$'s form a basis for a topology on $E$.

\noindent\emph{Step 3. $p$ is a continuous open map.} It is easy to see that the image of $B(U, \alpha)$ lies in $U$, conversely, given any $x\in U$, there is a path $\delta$ in $U$ from $\alpha(1)$ to $x$, whence the image of $(\alpha\ast\delta)^\sharp$ under $p$ is $x$. It follows that the image of $B(U,\alpha)$ under $p$ is all of $U$. Hence $p$ is an open map.

To show continuity of $p$, we show that it is continuous at each $\alpha^\sharp\in E$. Let $b = p(\alpha^\sharp)$ and let $W$ be a neighborhood of $b$ in $B$. Since $B$ is locally path connected, there is a path connected neighborhood $U$ of $b$ contained in $W$. Then as we have seen, $B(U, \alpha)$ is a neighborhood of $\alpha^\sharp$ that maps to $U$ under $p$, whence $p$ is continuous at $\alpha^\sharp$; consequently, $p$ is a continuous open map, as desired. 

\noindent\emph{Step 4. $p$ is a covering map.} We shall show that every $b\in B$ has an evenly covered neighborhood. Since $B$ is semilocally simply connected, there is a neighborhood $U$ of $b$ such that the induced map $\pi_1(U, b)\to\pi_1(B,b)$ is trivial. Let $S$ denote the set of all paths in $B$ from $b_0$ to $b$. We shall show that $p^{-1}(U)$ is the union of all $B(U,\alpha)$ where $\alpha\in S$.

Obviously, the inclusion $\displaystyle\bigcup_{\alpha\in S} B(U,\alpha)\subseteq p^{-1}(U)$. Conversely, if $\beta^\sharp\in p^{-1}(U)$, then $\beta(1)\in U$. Choose a path $\delta$ in $U$ from $b$ to $\beta(1)$, and let $\alpha$ be the path $\beta\ast\overline\delta\in\scrP$. Then $[\beta] = [\alpha\ast\delta]$, so that $\beta^\sharp = (\alpha\ast\delta)^\sharp$, that is, $\beta\in B(U,\alpha)$. It follows that 
\begin{equation*}
    p^{-1}(U) = \bigcup_{\alpha\in S} B(U,\alpha).
\end{equation*}
Further, note that if $B(U,\alpha_1)\cap B(U,\alpha_2)$ is non-empty, then as we have seen in \emph{Step 2}, $B(U, \alpha_1) = B(U, \alpha_2)$. It follows that $p^{-1}(U)$ is a disjoint union of $B(U, \alpha)$ where $\alpha$ ranges over a suitable subset of $S$.

Finally to show that $p$ is a covering map, we must show that the restriction $p: B(U, \alpha)\to U$ is a homeomorphism, where $\alpha\in S$. Since we know that it is a continuous open map, it suffices to show that it is a bijection. Surjectivity has already been shown so we must only establish injectivity. Indeed, suppose 
\begin{equation*}
    p\left((\alpha\ast\delta_1)^\sharp\right) = p\left((\alpha\ast\delta_2)^\sharp\right)
\end{equation*}
for some paths $\delta_1,\delta_2$ in $U$ that begin at $b = \alpha(1)$. Of course, we must have $\delta_1(1) = \delta_2(1)$. Further, since the inclusion induced map $\pi_1(U, b)\to \pi_1(B, b)$ is trivial, the loop $\delta_1\ast\overline\delta_2$ is path homotopic to the constant loop $e_b$ in $B$ based at $b$; consequently, $[\delta_1] = [\delta_2]$ in $B$, and hence $[\alpha\ast\delta_1] = [\alpha\ast\delta_2]$ in $B$. Using observation (1), we get that $(\alpha\ast\delta_1)^\sharp = (\alpha\ast\delta_2)^\sharp$, thereby proving injectivity. This gives us our original desideratum.

\emph{Step 5. Lifting paths to $E$.} Let $e_0\in E$ denote $(e_{b_0})^\sharp$, the equivalence class of the constant path at $b_0$. Let $\alpha: I\to B$ be a path in $B$ beginning at $b_0$. We shall construct a continuous lift $\wt\alpha: I\to E$ beginning at $e_0$. 

For $c\in[0, 1]$, let $\alpha_c: I\to B$ be given by $\alpha_c(t) = \alpha(ct)$. This is a path in $B$ beginning at $b_0$ and ending at $\alpha(c)$. Set $\wt\alpha(c) = (\alpha_c)^\sharp\in E$. Note that 
\begin{equation*}
    p\left(\wt\alpha(c)\right) = \alpha_c(1) = \alpha(c)\implies\alpha = p\circ\wt\alpha,
\end{equation*}
whence $\wt\alpha$ is indeed a lifting of $\alpha$. It remains to show that $\wt\alpha$ is a continuous map.

Indeed, let $c\in [0, 1]$ and choose a basic neighborhood $B(U,\alpha_c)$ of $\wt\alpha(c) = (\alpha_c)^\sharp$; where $U$ is a path connected neighborhood of $\alpha_c(1) = \alpha(c)$. Since $\alpha$ is continuous, there is an $\varepsilon > 0$ such that $\alpha(t)\in U$ whenever $|c - t| < \varepsilon$. Now, for any $d\in[0, 1]$ with $|c - d| < \varepsilon$, let $\delta: I\to B$ be the path given by 
\begin{equation*}
    \delta(t) = \alpha\left((1 - t)c + td\right).
\end{equation*}
It is easy to see that $[\alpha_c\ast\delta] = [\alpha_d]$, since the former is just a reparametrization of the latter. Consequently, using observation (1), 
\begin{equation*}
    \wt\alpha(d) = (\alpha_d)^\sharp = (\alpha_c\ast\delta)^\sharp = \alpha_d^\sharp\implies\alpha_d^\sharp\in B(U,\alpha_c).
\end{equation*}
This shows that $\wt\alpha$ is continuous, as desired.

\emph{Step 6. $E$ is path connected.} If $\alpha^\sharp\in E$, then the path $\alpha: I\to B$ lifts to a path $\wt\alpha: I\to E$ from $e_0$ to $\alpha^\sharp$, thereby establishing path connectedness.

\emph{Step 7. $p_\ast\left(\pi_1(E, e_0)\right) = H$.} Due to the uniqueness of path liftings (given an initial point), any loop in $E$ based at $e_0$ is of the form $\wt\alpha$ for some loop $\alpha$ in $B$ based at $b_0$. Note that $$(e_{b_0})^\sharp = e_0 = \wt\alpha(1) = \alpha^\sharp,$$
and hence, $[\alpha\ast\overline e_{b_0}]\in H$, that is, $[\alpha]\in H$. This shows that $p_\ast\left(\pi_1(E, e_0)\right)\subseteq H$.

Conversely, if $[\alpha]\in H$, then again $\alpha^\sharp = (e_{b_0})^\sharp$, since $[\alpha\ast\overline e_{b_0}]\in H$. Consequently, $\wt\alpha$ is a loop in $E$ based at $e_0$. It follows hence that $[\alpha]\in p_\ast\left(\pi_1(E, e_0)\right)$, consequently, $H = p_\ast\left(\pi_1(E, e_0)\right)$, as desired.
\end{proof}
\end{document}