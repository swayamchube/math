\documentclass[10pt]{article}

% \usepackage{./arxiv}

\title{Gorenstein Rings\\\small{Notes for the course MA 842: Topics in Algebra II}}
\author{H. Ananthnarayan\\\small\textsc{Scribe:} Swayam Chube}
\date{Last Updated: March 7, 2025}

\usepackage[utf8]{inputenc} % allow utf-8 input
\usepackage[T1]{fontenc}    % use 8-bit T1 fonts
\usepackage{hyperref}       % hyperlinks
\usepackage{url}            % simple URL typesetting
\usepackage{booktabs}       % professional-quality tables
\usepackage{amsfonts}       % blackboard math symbols
\usepackage{nicefrac}       % compact symbols for 1/2, etc.
\usepackage{microtype}      % microtypography
\usepackage{graphicx}
\usepackage{natbib}
\usepackage{doi}
\usepackage{amssymb}
\usepackage{bbm}
\usepackage{amsthm}
\usepackage{amsmath}
\usepackage{xcolor}
\usepackage{theoremref}
\usepackage{enumitem}
\usepackage{mathpazo}
% \usepackage{euler}
\usepackage{mathrsfs}
\setlength{\marginparwidth}{2cm}
\usepackage{todonotes}
\usepackage{stmaryrd}
\usepackage[all,cmtip]{xy} % For diagrams, praise the Freyd–Mitchell theorem 
\usepackage{marvosym}
\usepackage{geometry}
\usepackage{titlesec}
\usepackage{tikz}
\usetikzlibrary{cd}

\renewcommand{\qedsymbol}{$\blacksquare$}

% Uncomment to override  the `A preprint' in the header
% \renewcommand{\headeright}{}
% \renewcommand{\undertitle}{}
% \renewcommand{\shorttitle}{}

\hypersetup{
    pdfauthor={Lots of People},
    colorlinks=true,
}

\newtheoremstyle{thmstyle}%               % Name
  {}%                                     % Space above
  {}%                                     % Space below
  {}%                             % Body font
  {}%                                     % Indent amount
  {\bfseries\scshape}%                            % Theorem head font
  {.}%                                    % Punctuation after theorem head
  { }%                                    % Space after theorem head, ' ', or \newline
  {\thmname{#1}\thmnumber{ #2}\thmnote{ (#3)}}%                                     % Theorem head spec (can be left empty, meaning `normal')

\newtheoremstyle{defstyle}%               % Name
  {}%                                     % Space above
  {}%                                     % Space below
  {}%                                     % Body font
  {}%                                     % Indent amount
  {\bfseries\scshape}%                            % Theorem head font
  {.}%                                    % Punctuation after theorem head
  { }%                                    % Space after theorem head, ' ', or \newline
  {\thmname{#1}\thmnumber{ #2}\thmnote{ (#3)}}%                                     % Theorem head spec (can be left empty, meaning `normal')

\theoremstyle{thmstyle}
\newtheorem{theorem}{Theorem}[section]
\newtheorem{lemma}[theorem]{Lemma}
\newtheorem{proposition}[theorem]{Proposition}

\theoremstyle{defstyle}
\newtheorem{definition}[theorem]{Definition}
\newtheorem{corollary}[theorem]{Corollary}
\newtheorem{porism}[theorem]{Porism}
\newtheorem{remark}[theorem]{Remark}
\newtheorem{interlude}[theorem]{Interlude}
\newtheorem{example}[theorem]{Example}
\newtheorem*{notation}{Notation}

% Common Algebraic Structures
\newcommand{\R}{\mathbb{R}}
\newcommand{\Q}{\mathbb{Q}}
\newcommand{\Z}{\mathbb{Z}}
\newcommand{\N}{\mathbb{N}}
\newcommand{\bbC}{\mathbb{C}} 
\newcommand{\K}{\mathbb{K}} % Base field which is either \R or \bbC
\newcommand{\calA}{\mathcal{A}} % Banach Algebras
\newcommand{\calB}{\mathcal{B}} % Banach Algebras
\newcommand{\calI}{\mathcal{I}} % ideal in a Banach algebra
\newcommand{\calJ}{\mathcal{J}} % ideal in a Banach algebra
\newcommand{\frakM}{\mathfrak{M}} % sigma-algebra
\newcommand{\calO}{\mathcal{O}} % Ring of integers
\newcommand{\bbA}{\mathbb{A}} % Adele (or ring thereof)
\newcommand{\bbI}{\mathbb{I}} % Idele (or group thereof)

% Categories
\newcommand{\catTopp}{\mathbf{Top}_*}
\newcommand{\catGrp}{\mathbf{Grp}}
\newcommand{\catTopGrp}{\mathbf{TopGrp}}
\newcommand{\catSet}{\mathbf{Set}}
\newcommand{\catTop}{\mathbf{Top}}
\newcommand{\catRing}{\mathbf{Ring}}
\newcommand{\catCRing}{\mathbf{CRing}} % comm. rings
\newcommand{\catMod}{\mathbf{Mod}}
\newcommand{\catMon}{\mathbf{Mon}}
\newcommand{\catMan}{\mathbf{Man}} % manifolds
\newcommand{\catDiff}{\mathbf{Diff}} % smooth manifolds
\newcommand{\catAlg}{\mathbf{Alg}}
\newcommand{\catRep}{\mathbf{Rep}} % representations 
\newcommand{\catVec}{\mathbf{Vec}}

% Group and Representation Theory
\newcommand{\chr}{\operatorname{char}}
\newcommand{\Aut}{\operatorname{Aut}}
\newcommand{\GL}{\operatorname{GL}}
\newcommand{\im}{\operatorname{im}}
\newcommand{\tr}{\operatorname{tr}}
\newcommand{\id}{\mathbf{id}}
\newcommand{\cl}{\mathbf{cl}}
\newcommand{\Gal}{\operatorname{Gal}}
\newcommand{\Tr}{\operatorname{Tr}}
\newcommand{\sgn}{\operatorname{sgn}}
\newcommand{\Sym}{\operatorname{Sym}}
\newcommand{\Alt}{\operatorname{Alt}}

% Commutative and Homological Algebra
\newcommand{\spec}{\operatorname{spec}}
\newcommand{\mspec}{\operatorname{m-spec}}
\newcommand{\Tor}{\operatorname{Tor}}
\newcommand{\tor}{\operatorname{tor}}
\newcommand{\Ann}{\operatorname{Ann}}
\newcommand{\Supp}{\operatorname{Supp}}
\newcommand{\Hom}{\operatorname{Hom}}
\newcommand{\End}{\operatorname{End}}
\newcommand{\coker}{\operatorname{coker}}
\newcommand{\limit}{\varprojlim}
\newcommand{\colimit}{%
  \mathop{\mathpalette\colimit@{\rightarrowfill@\textstyle}}\nmlimits@
}
\makeatother


\newcommand{\fraka}{\mathfrak{a}} % ideal
\newcommand{\frakb}{\mathfrak{b}} % ideal
\newcommand{\frakc}{\mathfrak{c}} % ideal
\newcommand{\frakf}{\mathfrak{f}} % face map
\newcommand{\frakg}{\mathfrak{g}}
\newcommand{\frakh}{\mathfrak{h}}
\newcommand{\frakm}{\mathfrak{m}} % maximal ideal
\newcommand{\frakn}{\mathfrak{n}} % naximal ideal
\newcommand{\frakp}{\mathfrak{p}} % prime ideal
\newcommand{\frakq}{\mathfrak{q}} % qrime ideal
\newcommand{\fraks}{\mathfrak{s}}
\newcommand{\frakt}{\mathfrak{t}}
\newcommand{\frakz}{\mathfrak{z}}
\newcommand{\frakA}{\mathfrak{A}}
\newcommand{\frakI}{\mathfrak{I}}
\newcommand{\frakJ}{\mathfrak{J}}
\newcommand{\frakK}{\mathfrak{K}}
\newcommand{\frakL}{\mathfrak{L}}
\newcommand{\frakN}{\mathfrak{N}} % nilradical 
\newcommand{\frakO}{\mathfrak{O}} % dedekind domain
\newcommand{\frakP}{\mathfrak{P}} % Prime ideal above
\newcommand{\frakQ}{\mathfrak{Q}} % Qrime ideal above 
\newcommand{\frakR}{\mathfrak{R}} % jacobson radical
\newcommand{\frakU}{\mathfrak{U}}
\newcommand{\frakV}{\mathfrak{V}}
\newcommand{\frakW}{\mathfrak{W}}
\newcommand{\frakX}{\mathfrak{X}}

% General/Differential/Algebraic Topology 
\newcommand{\scrA}{\mathscr{A}}
\newcommand{\scrB}{\mathscr{B}}
\newcommand{\scrF}{\mathscr{F}}
\newcommand{\scrM}{\mathscr{M}}
\newcommand{\scrN}{\mathscr{N}}
\newcommand{\scrP}{\mathscr{P}}
\newcommand{\scrO}{\mathscr{O}} % sheaf
\newcommand{\scrR}{\mathscr{R}}
\newcommand{\scrS}{\mathscr{S}}
\newcommand{\bbH}{\mathbb H}
\newcommand{\Int}{\operatorname{Int}}
\newcommand{\psimeq}{\simeq_p}
\newcommand{\wt}[1]{\widetilde{#1}}
\newcommand{\RP}{\mathbb{R}\text{P}}
\newcommand{\CP}{\mathbb{C}\text{P}}

% Miscellaneous
\newcommand{\wh}[1]{\widehat{#1}}
\newcommand{\calM}{\mathcal{M}}
\newcommand{\calP}{\mathcal{P}}
\newcommand{\onto}{\twoheadrightarrow}
\newcommand{\into}{\hookrightarrow}
\newcommand{\Gr}{\operatorname{Gr}}
\newcommand{\Span}{\operatorname{Span}}
\newcommand{\ev}{\operatorname{ev}}
\newcommand{\weakto}{\stackrel{w}{\longrightarrow}}

\newcommand{\define}[1]{\textcolor{blue}{\textit{#1}}}
\newcommand{\caution}[1]{\textcolor{red}{\textit{#1}}}
\newcommand{\important}[1]{\textcolor{red}{\textit{#1}}}
\renewcommand{\mod}{~\mathrm{mod}~}
\renewcommand{\le}{\leqslant}
\renewcommand{\leq}{\leqslant}
\renewcommand{\ge}{\geqslant}
\renewcommand{\geq}{\geqslant}
\newcommand{\Res}{\operatorname{Res}}
\newcommand{\floor}[1]{\left\lfloor #1\right\rfloor}
\newcommand{\ceil}[1]{\left\lceil #1\right\rceil}
\newcommand{\gl}{\mathfrak{gl}}
\newcommand{\ad}{\operatorname{ad}}
\newcommand{\Stab}{\operatorname{Stab}}
\newcommand{\bfX}{\mathbf{X}}
\newcommand{\Ind}{\operatorname{Ind}}
\newcommand{\bfG}{\mathbf{G}}
\newcommand{\rank}{\operatorname{rank}}
\newcommand{\calo}{\mathcal{o}}
\newcommand{\frako}{\mathfrak{o}}
\newcommand{\Cl}{\operatorname{Cl}}

\newcommand{\idim}{\operatorname{idim}}
\newcommand{\pdim}{\operatorname{pdim}}
\newcommand{\Ext}{\operatorname{Ext}}
\newcommand{\co}{\operatorname{co}}
\newcommand{\bfO}{\mathbf{O}}
\newcommand{\bfF}{\mathbf{F}} % Fitting Subgroup
\newcommand{\Syl}{\operatorname{Syl}}
\newcommand{\nor}{\vartriangleleft}
\newcommand{\noreq}{\trianglelefteqslant}
\newcommand{\subnor}{\nor\!\nor}
\newcommand{\Soc}{\operatorname{Soc}}
\newcommand{\core}{\operatorname{core}}
\newcommand{\Sd}{\operatorname{Sd}}
\newcommand{\mesh}{\operatorname{mesh}}
\newcommand{\sminus}{\setminus}
\newcommand{\diam}{\operatorname{diam}}
\newcommand{\Ass}{\operatorname{Ass}}

% \renewcommand{\familydefault}{\sfdefault} % Do you want this font? 

\geometry {
    margin = 1in
}

\titleformat
{\section}
[block]
{\Large\bfseries\scshape}
{\S\thesection}
{0.5em}
{\centering}
[]


\titleformat
{\subsection}
[block]
{\normalfont\bfseries\sffamily}
{\S\S}
{0.5em}
{\centering}
[]


\begin{document}
\maketitle

\section{Injective Modules}

\subsection{Essential Extensions}

\begin{remark}\thlabel{rem:extension-of-injection}
    Let $M\subseteq N$ be an essential extension of $R$-modules and $\varphi: M\into P$ be an $R$-linear injective map. If $\varphi$ extends to an $R$-linear map $\wt\varphi: N\to P$, then $\wt\varphi$ is injective too. Indeed, if $K = \ker\wt\varphi\ne 0$, then $K\cap M\ne 0$, a contradiction.
\end{remark}

\begin{proposition}\thlabel{prop:essential-over-socle}
    Let $(R,\frakm, k)$ be a Noetherian local ring. Let $M$ be an Artinian $R$-module. Then $\Soc_R(M)\subseteq M$ is an essential extension.
\end{proposition}
\begin{proof}
    Let $0\ne K\subseteq M$ be a submodule. Choose $0\ne x\in K$. Since $M$ is Artinian, the descending chain $Rx\supseteq\frakm x\supseteq\frakm^2 x\supseteq\cdots$ stabilizes. Let $n\ge 0$ be the least positive integer such that $\frakm^n x = \frakm^{n + 1}x$. Due to Nakayama's lemma, $\frakm^n x = 0$, whence $n\ge 1$. It follows that $0\ne\frakm^{n  - 1}x\subseteq\Soc_R(M)\cap K$, as desired.
\end{proof}

\begin{corollary}\thlabel{cor:inj-hull-artinian}
    Let $(R,\frakm, k, E)$ be a Noetherian local ring and $M$ an Artinian $R$-module. If $\dim_k\Soc_R(M) = d$, then $E_R(M)\cong E^{\oplus d}$.
\end{corollary}
\begin{proof}
    Since $\Soc_R(M)\cong k^{\oplus d}$, it is clear that $E_R\left(\Soc_R(M)\right)\cong E^{\oplus d}$. The inclusion $\Soc_R(M)\into E^{\oplus d}$ can be extended to $M$ to obtain a commutative diagram:
    \begin{equation*}
        \xymatrix {
            M\ar@{^{(}->}[rd]\\
            \Soc_R(M)\ar@{^{(}->}[r]\ar@{^{(}->}[u] & E_R\left(\Soc_R(M)\right)\cong E^{\oplus d}
        }
    \end{equation*}
    where all maps are inclusion. It follows that $M\into E^{\oplus d}$ is an essential extension. Since $E^{\oplus d}$ is an injective module, we have that $E_R(M)\cong E^{\oplus d}$.
\end{proof}



\section{Matlis Duality}

\begin{definition}
    Let $(R,\frakm, k, E)$ be a Noetherian local ring. For an $R$-module $M$, set $M^\vee = \Hom_R(M, E)$. This is known as the \define{Matlis dual} of a module.
\end{definition}

Clearly $(-)^\vee$ is a contravariant exact functor on the category of $R$-modules. Note that if $I\subseteq\frakm$ is an ideal, then as we have seen earlier, 
\begin{equation*}
    E_{R/I}(k) = \Hom_R\left(R/I, E\right) = \left(R/I\right)^\vee.
\end{equation*}
In particular, taking $I = \frakm$, we see that $k^\vee \cong k$ as $R$-modules.

\begin{lemma}\thlabel{lem:properties-of-cech}
    Let $(R,\frakm,k, E)$ be a Noetherian local ring. Then 
    \begin{enumerate}[label=(\arabic*)]
        \item If $M\ne 0$, then $M^\vee\ne 0$. 
        \item If $\lambda_R(M) < \infty$, then $\lambda_R(M^\vee)\ne 0$. Moreover, $\lambda_R(M) = \lambda_R(M^\vee)$.
    \end{enumerate}
\end{lemma}
\begin{proof}
\begin{enumerate}[label=(\arabic*)]
    \item Let $0\ne x\in M$. If $I = \Ann_R(x)$, then there is a natural inclusion $R/I\into M$ sending $\overline 1\mapsto x$. Taking the Matlis dual, we have a surjection 
    \begin{equation*}
        M^\vee\onto \left(R/I\right)^\vee = E_{R/I}(k)\ne 0,
    \end{equation*}
    consequently $M^\vee\ne 0$.

    \item We shall prove both statements by induction on $\lambda_R(M)$. If $\lambda_R(M) = 0$, then $M = 0$, so that $M^\vee = 0$ and we get $\lambda_R(M) = 0 = \lambda_R(M^\vee)$. Suppose now that $0 < \lambda_R(M) < \infty$. Then $\frakm\in\Ass_R(M)$, and we have a short exact sequence 
    \begin{equation*}
        0\longrightarrow k \longrightarrow M\longrightarrow N\longrightarrow 0.
    \end{equation*}
    Since length is additive, $\lambda_R(N) = \lambda_R(M) - 1$; hence the induction hypothesis applies and $\lambda_R(N^\vee) = \lambda_R(N)$. Taking the Matlis dual of the above short exact sequence, we have
    \begin{equation*}
        0\longrightarrow N^\vee\longrightarrow M^\vee\longrightarrow k^\vee\longrightarrow 0.
    \end{equation*}
    Since $k^\vee = 0$, we see that 
    \begin{equation*}
        \lambda_R(M^\vee) = \lambda_R(N^\vee) + 1 = \lambda_R(N) + 1 = \lambda_R(M),
    \end{equation*}
    as desired. \qedhere
\end{enumerate}
\end{proof}

\begin{theorem}\thlabel{thm:properties-of-E-artinian}
    Let $(R,\frakm, k, E)$ be an Artinian local ring.
    \begin{enumerate}[label=(\arabic*)]
        \item $E$ is a faithful finite $R$-module.
        \item The map 
        \begin{equation*}
            \mu: R\longrightarrow \Hom_R(E, E)\qquad a\longmapsto\mu_a
        \end{equation*}
        is an isomorphism of $R$-modules and rings.
        \item Given a finite $R$-module $M$, the natural map 
        \begin{equation*}
            \varphi_M: M\longrightarrow M^{\vee\vee}\qquad m\longmapsto\ev_m
        \end{equation*}
        is an isomorphism.
    \end{enumerate}
\end{theorem}
\begin{proof}
\begin{enumerate}[label=(\arabic*)]
    \item Suppose $a\in R$ is such that $aE = 0$. Then 
    \begin{equation*}
        R^\vee = \Hom_R(R, E) = E = (E :_E a)\cong\Hom_R\left(R/aR, E\right) = \left(R/aR\right)^\vee.
    \end{equation*}
    Since $R$ is Artinian, we then have 
    \begin{equation*}
        \lambda_R(R) = \lambda_R(R^\vee) = \lambda_R\left((R/aR)^\vee\right) = \lambda_R(R/aR)\implies\lambda_R(aR) = 0, 
    \end{equation*}
    consequently, $a = 0$, i.e., $E$ is a faithful $R$-module.

    Next, since $R$ is Artinian, $\frakm\in\Ass_R(R)$, consequently, there is an injection $k = R/\frakm\into R$. Due to \thref{rem:extension-of-injection} extends to an inclusion $E\into R$, consequently, $E$ is a finite $R$-module.

    \item First note that $\mu$ is injective due to (1). But note that 
    \begin{equation*}
        \infty > \lambda_R(R) = \lambda_R(R^\vee) = \lambda_R(E) = \lambda_R(E^\vee) = \lambda_R\left(\Hom_R(E, E)\right),
    \end{equation*}
    consequently $\mu$ is an isomorphism.

    \item It suffices to show that $\varphi_M$ is injective since $\lambda_R(M) = \lambda_R(M^{\vee\vee})$. Suppose $0\ne x\in M$ is such that $\varphi_M(x) = 0$, that is, for all $f\in\Hom_R(M, E)$, $f(x) = 0$. Let $I = \Ann_R(x)$. Now, there is a non-zero map 
    \begin{equation*}
        \psi: R/I\onto R/\frakm = k\into E,
    \end{equation*}
    which extends to a non-zero map $f: M\to E$ since $R/I\into M$ through $\overline 1\mapsto x$. Thus, $f(x) = \psi(\overline 1)\ne 0$, a contradiction. \qedhere
\end{enumerate}
\end{proof}

\begin{interlude}[On $\wh R$-modules]\thlabel{inter:modules-over-completion}
    Let $(R,\frakm, k)$ be a local ring and $M$ an $R$-module such that $\Gamma_\frakm(M) = M$. We contend that $M$ is an $\wh R$-module in a natural way. To this end, we need only define $\wh a\cdot m$ for $\wh a\in\wh R$ and $m\in M$. 

    Let $\wh a = (a_1, a_2, \dots)$, where we are using the isomorphism
    \begin{equation*}
        \wh R = \varprojlim R/\frakm^n.
    \end{equation*}
    Since $\Gamma_\frakm(M) = M$, there is a positive integer $n\ge 1$ such that $\frakm^n m = 0$. Hence, for $k\ge n$, we have $a_k\cdot m = a_n\cdot m$, as $a_k - a_n\in\frakm^n$. In light of this, we define $\wh a\cdot m = a_n\cdot m$. We must show that this makes $M$ into an $\wh R$-module.

    Let $m_1, m_2\in M$ and $\wh a = (a_1, a_2, \dots)\in\wh R$. There are positive integers $n_1, n_2\ge 1$ such that $\frakm^{n_1}m_1 = 0 = \frakm^{n_2} m_2$; then $\frakm^n m_1 = 0 = \frakm^n m_2$ for all $n\ge\max\{n_1, n_2\}$. Hence, for all such $n\ge 1$, 
    \begin{equation*}
        \wh a\cdot(m_1 + m_2) = a_n\cdot (m_1 + m_2) =  a_n\cdot m_1 + a_n\cdot m_2 = \wh a\cdot m_1 + \wh a\cdot m_2.
    \end{equation*}
    Next, let $\wh a,\wh b\in\wh R$ and $m\in M$ with 
    \begin{equation*}
        \wh a = (a_1, a_2, \dots)\qquad\text{and}\qquad\wh b = (b_1, b_2, \dots).
    \end{equation*}
    There is a positive integer $n$ such that $\frakm^n m = 0$. Then 
    \begin{equation*}
        (\wh a + \wh b)\cdot m = (a_n + b_n)\cdot m = a_n\cdot m + b_n\cdot m = \wh a\cdot m + \wh b\cdot m.
    \end{equation*}
    Finally, note that $\wh b\cdot m = b_n m$ and $\frakm^n\left(\wh b\cdot m\right) = 0$, so that 
    \begin{equation*}
        \wh a\cdot\left(\wh b\cdot m\right) = \wh a\cdot\left(b_n\cdot m\right) = a_n\cdot\left(b_n\cdot m\right) = (a_nb_n)\cdot m =(\wh a\wh b)\cdot m.
    \end{equation*}
    This shows that $M$ is indeed an $\wh R$-module as described above. Further, since $R\to \wh R$ is the diagonal map, it follows that the $\wh R$-module structure on $M$ agrees with the $R$-module struture through the diagonal map. In particular, this means that:
    \begin{quotation}
        A subset of $M$ is an $R$-submodule if and only if it is an $\wh R$-submodule.
    \end{quotation}
    As a result, $M$ is Noetherian (resp. Artinian) as an $R$-module if and only if it is so as an $\wh R$-module.

\end{interlude}

\begin{interlude}[On maps between $\frakm$-power torsion modules]\thlabel{inter:maps-between-m-power-torsion-modules}
    Again, let $(R,\frakm, k)$ be a local ring and suppose $M$ and $N$ are $R$-modules such that $\Gamma_\frakm(M) = \Gamma_\frakm(N)$. By \thref{inter:modules-over-completion}, we know that they are $\wh R$-modules in a natural way. Let $\varphi: M\to N$ be an $R$-linear map. We contend that $\varphi$ is also $\wh R$-linear. Indeed, let $m\in M$ and $\wh a = (a_1, a_2, \dots)\in\wh R$. There is a positive integer $n\ge 1$ such that $\frakm^n m = 0$, and hence, $\frakm^n\varphi(m) = 0$. It follows that 
    \begin{equation*}
        \varphi(\wh a\cdot m) = \varphi(a_n\cdot m) = a_n\cdot\varphi(m) = \wh a\cdot\varphi(m),
    \end{equation*}
    as desired.
\end{interlude}

\begin{theorem}\thlabel{thm:properties-of-E-noetherian}
    Let $(R,\frakm, k, E)$ be a Noetherian local ring. 
    \begin{enumerate}[label=(\arabic*)]
        \item $\Gamma_{\frakm}(E) = E$, and hence $E$ is an $\wh R$ module and for every $R$-module $M$, $M^\vee$ is $\frakm$-power torsion.
        \item $E\cong E_{\wh R}(k)$ as $\wh R$-modules.
        \item $R^{\vee\vee} = \Hom_R(E, E)\cong\wh R$ as $R$-modules.
        \item $E$ is an Artinian $R$-module.
    \end{enumerate}
\end{theorem}
\begin{proof}
\begin{enumerate}[label=(\arabic*)]
    \item That $E$ is an $\wh R$-module follows immediately from \thref{inter:modules-over-completion}. Finally, $M^\vee = \Hom_R(M, E)$ is $\frakm$-power torsion because $E$ is so.

    \item The containment $k\subseteq E$ is an essential extension of $R$-modules, both of which are $\frakm$-power torsion. Due to \thref{inter:modules-over-completion}, it follows that it is an essential extension of $\wh R$-modules too. Now, due to \thref{rem:extension-of-injection}, there is a commutative diagram of inclusions 
    \begin{equation*}
        \xymatrix {
            E\ar[rd]\\
            k\ar[u]\ar[r] & E_{\wh R}(k),
        }
    \end{equation*}
    where all maps are $\wh R$-linear. It follows that $E\into E_{\wh R}(k)$ is an essential extension of $\wh R$-modules, and consequently, an essential extension of $R$-modules. Since $E$ is $R$-injective, we must have that the inclusion is an isomorphism of $R$-modules. Finally, due to \thref{inter:maps-between-m-power-torsion-modules}, this is an isomorphism of $\wh R$-modules.

    \item \textcolor{red}{TODO: Write this out in gory detail.}

    \item Let $M_1\supseteq M_2\supseteq\cdots$ be a chain of $R$-submodules in $E$. There are commutative diagrams 
    \begin{equation*}
        \xymatrix {
            M_{j + 1}\ar@{^{(}->}[r]^-{\iota_{j + 1}}\ar@{^{(}->}[d] & E\\
            M_j\ar@{^{(}->}[ru]_-{\iota_j}
        }
    \end{equation*}
    whose Matlis dual furnishes commutative diagrams 
    \begin{equation*}
        \xymatrix {
            \wh R = E^\vee\ar@{->>}[r]^-{\varphi_j}\ar@{->>}[rd]_-{\varphi_{j + 1}} & M_{j}^\vee\ar@{->>}[d] \\
            & M_{j + 1}^\vee
        }.
    \end{equation*}
    Note that all Matlis duals are $\frakm$-power torsions and hence due to \thref{inter:maps-between-m-power-torsion-modules}, the $\varphi_j$'s are $\wh R$-linear. Let $I_j = \ker\varphi_j\subseteq\wh R$, which is an ideal. Due to the commutative diagram, it is clear that there is an ascending chain $I_j\subseteq I_{j + 1}$. Since $\wh R$ is Noetherian, this chain stabilizes, say $I_n = I_{n + 1} = \dots$. 
    
    Then due to the first isomorphism theorem, $M_j^\vee\onto M_{j + 1}^\vee$ is an isomorphism for all $j\ge n$. Let $C_j = \coker\left(M_{j + 1}\into M_j\right)$. The exactness of the Matlis dual gives $C_j^\vee = 0$, which, due to \thref{lem:properties-of-cech}, implies that $C_j = 0$, that is, $M_{j + 1}\into M_j$ is an isomorphism for all $j\ge n$, i.e., the descending chain stabilizes, as desired. \qedhere
\end{enumerate}
\end{proof}

\begin{theorem}[Matlis Duality, version 1]
    Let $(R,\frakm,k,E)$ be a Noetherian local ring. Then there is a bijective correspondence 
    \begin{equation*}
        \left\{
        \begin{tabular}{@{} c @{}}
            isomorphism classes of\\
            finitely generated\\ 
            $\wh R$-modules
        \end{tabular}
        \right\}
        \;
        \begin{gathered}
            \overset{(-)^\vee}{\longrightarrow}\\[-2ex]
            \underset{(-)^\vee}{\longleftarrow}
        \end{gathered}
        \;
        \left\{
        \begin{tabular}{@{} c @{}}
            isomorphism classes of\\
            Artinian $R$-modules
        \end{tabular}
        \right\}.
    \end{equation*}
\end{theorem}
\begin{proof}
    Let $M$ be an Artinian $R$-module and let $d = \dim_k\Soc_R(M)$. Due to \thref{cor:inj-hull-artinian}, $E_R(M)\cong E^{\oplus d}$, so that there is an inclusion $M\into E^{\oplus d}$, which upon taking the Matlis dual furnishes an $\wh R$-linear surjection $\wh R^{\oplus d}\onto M^{\vee}$. Thus $M^{\vee}$ is a finite $\wh R$-module. 

    Conversely, suppose $M$ is a finite $\wh R$-module. Thus, there is a surjection $\wh R^{\oplus n}\onto M$. Taking the Matlis dual, we obtain an injection $M^\vee\into\left(\wh R^\vee\right)^{\oplus n}$.
    
    There is a natural ``evaluation map'' $\ev: M\to M^{\vee\vee}$, which we shall show is an isomorphism. That $\ev$ is injective follows in the same way as \thref{thm:properties-of-E-artinian} (3). Next, since $\lambda_R(M) < \infty$, we have that $\lambda_R(M) = \lambda_R(M^\vee) = \lambda_R(M^{\vee\vee})$, whence $\ev$ is an isomorphism.
\end{proof}
\end{document}