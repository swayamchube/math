\documentclass[10pt]{article}

% \usepackage{./arxiv}

\title{Selected Solutions to Lang's \emph{Algebra}}
\author{Swayam Chube}
\date{\today}

\usepackage[utf8]{inputenc} % allow utf-8 input
\usepackage[T1]{fontenc}    % use 8-bit T1 fonts
\usepackage{hyperref}       % hyperlinks
\usepackage{url}            % simple URL typesetting
\usepackage{booktabs}       % professional-quality tables
\usepackage{amsfonts}       % blackboard math symbols
\usepackage{nicefrac}       % compact symbols for 1/2, etc.
\usepackage{microtype}      % microtypography
\usepackage{graphicx}
\usepackage{natbib}
\usepackage{doi}
\usepackage{amssymb}
\usepackage{bbm}
\usepackage{amsthm}
\usepackage{amsmath}
\usepackage{xcolor}
\usepackage{theoremref}
\usepackage{enumitem}
% \usepackage{lmodern}
\usepackage{fouriernc}
% \usepackage{euler}
\usepackage{mathrsfs}
\usepackage{todonotes}
\usepackage{stmaryrd}
\usepackage[all,cmtip]{xy} % For diagrams, praise the Freyd–Mitchell theorem 
\usepackage{marvosym}
\usepackage{geometry}
\usepackage{calligra}
\usepackage{titlesec}
\usepackage{mathtools}

\renewcommand{\qedsymbol}{$\blacksquare$}

% Uncomment to override  the `A preprint' in the header
% \renewcommand{\headeright}{}
% \renewcommand{\undertitle}{}
% \renewcommand{\shorttitle}{}

\hypersetup{
    pdfauthor={Lots of People},
    colorlinks=true,
}

\newtheoremstyle{thmstyle}%               % Name
  {}%                                     % Space above
  {}%                                     % Space below
  {}%                             % Body font
  {}%                                     % Indent amount
  {\bfseries\scshape}%                            % Theorem head font
  {.}%                                    % Punctuation after theorem head
  { }%                                    % Space after theorem head, ' ', or \newline
  {\thmname{#1}\thmnumber{ #2}\thmnote{ (#3)}}%                                     % Theorem head spec (can be left empty, meaning `normal')

\newtheoremstyle{defstyle}%               % Name
  {}%                                     % Space above
  {}%                                     % Space below
  {}%                                     % Body font
  {}%                                     % Indent amount
  {\bfseries\scshape}%                            % Theorem head font
  {.}%                                    % Punctuation after theorem head
  { }%                                    % Space after theorem head, ' ', or \newline
  {\thmname{#1}\thmnumber{ #2}\thmnote{ (#3)}}%                                     % Theorem head spec (can be left empty, meaning `normal')

\theoremstyle{thmstyle}
\newtheorem{theorem}{Theorem}[section]
\newtheorem{lemma}[theorem]{Lemma}
\newtheorem{proposition}[theorem]{Proposition}

\theoremstyle{defstyle}
\newtheorem*{definition}{Definition}
\newtheorem*{corollary}{Corollary}
\newtheorem*{remark}{Remark}
\newtheorem{example}[theorem]{Example}
\newtheorem*{notation}{Notation}
\newtheorem{exercise}{Exercise}[section]
\newtheorem*{claim}{Claim}

% Common Algebraic Structures
\newcommand{\R}{\mathbb{R}}
\newcommand{\Q}{\mathbb{Q}}
\newcommand{\Z}{\mathbb{Z}}
\newcommand{\N}{\mathbb{N}}
\newcommand{\bbC}{\mathbb{C}}
\newcommand{\K}{\mathbb{K}}
\newcommand{\calA}{\mathcal{A}}
\newcommand{\frakM}{\mathfrak{M}}
\newcommand{\calO}{\mathcal{O}}
\newcommand{\bbA}{\mathbb{A}} % affine space or adele ring
\newcommand{\bbP}{\mathbb{P}} % projective space
\newcommand{\bbI}{\mathbb{I}} % idele group

% Categories
\newcommand{\catTopp}{\mathbf{Top}_*}
\newcommand{\catGrp}{\mathbf{Grp}}
\newcommand{\catTopGrp}{\mathbf{TopGrp}}
\newcommand{\catSet}{\mathbf{Set}}
\newcommand{\catTop}{\mathbf{Top}}
\newcommand{\catRing}{\mathbf{Ring}}
\newcommand{\catCRing}{\mathbf{CRing}} % comm. rings
\newcommand{\catMod}{\mathbf{Mod}}
\newcommand{\catMon}{\mathbf{Mon}}
\newcommand{\catMan}{\mathbf{Man}} % manifolds
\newcommand{\catDiff}{\mathbf{Diff}} % smooth manifolds
\newcommand{\catAlg}{\mathbf{Alg}}
\newcommand{\catRep}{\mathbf{Rep}} % representations 
\newcommand{\catVec}{\mathbf{Vec}}

% Group and Representation Theory
\newcommand{\chr}{\operatorname{char}}
\newcommand{\Aut}{\operatorname{Aut}}
\newcommand{\GL}{\operatorname{GL}}
\newcommand{\im}{\operatorname{im}}
\newcommand{\tr}{\operatorname{tr}}
\newcommand{\id}{\mathbf{id}}
\newcommand{\cl}{\mathbf{cl}}
\newcommand{\Gal}{\operatorname{Gal}}
\newcommand{\Tr}{\operatorname{Tr}}
\newcommand{\sgn}{\operatorname{sgn}}
\newcommand{\Sym}{\operatorname{Sym}}
\newcommand{\Alt}{\operatorname{Alt}}

% Commutative and Homological Algebra
\newcommand{\spec}{\operatorname{spec}}
\newcommand{\mspec}{\operatorname{m-spec}}
\newcommand{\Spec}{\operatorname{Spec}}
\newcommand{\mSpec}{\operatorname{MaxSpec}}
\newcommand{\Tor}{\operatorname{Tor}}
\newcommand{\tor}{\operatorname{tor}}
\newcommand{\Ann}{\operatorname{Ann}}
\newcommand{\Supp}{\operatorname{Supp}}
\newcommand{\Hom}{\operatorname{Hom}}
\newcommand{\End}{\operatorname{End}}
\newcommand{\coker}{\operatorname{coker}}
\newcommand{\limit}{\varprojlim}
\newcommand{\colimit}{%
  \mathop{\mathpalette\colimit@{\rightarrowfill@\textstyle}}\nmlimits@
}
\makeatother


\newcommand{\fraka}{\mathfrak{a}} % ideal
\newcommand{\frakb}{\mathfrak{b}} % ideal
\newcommand{\frakc}{\mathfrak{c}} % ideal
\newcommand{\frakf}{\mathfrak{f}} % face map
\newcommand{\frakg}{\mathfrak{g}}
\newcommand{\frakh}{\mathfrak{h}}
\newcommand{\frakm}{\mathfrak{m}} % maximal ideal
\newcommand{\frakn}{\mathfrak{n}} % naximal ideal
\newcommand{\frakp}{\mathfrak{p}} % prime ideal
\newcommand{\frakq}{\mathfrak{q}} % qrime ideal
\newcommand{\fraks}{\mathfrak{s}}
\newcommand{\frakt}{\mathfrak{t}}
\newcommand{\frakz}{\mathfrak{z}}
\newcommand{\frakI}{\mathfrak{I}}
\newcommand{\frakJ}{\mathfrak{J}}
\newcommand{\frakK}{\mathfrak{K}}
\newcommand{\frakL}{\mathfrak{L}}
\newcommand{\frakN}{\mathfrak{N}} % nilradical 
\newcommand{\frakP}{\mathfrak{P}} % nilradical 
\newcommand{\frakR}{\mathfrak{R}} % jacobson radical
\newcommand{\frakU}{\mathfrak{U}}
\newcommand{\frakX}{\mathfrak{X}}

% General/Differential/Algebraic Topology 
\newcommand{\scrA}{\mathscr{A}}
\newcommand{\scrB}{\mathscr{B}}
\newcommand{\scrF}{\mathscr{F}}
\newcommand{\scrG}{\mathscr{G}}
\newcommand{\scrO}{\mathscr{O}}
\newcommand{\scrP}{\mathscr{P}}
\newcommand{\scrS}{\mathscr{S}}
\newcommand{\bbH}{\mathbb H}
\newcommand{\Int}{\operatorname{Int}}
\newcommand{\psimeq}{\simeq_p}
\newcommand{\wt}[1]{\widetilde{#1}}
\newcommand{\RP}{\mathbb{R}\text{P}}
\newcommand{\CP}{\mathbb{C}\text{P}}

% Algebraic Geometry 
\newcommand{\res}{\operatorname{res}}

% Miscellaneous
\newcommand{\wh}[1]{\widehat{#1}}
\newcommand{\calM}{\mathcal{M}}
\newcommand{\calP}{\mathcal{P}}
\newcommand{\onto}{\twoheadrightarrow}
\newcommand{\into}{\hookrightarrow}
\newcommand{\Gr}{\operatorname{Gr}}
\newcommand{\Span}{\operatorname{Span}}
\newcommand{\ev}{\operatorname{ev}}
\newcommand{\weakto}{\stackrel{w}{\longrightarrow}}

\newcommand{\define}[1]{\textcolor{blue}{\textit{#1}}}
% \newcommand{\caution}[1]{\textcolor{red}{\textit{#1}}}
\renewcommand{\mod}{~\mathrm{mod}~}
\renewcommand{\le}{\leqslant}
\renewcommand{\leq}{\leqslant}
\renewcommand{\ge}{\geqslant}
\renewcommand{\geq}{\geqslant}
\newcommand{\Res}{\operatorname{Res}}
\newcommand{\floor}[1]{\left\lfloor #1\right\rfloor}
\newcommand{\ceil}[1]{\left\lceil #1\right\rceil}
\newcommand{\gl}{\mathfrak{gl}}
\newcommand{\ad}{\operatorname{ad}}
\newcommand{\Stab}{\operatorname{Stab}}
\newcommand{\bfX}{\mathbf{X}}
\newcommand{\Ind}{\operatorname{Ind}}
\newcommand{\bfG}{\mathbf{G}}

\newcommand{\sheafHom}{\mathscr{H}\kern-3.5pt \text{\calligra\Large om}\,}
\newcommand{\hght}{\operatorname{ht}}
\newcommand{\pre}{\mathrm{pre}}
\newcommand{\scrH}{\mathscr{H}}
\newcommand{\F}{\mathbb{F}}

\geometry {
    margin = 1in
}

\titleformat
{\section}
[block]
{\Large\bfseries\scshape}
{\S\thesection}
{0.5em}
{\centering}
[]


\titleformat
{\subsection}
[block]
{\normalfont\bfseries\sffamily}
{\S\S}
{0.5em}
{\centering}
[]


\renewcommand{\thesection}{\textsc{\Roman{section}}}

\begin{document}
\maketitle
\tableofcontents

\setcounter{section}{4}

\section{Algebraic Extensions}
\setcounter{exercise}{27}

\begin{exercise}
    \textbf{\emph{Part 1.}} Let $f(X_1,\dots,X_n)$ be a homogeneous polynomial of degree $2$ over $k$, i.e., a quadratic form. Suppose $f$ is \emph{anisotropic} over $k$, i.e., the only non-trivial zero of $f$ over $k$ is the vector $(0,\dots,0)$. Let $K/k$ be an extension of odd degree. Using induction on the degree we shall show that $f$ is anisotropic when viewed as a quadratic form over $K$. In literature, this is a theorem attributed to Springer.

    Throughout, we shall fix an algebraic closure $k^a$ of $k$ and consider all extensions to be embedded inside $k^a/k$. The base case $K = k$ is clear. Suppose now that $[K : k]\ge 3$ and that the hypothesis has been proven for all odd degrees less than $[K : k]$. If the extension $K/k$ admits a proper intermediate field, say $L$, then due to the inductive hypothesis, $f$ is anisotropic when viewed over $L$ and then again due to the inductive hypothesis, $f$ is anisotropic when viewed over $K$. Suppose henceforth that $K/k$ admits no proper intermediate fields. In particular, due to the Primitive Element Theorem, this means that the extension $K/k$ is simple, i.e., there exists $\alpha\in K$ such that $K = k(\alpha)$.

    Let $d = [K : k]\ge 3$ and let $p(X)$ be the minimal polynomial of $\alpha$ over $k$. Suppose $f$ is not anisotropic over $K$, which means that there is a non-zero vector in $K^n$ on which $f$ vanishes. Thus, there exist polynomials $g_1,\dots,g_n\in k[T]$ such that $\deg g_i\le d - 1$ for $1\le i\le n$ and 
    \begin{equation*}
        f\left(g_1(\alpha),\dots,g_n(\alpha)\right) = 0.
    \end{equation*}
    Consider the polynomial 
    \begin{equation*}
        h(T) = f\left(g_1(T),\dots,g_n(T)\right).
    \end{equation*}
    Since $k[T]$ is a PID, we can further impose the condition that $\left(g_1(T),\dots,g_n(T)\right) = (1)$. Indeed, if their gcd is some polynomial $g(T)$, then $g(\alpha)\ne 0$, and hence, dividing all the $g_i$'s by $g(T)$, we obtain the desired tuple.

    Let $M = \max\deg g_i\le d - 1$. The coefficient of $T^{2M}$ on the left hand side is $f(a_{1m},\dots,a_{nm})$ where $a_{im}$ is the coefficient of $T^m$ in $g_i(T)$. Since the vector $(a_{1m},\dots,a_{nm})$ is not identically zero, and $f$ is anisotropic over $k$, it is clear that $\deg h(T) = 2M\le 2d - 2$. 

    Next, since $h(\alpha) = 0$, we can write $h(T) = p(T)q(T)$ for some polynomial $q(T)\in k[T]$. Note that $\deg q = 2M - d\le d - 2$ and is an odd number. As a result, $q$ has an irreducible factor $\wt q$ of odd degree, and let $\beta\in k^a$ be a root of $\wt q$. Due to the inductive hypothesis and the fact that $h(\beta) = 0$, we must have that $g_1(\beta) = \dots = g_n(\beta) = 0$, and hence, $\wt q$ divides $g_1,\dots, g_n$ in $k[T]$, which is absurd. Thus $f$ is anisotropic over $K$.

    \textbf{\emph{Part 2.}} Let $f(X_1,\dots,X_n)$ be a homogeneous polynomial of degree $3$ over $k$ and $K/k$ a quadratic extension. Note that $K = k(\alpha)$ for any $\alpha\in K\setminus k$. Let $p(T)\in k[T]$ be the minimal polynomial of $\alpha$ over $k$. This is clearly a quadratic polynomial. Suppose $f$ were isotropic over $K$, then one can find linear polynomials $g_1,\dots, g_n\in k[T]$ such that 
    \begin{equation*}
        f(g_1(\alpha),\dots,g_n(\alpha)) = 0.
    \end{equation*}
    As in Part 1, since $k[T]$ is a PID, we can further impose the condition that $(g_1(T),\dots,g_n(T)) = (1)$. Let 
    \begin{equation*}
        h(T) = f(g_1(T),\dots,g_n(T))\in k[T].
    \end{equation*}
    Again, since $f$ is anisotropic over $k$, just as argued in Part 1, it follows that $h(T)$ is a cubic polynomial in $k[T]$. Note that $h(\alpha) = 0$, and thus $h(T) = A p(T)(T - \beta)$ for some $A,\beta\in k$. It follows that $h(\beta) = 0$, i.e., $g_i(\beta) = 0$ for all $1\le i\le n$. But this is absurd, since $T - \beta$ cannot divide all the $g_i$'s simultaneously. Thus $f$ is anisotropic over $K$, as desired.
\end{exercise}



\section{Galois Theory}
\setcounter{exercise}{20}
\begin{exercise}\thlabel{5.21}
    
\end{exercise}

\setcounter{exercise}{22}

\begin{exercise}
\begin{enumerate}[label=(\alph*)]
\item The standard way to do this is to first write 
\begin{equation*}
    G\cong\bigoplus_{i = 1}^r \Z/n_i\Z,
\end{equation*}
where $n_i\ge 2$. Using either Dirichlet's theorem on primes in AP or \thref{5.21}(b), choose primes $p_i\equiv 1\pmod{n_i}$. Set $\displaystyle N = \prod_{i = 1}^{r}p_i$ and note that 
\begin{equation*}
    \Gal\left(\Q\left(\zeta_N\right)/\Q\right)\cong\left(\Z/N\Z\right)^\times\cong\bigoplus_{i = 1}^r \Z/(p_i - 1)\Z.
\end{equation*}
Since $G$ is a quotient of the above group, it is clear that $G$ can be realized as a Galois group over $\Q$.

\item Again, begin by writing 
\begin{equation*}
    G\cong\bigoplus_{i = 1}^r \Z/n_i\Z.
\end{equation*}
Using either Dirichlet's theorem on primes in AP or \thref{5.21}, for each positive integer $i\ge 1$, choose a tuple of primes $(p_{i1},\dots,p_{ir})$ such that $p_{ij}\equiv 1\pmod{n_j}$. Further, setting $N_i = \displaystyle\prod_{j = 1}^r p_{ij}$, we may further impose the condition that $\gcd(N_i, N_j) = 1$ whenever $i\ne j$. In particular, this means that $\Q(\zeta_{N_i})\cap\Q(\zeta_{N_j}) = \Q$. As in part (a), we can find a subfield $E_i\subseteq\Q(\zeta_{N_i})$ such that $\Gal(E_i/\Q)\cong G$.
\begin{equation*}
    \xymatrix {
        & kE_i & \\
        k\ar@{-}[ru] & & E_i\ar@{-}[lu]\\
        & k\cap E_i\ar@{-}[lu]\ar@{-}[ru] & \\
            & \Q\ar@{-}[u] & 
    }
\end{equation*}
We know that $\Gal(kE_i/k)\cong\Gal(E_i/k\cap E_i)$ for all $i\ge 1$. We contend that $k\cap E_i = \Q$ for infinitely many $i\ge 1$. Indeed, since $k/\Q$ is separable, due to the Primitive Element Theorem, there are only finitely many intermediate fields in the extension $k/\Q$. Thus, there is an infinite subset $I\subseteq\N$ such that $k\cap E_i = k\cap E_j$ for all $i,j\in I$. Then, for $i,j\in I$, we have 
\begin{equation*}
    k\cap E_i = (k\cap E_i)\cap(k\cap E_i) = k\cap (E_i\cap E_j) = k\cap\Q = \Q.
\end{equation*}
Thus, $\Gal(kE_i/k)\cong\Gal(E_i/\Q)\cong G$.

All that remains to be shown is that the set $\{kE_i\colon i\in I\}$ is infinite. Suppose not, then there is an extension $F/k$ and an infinite subset $J\subseteq I$ such that $kE_j = F$ for all $j\in J$. In particular, $E_j\subseteq F$ for all $j\in J$. Note that $F/\Q$ is a finite separable extension, and hence, due to the Primitive Element Theorem, has at most finitely many intermediate fields, but this is absurd, since $E_i\ne E_j$ for $i,j\in J$. Thus, the set $\{kE_i\colon i\in I\}$ is infinite, as desired.
\end{enumerate}
\end{exercise}


\setcounter{exercise}{24}

\begin{exercise}
    First note that every finite extension of $k$ is Galois, and hence $k$ is perfect. Further, since any algebraic extension of $k$ is a union of finite subextensions (each of which is Galois), we have that every algebraic extension of $k$ is Galois so we can freely talk about its Galois group. Finally, we make note of the fact that $k$ can have at most one finite extension of a given degree in $k^a$. Indeed, if $E$ and $F$ are two finite extensions of $k$ in $k^a$ of the same degree, then $\Gal(EF/E)$ and $\Gal(EF/F)$ are subgroups of $\Gal(EF/k)$ of the same order. Since $\Gal(EF/k)$ is cyclic, it has at most one subgroup of a given order, and hence, $\Gal(EF/E) = \Gal(EF/F)$, i.e., $E = F$.
    
    Let 
    \begin{equation*}
        \Sigma \coloneq \left\{(E, \sigma_E)\colon k\subseteq E\subseteq k^a\text{ and }\sigma_E\in\Gal(E/k)\text{ such that } E^{\sigma_E} = k\right\}.
    \end{equation*}
    This is clearly a poset under the relation $(F,\sigma_F)\leqq(E,\sigma_E)$ if and only if $F\subseteq E$ and $\sigma_E|_F = \sigma_F$. Clearly, Zorn's lemma is applicable and let $(M,\sigma_M)$ be a maximal element in $\Sigma$. We contend that $M = k^a$. 

    Suppose $M\subsetneq k^a$ and choose an element $\alpha\in k^a\setminus M$ of minimum degree over $M$. Since $M(\alpha)/k$ is Galois, we can extend $\sigma_M$ to an automorphism $\sigma_1\in\Gal(M(\alpha)/k)$. The maximality of $(M,\sigma_M)$ implies the existence of some $\beta\in M(\alpha)\setminus M$ which is fixed by $\sigma_1$. Note that the minimality of the degree of $\alpha$ over $M$ further implies that $M(\alpha) = M(\beta)$.

    \begin{equation*}
        \xymatrix {
            & M(\alpha) = M(\beta)\ar@{-}[rd] & \\
            M\ar@{-}[ru]\ar@{-}[rd] & & k(\beta)\\
            & k\ar@{-}[ru] & 
        }
    \end{equation*}

    We contend that $[M(\beta) : M] = [k(\beta) : k]$. Indeed, let $f(X) = \mathrm{Irr}(\beta, M, X)$ be the irreducible polynomial of $\beta$ over $M$. Since $\sigma_1$ fixes $\beta$, we see that $\beta$ is a root of $f^{\sigma_1}\in M[X]$. Again, since $\deg f = \deg f^{\sigma_1}$, it follows that $f = f^{\sigma_1}$. In particular, the coefficients of $f$ lie in the fixed field $M^{\sigma_1} = M^{\sigma_M} = k$. Thus, $f(X) = \mathrm{Irr}(\beta, k, X)$, so that $[k(\beta) : k] = [M(\beta) : M]$.

    Now note that $f(X)$ is a separable polynomial and has degree at least $2$. Let $\beta'\ne\beta$ be another root of $f(X)$ in $k^a$ and extend the automorphism $\sigma_M$ to an automorphism $\sigma_2$ of $M(\beta)$ sending $\beta\mapsto\beta'$. Again, due to maximality, $\sigma_2$ must fix some $\gamma\in M(\beta)\setminus M$. Furthermore, as we argued above, we must have $M(\beta) = M(\gamma)$ and $[k(\gamma) : k] = [M(\gamma) : M] = [M(\beta) : M] = [k(\beta) : k]$. 

    Note that we cannot have $k(\beta) = k(\gamma)$, else $\beta\in k(\gamma)$ would be fixed by $\sigma_2$, which is absurd, since $\sigma_2\beta = \beta'$. Thus, $k(\beta)$ and $k(\gamma)$ are distinct Galois extensions of $k$ having the same degree, a contradiction. In conclusion, $M = k^a$, and we have our desired automorphism in $\Gal(k^a/k)$.
\end{exercise}

\begin{exercise}
    Let $\alpha\in\Q^a\setminus\Q$ be an algebraic irrational and $E$ a maximal subfield of $\Q^a$ not containing $\alpha$. We shall show that every finite extension of $E$ contained in $\Q^a$ is cyclic. Since every finite extension of $E$ is contained in a finite Galois extension, and quotients of cyclic groups are cyclic, it suffices to show that every finite Galois extension of $E$ is cyclic.

    Let $K$ be a finite Galois extension of $E$ contained in $\Q^a$ and let $G =\Gal(K/E)$. If $F$ is an intermediate field properly containing $E$, then it must contain $\alpha$ due to maximality of $E$, i.e., $E(\alpha)\subseteq F$. Let $H = \Gal(K/E(\alpha))$. From the Galois correspondence, it is clear that $H$ is \emph{the} unique maximal subgroup of $G$. We shall be done by proving the following:

    \begin{claim}
        Let $G$ be a finite group. If $G$ admits a unique maximal subgroup $H$, then $G$ is cyclic.\footnote{We can further say that $G$ must be a $p$-group. This follows immediately from the fact that it has a \emph{unique} maximal subgroup.}
    \end{claim}
    To see this, let $a\in G\setminus H$. If $G\ne\langle a\rangle$, then $\langle a\rangle$ is contained in a maximal subgroup $M$ of $G$. But since $H$ is the unique maximal subgroup of $G$, we must have $M = H$, that is, $a\in H$, a contradiction. Thus $G = \langle a\rangle$, as desired. 
\end{exercise}

\begin{exercise}

\end{exercise}

\setcounter{exercise}{33}

\begin{exercise}
    Consider two automorphisms $\sigma\colon x\mapsto -x$ and $\tau\colon x\mapsto 1 - x$ of $K\coloneq\bbC(X)$ over $\bbC$. Let $E$ and $F$ denote the fixed fields of $\sigma$ and $\tau$ respectively. Since both $\sigma$ and $\tau$ are order $2$ automorphisms, we have that $[K : E] = [K : F] = 2$. Let $k = E\cap F$. Note that $k$ is invariant under the action of $\varphi = \tau\circ\sigma\colon x\mapsto 1 + x$. It is clear that $\varphi$ is an infinite order automorphism of $K$ and that $k$ is contained in the fixed field $K^{\varphi}$. Finally, since $K$ is finite degree over any intermediate field properly containing $\bbC$, it follows that the fixed field $K^\varphi = \bbC$. Hence, $k = \bbC$, so that $K$ is not algebraic over $k$.
\end{exercise}

\end{document}