\begin{definition}
    An $\scrO_X$-module $\scrF$ is said to be \define{free} if it is isomorphic to a direct sum of copies of $\scrO_X$. It is said to be \define{locally free} if $X$ has an open cover by sets $U$ for which $\scrF|_U$ is a free $\scrO_X|_U$-module.
\end{definition}

\setcounter{exercise}{6}
\begin{exercise}\thlabel{exer:5.7} \hfill % Exercise 5.7
\begin{enumerate}[label=(\alph*)]
\item We reduce this to the affine case since $\scrF$ is coherent on a noetherian scheme. Thus, we have a finitely generated $A$-module $M$ and a prime ideal $\frakp\in\Spec A$ such that $M_\frakp$ is a free $A_\frakp$-module. 

Choose a basis $\left\{\frac{m_1}{1},\dots,\frac{m_n}{1}\right\}$ of $M_\frakp$ over $A_\frakp$ and consider the exact sequence 
\begin{equation*}
    0\to K\to A^n\to M\to Q\to 0,
\end{equation*}
where the map $A^n\to M$ is the natural map sending $e_i\mapsto m_i$ for $1\le i\le n$. Localising, we see that $K_\frakp = Q_\frakp = 0$ and hence, there is an $f\in A\setminus\frakp$ such that $K_f = Q_f = 0$ (since both $K$ and $Q$ are finitely generated). Localising the above exact sequence at $f$, we obtain an isomorphism $A^n_f\xrightarrow{\sim} M_f$. It follows that $M_\frakq$ is a free $A_\frakq$ module for all $\frakq\in D(f)$.

\item Follows immediately from (a). 

\item Let $\scrF^\vee$ denote the dual sheaf. Recall that 
\begin{equation*}
    \scrF^\vee(U) = \Hom_{\scrO_X|_U}\left(\scrF|_U,\scrO_X|_U\right).
\end{equation*}
This gives a natural map $\scrF(U)\otimes\scrF^\vee(U) \to\scrO_X(U)$ given by 
\begin{equation*}
    s\otimes\varphi\mapsto\varphi_U(s).
\end{equation*}
It is easy to check that this is a morphism of presheaves $\scrF\otimes\scrF^\vee\to\scrO_X$ and since the latter is a sheaf, it factors through the sheafification inducing a map on the tensor sheaf.

We contend that this induced map is an isomorphism. To this end, it suffices to show that the induced morphism on stalks is an isomorphism.
\end{enumerate}
\end{exercise}
