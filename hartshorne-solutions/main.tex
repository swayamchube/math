\documentclass[12pt]{report}

% \usepackage{./arxiv}

\title{Hartshorne Exercises}
\author{Swayam Chube}
\date{\today}

\usepackage[utf8]{inputenc} % allow utf-8 input
\usepackage[T1]{fontenc}    % use 8-bit T1 fonts
\usepackage{hyperref}       % hyperlinks
\usepackage{url}            % simple URL typesetting
\usepackage{booktabs}       % professional-quality tables
\usepackage{amsfonts}       % blackboard math symbols
\usepackage{nicefrac}       % compact symbols for 1/2, etc.
\usepackage{microtype}      % microtypography
\usepackage{graphicx}
\usepackage{natbib}
\usepackage{doi}
\usepackage{amssymb}
\usepackage{bbm}
\usepackage{amsthm}
\usepackage{amsmath}
\usepackage{xcolor}
\usepackage{theoremref}
\usepackage{enumitem}
\usepackage{mathpazo}
% \usepackage{euler}
\usepackage{mathrsfs}
\usepackage{todonotes}
\usepackage{stmaryrd}
\usepackage[all,cmtip]{xy} % For diagrams, praise the Freyd–Mitchell theorem 
\usepackage{marvosym}
\usepackage{geometry}
\usepackage{calligra}
\usepackage{titlesec}

\renewcommand{\qedsymbol}{$\blacksquare$}

% Uncomment to override  the `A preprint' in the header
% \renewcommand{\headeright}{}
% \renewcommand{\undertitle}{}
% \renewcommand{\shorttitle}{}

\hypersetup{
    pdfauthor={Lots of People},
    colorlinks=true,
}

\newtheoremstyle{thmstyle}%               % Name
  {}%                                     % Space above
  {}%                                     % Space below
  {}%                             % Body font
  {}%                                     % Indent amount
  {\bfseries\scshape}%                            % Theorem head font
  {.}%                                    % Punctuation after theorem head
  { }%                                    % Space after theorem head, ' ', or \newline
  {\thmname{#1}\thmnumber{ #2}\thmnote{ (#3)}}%                                     % Theorem head spec (can be left empty, meaning `normal')

\newtheoremstyle{defstyle}%               % Name
  {}%                                     % Space above
  {}%                                     % Space below
  {}%                                     % Body font
  {}%                                     % Indent amount
  {\bfseries\scshape}%                            % Theorem head font
  {.}%                                    % Punctuation after theorem head
  { }%                                    % Space after theorem head, ' ', or \newline
  {\thmname{#1}\thmnumber{ #2}\thmnote{ (#3)}}%                                     % Theorem head spec (can be left empty, meaning `normal')

\theoremstyle{thmstyle}
\newtheorem{theorem}{Theorem}[section]
\newtheorem{lemma}[theorem]{Lemma}
\newtheorem{proposition}[theorem]{Proposition}

\theoremstyle{defstyle}
\newtheorem*{definition}{Definition}
\newtheorem*{corollary}{Corollary}
\newtheorem*{remark}{Remark}
\newtheorem{example}[theorem]{Example}
\newtheorem*{notation}{Notation}
\newtheorem{exercise}{Exercise}[section]

% Common Algebraic Structures
\newcommand{\R}{\mathbb{R}}
\newcommand{\Q}{\mathbb{Q}}
\newcommand{\Z}{\mathbb{Z}}
\newcommand{\N}{\mathbb{N}}
\newcommand{\bbC}{\mathbb{C}}
\newcommand{\K}{\mathbb{K}}
\newcommand{\calA}{\mathcal{A}}
\newcommand{\frakM}{\mathfrak{M}}
\newcommand{\calO}{\mathcal{O}}
\newcommand{\bbA}{\mathbb{A}} % affine space or adele ring
\newcommand{\bbP}{\mathbb{P}} % projective space
\newcommand{\bbI}{\mathbb{I}} % idele group

% Categories
\newcommand{\catTopp}{\mathbf{Top}_*}
\newcommand{\catGrp}{\mathbf{Grp}}
\newcommand{\catTopGrp}{\mathbf{TopGrp}}
\newcommand{\catSet}{\mathbf{Set}}
\newcommand{\catTop}{\mathbf{Top}}
\newcommand{\catRing}{\mathbf{Ring}}
\newcommand{\catCRing}{\mathbf{CRing}} % comm. rings
\newcommand{\catMod}{\mathbf{Mod}}
\newcommand{\catMon}{\mathbf{Mon}}
\newcommand{\catMan}{\mathbf{Man}} % manifolds
\newcommand{\catDiff}{\mathbf{Diff}} % smooth manifolds
\newcommand{\catAlg}{\mathbf{Alg}}
\newcommand{\catRep}{\mathbf{Rep}} % representations 
\newcommand{\catVec}{\mathbf{Vec}}

% Group and Representation Theory
\newcommand{\chr}{\operatorname{char}}
\newcommand{\Aut}{\operatorname{Aut}}
\newcommand{\GL}{\operatorname{GL}}
\newcommand{\im}{\operatorname{im}}
\newcommand{\tr}{\operatorname{tr}}
\newcommand{\id}{\mathbf{id}}
\newcommand{\cl}{\mathbf{cl}}
\newcommand{\Gal}{\operatorname{Gal}}
\newcommand{\Tr}{\operatorname{Tr}}
\newcommand{\sgn}{\operatorname{sgn}}
\newcommand{\Sym}{\operatorname{Sym}}
\newcommand{\Alt}{\operatorname{Alt}}

% Commutative and Homological Algebra
\newcommand{\spec}{\operatorname{spec}}
\newcommand{\mspec}{\operatorname{m-spec}}
\newcommand{\Spec}{\operatorname{Spec}}
\newcommand{\mSpec}{\operatorname{MaxSpec}}
\newcommand{\Tor}{\operatorname{Tor}}
\newcommand{\tor}{\operatorname{tor}}
\newcommand{\Ann}{\operatorname{Ann}}
\newcommand{\Supp}{\operatorname{Supp}}
\newcommand{\Hom}{\operatorname{Hom}}
\newcommand{\End}{\operatorname{End}}
\newcommand{\coker}{\operatorname{coker}}
\newcommand{\limit}{\varprojlim}
\newcommand{\colimit}{%
  \mathop{\mathpalette\colimit@{\rightarrowfill@\textstyle}}\nmlimits@
}
\makeatother


\newcommand{\fraka}{\mathfrak{a}} % ideal
\newcommand{\frakb}{\mathfrak{b}} % ideal
\newcommand{\frakc}{\mathfrak{c}} % ideal
\newcommand{\frakf}{\mathfrak{f}} % face map
\newcommand{\frakg}{\mathfrak{g}}
\newcommand{\frakh}{\mathfrak{h}}
\newcommand{\frakm}{\mathfrak{m}} % maximal ideal
\newcommand{\frakn}{\mathfrak{n}} % naximal ideal
\newcommand{\frakp}{\mathfrak{p}} % prime ideal
\newcommand{\frakq}{\mathfrak{q}} % qrime ideal
\newcommand{\fraks}{\mathfrak{s}}
\newcommand{\frakt}{\mathfrak{t}}
\newcommand{\frakz}{\mathfrak{z}}
\newcommand{\frakI}{\mathfrak{I}}
\newcommand{\frakJ}{\mathfrak{J}}
\newcommand{\frakK}{\mathfrak{K}}
\newcommand{\frakL}{\mathfrak{L}}
\newcommand{\frakN}{\mathfrak{N}} % nilradical 
\newcommand{\frakP}{\mathfrak{P}} % nilradical 
\newcommand{\frakR}{\mathfrak{R}} % jacobson radical
\newcommand{\frakU}{\mathfrak{U}}
\newcommand{\frakX}{\mathfrak{X}}

% General/Differential/Algebraic Topology 
\newcommand{\scrA}{\mathscr{A}}
\newcommand{\scrB}{\mathscr{B}}
\newcommand{\scrF}{\mathscr{F}}
\newcommand{\scrG}{\mathscr{G}}
\newcommand{\scrO}{\mathscr{O}}
\newcommand{\scrP}{\mathscr{P}}
\newcommand{\scrS}{\mathscr{S}}
\newcommand{\bbH}{\mathbb H}
\newcommand{\Int}{\operatorname{Int}}
\newcommand{\psimeq}{\simeq_p}
\newcommand{\wt}[1]{\widetilde{#1}}
\newcommand{\RP}{\mathbb{R}\text{P}}
\newcommand{\CP}{\mathbb{C}\text{P}}

% Algebraic Geometry 
\newcommand{\res}{\operatorname{res}}

% Miscellaneous
\newcommand{\wh}[1]{\widehat{#1}}
\newcommand{\calM}{\mathcal{M}}
\newcommand{\calP}{\mathcal{P}}
\newcommand{\onto}{\twoheadrightarrow}
\newcommand{\into}{\hookrightarrow}
\newcommand{\Gr}{\operatorname{Gr}}
\newcommand{\Span}{\operatorname{Span}}
\newcommand{\ev}{\operatorname{ev}}
\newcommand{\weakto}{\stackrel{w}{\longrightarrow}}

\newcommand{\define}[1]{\textcolor{blue}{\textit{#1}}}
\newcommand{\caution}[1]{\textcolor{red}{\textit{#1}}}
\renewcommand{\mod}{~\mathrm{mod}~}
\renewcommand{\le}{\leqslant}
\renewcommand{\leq}{\leqslant}
\renewcommand{\ge}{\geqslant}
\renewcommand{\geq}{\geqslant}
\newcommand{\Res}{\operatorname{Res}}
\newcommand{\floor}[1]{\left\lfloor #1\right\rfloor}
\newcommand{\ceil}[1]{\left\lceil #1\right\rceil}
\newcommand{\gl}{\mathfrak{gl}}
\newcommand{\ad}{\operatorname{ad}}
\newcommand{\Stab}{\operatorname{Stab}}
\newcommand{\bfX}{\mathbf{X}}
\newcommand{\Ind}{\operatorname{Ind}}
\newcommand{\bfG}{\mathbf{G}}

\newcommand{\sheafHom}{\mathscr{H}\kern-3.5pt \text{\calligra\Large om}\,}
\newcommand{\hght}{\operatorname{ht}}
\newcommand{\pre}{\mathrm{pre}}
\newcommand{\scrH}{\mathscr{H}}

\geometry {
    margin = 1in
}

\titleformat
{\section}
[block]
{\Large\bfseries\scshape}
{\S\thesection}
{0.5em}
{\centering}
[]


\titleformat
{\subsection}
[block]
{\normalfont\bfseries\sffamily}
{\S\S}
{0.5em}
{\centering}
[]


\renewcommand{\thechapter}{\Roman{chapter}}

\begin{document}
\maketitle
\tableofcontents

\newpage

\chapter{Varieties}

\section{Affine Varieties}
\begin{definition}
    Let $\scrF$ and $\scrG$ be sheaves of abelian groups on $X$. The association $U\mapsto\Hom(\scrF|_U, \scrG|_U)$ is a sheaf on $X$. It is called the \define{sheaf Hom} and is denoted by $\sheafHom(\scrF, \scrG)$.
\end{definition}

\setcounter{exercise}{14}

\begin{exercise}
\end{exercise}


\newpage
\section{Projective Varieties}
\setcounter{exercise}{2}
\begin{exercise}[Reduced Schemes]\hfill
\begin{enumerate}[label=(\alph*)]
    \item Suppose $X$ is reduced. Then, every open affine corresponds to a reduced ring. Consequently, the local ring of any point on $X$ is the localisation of a reduced ring and hence, is reduced.

    Conversely, suppose $\scrO_{X, P}$ is reduced for every $P\in X$. Let $U = \Spec A$ be an affine open. The local ring of any point $P\in U$ is a localisation of $A$ at a prime. Since all these rings are reduced, so is $A$.

    Let $U\subseteq X$ be open. Cover $U$ with affine opens $U_i = \Spec A_i$ and let $s\in\scrO(U)$ be nilpotent. Its image $s_i = \res_{U, U_i}(s)$ is nilpotent in $\scrO(U_i) = A_i$ and hence, $s_i = 0$. Consequently $s = 0$ due to the identity axiom. This shows that $\scrO(U)$ is reduced.

    \item The first part follows immediately from the fact that there is a commutative diagram 
    \begin{equation*}
        \xymatrix {
            A\ar[r]^\phi\ar[d] & B\ar[d]\\
            A_{red}\ar[r]_{\phi_{red}} & B_{red}.
        }
    \end{equation*}
    Consider the map of locally ringed spaces $(\id, f^\sharp)$, where $f^\sharp: \scrO_X\to\scrO_X^{red}$ is the collection of the canonical maps $\scrO_X(U)\to\scrO_X^{red}(U)$.

    \item Follows from the fact that any morphism of rings $\phi: A\to B$ with $B$ reduced factors through the natural map $A\to A_{red}$.
\end{enumerate}
\end{exercise}

\begin{exercise}\thlabel{exer:2.4} % Exercise 2.4
    Let $\varphi\in\Hom_{\mathfrak{Rings}}(A,\Gamma(X,\scrO_X))$. Cover $X$ with affine opens $U_i = \Spec A_i$. The restriction map gives us a homomorphism 
    \begin{equation*}
        A\stackrel{\varphi}{\longrightarrow}\Gamma(X,\scrO_X)\xrightarrow{\res^{X}_{U_i}}\Gamma(U_i,\scrO_X) = A_i,
    \end{equation*}
    which induces a map on schemes $\pi_i: U_i\to\Spec A$ where $\pi_i = \Spec(\res^X_{U_i}\circ\varphi)$. 

    We contend that the maps $\pi_i$ can be glued. Indeed, for $i\ne j$, cover $U_i\cap U_j$ with affine opens $U_{ijk} = \Spec A_{ijk}$. Now, 
    \begin{equation*}
        \pi_i|_{U_{ijk}} = \Spec(\res^{U_i}_{U_{ijk}})\circ\pi_i = \Spec(\res^{U_i}_{U_{ijk}}\circ\res^X_{U_i}\circ\varphi) = \Spec(\res^{X}_{U_{ijk}}\circ\varphi).
    \end{equation*}
    Similarly, $\pi_j|_{U_{ijk}} = \Spec(\res^X_{U_{ijk}}\circ\varphi)$, consequently, the family of morphisms $\{\pi_i\}$ can be glued to a morphism $\pi: X\to\Spec A$. This gives a map 
    \begin{equation*}
        \beta:\Hom_{\mathfrak{Rings}}(A,\Gamma(X,\scrO_X))\to\Hom_{\mathfrak{Sch}}(X,\Spec A).
    \end{equation*}
    It is straightforward to verify that $\alpha$ and $\beta$ are inverses to one another.
\end{exercise}

\begin{exercise}
    Follows from the previous exercise and the fact that $\Z$ is an initial object in the category of rings.
\end{exercise}

\setcounter{exercise}{6}
\begin{exercise}
    Let $(f,f^\sharp): \Spec K\to X$ is a morphism of schemes which sends the unique point in $\Spec K$ to $x\in X$. Then, there is an induced map on local rings $f^\sharp_x: \scrO_x\to K$, which must be local and hence, factor through the maximal ideal of $\scrO_x$, thereby inducing a map $k(x)\to K$. It is easy to see that this process is reversible.
\end{exercise}

\setcounter{exercise}{8}
\begin{exercise}
    Let $Z\subseteq X$ be irreducible and closed. Let $U = \Spec A$ be an open affine intersecting $Z$. Then, $Z\cap U$ is open in $Z$ and hence, is irreducible. Further, it is closed in $U$ and hence, corresponds to a prime ideal $\xi = \frakp\in\Spec A$. Note that $\overline{\{\xi\}}\cap U = Z\cap U$ and $\overline{\{\xi\}}\subseteq Z$ since $Z$ is closed.

    Let $V$ be any other open set intersecting $Z$. Then, one can replace $V$ with an open affine $\Spec B$ intersecting $Z$. Suppose $\xi\notin V$. Then, 
    \begin{equation*}
        (Z\cap U)\cap (Z\cap V) = Z\cap U\cap V = \overline{\{\xi\}}\cap U\cap V = \emptyset,
    \end{equation*}
    since the closure of $\{\xi\}$ in $U$ is contained in $U\setminus V$. This is not possible since $Z\cap U$ and $Z\cap V$ are nonempty open sets in an irreducible space. Hence, $\xi$ is a generic point.


    Now we argue for uniqueness. Suppose $\xi_1$ and $\xi_2$ were two generic points in $Z$. Consider an affine neighborhood $U = \Spec A$ intersecting $Z$. Then, $Z\cap U$ must contain $\xi_1$ and $\xi_2$. Let $\xi_i$ correspond to a prime $\frakp_i$ in $A$ for $i = 1,2$. Now, $Z\cap U = V(\frakp_1) = V(\frakp_2)$, consequently, $\frakp_1 = \frakp_2$, that is, $\xi_1 = \xi_2$. This completes the proof.
\end{exercise}

\begin{definition}
    Let $(X,\scrO_X)$ be a scheme and let $f\in\Gamma(X,\scrO_X)$. Define $X_f$ to be the set of all $x\in X$ such that the stalk $f_x$ of $f$ at $x$ is not contained in the maximal ideal $\frakm_x$ of the local ring $\scrO_{X, x}$. This is known as the \define{support} of $f$ on $X$.
\end{definition}

\setcounter{exercise}{15}
\begin{exercise}\thlabel{exer:2.16}\hfill % Exercise 2.16
\begin{enumerate}[label=(\alph*)]
\item The set of all $x\in U$ such that $f_x\notin\frakm_x$ is the set of all prime ideals $\frakp$ in $B$ such that $f/1$ is not in the maximal ideal $\frakp B_\frakp$ in $B_\frakp$. Equivalently, $f\notin\frakp$. Thus, $X_f\cap U = D(\overline f)$. Now, since $X$ can be covered with open affines and the intersection of $X_f$ with every open affine is open, $X_f$ must also be open.

\item Pick a finite open cover $\{U_i = \Spec A_i\}_{i = 1}^m$. The restriction of $a$ to $X_f\cap U_i = D(\res^X_{U_i}(f))$ is zero and hence, there is a positive integer $n_i$ such that $\res^X_{U_i}(f^{n_i}a) = 0$. Let $N = \max\limits_{1\le i\le m} n_i$. Then, $\res^X_{U_i}(f^Na) = 0$. Due to the identity axiom, we must have $f^Na = 0$.

\item Let $U_i = \Spec A_i$ and let $f_i = \res^{X}_{U_i}(f)$. Since $X_f\cap U_i = D(f_i)$, there is a $b_i\in A_i = \Gamma(U_i,\scrO_X)$ such that $\res^X_{U_i\cap X_f}(b) = \frac{b_i}{f_i^{n_i}}$ for some nonnegative integer $n_i$. Choosing $n$ to be larger than all the $n_i$'s, we get that there is a $b_i\in A_i$ such that $\res^{X}_{U_i\cap X_f}(f^nb) = \res^{U_i}_{U_i\cap X_f}(b_i)$.

Now consider $b_i - b_j$ on $U_i\cap U_j$, which can be covered by finitely many affine opens $U_{ijk} = \Spec A_{ijk}$. Since $\res^{X}_{U_i\cap U_j\cap X_f}(b_i - b_j) = 0$, using a similar argument as in (b), there is a positive integer $m_{ij}$ such that $f^{m_{ij}}(b_i - b_j)$ restricts to $0$ On $U_i\cap U_j$. Choosing $m$ larger than $m_{ij}$ for all pairs $i,j$, we have that $f^m(b_i - b_j)$ restricts to $0$ on $U_i\cap U_j$. Consequently, $\res^{U_i}_{U_i\cap U_j}(f^mb_i) = \res^{U_j}_{U_i\cap U_j}(f^mb_j)$ and hence, there is a $c\in\Gamma(X,\scrO_X)$ such that $\res^{X}_{U_i}(c) = f^mb_i$. Hence, $\res^X_{U_i\cap X_f}(c) = \res^X_{U_i\cap X_f}(f^{n + m}b)$. This completes the proof.

\item First, we show that $\res^X_{X_f}(f)$ is invertible. Since $f_x\notin\frakm_x\subseteq\scrO_x$ for every $x\in X_f$, we see that the restriction of $f$ to every affine open contained in $X_f$ must be invertible (else it would lie in a prime ideal and hence, in the stalk of some point). Consider an open cover $U_i$ of $X_f$ using affine opens. There is a $g_i\in\Gamma(U_i,\scrO)$ such that $g_i\res^X_{U_i}(f) = 1$. For $i\ne j$, we have 
\begin{equation*}
    \res^{U_i}_{U_i\cap U_j}(g_i)\res^X_{U_i\cap U_j}(f) = 1 = \res^{U_j}_{U_i\cap U_j}(g_j)\res^X_{U_i\cap U_j}(f)
\end{equation*}
and hence, $\res^{U_i}_{U_i\cap U_j}(g_i) = \res^{U_j}_{U_i\cap U_j}(g_j)$ and hence, the $g_i$'s can be lifted to some $g\in\Gamma(X_f,\scrO_X)$, furthermore $\res^X_{X_f}(f)g = 1$, whence invertibility follows.

Consider the map $\Phi: A_f\to\Gamma(X_f,\scrO_X)$ given by 
\begin{equation*}
    \frac{a}{f^n}\mapsto\frac{\res^X_{X_f}(a)}{\res^{X}_{X_f}(f^n)}.
\end{equation*} 
If $\Phi(a/f^n) = 0$, then $\res^X_{X_f}(a) = 0$, consequently, due to part (b), there is a positive integer $m$ such that $f^ma = 0$, equivalently, $a/f^n = 0$ in $A_f$. Hence, $\Phi$ is injective.

As for surjectivity, let $b\in\Gamma(X_f,\scrO_X)$. Due to part (c), there is a positive integer $m$ such that $f^mb = \res^X_{X_f}(a)$ for some $a\in A$ whence $\Phi(a/f^m) = b$. This completes the proof.
\end{enumerate} 
\end{exercise}

\begin{exercise}[A Criterion for Affineness]\thlabel{exer:2.17}\hfill % Exercise 2.17
\begin{enumerate}[label=(\alph*)]
    \item Each $f: f^{-1}U_i\to U_i$ has an inverse $g_i: U_i\to f^{-1}U_i$ that agrees on intersections since inverses are unique. These maps can be glued to give an inverse $g: Y\to X$ of $f$.

    \item First, note that $X = \bigcup\limits_{i = 1}^n X_{f_i}$, for if not, then there is an $x\in X$ such that $x\notin X_{f_i}$ for $1\le i\le n$. Consider an affine open $U = \Spec B$ containing $x$ and let $\frakp$ be the prime corresponding to $x$. According to our hypothesis, $\res^X_{U}(f_i)\in\frakp$ for $1\le i\le n$. But these restrictions generate the unit ideal, a contradiction. 

    Being a finite union of affine opens, $X$ is quasi-compact. Further, $X_{f_i}\cap X_{f_j}$ is a distinguished open in $X_{f_i}$ and hence, is quasi-compact. As a result, \thref{exer:2.16} (d) is applicable. Using \thref{exer:2.4} and glueing morphisms just as in part (a), we are done.
\end{enumerate}
\end{exercise}

\begin{definition}
    A morphism $f: X\to Y$ of schemes is said to be \define{dominant} if $f(X)$ is dense in $Y$.
\end{definition}

\begin{exercise}\thlabel{exer:2.18}\hfill % Exercise 2.18
\begin{enumerate}[label=(\alph*)]
\item Intersection of all prime ideals is the nilradical.

\item We denote the morphism by $\pi: Y\to X$. If $\pi^\sharp$ is injective, then taking global sections, we obtain that $\varphi$ is injective. Conversely, suppose $\varphi$ is injective. It suffices to show that $\varphi^\sharp$ is injective on the $D(f)$'s since these form a base on $X$. We have 
\begin{equation*}
    \pi^\sharp_{D(f)}:\scrO_X(D(f)) \to \scrO(\pi^{-1}(D(f))\equiv \pi^\sharp_{D(f)}: A_f\to B_f,
\end{equation*}
which is injective. This proves the first part.

Next, we must show that $\pi$ is dominant if $\varphi$ is injective. Indeed, suppose $\pi(Y)$ were not dense, then there would be a basic open set $D(f)$ in $\Spec A$ such that $\pi^{-1}D(f) = \emptyset$, equivalently, $f\in\frakq$ for every prime ideal $\frakq$ of $B$. Hence, $f$ is nilpotent in $B$, whence nilpotent in $A$, consequently, $D(f) = \emptyset$. This completes the proof.

\item We denote the morphism by $\pi$. The first part follows from the fact that $\Spec A/\fraka\into\Spec A$ is a topological imbedding. The second part is argued in a similar way as (b) by first concluding surjectivity on basic opens $D(f)$. Then, taking stalks, it follows that $\pi^\sharp$ is surjective.

\item 
\end{enumerate}
\end{exercise}

\newpage
\section{Morphisms}
\begin{definition}
    A \define{variety over $k$} is any affine, quasi-affine, projective, or quasi-projective variety as defined above. If $X, Y$ are two varieties, a \define{morphism} $\varphi: X\to Y$ is a \emph{continuous} map such that for every open set $V\subseteq Y$, and for every regular function $f: V\to k$, the function $f\circ\varphi: \varphi^{-1}(V)\to k$ is regular.
\end{definition}

\setcounter{exercise}{16}
\begin{exercise}[Normal Varieties]
\begin{enumerate}[label=(\alph*)]
    \item
    \item
    \item The coordinate ring $k[t^2, t^3]$ is not integrally closed in its fraction field $k(t)$, whence due to (d), the variety is not normal.
    \item This is immediate from the fact that being integrally closed is a local property, see \cite[Chapter V]{atiyah}.
    \item The construction of $\wt Y$ is obvious. Consider $A(Y)\subseteq K(Y)$ and let $\overline{A(Y)}$ denote the integral closure of the former in the latter. By \textbf{Theorem I.3.9A}, this is an affine $k$-domain, consequently, there is an affine variety $\wt Y$ such that $\overline{A(Y)}\cong A(\wt Y)$ and the \emph{integral} morphism $A(Y)\to A(\wt Y)$ corresponds to a surjection $\wt Y\onto Y$.
    
    Due to \textbf{Theorem I.4.3}, we we may first assume that $Z$ is affine and $\varphi: Z\to Y$ is dominant. Since $\varphi(Z)$ is dense in $Y$, the map $\varphi^\ast: K(Y)\to K(X)$ is well-defined and injective since it is a morphism of fields. The restriction of this map to $A(\wt Y)$ must have image contained in $A(Z)$, since $A(\wt Y)$ is integral over $A(Y)$ and $A(Z)$ is integrally closed in $K(Z)$. This gives a (unique) map $A(\wt Y)\to A(Z)$ extending $\varphi^\ast: A(Y)\to A(Z)$, where uniqueness follows from the fact that $A(\wt Y)\subseteq K(Y)$, the fraction field of $A(Y)$. 

    Once we have shown that a uique lift exists for each affine open in $Z$, it is obvious that these morphisms glue to a global morphism on all of $Z$, where to glue the morphisms on intersections, we make use of the uniqueness on affine opens; recall again that affine opens constitute a base for the topology on $Z$.
\end{enumerate}
\end{exercise}

\setcounter{exercise}{19}
\begin{exercise}
\begin{enumerate}[label=(\alph*)]
    \item Since every variety has a basis of open affine sets (\textbf{Theorem I.4.3}), we may assume that $Y$ is affine. Since $A(Y)_{\frakm_P}$ is an integrally closed domain in its fraction field $K(Y)$, we have 
    \begin{equation*}
        A(Y)_{\frakm_P} = \bigcap_{\frakq\text{ height $1$ in }A(Y)_{\frakm_P}} \left(A(Y)_{\frakm_P}\right)_{\frakq} = \bigcap_{\substack{\hght\frakq = 1\\\frakq\subseteq\frakm_P}} A(Y)_{\frakq}.
    \end{equation*}
    Now, $f$ is a rational function and hence, is equal to $\frac gh$ where $g,h\in A(Y)$ and $h\ne 0$ on $Y\setminus P$. We contend that $h$ is not in any height $1$ prime $\frakq\subseteq\frakm_P$. For if it were, then we could choose a $Q\ne P$ with $\frakq\subseteq\frakm_Q$, since $\frakq\ne\frakm_P$, owing to $\hght\frakm_P = 2$. It follows that $h$ vanishes at $Q$, a contradiction. Hence, $h\notin\frakq$ for all height $1$ primes $\frakq\subseteq\frakm_P$. Consequently, in the above intersection, $\frac gh\in A(Y)_{\frakq}$ for every such $\frakq$, and thus $f\in A(Y)_{\frakm_P}$, that is, $f$ is regular at $P$, as desired.

    \item The rational function $\frac{1}{x}$ on $\bbA^1$ is regular on $\bbA^1\setminus\{0\}$ but does not have a regular extension to $\bbA^1$. Another way to see this is that the inclusion $\bbA^1\setminus\{0\}\into\bbA^1$ correponds to the ring homomorphism $k[x]\to k[x, x^{-1}]$.
\end{enumerate}
\end{exercise}

\chapter{Schemes}

\section{Sheaves}
\begin{definition}
    Let $\scrF$ and $\scrG$ be sheaves of abelian groups on $X$. The association $U\mapsto\Hom(\scrF|_U, \scrG|_U)$ is a sheaf on $X$. It is called the \define{sheaf Hom} and is denoted by $\sheafHom(\scrF, \scrG)$.
\end{definition}

\setcounter{exercise}{14}

\begin{exercise}
\end{exercise}


\newpage 
\section{Schemes}
\setcounter{exercise}{2}
\begin{exercise}[Reduced Schemes]\hfill
\begin{enumerate}[label=(\alph*)]
    \item Suppose $X$ is reduced. Then, every open affine corresponds to a reduced ring. Consequently, the local ring of any point on $X$ is the localisation of a reduced ring and hence, is reduced.

    Conversely, suppose $\scrO_{X, P}$ is reduced for every $P\in X$. Let $U = \Spec A$ be an affine open. The local ring of any point $P\in U$ is a localisation of $A$ at a prime. Since all these rings are reduced, so is $A$.

    Let $U\subseteq X$ be open. Cover $U$ with affine opens $U_i = \Spec A_i$ and let $s\in\scrO(U)$ be nilpotent. Its image $s_i = \res_{U, U_i}(s)$ is nilpotent in $\scrO(U_i) = A_i$ and hence, $s_i = 0$. Consequently $s = 0$ due to the identity axiom. This shows that $\scrO(U)$ is reduced.

    \item The first part follows immediately from the fact that there is a commutative diagram 
    \begin{equation*}
        \xymatrix {
            A\ar[r]^\phi\ar[d] & B\ar[d]\\
            A_{red}\ar[r]_{\phi_{red}} & B_{red}.
        }
    \end{equation*}
    Consider the map of locally ringed spaces $(\id, f^\sharp)$, where $f^\sharp: \scrO_X\to\scrO_X^{red}$ is the collection of the canonical maps $\scrO_X(U)\to\scrO_X^{red}(U)$.

    \item Follows from the fact that any morphism of rings $\phi: A\to B$ with $B$ reduced factors through the natural map $A\to A_{red}$.
\end{enumerate}
\end{exercise}

\begin{exercise}\thlabel{exer:2.4} % Exercise 2.4
    Let $\varphi\in\Hom_{\mathfrak{Rings}}(A,\Gamma(X,\scrO_X))$. Cover $X$ with affine opens $U_i = \Spec A_i$. The restriction map gives us a homomorphism 
    \begin{equation*}
        A\stackrel{\varphi}{\longrightarrow}\Gamma(X,\scrO_X)\xrightarrow{\res^{X}_{U_i}}\Gamma(U_i,\scrO_X) = A_i,
    \end{equation*}
    which induces a map on schemes $\pi_i: U_i\to\Spec A$ where $\pi_i = \Spec(\res^X_{U_i}\circ\varphi)$. 

    We contend that the maps $\pi_i$ can be glued. Indeed, for $i\ne j$, cover $U_i\cap U_j$ with affine opens $U_{ijk} = \Spec A_{ijk}$. Now, 
    \begin{equation*}
        \pi_i|_{U_{ijk}} = \Spec(\res^{U_i}_{U_{ijk}})\circ\pi_i = \Spec(\res^{U_i}_{U_{ijk}}\circ\res^X_{U_i}\circ\varphi) = \Spec(\res^{X}_{U_{ijk}}\circ\varphi).
    \end{equation*}
    Similarly, $\pi_j|_{U_{ijk}} = \Spec(\res^X_{U_{ijk}}\circ\varphi)$, consequently, the family of morphisms $\{\pi_i\}$ can be glued to a morphism $\pi: X\to\Spec A$. This gives a map 
    \begin{equation*}
        \beta:\Hom_{\mathfrak{Rings}}(A,\Gamma(X,\scrO_X))\to\Hom_{\mathfrak{Sch}}(X,\Spec A).
    \end{equation*}
    It is straightforward to verify that $\alpha$ and $\beta$ are inverses to one another.
\end{exercise}

\begin{exercise}
    Follows from the previous exercise and the fact that $\Z$ is an initial object in the category of rings.
\end{exercise}

\setcounter{exercise}{6}
\begin{exercise}
    Let $(f,f^\sharp): \Spec K\to X$ is a morphism of schemes which sends the unique point in $\Spec K$ to $x\in X$. Then, there is an induced map on local rings $f^\sharp_x: \scrO_x\to K$, which must be local and hence, factor through the maximal ideal of $\scrO_x$, thereby inducing a map $k(x)\to K$. It is easy to see that this process is reversible.
\end{exercise}

\setcounter{exercise}{8}
\begin{exercise}
    Let $Z\subseteq X$ be irreducible and closed. Let $U = \Spec A$ be an open affine intersecting $Z$. Then, $Z\cap U$ is open in $Z$ and hence, is irreducible. Further, it is closed in $U$ and hence, corresponds to a prime ideal $\xi = \frakp\in\Spec A$. Note that $\overline{\{\xi\}}\cap U = Z\cap U$ and $\overline{\{\xi\}}\subseteq Z$ since $Z$ is closed.

    Let $V$ be any other open set intersecting $Z$. Then, one can replace $V$ with an open affine $\Spec B$ intersecting $Z$. Suppose $\xi\notin V$. Then, 
    \begin{equation*}
        (Z\cap U)\cap (Z\cap V) = Z\cap U\cap V = \overline{\{\xi\}}\cap U\cap V = \emptyset,
    \end{equation*}
    since the closure of $\{\xi\}$ in $U$ is contained in $U\setminus V$. This is not possible since $Z\cap U$ and $Z\cap V$ are nonempty open sets in an irreducible space. Hence, $\xi$ is a generic point.


    Now we argue for uniqueness. Suppose $\xi_1$ and $\xi_2$ were two generic points in $Z$. Consider an affine neighborhood $U = \Spec A$ intersecting $Z$. Then, $Z\cap U$ must contain $\xi_1$ and $\xi_2$. Let $\xi_i$ correspond to a prime $\frakp_i$ in $A$ for $i = 1,2$. Now, $Z\cap U = V(\frakp_1) = V(\frakp_2)$, consequently, $\frakp_1 = \frakp_2$, that is, $\xi_1 = \xi_2$. This completes the proof.
\end{exercise}

\begin{definition}
    Let $(X,\scrO_X)$ be a scheme and let $f\in\Gamma(X,\scrO_X)$. Define $X_f$ to be the set of all $x\in X$ such that the stalk $f_x$ of $f$ at $x$ is not contained in the maximal ideal $\frakm_x$ of the local ring $\scrO_{X, x}$. This is known as the \define{support} of $f$ on $X$.
\end{definition}

\setcounter{exercise}{15}
\begin{exercise}\thlabel{exer:2.16}\hfill % Exercise 2.16
\begin{enumerate}[label=(\alph*)]
\item The set of all $x\in U$ such that $f_x\notin\frakm_x$ is the set of all prime ideals $\frakp$ in $B$ such that $f/1$ is not in the maximal ideal $\frakp B_\frakp$ in $B_\frakp$. Equivalently, $f\notin\frakp$. Thus, $X_f\cap U = D(\overline f)$. Now, since $X$ can be covered with open affines and the intersection of $X_f$ with every open affine is open, $X_f$ must also be open.

\item Pick a finite open cover $\{U_i = \Spec A_i\}_{i = 1}^m$. The restriction of $a$ to $X_f\cap U_i = D(\res^X_{U_i}(f))$ is zero and hence, there is a positive integer $n_i$ such that $\res^X_{U_i}(f^{n_i}a) = 0$. Let $N = \max\limits_{1\le i\le m} n_i$. Then, $\res^X_{U_i}(f^Na) = 0$. Due to the identity axiom, we must have $f^Na = 0$.

\item Let $U_i = \Spec A_i$ and let $f_i = \res^{X}_{U_i}(f)$. Since $X_f\cap U_i = D(f_i)$, there is a $b_i\in A_i = \Gamma(U_i,\scrO_X)$ such that $\res^X_{U_i\cap X_f}(b) = \frac{b_i}{f_i^{n_i}}$ for some nonnegative integer $n_i$. Choosing $n$ to be larger than all the $n_i$'s, we get that there is a $b_i\in A_i$ such that $\res^{X}_{U_i\cap X_f}(f^nb) = \res^{U_i}_{U_i\cap X_f}(b_i)$.

Now consider $b_i - b_j$ on $U_i\cap U_j$, which can be covered by finitely many affine opens $U_{ijk} = \Spec A_{ijk}$. Since $\res^{X}_{U_i\cap U_j\cap X_f}(b_i - b_j) = 0$, using a similar argument as in (b), there is a positive integer $m_{ij}$ such that $f^{m_{ij}}(b_i - b_j)$ restricts to $0$ On $U_i\cap U_j$. Choosing $m$ larger than $m_{ij}$ for all pairs $i,j$, we have that $f^m(b_i - b_j)$ restricts to $0$ on $U_i\cap U_j$. Consequently, $\res^{U_i}_{U_i\cap U_j}(f^mb_i) = \res^{U_j}_{U_i\cap U_j}(f^mb_j)$ and hence, there is a $c\in\Gamma(X,\scrO_X)$ such that $\res^{X}_{U_i}(c) = f^mb_i$. Hence, $\res^X_{U_i\cap X_f}(c) = \res^X_{U_i\cap X_f}(f^{n + m}b)$. This completes the proof.

\item First, we show that $\res^X_{X_f}(f)$ is invertible. Since $f_x\notin\frakm_x\subseteq\scrO_x$ for every $x\in X_f$, we see that the restriction of $f$ to every affine open contained in $X_f$ must be invertible (else it would lie in a prime ideal and hence, in the stalk of some point). Consider an open cover $U_i$ of $X_f$ using affine opens. There is a $g_i\in\Gamma(U_i,\scrO)$ such that $g_i\res^X_{U_i}(f) = 1$. For $i\ne j$, we have 
\begin{equation*}
    \res^{U_i}_{U_i\cap U_j}(g_i)\res^X_{U_i\cap U_j}(f) = 1 = \res^{U_j}_{U_i\cap U_j}(g_j)\res^X_{U_i\cap U_j}(f)
\end{equation*}
and hence, $\res^{U_i}_{U_i\cap U_j}(g_i) = \res^{U_j}_{U_i\cap U_j}(g_j)$ and hence, the $g_i$'s can be lifted to some $g\in\Gamma(X_f,\scrO_X)$, furthermore $\res^X_{X_f}(f)g = 1$, whence invertibility follows.

Consider the map $\Phi: A_f\to\Gamma(X_f,\scrO_X)$ given by 
\begin{equation*}
    \frac{a}{f^n}\mapsto\frac{\res^X_{X_f}(a)}{\res^{X}_{X_f}(f^n)}.
\end{equation*} 
If $\Phi(a/f^n) = 0$, then $\res^X_{X_f}(a) = 0$, consequently, due to part (b), there is a positive integer $m$ such that $f^ma = 0$, equivalently, $a/f^n = 0$ in $A_f$. Hence, $\Phi$ is injective.

As for surjectivity, let $b\in\Gamma(X_f,\scrO_X)$. Due to part (c), there is a positive integer $m$ such that $f^mb = \res^X_{X_f}(a)$ for some $a\in A$ whence $\Phi(a/f^m) = b$. This completes the proof.
\end{enumerate} 
\end{exercise}

\begin{exercise}[A Criterion for Affineness]\thlabel{exer:2.17}\hfill % Exercise 2.17
\begin{enumerate}[label=(\alph*)]
    \item Each $f: f^{-1}U_i\to U_i$ has an inverse $g_i: U_i\to f^{-1}U_i$ that agrees on intersections since inverses are unique. These maps can be glued to give an inverse $g: Y\to X$ of $f$.

    \item First, note that $X = \bigcup\limits_{i = 1}^n X_{f_i}$, for if not, then there is an $x\in X$ such that $x\notin X_{f_i}$ for $1\le i\le n$. Consider an affine open $U = \Spec B$ containing $x$ and let $\frakp$ be the prime corresponding to $x$. According to our hypothesis, $\res^X_{U}(f_i)\in\frakp$ for $1\le i\le n$. But these restrictions generate the unit ideal, a contradiction. 

    Being a finite union of affine opens, $X$ is quasi-compact. Further, $X_{f_i}\cap X_{f_j}$ is a distinguished open in $X_{f_i}$ and hence, is quasi-compact. As a result, \thref{exer:2.16} (d) is applicable. Using \thref{exer:2.4} and glueing morphisms just as in part (a), we are done.
\end{enumerate}
\end{exercise}

\begin{definition}
    A morphism $f: X\to Y$ of schemes is said to be \define{dominant} if $f(X)$ is dense in $Y$.
\end{definition}

\begin{exercise}\thlabel{exer:2.18}\hfill % Exercise 2.18
\begin{enumerate}[label=(\alph*)]
\item Intersection of all prime ideals is the nilradical.

\item We denote the morphism by $\pi: Y\to X$. If $\pi^\sharp$ is injective, then taking global sections, we obtain that $\varphi$ is injective. Conversely, suppose $\varphi$ is injective. It suffices to show that $\varphi^\sharp$ is injective on the $D(f)$'s since these form a base on $X$. We have 
\begin{equation*}
    \pi^\sharp_{D(f)}:\scrO_X(D(f)) \to \scrO(\pi^{-1}(D(f))\equiv \pi^\sharp_{D(f)}: A_f\to B_f,
\end{equation*}
which is injective. This proves the first part.

Next, we must show that $\pi$ is dominant if $\varphi$ is injective. Indeed, suppose $\pi(Y)$ were not dense, then there would be a basic open set $D(f)$ in $\Spec A$ such that $\pi^{-1}D(f) = \emptyset$, equivalently, $f\in\frakq$ for every prime ideal $\frakq$ of $B$. Hence, $f$ is nilpotent in $B$, whence nilpotent in $A$, consequently, $D(f) = \emptyset$. This completes the proof.

\item We denote the morphism by $\pi$. The first part follows from the fact that $\Spec A/\fraka\into\Spec A$ is a topological imbedding. The second part is argued in a similar way as (b) by first concluding surjectivity on basic opens $D(f)$. Then, taking stalks, it follows that $\pi^\sharp$ is surjective.

\item 
\end{enumerate}
\end{exercise}

\newpage 

\section{First Properties of Schemes}
\begin{definition}
    A \define{variety over $k$} is any affine, quasi-affine, projective, or quasi-projective variety as defined above. If $X, Y$ are two varieties, a \define{morphism} $\varphi: X\to Y$ is a \emph{continuous} map such that for every open set $V\subseteq Y$, and for every regular function $f: V\to k$, the function $f\circ\varphi: \varphi^{-1}(V)\to k$ is regular.
\end{definition}

\setcounter{exercise}{16}
\begin{exercise}[Normal Varieties]
\begin{enumerate}[label=(\alph*)]
    \item
    \item
    \item The coordinate ring $k[t^2, t^3]$ is not integrally closed in its fraction field $k(t)$, whence due to (d), the variety is not normal.
    \item This is immediate from the fact that being integrally closed is a local property, see \cite[Chapter V]{atiyah}.
    \item The construction of $\wt Y$ is obvious. Consider $A(Y)\subseteq K(Y)$ and let $\overline{A(Y)}$ denote the integral closure of the former in the latter. By \textbf{Theorem I.3.9A}, this is an affine $k$-domain, consequently, there is an affine variety $\wt Y$ such that $\overline{A(Y)}\cong A(\wt Y)$ and the \emph{integral} morphism $A(Y)\to A(\wt Y)$ corresponds to a surjection $\wt Y\onto Y$.
    
    Due to \textbf{Theorem I.4.3}, we we may first assume that $Z$ is affine and $\varphi: Z\to Y$ is dominant. Since $\varphi(Z)$ is dense in $Y$, the map $\varphi^\ast: K(Y)\to K(X)$ is well-defined and injective since it is a morphism of fields. The restriction of this map to $A(\wt Y)$ must have image contained in $A(Z)$, since $A(\wt Y)$ is integral over $A(Y)$ and $A(Z)$ is integrally closed in $K(Z)$. This gives a (unique) map $A(\wt Y)\to A(Z)$ extending $\varphi^\ast: A(Y)\to A(Z)$, where uniqueness follows from the fact that $A(\wt Y)\subseteq K(Y)$, the fraction field of $A(Y)$. 

    Once we have shown that a uique lift exists for each affine open in $Z$, it is obvious that these morphisms glue to a global morphism on all of $Z$, where to glue the morphisms on intersections, we make use of the uniqueness on affine opens; recall again that affine opens constitute a base for the topology on $Z$.
\end{enumerate}
\end{exercise}

\setcounter{exercise}{19}
\begin{exercise}
\begin{enumerate}[label=(\alph*)]
    \item Since every variety has a basis of open affine sets (\textbf{Theorem I.4.3}), we may assume that $Y$ is affine. Since $A(Y)_{\frakm_P}$ is an integrally closed domain in its fraction field $K(Y)$, we have 
    \begin{equation*}
        A(Y)_{\frakm_P} = \bigcap_{\frakq\text{ height $1$ in }A(Y)_{\frakm_P}} \left(A(Y)_{\frakm_P}\right)_{\frakq} = \bigcap_{\substack{\hght\frakq = 1\\\frakq\subseteq\frakm_P}} A(Y)_{\frakq}.
    \end{equation*}
    Now, $f$ is a rational function and hence, is equal to $\frac gh$ where $g,h\in A(Y)$ and $h\ne 0$ on $Y\setminus P$. We contend that $h$ is not in any height $1$ prime $\frakq\subseteq\frakm_P$. For if it were, then we could choose a $Q\ne P$ with $\frakq\subseteq\frakm_Q$, since $\frakq\ne\frakm_P$, owing to $\hght\frakm_P = 2$. It follows that $h$ vanishes at $Q$, a contradiction. Hence, $h\notin\frakq$ for all height $1$ primes $\frakq\subseteq\frakm_P$. Consequently, in the above intersection, $\frac gh\in A(Y)_{\frakq}$ for every such $\frakq$, and thus $f\in A(Y)_{\frakm_P}$, that is, $f$ is regular at $P$, as desired.

    \item The rational function $\frac{1}{x}$ on $\bbA^1$ is regular on $\bbA^1\setminus\{0\}$ but does not have a regular extension to $\bbA^1$. Another way to see this is that the inclusion $\bbA^1\setminus\{0\}\into\bbA^1$ correponds to the ring homomorphism $k[x]\to k[x, x^{-1}]$.
\end{enumerate}
\end{exercise}

\newpage

\section{Separated and Proper Morphisms}
\begin{definition}
    A morphism $\pi:X\to Y$ of schemes is said to be \define{separated} if the diagonal morphism $\Delta: X\to X\times_Y X$ is a closed immersion.
    \begin{equation*}
        \xymatrix{
            X\ar[rd]^\Delta\ar@/^/[rrd]^{\id_X}\ar@/_/[rdd]_{\id_X} & & \\
            & X\times_Y X\ar[r]^-{p_1}\ar[d]_{p_2} & X\ar[d]^{\pi}\\
            & X\ar[r]_\pi & Y
        }
    \end{equation*}
\end{definition}

\begin{definition}
    A morphism $\pi: X\to Y$ is said to be \define{universally closed} if it is closed as a continuous map on the underlying topological spaces and for every morphism $Y'\to Y$, the map obtained by \emph{base extension} $X\times_Y Y'\to Y'$ is also closed.
\end{definition}

\begin{definition}
    A morphism $\pi: X\to Y$ is said to be \define{proper} if it is separated, of finite type and universally closed.
\end{definition}

Since a complete proof of the following is not provided in the text, I reproduce it here.

\begin{corollary}[Hartshorne, II.4.6]
    Assume that all schemes are noetherian in the following statements.
    \begin{enumerate}[label=(\alph*)]
        \item Open and closed immersions are separated.
        \item A composition of two separated morphisms is seprated. 
        \item Separated morphisms are stable under base extension.
        \item If $\pi: X\to Y$ and $\pi': X'\to Y'$ are separated morphisms of schemes over a base scheme $S$, then the \define{product morphism} $\pi\times\pi': X\times_S X'\to Y\times_S Y'$ is also separated.
        \item If $\pi: X\to Y$ and $\varphi: Y\to Z$ are two morphisms and if $\varphi\circ\pi$ is separated, then $\pi$ is separated.
        \item A morphism $\pi: X\to Y$ is separated if and only if $Y$ can be covered by open subsets $V_i$ such that $\pi^{-1}V_i\to V_i$ is separated for each $i$.
    \end{enumerate}
\end{corollary}
\begin{proof}
\begin{enumerate}[label=(\alph*)]
\item We show more generally that ``a monomorphism of schemes is separated''. Let $Y\into X$ be a monomorphism in $\mathfrak{Sch}_\Z$. Then, the fiber product $Y\times_X Y$ is precisely $Y$, given by the following diagram. 
\begin{equation*}
    \xymatrix {
    Z\ar@/^/[rrd]\ar@/_/[rdd] & & \\
        &Y\ar[d]_\id\ar[r]^\id & Y\ar@{^(->}[d]\\
        &Y\ar@{^(->}[r] & X
    }
\end{equation*}
Since $Y\into X$ is a monomorphism, the two maps $Z\to Y$ in the above diagram must be the same and it follows that $Y = Y\times_X Y$. Hence, the diagonal morphism $\Delta: Y\to Y\times_X Y$ is the identity map, whence is a closed immersion.

\item We use the valuative criterion. Let $R$ be a DVR and $K$ its fraction field. Let $U = \Spec K$ and $T = \Spec R$ and suppose $\pi: X\to Y$ and $\varphi: Y\to Z$ are separated. Let there be a commutative diagram 
\begin{equation*}
    \xymatrix {
        U\ar[r]\ar[dd] & X\ar[d]\\
        & Y\ar[d]\\
        T\ar[r] & Z.
    }
\end{equation*}
Suppose there are two lifts $\psi_1,\psi_2: T\to X$ making the diagram commute. Then, $\pi\circ\psi_1 = \pi\circ\psi_2$ since $Y\to Z$ is separated. Finally, since $X\to Y$ is separated, we must have $\psi_1 = \psi_2$. This shows that $X\to Z$ is separated.

\item This is done in the book. 

\item The same idea as in (b) works. Not writing this up because the diagram is too complicated to draw and I'm too lazy.

\item Again, begin with a commutative diagram 
\begin{equation*}
    \xymatrix {
        U\ar[r]\ar[d] & X\ar[d]\\
        T\ar[r]\ar[rd] & Y\ar[d]\\
        & Z
    }
\end{equation*}
and suppose there are two lifts $\psi_1,\psi_2: T\to X$ making the diagram commute. Since $X\to Z$ is separated, we must have that $\psi_1 = \psi_2$. Hence, $X\to Y$ is separated.

\item 
\end{enumerate}
\end{proof}

\newpage 

\section{Sheaves of Modules}
\begin{definition}
    An $\scrO_X$-module $\scrF$ is said to be \define{free} if it is isomorphic to a direct sum of copies of $\scrO_X$. It is said to be \define{locally free} if $X$ has an open cover by sets $U$ for which $\scrF|_U$ is a free $\scrO_X|_U$-module.
\end{definition}

\setcounter{exercise}{6}
\begin{exercise}\thlabel{exer:5.7} \hfill % Exercise 5.7
\begin{enumerate}[label=(\alph*)]
\item We reduce this to the affine case since $\scrF$ is coherent on a noetherian scheme. Thus, we have a finitely generated $A$-module $M$ and a prime ideal $\frakp\in\Spec A$ such that $M_\frakp$ is a free $A_\frakp$-module. 

Choose a basis $\left\{\frac{m_1}{1},\dots,\frac{m_n}{1}\right\}$ of $M_\frakp$ over $A_\frakp$ and consider the exact sequence 
\begin{equation*}
    0\to K\to A^n\to M\to Q\to 0,
\end{equation*}
where the map $A^n\to M$ is the natural map sending $e_i\mapsto m_i$ for $1\le i\le n$. Localising, we see that $K_\frakp = Q_\frakp = 0$ and hence, there is an $f\in A\setminus\frakp$ such that $K_f = Q_f = 0$ (since both $K$ and $Q$ are finitely generated). Localising the above exact sequence at $f$, we obtain an isomorphism $A^n_f\xrightarrow{\sim} M_f$. It follows that $\scrF|_{D(f)}$ is a free sheaf.

\item Follows immediately from (a). 

% \item Let $\scrF^\vee$ denote the dual sheaf. Recall that 
% \begin{equation*}
%     \scrF^\vee(U) = \Hom_{\scrO_X|_U}\left(\scrF|_U,\scrO_X|_U\right).
% \end{equation*}
% This gives a natural map $\scrF(U)\otimes\scrF^\vee(U) \to\scrO_X(U)$ given by 
% \begin{equation*}
%     s\otimes\varphi\mapsto\varphi_U(s).
% \end{equation*}
% It is easy to check that this is a morphism of presheaves $\scrF\otimes\scrF^\vee\to\scrO_X$ and since the latter is a sheaf, it factors through the sheafification inducing a map on the tensor sheaf.

% We contend that this induced map is an isomorphism. To this end, it suffices to show that the induced morphism on stalks is an isomorphism.
\end{enumerate}
\end{exercise}

\begin{exercise}\hfill 
\begin{enumerate}[label=(\alph*)]
    \item 
    \item This is a topological property of connected spaces and has nothing to do with algebraic geometry.
    \item We shall use \thref{exer:5.7} (b) to show that $\scrF$ is locally free. To this end, we need to show that $\scrF_x$ is a free $\scrO_{X, x}$-module for each $x\in X$. Let $U = \Spec A$ be an open affine neighborhood of $x$ in $X$ on which $\varphi$ is constant. Let $\frakp\in\Spec A$ be the prime corresponding to the point $x\in U$. Thus, we have a finite $A$-module $M$ such that $\scrF|_{U} = \wt M$. Using Nakayama's lemma, we can find a minimal generating set $m_1,\dots,m_r\in M_\frakp$, where $r = \varphi(x)$ , which gives a surjection $A_\frakp^r\onto M_\frakp$. This can be localized at each prime $\frakq\subseteq\frakp$, and hence $m_1,\dots,m_r\in M_\frakq$ generate it as an $A_\frakq$-module. But since $\varphi(\frakq) = r$, it follows that $m_1,\dots,m_r\in M_\frakq$ is a minimal generating set for each prime $\frakq\subseteq\frakp$.

    Finally, we claim that $m_1,\dots,m_r$ freely generate $M_\frakp$. Indeed, suppose $a_1m_1 + \dots + a_rm_r = 0$ for $a_i\in A_\frakp$. This equality is true for $M_\frakq$ as an $A_\frakq$-module and hence, all the coefficients lie in $\frakq A_\frakq$, therefore, all the coefficients lie in $\frakq A_\frakp$ for all primes $\frakq\subseteq\frakp$. But since $A_\frakp$ is reduced, it follows that $a_i = 0$ for all $1\le i\le r$ in $A_\frakp$. Hence $M_\frakp = \scrF_x$ is a free $A_\frakp = \scrO_{X, x}$-module, as desired.
\end{enumerate}
\end{exercise}

\bibliographystyle{alpha}
\bibliography{references}
\end{document}