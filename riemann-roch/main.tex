\documentclass[12pt]{article}

% \usepackage{./arxiv}

\title{Riemann-Roch for Riemann Surfaces}
\author{Swayam Chube}
\date{\today}

\usepackage[utf8]{inputenc} % allow utf-8 input
\usepackage[T1]{fontenc}    % use 8-bit T1 fonts
\usepackage{hyperref}       % hyperlinks
\usepackage{url}            % simple URL typesetting
\usepackage{booktabs}       % professional-quality tables
\usepackage{amsfonts}       % blackboard math symbols
\usepackage{nicefrac}       % compact symbols for 1/2, etc.
\usepackage{microtype}      % microtypography
\usepackage{graphicx}
\usepackage{natbib}
\usepackage{doi}
\usepackage{amssymb}
\usepackage{bbm}
\usepackage{amsthm}
\usepackage{amsmath}
\usepackage{xcolor}
\usepackage{theoremref}
\usepackage{enumitem}
\usepackage{mathpazo}
% \usepackage{euler}
\usepackage{mathrsfs}
\setlength{\marginparwidth}{2cm}
\usepackage{todonotes}
\usepackage{stmaryrd}
\usepackage[all,cmtip]{xy} % For diagrams, praise the Freyd–Mitchell theorem 
\usepackage{marvosym}
\usepackage{geometry}
\usepackage{titlesec}
\usepackage{tikz}
\usetikzlibrary{cd}

\renewcommand{\qedsymbol}{$\blacksquare$}

% Uncomment to override  the `A preprint' in the header
% \renewcommand{\headeright}{}
% \renewcommand{\undertitle}{}
% \renewcommand{\shorttitle}{}

\hypersetup{
    pdfauthor={Lots of People},
    colorlinks=true,
}

\newtheoremstyle{thmstyle}%               % Name
  {}%                                     % Space above
  {}%                                     % Space below
  {}%                             % Body font
  {}%                                     % Indent amount
  {\bfseries\scshape}%                            % Theorem head font
  {.}%                                    % Punctuation after theorem head
  { }%                                    % Space after theorem head, ' ', or \newline
  {\thmname{#1}\thmnumber{ #2}\thmnote{ (#3)}}%                                     % Theorem head spec (can be left empty, meaning `normal')

\newtheoremstyle{defstyle}%               % Name
  {}%                                     % Space above
  {}%                                     % Space below
  {}%                                     % Body font
  {}%                                     % Indent amount
  {\bfseries\scshape}%                            % Theorem head font
  {.}%                                    % Punctuation after theorem head
  { }%                                    % Space after theorem head, ' ', or \newline
  {\thmname{#1}\thmnumber{ #2}\thmnote{ (#3)}}%                                     % Theorem head spec (can be left empty, meaning `normal')

\theoremstyle{thmstyle}
\newtheorem{theorem}{Theorem}[section]
\newtheorem{lemma}[theorem]{Lemma}
\newtheorem{proposition}[theorem]{Proposition}

\theoremstyle{defstyle}
\newtheorem{definition}[theorem]{Definition}
\newtheorem{corollary}[theorem]{Corollary}
\newtheorem{remark}[theorem]{Remark}
\newtheorem{example}[theorem]{Example}
\newtheorem*{notation}{Notation}

% Common Algebraic Structures
\newcommand{\R}{\mathbb{R}}
\newcommand{\Q}{\mathbb{Q}}
\newcommand{\Z}{\mathbb{Z}}
\newcommand{\N}{\mathbb{N}}
\newcommand{\bbC}{\mathbb{C}} 
\newcommand{\K}{\mathbb{K}} % Base field which is either \R or \bbC
\newcommand{\calA}{\mathcal{A}} % Banach Algebras
\newcommand{\calB}{\mathcal{B}} % Banach Algebras
\newcommand{\calI}{\mathcal{I}} % ideal in a Banach algebra
\newcommand{\calJ}{\mathcal{J}} % ideal in a Banach algebra
\newcommand{\frakM}{\mathfrak{M}} % sigma-algebra
\newcommand{\calO}{\mathcal{O}} % Ring of integers
\newcommand{\bbA}{\mathbb{A}} % Adele (or ring thereof)
\newcommand{\bbI}{\mathbb{I}} % Idele (or group thereof)

% Categories
\newcommand{\catTopp}{\mathbf{Top}_*}
\newcommand{\catGrp}{\mathbf{Grp}}
\newcommand{\catTopGrp}{\mathbf{TopGrp}}
\newcommand{\catSet}{\mathbf{Set}}
\newcommand{\catTop}{\mathbf{Top}}
\newcommand{\catRing}{\mathbf{Ring}}
\newcommand{\catCRing}{\mathbf{CRing}} % comm. rings
\newcommand{\catMod}{\mathbf{Mod}}
\newcommand{\catMon}{\mathbf{Mon}}
\newcommand{\catMan}{\mathbf{Man}} % manifolds
\newcommand{\catDiff}{\mathbf{Diff}} % smooth manifolds
\newcommand{\catAlg}{\mathbf{Alg}}
\newcommand{\catRep}{\mathbf{Rep}} % representations 
\newcommand{\catVec}{\mathbf{Vec}}

% Group and Representation Theory
\newcommand{\chr}{\operatorname{char}}
\newcommand{\Aut}{\operatorname{Aut}}
\newcommand{\GL}{\operatorname{GL}}
\newcommand{\im}{\operatorname{im}}
\newcommand{\tr}{\operatorname{tr}}
\newcommand{\id}{\mathbf{id}}
\newcommand{\cl}{\mathbf{cl}}
\newcommand{\Gal}{\operatorname{Gal}}
\newcommand{\Tr}{\operatorname{Tr}}
\newcommand{\sgn}{\operatorname{sgn}}
\newcommand{\Sym}{\operatorname{Sym}}
\newcommand{\Alt}{\operatorname{Alt}}

% Commutative and Homological Algebra
\newcommand{\spec}{\operatorname{spec}}
\newcommand{\mspec}{\operatorname{m-spec}}
\newcommand{\Tor}{\operatorname{Tor}}
\newcommand{\tor}{\operatorname{tor}}
\newcommand{\Ann}{\operatorname{Ann}}
\newcommand{\Supp}{\operatorname{Supp}}
\newcommand{\Hom}{\operatorname{Hom}}
\newcommand{\End}{\operatorname{End}}
\newcommand{\coker}{\operatorname{coker}}
\newcommand{\limit}{\varprojlim}
\newcommand{\colimit}{%
  \mathop{\mathpalette\colimit@{\rightarrowfill@\textstyle}}\nmlimits@
}
\makeatother


\newcommand{\fraka}{\mathfrak{a}} % ideal
\newcommand{\frakb}{\mathfrak{b}} % ideal
\newcommand{\frakc}{\mathfrak{c}} % ideal
\newcommand{\frakf}{\mathfrak{f}} % face map
\newcommand{\frakg}{\mathfrak{g}}
\newcommand{\frakh}{\mathfrak{h}}
\newcommand{\frakm}{\mathfrak{m}} % maximal ideal
\newcommand{\frakn}{\mathfrak{n}} % naximal ideal
\newcommand{\frakp}{\mathfrak{p}} % prime ideal
\newcommand{\frakq}{\mathfrak{q}} % qrime ideal
\newcommand{\fraks}{\mathfrak{s}}
\newcommand{\frakt}{\mathfrak{t}}
\newcommand{\frakz}{\mathfrak{z}}
\newcommand{\frakA}{\mathfrak{A}}
\newcommand{\frakI}{\mathfrak{I}}
\newcommand{\frakJ}{\mathfrak{J}}
\newcommand{\frakK}{\mathfrak{K}}
\newcommand{\frakL}{\mathfrak{L}}
\newcommand{\frakN}{\mathfrak{N}} % nilradical 
\newcommand{\frakO}{\mathfrak{O}} % dedekind domain
\newcommand{\frakP}{\mathfrak{P}} % Prime ideal above
\newcommand{\frakQ}{\mathfrak{Q}} % Qrime ideal above 
\newcommand{\frakR}{\mathfrak{R}} % jacobson radical
\newcommand{\frakU}{\mathfrak{U}}
\newcommand{\frakV}{\mathfrak{V}}
\newcommand{\frakW}{\mathfrak{W}}
\newcommand{\frakX}{\mathfrak{X}}

% General/Differential/Algebraic Topology 
\newcommand{\scrA}{\mathscr A}
\newcommand{\scrB}{\mathscr B}
\newcommand{\scrF}{\mathscr F}
\newcommand{\scrN}{\mathscr N}
\newcommand{\scrP}{\mathscr P}
\newcommand{\scrR}{\mathscr R}
\newcommand{\scrS}{\mathscr S}
\newcommand{\bbH}{\mathbb H}
\newcommand{\Int}{\operatorname{Int}}
\newcommand{\psimeq}{\simeq_p}
\newcommand{\wt}[1]{\widetilde{#1}}
\newcommand{\RP}{\mathbb{R}\text{P}}
\newcommand{\CP}{\mathbb{C}\text{P}}

% Miscellaneous
\newcommand{\wh}[1]{\widehat{#1}}
\newcommand{\calM}{\mathcal{M}}
\newcommand{\calP}{\mathcal{P}}
\newcommand{\onto}{\twoheadrightarrow}
\newcommand{\into}{\hookrightarrow}
\newcommand{\Gr}{\operatorname{Gr}}
\newcommand{\Span}{\operatorname{Span}}
\newcommand{\ev}{\operatorname{ev}}
\newcommand{\weakto}{\stackrel{w}{\longrightarrow}}

\newcommand{\define}[1]{\textcolor{blue}{\textit{#1}}}
\newcommand{\caution}[1]{\textcolor{red}{\textit{#1}}}
\renewcommand{\mod}{~\mathrm{mod}~}
\renewcommand{\le}{\leqslant}
\renewcommand{\leq}{\leqslant}
\renewcommand{\ge}{\geqslant}
\renewcommand{\geq}{\geqslant}
\newcommand{\Res}{\operatorname{Res}}
\newcommand{\floor}[1]{\left\lfloor #1\right\rfloor}
\newcommand{\ceil}[1]{\left\lceil #1\right\rceil}
\newcommand{\gl}{\mathfrak{gl}}
\newcommand{\ad}{\operatorname{ad}}
\newcommand{\Stab}{\operatorname{Stab}}
\newcommand{\bfX}{\mathbf{X}}
\newcommand{\Ind}{\operatorname{Ind}}
\newcommand{\bfG}{\mathbf{G}}
\newcommand{\rank}{\operatorname{rank}}
\newcommand{\calo}{\mathcal{o}}
\newcommand{\frako}{\mathfrak{o}}
\newcommand{\Cl}{\operatorname{Cl}}

\newcommand{\idim}{\operatorname{idim}}
\newcommand{\pdim}{\operatorname{pdim}}
\newcommand{\Ext}{\operatorname{Ext}}
\newcommand{\co}{\operatorname{co}}
\newcommand{\bfO}{\mathbf{O}}
\newcommand{\bfF}{\mathbf{F}} % Fitting Subgroup
\newcommand{\Syl}{\operatorname{Syl}}
\newcommand{\nor}{\vartriangleleft}
\newcommand{\noreq}{\trianglelefteqslant}
\newcommand{\subnor}{\nor\!\nor}
\newcommand{\Soc}{\operatorname{Soc}}
\newcommand{\core}{\operatorname{core}}

\geometry {
    margin = 1in
}

\titleformat
{\section}
[block]
{\Large\bfseries\scshape}
{\S\thesection}
{0.5em}
{\centering}
[]


\titleformat
{\subsection}
[block]
{\normalfont\bfseries\sffamily}
{\S\S}
{0.5em}
{\centering}
[]


\begin{document}
\maketitle
\tableofcontents

\section{\v{C}ech Cohomology}

\begin{definition}
    Let $X$ be a topological space and $\scrF$ a sheaf of abelian groups on $X$. Let $\frakU = (U_i)_{i\in I}$ be an open cover of $X$. Define the \define{$q$-th cochain group} of $\scrF$ with respect to $\frakU$ as 
    \begin{equation*}
        C^q(\frakU,\scrF) = \prod_{(i_0,\dots,i_q)\in I^{q + 1}}\scrF(U_{i_0}\cap\dots\cap U_{i_q}).
    \end{equation*}
    We denote elements of this cochain group by $\displaystyle \left(f_{i_0,\dots,i_q}\right)_{(i_0,\dots,i_{q})\in I^{q + 1}}$ and these are called $q$-cochains.

    Next, define the \define{coboundary operators} 
    \begin{equation*}
        \delta: C^0(\frakU,\scrF)\to C^1(\frakU,\scrF)\qquad\delta: C^1(\frakU,\scrF)\to C^2(\frakU,\scrF)
    \end{equation*}
    as follows: 
    \begin{itemize}
        \item For $(f_i)_{i\in I}\in C^0(\frakU,\scrF)$, let $\delta\left((f_i)_{i\in I}\right) = (g_{ij})_{i,j\in I}\in C^2(\frakU,\scrF)$ where 
        \begin{equation*}
            g_{ij} = f_j|_{U_i\cap U_j} - f_i|_{U_i\cap U_j}\in\scrF(U_i\cap U_j).
        \end{equation*}

        \item For $(f_{ij})_{i, j\in I}\in C^1(\frakU,\scrF)$, let $\delta\left((f_{ij})_{i, j\in I}\right) = (g_{ijk})_{i, j,k\in I}$ where 
        \begin{equation*}
            g_{ijk} = f_{jk}|_{U_i\cap U_j\cap U_k} - f_{ik}|_{U_i\cap U_j\cap U_k} + f_{ij}|_{U_i\cap U_j\cap U_k}\in\scrF(U_i\cap U_j\cap U_k).
        \end{equation*}
    \end{itemize}
\end{definition}

The coboundary operators are group homomorphisms and let
\begin{equation*}
    Z^1(\frakU,\scrF) = \ker\left(C^1(\frakU,\scrF)\xrightarrow\delta C^2(\frakU,\scrF)\right)\qquad B^1(\frakU,\scrF) = \im\left(C^0(\frakU,\scrF)\xrightarrow{\delta} C^1(\frakU,\scrF)\right).
\end{equation*}
The elements of $Z^1(\frakU,\scrF)$ are called $1$-cocycles and the elements of $B^1(\frakU,\scrF)$ are called $1$-coboundaries. It is easy to see that $(f_{ij})_{i,j\in I}$ is a $1$-cocycle if and only if 
\begin{equation*}
    f_{ik}|_{U_i\cap U_j\cap U_k} = f_{jk}|_{U_i\cap U_j\cap U_k} + f_{ij}|_{U_i\cap U_j\cap U_k}\in\scrF(U_i\cap U_j\cap U_k).
\end{equation*}
The above relation is called the \define{cocycle relation}. Indeed, if $(f_{ij})_{i, j\in I}$ is a $1$-cocycle, then taking $i = j$, we see that 
\begin{equation*}
    f_{ii}|_{U_i\cap U_k} = 0\qquad\forall k\in I.
\end{equation*}
Since the $U_k$'s cover $U_i$, using the identity axiom, we have that $f_{ii} = 0\in\scrF(U_i)$. As a consequence, we also see that 
\begin{equation*}
    f_{ji}|_{U_i\cap U_j\cap U_k} + f_{ij}|_{U_i\cap U_j\cap U_k} = 0.
\end{equation*}
Again, using the same argument, we have that $f_{ij} + f_{ji} = 0\in\scrF(U_i\cap U_j)$. It immediately follows from the above discussion that $\delta\circ\delta = 0$ as a map $C^0(\frakU,\scrF)\to C^2(\frakU,\scrF)$.

\begin{definition}
    The group 
    \begin{equation*}
        H^1(\frakU,\scrF) := \frac{Z^1(\frakU,\scrF)}{B^1(\frakU,\scrF)}
    \end{equation*}
    is called the \define{$1$-st cohomology group} with coefficients in $\scrF$ \define{with respect to the covering} $\frakU$.
\end{definition}

\begin{definition}
    Let $\frakU = (U_i)_{i\in I}$ and $\frakV = (V_k)_{k\in K}$ be two open covers of $X$. We say that $\frakV$ is \define{finer} than $\frakU$ if every $V_k$ is contained in some $U_i$.
\end{definition}

Thus, there is a map $\tau: K\to I$ such that $V_k\subseteq U_{\tau(k)}$. This defines a mapping 
\begin{equation*}
    t^{\frakU}_{\frakV}: Z^1(\frakU,\scrF)\to Z^1(\frakV,\scrF)
\end{equation*}
as follows: for $(f_{ij})\in Z^1(\frakU,\scrF)$, let $t^{\frakU}_{\frakV}\left((f_{ij})\right) = (g_{kl})$, where 
\begin{equation*}
    g_{kl} = f_{\tau(k)\tau(l)}|_{V_k\cap V_l}\qquad\forall~k, l\in K.
\end{equation*}
To see that this map is indeed well-defined, we need only check that $(g_{kl})$ is a $1$-cocycle. To this end, we must check that the cocycle condition is satisfied. Indeed, for indices $k,l,m\in K$, we have 
\begin{align*}
    g_{km}|_{V_k\cap V_l\cap V_m} &= f_{\tau(k)\tau(m)}|_{V_k\cap V_l\cap V_m}\\
    &= \left(f_{\tau(k)\tau(l)}|_{U_{\tau(k)}\cap U_{\tau(l)}\cap U_{\tau(m)}} + f_{\tau(l)\tau(m)}|_{U_{\tau(k)}\cap U_{\tau(l)}\cap U_{\tau(m)}}\right)|_{V_k\cap V_l\cap V_m}\\
    &= f_{\tau(k)\tau(l)}|_{V_k\cap V_l\cap V_m} + f_{\tau(k)\tau(l)}|_{V_k\cap V_l\cap V_m}\\
    &= g_{kl}|_{V_k\cap V_l\cap V_m} + g_{lm}|_{V_k\cap V_l\cap V_m},
\end{align*}
as desired. Further, we claim that the above map takes $1$-coboundaries to $1$-coboundaries. Indeed, suppose $(f_{ij})_{i,j\in I}$ is a $1$-coboundary, that is, there is some $(f_i)_{i\in I}$ such that 
\begin{equation*}
    f_{ij} = f_i|_{U_i\cap U_j} - f_j|_{U_i\cap U_j}.
\end{equation*}
Let $(g_k)_{k\in K}$ be such that $g_k = f_{\tau(k)}|_{V_k}$. Then $\delta\left((g_k)\right) = (g_{kl})$ where 
\begin{align*}
    g_{kl} &= g_k|_{V_k\cap V_l} - g_l|_{V_k\cap V_l}\\
    &= f_{\tau(k)}|_{V_k\cap V_l} - f_{\tau(l)}|_{V_k\cap V_l}\\
    &= f_{\tau(k)\tau(l)}|_{V_k\cap V_l},
\end{align*}
that is, $(g_kl) = t^{\frakU}_{\frakV}\left((f_{ij})\right)$, that is, $t^{\frakU}_{\frakV}$ takes $1$-coboundaries to $1$-coboundaries. This induces a map 
\begin{equation*}
    t^{\frakU}_{\frakV}: H^1(\frakU,\scrF)\to H^1(\frakV,\scrF).
\end{equation*}

\begin{lemma}
    The map $t^{\frakU}_{\frakV}$ induced on cohomology is independent of the choice of $\tau: K\to I$.
\end{lemma}
\begin{proof}
    Suppose $\wt\tau: K\to I$ is another such mapping. Suppose $(f_{ij})\in Z^1(\frakU,\scrF)$ and let 
    \begin{equation*}
        g_{kl} = f_{\tau(k)\tau(l)}|_{V_k\cap V_l}\qquad\text{and}\qquad\wt g_{kl} = f_{\wt\tau(k)\wt\tau(l)}|_{V_k\cap V_l}.
    \end{equation*}
    We must show that the cocycles $(g_{kl})$ and $(\wt g_{kl})$ are cohomologous, that is, their difference lies in $B^1(\frakV,\scrF)$. Define
    \begin{equation*}
        h_k = f_{\tau(k),\wt\tau(k)}|_{V_k}\in\scrF(V_k).
    \end{equation*}
    Then, on $V_k\cap V_l$, we have 
    \begin{align*}
        g_{kl} - \wt g_{kl} &= f_{\tau(k)\tau(l)} - f_{\wt\tau(k)\wt\tau(l)}\\
        &= f_{\tau(k)\tau(l)} + f_{\tau(l),\wt\tau(k)} - f_{\tau(l)\wt\tau(k)} - f_{\wt\tau(k)\wt\tau(l)}\\
        &= f_{\tau(k)\wt\tau(k)} - f_{\tau(l)\wt\tau(l)}\\
        &= h_k - h_l.
    \end{align*}
    Whence $(g_{kl} - \wt g_{kl})$ is a coboundary, as desired.
\end{proof}

\begin{lemma}
    The map $t^{\frakU}_{\frakV}: H^1(\frakU,\scrF)\to H^1(\frakV,\scrF)$ is injective.
\end{lemma}
\begin{proof}
    Suppose $(f_{ij})\in Z^1(\frakU,\scrF)$ is a $1$-cocycle whose image in $Z^1(\frakV,\scrF)$ is a $1$-coboundary. Then, there is some $(g_k)\in C^1(\frakU,\scrF)$ such that 
    \begin{equation*}
        f_{\tau(k)\tau(l)}|_{V_k\cap V_l} = g_k|_{V_k\cap V_l} - g_l|_{V_k\cap V_l}.
    \end{equation*}
    Then on $U_i\cap V_k\cap V_l$, we have 
    \begin{equation*}
        g_k - g_l = f_{\tau(k)\tau(l)} = f_{\tau(k)i} + f_{i\tau(l)} = f_{i\tau(l)} - f_{i\tau(k)}.
    \end{equation*}
    Hence $f_{i\tau(k)} + g_k = f_{i\tau(l)} + g_l$ on $U_i\cap V_k\cap V_l$. The gluability axiom applied to the open cover $\{U_i\cap V_k\}_{k\in K}$ furnishes a $h_i\in\scrF(U_i)$ such that 
    \begin{equation*}
        h_i = f_{i\tau(k)} + g_k\qquad\text{on } U_i\cap V_k\text{ for all }k\in K.
    \end{equation*}
    Then, on $U_i\cap U_j\cap V_k$ we have 
    \begin{equation*}
        f_{ij} = f_{i\tau(k)} + f_{\tau(k)j} = f_{i\tau(k)} + g_k - f_{j\tau(k)} - g_k = h_i - h_j.
    \end{equation*}
    Since $\{U_i\cap U_j\cap V_k\}$ forms an open cover of $U_i\cap U_j$, using the identity axiom, we have that $f_{ij} = h_i - h_j$ on $U_i\cap U_j$. Thus, $(f_{ij})\in B^1(\frakU,\scrF)$, thereby completing the proof.
\end{proof}

\begin{definition}
    If $\frakW < \frakV < \frakU$ are open covers of $X$, then it is easy to see that $t^{\frakV}_{\frakW}\circ t^{\frakU}_{\frakV} = t^{\frakU}_{\frakW}$. This gives us a directed system of cohomology groups, and we define 
    \begin{equation*}
        H^1(X,\scrF) = \varinjlim_{\frakU} H^1(\frakU,\scrF).
    \end{equation*}
\end{definition}
\begin{remark}\thlabel{rem:vanishing-cohomology}
    Note that since the $t^{\frakU}_{\frakV}$ are all injective, the natural map $H^1(\frakU,\scrF)\to H^1(X,\scrF)$ is injective. In particular, this means that $H^1(X,\scrF) = 0$ if and only if $H^1(\frakU,\scrF) = 0$ for every open cover $\frakU$ of $X$.
\end{remark}

\begin{theorem}[Leray]
    Let $\scrF$ be a sheaf of abelian groups on the topological space $X$ and $\frakU = (U_i)_{i\in I}$ be an open cover of $X$ such that $H^1(U_i,\scrF) = 0$ for every $i\in I$. Then for every open covering $\frakV = (V_\alpha)_{\alpha\in A} < \frakU$, the mapping 
    \begin{equation*}
        t^{\frakU}_{\frakV} : H^1(\frakU,\scrF)\to H^1(\frakV,\scrF)
    \end{equation*}
    is an isomorphism. The covering $\frakU$ is called a \define{Leray covering} of $X$ for the sheaf $\scrF$.
\end{theorem}
\begin{proof}
    Let $\tau: A\to I$ be such that $V_\alpha\subseteq U_{\tau(\alpha)}$ for every $\alpha\in A$. Since we know that $t^{\frakU}_{\frakV}$ is injective, we would like to show that it is surjective. Let $(f_{\alpha\beta})\in Z^1(\frakV,\scrF)$. The family $(U_i\cap V_\alpha)_{\alpha\in A}$ is an open covering of $U_i$, which we denote by $U_i\cap\frakV$. By assumption and \thref{rem:vanishing-cohomology}, we know that $H^1(U_i\cap\frakV,\scrF) = 0$, that is, there exist $g_{i\alpha}\in\scrF(U_i\cap V_\alpha)$ such that 
    \begin{equation*}
        f_{\alpha\beta} = g_{i\alpha} - g_{i\beta}\quad\text{ on } U_i\cap V_\alpha\cap V_\beta.
    \end{equation*}
    On the intersection $U_i\cap U_j\cap V_\alpha\cap V_\beta$, we have 
    \begin{equation*}
        g_{j\alpha} - g_{i\alpha} = g_{j\beta} - g_{i\beta}.
    \end{equation*}
    Using the gluability axiom on the open cover $\{U_i\cap U_j\cap V_\alpha\}_{\alpha\in A}$, there exist elements $F_{ij}\in\scrF(U_i\cap U_j)$ such that 
    \begin{equation*}
        F_{ij} = g_{j\alpha} - g_{i\alpha}\quad\text{ on } U_i\cap U_j\cap V_\alpha.
    \end{equation*}
    We claim that $f_{ij}$ satisfies the cocycle condition. Obviously, from the above description, on $U_i\cap U_j\cap U_k\cap V_\alpha$ we have that $F_{ik} = F_{ij} + F_{jk}$. Using the identity axiom, we see that this equality holds on $U_i\cap U_j\cap U_k$. Thus, $(F_{ij})\in Z^1(\frakU,\scrF)$. Let $h_\alpha = g_{\tau(\alpha)\alpha}|_{V_\alpha}\in\scrF(V_\alpha)$. The on $V_\alpha\cap V_\beta$, we have 
    \begin{align*}
        F_{\tau(\alpha)\tau(\beta)} - f_{\alpha\beta} &= \left(g_{\tau(\beta)\alpha} - g_{\tau(\alpha)\alpha}\right) - \left(g_{\tau(\beta)\alpha} - g_{\tau(\beta)\beta}\right)\\
        &= g_{\tau(\beta)\beta} - g_{\tau(\alpha)\alpha}\\
        &= h_\beta - h_\alpha,
    \end{align*}
    whence $(F_{\tau(\alpha)\tau(\beta)}) - (f_{\alpha\beta})$ is a coboundary, thereby completing the proof.
\end{proof}

\begin{corollary}
    If $\frakU$ is a Leray covering of $X$, then $H^1(\frakU,\scrF)\cong H^1(X,\scrF)$.
\end{corollary}

\section{The Finiteness Theorem}

\subsection{Laurent Schwartz's Theorem}

\begin{theorem}[Closed Range Theorem]\thlabel{thm:closed-range}
    Let $u: E\to F$ be a continuous linear map between Banach spaces. Then $u(E)$ is closed in $F$ if and only if $u^{\ast}(F^\ast)$ is closed in $E^\ast$.
\end{theorem}
\begin{proof}
    See \cite[Theorem 4.14]{grandpa-rudin}.
\end{proof}

\begin{theorem}[Schauder]\thlabel{thm:schauder}
    Let $u: E\to F$ be a continuous linear map between Banach spaces. Then $u$ is compact if and only if $u^\ast$ is.
\end{theorem}
\begin{proof}
    See \cite[Theorem 4.19]{grandpa-rudin}.
\end{proof}

\begin{lemma}\thlabel{lem:schwartz}
    Let $E$, $F$ be Banach spaces and let $u: E\to F$ be a continuous linear map. Suppose that $u$ is injective and that $u(E)$ is closed. Let $v: E\to F$ be a compact continuous linear map. Then $\ker(u + v)$ is finite-dimensional and $(u + v)(E)$ is closed in $F$.
\end{lemma}
\begin{proof}
    Let $N = \ker(u + v)$. To see that this is finite-dimensional, it suffices to show that the closed unit ball in $N$ is compact. To this end, let $(x_n)$ be a sequence in the closed unit ball of $N$. Since $v$ is compact, there is a subsequence $(x_{n_k})$ such that $\left(v(x_{n_k})\right)$ converges, as a result, $u(x_{n_k}) = -v(x_{n_k})$ also converges. Since $u(E)$ is closed in $F$, it is a Banach space and $u: E\to u(E)$ is a bijection, whence, due to the ``bounded inverse theorem'', there is a constant $c > 0$ such that $\|u(x)\|\ge c\|x\|$ for all $x\in E$, consequently, for $k,l\ge 1$, 
    \begin{equation*}
        \|x_{n_k} - x_{n_l}\|\le\frac{1}{c}\|u(x_{n_k}) - u(x_{n_l})\|,
    \end{equation*}
    whence $(x_{n_k})$ is Cauchy, and thus converges. This shows that $N$ is finite-dimensional.

    Owing to $N$ being finite-dimensional, there is a closed subspace $N'$ of $E$ such that $E = N\oplus N'$. Since $(u + v)(E) = (u + v)(N')$, it suffices to show that the latter is closed in $F$. Let $(x_n)$ be a sequence in $N'$ such that $(u + v)(x_n)$ converges in $F$; we show that the limit lies in $(u + v)(N')$. First, we claim that the sequence $(x_n)$ is bounded. If not, then we can move to a subsequence and assume that $0\ne x_n$ for all $n$ and $\|x_n\|\to\infty$ as $n\to\infty$. Set $x_n' = x_n/\|x_n\|$. Then $\|x_n'\| = 1$ and $(u + v)(x_n')\to 0$ as $n\to\infty$. Since $v$ is compact, there is a subsequence $(x_{n_k}')$ such that $v(x_{n_k}')$ converges, consequently, 
    \begin{equation*}
        u(x_{n_k}') = (u + v)(x_{n_k}') - v(x_{n_k}')
    \end{equation*}
    also converges. As we argued in the preceding paragraph using the ``bounded inverse theorem'', this means that $(x_{n_k}')$ converges. It follows that there is some $x_0\in N'$ with $\|x_0\| = 1$ and $(u + v)(x_0) = 0$, that is, $x_0\in N\cap N' = \{0\}$, a contradiction. Hence, $(x_n)$ is a bounded sequence in $E$.

    Compactness of $v$ implies that there is a subsequence $(x_{n_k})$ such that $v(x_{n_k})$ converges in $F$; and since $(u + v)(x_n)$ was assumed to be convergent, we see that $u(x_{n_k})$ is convergent too. Again, using the ``bounded inverse theorem'', we have that $(x_{n_k})$ is convergent to some $x_0\in N'$. Hence, 
    \begin{equation*}
        \lim_{n\to\infty} (u + v)(x_n) = \lim_{k\to\infty}(u + v)(x_{n_k}) = (u + v)(x_0)\in (u + v)(E),
    \end{equation*}
    as desired.
\end{proof}

\begin{theorem}[L. Schwartz]\thlabel{thm:schwartz}
    Let $E$, $F$ be Banach space and let $u, v: E\to F$ be continuous linear maps. Suppose that $u$ is surjective and that $v$ is compact. Then $F' = (u + v)(E)$ is closed and $F/F'$ is finite-dimensional.
\end{theorem}
\begin{proof}
    Due to \thref{thm:closed-range}, it suffices to show that $(u^\ast + v^\ast)(F^\ast)$ is closed in $E^\ast$. Due to \thref{thm:schauder}, we know that $v^\ast$ is compact, and due to \thref{thm:closed-range}, we know that $u^\ast(F^\ast)$ is closed in $E^\ast$. Further, since $u$ is surjective, it is easy to see that $u^\ast$ must be injective. Thus, due to \thref{lem:schwartz}, we see that $(u^\ast + v^\ast)(F^\ast)$ is closed in $E^\ast$, as desired. We have shown that $F'$ is closed in $F$.

    To show that $F/F'$ is finite-dimensional, we shall show that its closed unit ball is compact. Indeed, let $(w_n)$ be a sequence in the closed unit ball of $F/F'$, and choose preimages $(x_n)$ in $F$ satisfying $\|x_n\|\le 2$. Since $u: E\to F$ is surjective, it is a consequence of the open mapping theorem, that there is a constant $M > 0$, independent of the sequence chosen, and a sequence $(y_n)$ in $E$ such that $\|y_n\|\le M$ and $u(y_n) = x_n$. Since $v$ is compact, there is a subsequence $(x_{n_k})$ such that $z_k = v(x_{n_k})$ converges in $F$ to some $\wt z$. We can write 
    \begin{equation*}
        y_{n_k} = u(x_{n_k}) + v(x_{n_k}) - z_k = (u + v)(x_{n_k}) - z_k,
    \end{equation*}
    and hence, $-z_k$ maps to $w_{n_k}$ in $F/F'$. Since the former converges, so does the latter. It follows that $(w_n)$ admits a convergent subsequence, consequently, the closed unit ball in $F/F'$ is compact, whence $F/F'$ is finite-dimensional. This completes the proof.
\end{proof}

\subsection{The Finiteness Theorem}

\bibliographystyle{alpha}
\bibliography{references}
\end{document}