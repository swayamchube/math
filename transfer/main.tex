\documentclass[12pt]{article}

% \usepackage{./arxiv}

\title{Group Transfer}
\author{Swayam Chube}
\date{\today}

\usepackage[utf8]{inputenc} % allow utf-8 input
\usepackage[T1]{fontenc}    % use 8-bit T1 fonts
\usepackage{hyperref}       % hyperlinks
\usepackage{url}            % simple URL typesetting
\usepackage{booktabs}       % professional-quality tables
\usepackage{amsfonts}       % blackboard math symbols
\usepackage{nicefrac}       % compact symbols for 1/2, etc.
\usepackage{microtype}      % microtypography
\usepackage{graphicx}
\usepackage{natbib}
\usepackage{doi}
\usepackage{amssymb}
\usepackage{bbm}
\usepackage{amsthm}
\usepackage{amsmath}
\usepackage{xcolor}
\usepackage{theoremref}
\usepackage{enumitem}
\usepackage{mathpazo}
% \usepackage{euler}
\usepackage{mathrsfs}
\usepackage{todonotes}
\usepackage{stmaryrd}
\usepackage[all,cmtip]{xy} % For diagrams, praise the Freyd–Mitchell theorem 
\usepackage{marvosym}
\usepackage{geometry}
\usepackage{titlesec}

\renewcommand{\qedsymbol}{$\blacksquare$}

% Uncomment to override  the `A preprint' in the header
% \renewcommand{\headeright}{}
% \renewcommand{\undertitle}{}
% \renewcommand{\shorttitle}{}

\hypersetup{
    pdfauthor={Lots of People},
    colorlinks=true,
}

\newtheoremstyle{thmstyle}%               % Name
  {}%                                     % Space above
  {}%                                     % Space below
  {}%                             % Body font
  {}%                                     % Indent amount
  {\bfseries\scshape}%                            % Theorem head font
  {.}%                                    % Punctuation after theorem head
  { }%                                    % Space after theorem head, ' ', or \newline
  {\thmname{#1}\thmnumber{ #2}\thmnote{ (#3)}}%                                     % Theorem head spec (can be left empty, meaning `normal')

\newtheoremstyle{defstyle}%               % Name
  {}%                                     % Space above
  {}%                                     % Space below
  {}%                                     % Body font
  {}%                                     % Indent amount
  {\bfseries\scshape}%                            % Theorem head font
  {.}%                                    % Punctuation after theorem head
  { }%                                    % Space after theorem head, ' ', or \newline
  {\thmname{#1}\thmnumber{ #2}\thmnote{ (#3)}}%                                     % Theorem head spec (can be left empty, meaning `normal')

\theoremstyle{thmstyle}
\newtheorem{theorem}{Theorem}[section]
\newtheorem{lemma}[theorem]{Lemma}
\newtheorem{proposition}[theorem]{Proposition}

\theoremstyle{defstyle}
\newtheorem{definition}[theorem]{Definition}
\newtheorem*{corollary}{Corollary}
\newtheorem{remark}[theorem]{Remark}
\newtheorem{example}[theorem]{Example}
\newtheorem*{notation}{Notation}

% Common Algebraic Structures
\newcommand{\R}{\mathbb{R}}
\newcommand{\Q}{\mathbb{Q}}
\newcommand{\Z}{\mathbb{Z}}
\newcommand{\N}{\mathbb{N}}
\newcommand{\bbC}{\mathbb{C}} 
\newcommand{\K}{\mathbb{K}} % Base field which is either \R or \bbC
\newcommand{\calA}{\mathcal{A}} % Banach Algebras
\newcommand{\calB}{\mathcal{B}} % Banach Algebras
\newcommand{\calI}{\mathcal{I}} % ideal in a Banach algebra
\newcommand{\calJ}{\mathcal{J}} % ideal in a Banach algebra
\newcommand{\frakM}{\mathfrak{M}} % sigma-algebra
\newcommand{\calO}{\mathcal{O}} % Ring of integers
\newcommand{\bbA}{\mathbb{A}} % Adele (or ring thereof)
\newcommand{\bbI}{\mathbb{I}} % Idele (or group thereof)

% Categories
\newcommand{\catTopp}{\mathbf{Top}_*}
\newcommand{\catGrp}{\mathbf{Grp}}
\newcommand{\catTopGrp}{\mathbf{TopGrp}}
\newcommand{\catSet}{\mathbf{Set}}
\newcommand{\catTop}{\mathbf{Top}}
\newcommand{\catRing}{\mathbf{Ring}}
\newcommand{\catCRing}{\mathbf{CRing}} % comm. rings
\newcommand{\catMod}{\mathbf{Mod}}
\newcommand{\catMon}{\mathbf{Mon}}
\newcommand{\catMan}{\mathbf{Man}} % manifolds
\newcommand{\catDiff}{\mathbf{Diff}} % smooth manifolds
\newcommand{\catAlg}{\mathbf{Alg}}
\newcommand{\catRep}{\mathbf{Rep}} % representations 
\newcommand{\catVec}{\mathbf{Vec}}

% Group and Representation Theory
\newcommand{\chr}{\operatorname{char}}
\newcommand{\Aut}{\operatorname{Aut}}
\newcommand{\GL}{\operatorname{GL}}
\newcommand{\im}{\operatorname{im}}
\newcommand{\tr}{\operatorname{tr}}
\newcommand{\id}{\mathbf{id}}
\newcommand{\cl}{\mathbf{cl}}
\newcommand{\Gal}{\operatorname{Gal}}
\newcommand{\Tr}{\operatorname{Tr}}
\newcommand{\sgn}{\operatorname{sgn}}
\newcommand{\Sym}{\operatorname{Sym}}
\newcommand{\Alt}{\operatorname{Alt}}

% Commutative and Homological Algebra
\newcommand{\spec}{\operatorname{spec}}
\newcommand{\mspec}{\operatorname{m-spec}}
\newcommand{\Tor}{\operatorname{Tor}}
\newcommand{\tor}{\operatorname{tor}}
\newcommand{\Ann}{\operatorname{Ann}}
\newcommand{\Supp}{\operatorname{Supp}}
\newcommand{\Hom}{\operatorname{Hom}}
\newcommand{\End}{\operatorname{End}}
\newcommand{\coker}{\operatorname{coker}}
\newcommand{\limit}{\varprojlim}
\newcommand{\colimit}{%
  \mathop{\mathpalette\colimit@{\rightarrowfill@\textstyle}}\nmlimits@
}
\makeatother


\newcommand{\fraka}{\mathfrak{a}} % ideal
\newcommand{\frakb}{\mathfrak{b}} % ideal
\newcommand{\frakc}{\mathfrak{c}} % ideal
\newcommand{\frakf}{\mathfrak{f}} % face map
\newcommand{\frakg}{\mathfrak{g}}
\newcommand{\frakh}{\mathfrak{h}}
\newcommand{\frakm}{\mathfrak{m}} % maximal ideal
\newcommand{\frakn}{\mathfrak{n}} % naximal ideal
\newcommand{\frakp}{\mathfrak{p}} % prime ideal
\newcommand{\frakq}{\mathfrak{q}} % qrime ideal
\newcommand{\fraks}{\mathfrak{s}}
\newcommand{\frakt}{\mathfrak{t}}
\newcommand{\frakz}{\mathfrak{z}}
\newcommand{\frakA}{\mathfrak{A}}
\newcommand{\frakI}{\mathfrak{I}}
\newcommand{\frakJ}{\mathfrak{J}}
\newcommand{\frakK}{\mathfrak{K}}
\newcommand{\frakL}{\mathfrak{L}}
\newcommand{\frakN}{\mathfrak{N}} % nilradical 
\newcommand{\frakO}{\mathfrak{O}} % dedekind domain
\newcommand{\frakP}{\mathfrak{P}} % Prime ideal above
\newcommand{\frakQ}{\mathfrak{Q}} % Qrime ideal above 
\newcommand{\frakR}{\mathfrak{R}} % jacobson radical
\newcommand{\frakU}{\mathfrak{U}}
\newcommand{\frakX}{\mathfrak{X}}

% General/Differential/Algebraic Topology 
\newcommand{\scrA}{\mathscr A}
\newcommand{\scrB}{\mathscr B}
\newcommand{\scrF}{\mathscr F}
\newcommand{\scrN}{\mathscr N}
\newcommand{\scrP}{\mathscr P}
\newcommand{\scrR}{\mathscr R}
\newcommand{\scrS}{\mathscr S}
\newcommand{\bbH}{\mathbb H}
\newcommand{\Int}{\operatorname{Int}}
\newcommand{\psimeq}{\simeq_p}
\newcommand{\wt}[1]{\widetilde{#1}}
\newcommand{\RP}{\mathbb{R}\text{P}}
\newcommand{\CP}{\mathbb{C}\text{P}}

% Miscellaneous
\newcommand{\wh}[1]{\widehat{#1}}
\newcommand{\calM}{\mathcal{M}}
\newcommand{\calP}{\mathcal{P}}
\newcommand{\onto}{\twoheadrightarrow}
\newcommand{\into}{\hookrightarrow}
\newcommand{\Gr}{\operatorname{Gr}}
\newcommand{\Span}{\operatorname{Span}}
\newcommand{\ev}{\operatorname{ev}}
\newcommand{\weakto}{\stackrel{w}{\longrightarrow}}

\newcommand{\define}[1]{\textcolor{blue}{\textit{#1}}}
\newcommand{\caution}[1]{\textcolor{red}{\textit{#1}}}
\renewcommand{\mod}{~\mathrm{mod}~}
\renewcommand{\le}{\leqslant}
\renewcommand{\leq}{\leqslant}
\renewcommand{\ge}{\geqslant}
\renewcommand{\geq}{\geqslant}
\newcommand{\Res}{\operatorname{Res}}
\newcommand{\floor}[1]{\left\lfloor #1\right\rfloor}
\newcommand{\ceil}[1]{\left\lceil #1\right\rceil}
\newcommand{\gl}{\mathfrak{gl}}
\newcommand{\ad}{\operatorname{ad}}
\newcommand{\Stab}{\operatorname{Stab}}
\newcommand{\bfX}{\mathbf{X}}
\newcommand{\Ind}{\operatorname{Ind}}
\newcommand{\bfG}{\mathbf{G}}
\newcommand{\rank}{\operatorname{rank}}
\newcommand{\calo}{\mathcal{o}}
\newcommand{\frako}{\mathfrak{o}}
\newcommand{\Cl}{\operatorname{Cl}}

\newcommand{\idim}{\operatorname{idim}}
\newcommand{\pdim}{\operatorname{pdim}}
\newcommand{\Ext}{\operatorname{Ext}}
\newcommand{\co}{\operatorname{co}}
\newcommand{\Foc}{\operatorname{Foc}}

\geometry {
    margin = 1in
}

\titleformat
{\section}
[block]
{\Large\bfseries\scshape}
{\S\thesection}
{0.5em}
{\centering}
[]


\titleformat
{\subsection}
[block]
{\normalfont\bfseries\sffamily}
{\S\S}
{0.5em}
{\centering}
[]



\begin{document}
\maketitle

\section{Introduction}

\begin{definition}[The Transfer Map]
    Let $G$ be a group and $H\le G$ be a subgroup of finite index, say $n$. Let $t_1,\dots,t_n$ be a left traversal for $H$ in $G$. For every $g\in G$, and $1\le i\le n$, 
    \begin{equation*}
        gt_i = t_{j_i}h_i
    \end{equation*}
    for some $1\le j_i\le n$ and $h_i\in H$. Define 
    \begin{equation*}
        \psi(g) = \prod_{i = 1}^n h_i\pmod{H'}
    \end{equation*}

    This defines a map $\psi: G\to H^{ab}$ called the \emph{transfer}.
\end{definition}

\begin{proposition}
    The map $\psi$ is independent of the choice of coset traversal of $H$ in $G$.
\end{proposition}
\begin{proof}
    Gadha mehnat.
\end{proof}

\begin{theorem}\thlabel{thm:transfer-evaluation}
    Let $T = \{t_1,\dots,t_n\}$ be a left traversal for $H$ in $G$. Then, for each $g\in G$, there is a subset $T_0\subseteq T$ and positive integers $n_t$ for each $t\in T_0$ such that 
    \begin{enumerate}[label=(\alph*)]
        \item $\displaystyle\sum_{t\in T_0} n_t = n$.
        \item $\displaystyle t^{-1}g^{n_t}t\in H$ for all $t\in T_0$. 
        \item $\displaystyle\psi(g) = \prod_{t\in T_0} t^{-1}g^{n_t}t\pmod{H'}$.
        \item If $g$ has finite order, then each $n_t$ divides $|g|$.
    \end{enumerate}
\end{theorem}
\begin{proof}
    The group $\langle g\rangle$ acts on $T$ by left multiplication and decomposes $T$ into orbits. Let $T_0$ be a set of representatives of these orbits. For each $t\in T_0$, let $n_t$ denote the size of the orbit containing $t$. Then, note that 
    \begin{equation*}
        g^{n_t}t = tH.
    \end{equation*}
    There is some $h_t\in H$ such that $h_t = t^{-1} g^{n_t}t$. It follows that 
    \begin{equation*}
        \psi(g) = \prod_{t\in T_0} t^{-1}g^{n_t}t\pmod{H'},
    \end{equation*}
    which proves all four parts of the theorem.
\end{proof}

\begin{corollary}
    If $H$ is central and of finite index in $G$, then the transfer map $\psi: G\to H^{ab}$ is given by $g\mapsto g^n\pmod{H'}$ where $n = [G : H]$.
\end{corollary}

\begin{corollary}
    If $H$ is of finite index in $G$ such that no two elements of $H$ are conjugate in $G$, then the restriction of the transfer map $\psi|_H$ is given by $h\mapsto h^n$ where $n = [G: H]$.
\end{corollary}

\section{Some Applications}

\begin{proposition}
    Let $A\unlhd G$ be abelian of finite index and $\psi: G\to A$ the transfer map. 
    \begin{enumerate}[label=(\alph*)]
        \item $\psi(G)\subseteq Z(G)$.
        \item If $G$ is finite and $A$ is a Hall subgroup of $G$, then $\psi(G) = \psi(A) = A\cap Z(G)$. In this case, $G\cong\psi(G)\times\ker\psi$.
    \end{enumerate}
\end{proposition}
\begin{proof}
\begin{enumerate}[label=(\alph*)]
    \item Let $t_1,\dots,t_n$ be a left traversal for $A$ in $G$ and choose $a\in G$. Let $t_{j_i}H = at_i H$. Let $g\in G$ be arbitrary. We know 
    \begin{equation*}
        \psi(a) = \prod_{i = 1}^n t_{j_i}^{-1}at_i.
    \end{equation*}
    Then, 
    \begin{equation*}
        g^{-1}\psi(a)g = \prod_{i = 1}^n (t_{j_i}g)^{-1}a(t_ig).
    \end{equation*}
    Since $A$ is normal, it follows that $\{t_ig\mid 1\le i\le n\}$ is a left traversal for $A$ in $G$. This shows that $g^{-1}\psi(a)g = \psi(a)$ whence, $\psi(a)$ is central in $G$.

    \item From (a), it follows that $\psi(A)\subseteq\psi(G)\subseteq A\cap Z(G)$. Note that the restriction of $\psi$ to $A\cap Z(G)$ is $a\mapsto a^n$ where $n = [G : A]$. Since $A$ is a Hall subgroup, $n$ is coprime to $|A|$, hence, to $|A\cap Z(G)|$. Consequently, the restriction of $\psi$ to $A\cap Z(G)$ is an automorphism. It now follows that $A\cap Z(G)\subseteq\psi(A)$.

    Finally, consider the exact sequence 
    \begin{equation*}
        1\to \ker\psi\to G\xrightarrow{\psi} A\cap Z(G)\to 1.
    \end{equation*}
    This splits on the right and the splitting is central. Hence, $G\cong\ker\psi\times\psi(G)$. This completes the proof.
\end{enumerate}
\end{proof}

\begin{theorem}[Schur]
    Let $[G: Z(G)] < \infty$. Then, $G'$, the commutator subgroup, is a finite subgroup of $G$.
\end{theorem}
\begin{proof}
    Let $g_1,\dots,g_n$ be a left traversal for $Z(G)$ in $G$. Then, $G'$ is generated by $\{[g_i, g_j]\mid 1\le i, j\le n\}$, that is, $G'$ is finitely generated. Further, the transfer map $\psi: G\to Z(G)$ is given by $\psi(g) = g^n$. Since $Z(G)$ is abelian, $G'\subseteq\ker\psi$. Hence, every element of $G'$ is killed by $n$.

    Consider $H = G'\cap Z(G)$. This is a finite index abelian subgroup of $G'$, hence, is finitely generated. Further, it is killed by $n$, whence it is finite. It follows that $G'$ is finite.
\end{proof}

\begin{proposition}
    Let $S\subseteq G$ be the set of elements of finite order in $G$. If $S$ is finite, then it is a subgroup of $G$.
\end{proposition}
\begin{proof}
    Replace $G$ by the subgroup generated by $S$. It suffices to show that $G$ is finite, since then it would follow that $G = S$. Being the intersection of finitely many groups of finite index, we can conclude that $H = \bigcap_{s\in S} C_G(s)$ has finite index in $G$. But $H\subseteq Z(G)$ and hence, $[G:Z(G)] < \infty$, consequently, $|G'| < \infty$. Finally, note that $G^{ab}$ is a finitely generated torsion abelian group, hence, finite. This shows that $G$ is finite, thereby completing the proof.
\end{proof}

\begin{proposition}
    Let $G$ be a finite group of square free order. 
\end{proposition}

\section{Burnside's Complement Theorem}

\begin{definition}[Focal Subgroup]
    Let $H\le G$ be a subgroup. Then the \emph{focal subgroup} of $H$ in $G$ is defined as 
    \begin{equation*}
        \operatorname{Foc}_G(H) = \langle x^{-1}y\mid x,y\in H,\text{ and are }G-\text{conjugate}\rangle.
    \end{equation*}
\end{definition}

\begin{theorem}\thlabel{thm:focal-subgroup}
    Let $G$ be finite, $H\le G$ a Hall subgroup and $\psi: G\to H^{ab}$ the transfer map. Then, 
    \begin{equation*}
        \Foc_G(H) = H\cap G' = H\cap\ker\psi.
    \end{equation*}
\end{theorem}
\begin{proof}
    If $y = gxg^{-1}$ for some $g\in G$, then $x^{-1}y = [x^{-1}, g]\in G'$ and hence, $\Foc_G(H)\subseteq H\cap G'$. On the other hand, $G'\subseteq\ker\psi$ since $\psi$ is a homomorphism to an abelian group. Apriori, we have the following inclusions
    \begin{equation*}
        \Foc_G(H)\subseteq H\cap G'\subseteq H\cap\ker\psi.
    \end{equation*}
    Let $g\in H\cap\ker\psi$. It suffices to show that $g\in\Foc_G(H)$. Using \thref{thm:transfer-evaluation}, 
    \begin{equation*}
        \psi(g) = \prod_{t\in T_0} t^{-1}g^{n_t}t\pmod{H'} = g^n\prod_{t\in T_0}g^{-n_t}t^{-1}g^{n_t}t\pmod{H'}.
    \end{equation*}
    According to \thref{thm:transfer-evaluation}, we also know that $t^{-1}g^{n_t}t\in H$ and hence, each factor $g^{-n_t}t^{-1}g^{n_t}t\in\Foc_G(H)$. 
    
    But since $g\in\ker(\psi)$, we must have that the product $\psi(g)$ as an element of $H$, lies in $H'\subseteq\Foc_G(H)$. But since each factor $g^{-n_t}t^{-1}g^{n_t}t\in\Foc_G(H)$, we must have $g^n\in\Foc_G(H)$. Recall that $H$ is a Hall subgroup and hence, $n$ is relatively prime to $|H|$, consequently, relatively prime to $|g|$. As a result, $g\in\langle g^n\rangle\subseteq\Foc_G(H)$. This completes the proof.
\end{proof}

\begin{lemma}[Burnside]
    Let $P$ be a $p$-Sylow subgroup of $G$ and suppose $x,y\in C_G(P)$ are conjugate in $G$. Then $x$ and $y$ are conjugate in $N_G(P)$.
\end{lemma}
\begin{proof}
    Suppose $y = x^g$ for some $g\in G$. Then, $P\subseteq C_G(y)\cap C_G(x)$. Consequently, 
    \begin{equation*}
        P^g\subseteq C_G(x)^g = C_G(x^g) = C_G(y).
    \end{equation*}
    Since both $P$ and $P^g$ are Sylow $p$-subgroups of $C_G(y)$, there is a $c\in C_G(y)$ such that $P^{cg} = P$. Therefore, $cg\in N_G(P)$, and 
    \begin{equation*}
        x^{cg} = (x^g)^c = y^c = y.
    \end{equation*}
    This completes the proof.
\end{proof}

\begin{definition}
    A group $G$ is said to have a \emph{normal $p$-complement} if there is a normal subgroup $N\unlhd G$ such that $[G: N] = p^n$ where $n = v_p(|G|)$.
\end{definition}

\begin{theorem}[Burnside]\thlabel{thm:burnside}
    Let $P$ be a Sylow $p$-subgroup of $G$ and suppose $P\subseteq Z(N_G(P))$. Then, $G$ has a normal $p$-complement.
\end{theorem}
\begin{proof}
    We contend that $\Foc_G(P) = 1$. Indeed, suppose $x, y\in P$ are conjugate in $G$. According to our assumption on $P$, $P\subseteq C_G(P)$, therefore, there is some $g\in N_G(P)$ such that $y = gxg^{-1}$. But since $P\subseteq Z(N_G(P))$, we must have $y = x$ and hence, $\Foc_G(H) = 1$. Using \thref{thm:focal-subgroup}, we see that $P\cap\ker\psi = 1$ where $\psi: G\to P^{ab} = P$ is the transfer map. Therefore, $|\psi(P)| = |P|$, whence $\psi$ is surjective. This shows that $\ker\psi$ is a normal $p$-complement in $G$.
\end{proof}

\begin{theorem}
    Let $G$ be a finite group such that every Sylow subgroup of $G$ is cyclic. Then $G$ is solvable.
\end{theorem}
\begin{proof}
    Let $p$ be the smallest prime dividing the order of $G$ and $P$ be a Sylow $p$-subgroup. Due to the $N/C$-theorem, there is an injection $N_G(P)/C_G(P)\hookrightarrow\Aut(P)$. If $|P| = p^r$, then $\Aut(P)$ has order $p^{r - 1}(p - 1)$. But since both $N_G(P)$ and $C_G(P)$ contain $P$, the order of the quotient $N_G(P)/C_G(P)$ cannot be divisible by $p$, hence, must be $1$. Thus, $P\subseteq Z(N_G(P))$. Due to \thref{thm:burnside}, $G$ has a normal $p$-complement, say $N$. 

    This fits into a short exact sequence 
    \begin{equation*}
        1\to N\to G\to G/N\to 1,
    \end{equation*}
    where $G/N$ is a $p$-group, hence, solvable and $N\subsetneq G$ is a proper subgroup divisible by one less prime and hence, solvable due to an inductive argument. This completes the proof.
\end{proof}
\end{document}