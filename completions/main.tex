\documentclass[11pt]{article}

\usepackage[utf8]{inputenc} % allow utf-8 input
\usepackage[T1]{fontenc}    % use 8-bit T1 fonts
\usepackage{hyperref}       % hyperlinks
\usepackage{url}            % simple URL typesetting
\usepackage{booktabs}       % professional-quality tables
\usepackage{amsfonts}       % blackboard math symbols
\usepackage{nicefrac}       % compact symbols for 1/2, etc.
\usepackage{microtype}      % microtypography
\usepackage{graphicx}
\usepackage{natbib}
\usepackage{doi}
\usepackage{amssymb}
\usepackage{bbm}
\usepackage{amsthm}
\usepackage{amsmath}
\usepackage{xcolor}
\usepackage{theoremref}
\usepackage{enumitem}
\usepackage{fouriernc}
\usepackage{mdframed}
\usepackage{mathrsfs}
\setlength{\marginparwidth}{2cm}
\usepackage{todonotes}
\usepackage{stmaryrd}
\usepackage[all,cmtip]{xy} % For diagrams, praise the Freyd-Mitchell theorem 
\usepackage{marvosym}
\usepackage{geometry}
\usepackage{titlesec}
\usepackage{mathtools}
\usepackage{tikz}
\usetikzlibrary{cd}
\usepackage{epigraph}
\setlength\epigraphwidth{0.4\textwidth}

\renewcommand{\qedsymbol}{$\blacksquare$}
% \renewcommand{\familydefault}{\sfdefault} % Do you want this font? 

% Uncomment to override  the `A preprint' in the header
% \renewcommand{\headeright}{}
% \renewcommand{\undertitle}{}
% \renewcommand{\shorttitle}{}

\hypersetup{
    pdfauthor={Swayam Chube},
    colorlinks=true,
	citecolor=blue,
}

\newtheoremstyle{thmstyle}%               % Name
  {}%                                     % Space above
  {}%                                     % Space below
  {}%                             % Body font
  {}%                                     % Indent amount
  {\bfseries\scshape}%                            % Theorem head font
  {.}%                                    % Punctuation after theorem head
  { }%                                    % Space after theorem head, ' ', or \newline
  {\thmname{#1}\thmnumber{ #2}\thmnote{ (#3)}}%                                     % Theorem head spec (can be left empty, meaning `normal')

\newtheoremstyle{defstyle}%               % Name
  {}%                                     % Space above
  {}%                                     % Space below
  {}%                                     % Body font
  {}%                                     % Indent amount
  {\bfseries\scshape}%                            % Theorem head font
  {.}%                                    % Punctuation after theorem head
  { }%                                    % Space after theorem head, ' ', or \newline
  {\thmname{#1}\thmnumber{ #2}\thmnote{ (#3)}}%                                     % Theorem head spec (can be left empty, meaning `normal')

\theoremstyle{thmstyle}
\newtheorem{theorem}{Theorem}[section]
\newtheorem{lemma}[theorem]{Lemma}
\newtheorem{proposition}[theorem]{Proposition}

\theoremstyle{defstyle}
\newtheorem{definition}[theorem]{Definition}
\newtheorem{corollary}[theorem]{Corollary}
\newtheorem{porism}[theorem]{Porism}
\newtheorem{remark}[theorem]{Remark}
\newtheorem{interlude}[theorem]{Interlude}
\newtheorem{example}[theorem]{Example}
\newtheorem*{notation}{Notation}
\newtheorem*{claim}{Claim}

% Common Algebraic Structures
\newcommand{\R}{\mathbb{R}}
\newcommand{\Q}{\mathbb{Q}}
\newcommand{\Z}{\mathbb{Z}}
\newcommand{\N}{\mathbb{N}}
\newcommand{\bbC}{\mathbb{C}} 
\newcommand{\K}{\mathbb{K}} % Base field which is either \R or \bbC
\newcommand{\F}{\mathbb{F}} % Base field which is either \R or \bbC
\newcommand{\calA}{\mathcal{A}} % Banach Algebras
\newcommand{\calB}{\mathcal{B}} % Banach Algebras
\newcommand{\calI}{\mathcal{I}} % ideal in a Banach algebra
\newcommand{\calJ}{\mathcal{J}} % ideal in a Banach algebra
\newcommand{\frakM}{\mathfrak{M}} % sigma-algebra
\newcommand{\calO}{\mathcal{O}} % Ring of integers
\newcommand{\bbA}{\mathbb{A}} % Adele (or ring thereof)
\newcommand{\bbI}{\mathbb{I}} % Idele (or group thereof)

% Categories
\newcommand{\catTopp}{\mathbf{Top}_*}
\newcommand{\catGrp}{\mathbf{Grp}}
\newcommand{\catTopGrp}{\mathbf{TopGrp}}
\newcommand{\catSet}{\mathbf{Set}}
\newcommand{\catTop}{\mathbf{Top}}
\newcommand{\catRing}{\mathbf{Ring}}
\newcommand{\catCRing}{\mathbf{CRing}} % comm. rings
\newcommand{\catMod}{\mathbf{Mod}}
\newcommand{\catMon}{\mathbf{Mon}}
\newcommand{\catMan}{\mathbf{Man}} % manifolds
\newcommand{\catDiff}{\mathbf{Diff}} % smooth manifolds
\newcommand{\catAlg}{\mathbf{Alg}}
\newcommand{\catRep}{\mathbf{Rep}} % representations 
\newcommand{\catVec}{\mathbf{Vec}}

% Group and Representation Theory
\newcommand{\chr}{\operatorname{char}}
\newcommand{\Aut}{\operatorname{Aut}}
\newcommand{\GL}{\operatorname{GL}}
\newcommand{\im}{\operatorname{im}}
\newcommand{\tr}{\operatorname{tr}}
\newcommand{\id}{\mathbf{id}}
\newcommand{\cl}{\mathbf{cl}}
\newcommand{\Gal}{\operatorname{Gal}}
\newcommand{\Tr}{\operatorname{Tr}}
\newcommand{\sgn}{\operatorname{sgn}}
\newcommand{\Sym}{\operatorname{Sym}}
\newcommand{\Alt}{\operatorname{Alt}}

% Commutative and Homological Algebra
\newcommand{\spec}{\operatorname{spec}}
\newcommand{\mspec}{\operatorname{m-spec}}
\newcommand{\Spec}{\operatorname{Spec}}
\newcommand{\MaxSpec}{\operatorname{MaxSpec}}
\newcommand{\Tor}{\operatorname{Tor}}
\newcommand{\tor}{\operatorname{tor}}
\newcommand{\Ann}{\operatorname{Ann}}
\newcommand{\Supp}{\operatorname{Supp}}
\newcommand{\Hom}{\operatorname{Hom}}
\newcommand{\End}{\operatorname{End}}
\newcommand{\coker}{\operatorname{coker}}
\newcommand{\limit}{\varprojlim}
\newcommand{\colimit}{%
  \mathop{\mathpalette\colimit@{\rightarrowfill@\textstyle}}\nmlimits@
}
\makeatother


\newcommand{\fraka}{\mathfrak{a}} % ideal
\newcommand{\frakb}{\mathfrak{b}} % ideal
\newcommand{\frakc}{\mathfrak{c}} % ideal
\newcommand{\frakf}{\mathfrak{f}} % face map
\newcommand{\frakg}{\mathfrak{g}}
\newcommand{\frakh}{\mathfrak{h}}
\newcommand{\frakm}{\mathfrak{m}} % maximal ideal
\newcommand{\frakn}{\mathfrak{n}} % naximal ideal
\newcommand{\frakp}{\mathfrak{p}} % prime ideal
\newcommand{\frakq}{\mathfrak{q}} % qrime ideal
\newcommand{\fraks}{\mathfrak{s}}
\newcommand{\frakt}{\mathfrak{t}}
\newcommand{\frakz}{\mathfrak{z}}
\newcommand{\frakA}{\mathfrak{A}}
\newcommand{\frakB}{\mathfrak{B}}
\newcommand{\frakI}{\mathfrak{I}}
\newcommand{\frakJ}{\mathfrak{J}}
\newcommand{\frakK}{\mathfrak{K}}
\newcommand{\frakL}{\mathfrak{L}}
\newcommand{\frakN}{\mathfrak{N}} % nilradical 
\newcommand{\frakO}{\mathfrak{O}} % dedekind domain
\newcommand{\frakP}{\mathfrak{P}} % Prime ideal above
\newcommand{\frakQ}{\mathfrak{Q}} % Qrime ideal above 
\newcommand{\frakR}{\mathfrak{R}} % jacobson radical
\newcommand{\frakU}{\mathfrak{U}}
\newcommand{\frakV}{\mathfrak{V}}
\newcommand{\frakW}{\mathfrak{W}}
\newcommand{\frakX}{\mathfrak{X}}

% General/Differential/Algebraic Topology 
\newcommand{\scrA}{\mathscr{A}}
\newcommand{\scrB}{\mathscr{B}}
\newcommand{\scrF}{\mathscr{F}}
\newcommand{\scrM}{\mathscr{M}}
\newcommand{\scrN}{\mathscr{N}}
\newcommand{\scrP}{\mathscr{P}}
\newcommand{\scrO}{\mathscr{O}} % sheaf
\newcommand{\scrR}{\mathscr{R}}
\newcommand{\scrS}{\mathscr{S}}
\newcommand{\bbH}{\mathbb H}
\newcommand{\Int}{\operatorname{Int}}
\newcommand{\psimeq}{\simeq_p}
\newcommand{\wt}[1]{\widetilde{#1}}
\newcommand{\RP}{\mathbb{R}\text{P}}
\newcommand{\CP}{\mathbb{C}\text{P}}

% Miscellaneous
\newcommand{\wh}[1]{\widehat{#1}}
\newcommand{\calM}{\mathcal{M}}
\newcommand{\calP}{\mathcal{P}}
\newcommand{\onto}{\twoheadrightarrow}
\newcommand{\into}{\hookrightarrow}
\newcommand{\Gr}{\operatorname{Gr}}
\newcommand{\Span}{\operatorname{Span}}
\newcommand{\ev}{\operatorname{ev}}
\newcommand{\weakto}{\stackrel{w}{\longrightarrow}}

\newcommand{\define}[1]{\textcolor{blue}{\textit{#1}}}
% \newcommand{\caution}[1]{\textcolor{red}{\textit{#1}}}
\newcommand{\important}[1]{\textcolor{red}{\textit{#1}}}
\renewcommand{\mod}{~\mathrm{mod}~}
\renewcommand{\le}{\leqslant}
\renewcommand{\leq}{\leqslant}
\renewcommand{\ge}{\geqslant}
\renewcommand{\geq}{\geqslant}
\newcommand{\Res}{\operatorname{Res}}
\newcommand{\floor}[1]{\left\lfloor #1\right\rfloor}
\newcommand{\ceil}[1]{\left\lceil #1\right\rceil}
\newcommand{\gl}{\mathfrak{gl}}
\newcommand{\ad}{\operatorname{ad}}
\newcommand{\Stab}{\operatorname{Stab}}
\newcommand{\bfX}{\mathbf{X}}
\newcommand{\Ind}{\operatorname{Ind}}
\newcommand{\bfG}{\mathbf{G}}
\newcommand{\rank}{\operatorname{rank}}
\newcommand{\calo}{\mathcal{o}}
\newcommand{\frako}{\mathfrak{o}}
\newcommand{\Cl}{\operatorname{Cl}}

\newcommand{\idim}{\operatorname{idim}}
\newcommand{\pdim}{\operatorname{pdim}}
\newcommand{\Ext}{\operatorname{Ext}}
\newcommand{\co}{\operatorname{co}}
\newcommand{\bfO}{\mathbf{O}}
\newcommand{\bfF}{\mathbf{F}} % Fitting Subgroup
\newcommand{\Syl}{\operatorname{Syl}}
\newcommand{\nor}{\vartriangleleft}
\newcommand{\noreq}{\trianglelefteqslant}
\newcommand{\subnor}{\nor\!\nor}
\newcommand{\Soc}{\operatorname{Soc}}
\newcommand{\core}{\operatorname{core}}
\newcommand{\Sd}{\operatorname{Sd}}
\newcommand{\mesh}{\operatorname{mesh}}
\newcommand{\sminus}{\setminus}
\newcommand{\diam}{\operatorname{diam}}
\newcommand{\Ass}{\operatorname{Ass}}
\newcommand{\projdim}{\operatorname{proj~dim}}
\newcommand{\injdim}{\operatorname{inj~dim}}
\newcommand{\gldim}{\operatorname{gl~dim}}
\newcommand{\embdim}{\operatorname{emb~dim}}
\newcommand{\hght}{\operatorname{ht}}
\newcommand{\depth}{\operatorname{depth}}
\newcommand{\ul}[1]{\underline{#1}}
\newcommand{\type}{\operatorname{type}}
\newcommand{\CM}{\operatorname{CM}}
\newcommand{\cech}[1]{\mathbin{\check{#1}}}
\newcommand{\cdim}{\operatorname{cdim}}
\newcommand{\Der}{\operatorname{Der}}
\newcommand{\trdeg}{\operatorname{trdeg}}
\newcommand{\gr}{\operatorname{gr}}

\geometry {
    margin = 1in
}

\titleformat
{\section}
[block]
{\Large\bfseries\sffamily}
{\S\thesection}
{0.5em}
{\centering}
[]


\titleformat
{\subsection}
[block]
{\normalfont\bfseries\sffamily}
{\S\S}
{0.5em}
{\centering}
[]


\begin{document}
\title{Completions}
\author{Swayam Chube}
\date{Last Updated: \today}

\maketitle

\section{Graded and Filtered Objects}

\begin{definition}
    Let $(G, +)$ be an Abelian monoid with identity element $0\in G$. A \define{$G$-graded} ring is a ring $R$ together with a direct sum decomposition 
    \begin{equation*}
        R = \bigoplus_{i\in G} R_i
    \end{equation*}
    into additive subgroups, such that $R_iR_j\subseteq R_{i + j}$ for all $i, j\in G$.

    Similarly, a \define{graded $R$-module} is an $R$-module $M$ together with a direct sum decomposition 
    \begin{equation*}
        M = \bigoplus_{i\in G} M_i
    \end{equation*}
    into additive subgroups, such that $R_i M_j\subseteq M_{i + j}$ for all $i,j\in G$. An element $x\in M$ is said to be \define{homogeneous} if $x\in M_i$ for some $i\in G$. 
    
    Note that any element $x\in M$ can be written uniquely as a sum 
    \begin{equation*}
        x = \sum_{i\in G} x_i,
    \end{equation*}
    where $x_i\in M_i$ for all $i\in G$. The $x_i$'s are called the \define{homogeneous components} of $x$.
    A submodule $N\subseteq M$ is said to be \define{homogeneous} if it can be generated by homogeneous elements in $M$.

    A homomorphism $f\colon M\to N$ of graded $R$-modules is said to be \define{graded of degree $d\in G$} if $f(M_i)\subseteq N_{i + d}$ for all $i\in G$. A graded homomorphism of degree $0$ is said to be just \emph{graded}.
\end{definition}

\begin{proposition}
    Let $\displaystyle M = \bigoplus_{i\in G} M_i$ be a $G$-graded $R$-module and $N\subseteq M$ a submodule. The following are equivalent: 
    \begin{enumerate}[label=(\arabic*)]
        \item $N$ is a homogeneous submodule of $M$.
        \item For all $x\in N$, each homogeneous component of $x$ lies in $N$.
        \item $\displaystyle N = \sum_{i\in G}(N\cap M_i)$.
    \end{enumerate}
\end{proposition}
\begin{proof}
    $(1)\implies(2)$ Suppose $N$ is generated as an $R$-module by the homogeneous elements $\{z_j\}$. Then we can write 
    \begin{equation*}
        x = \sum_j r_j z_j = \sum_{j}\left(\sum_{g\in G} r_j^g z_j\right),
    \end{equation*}
    where we can decompose $r_j$ as
    \begin{equation*}
        r_j = \sum_{g\in G} r_j^g
    \end{equation*}
    into its homogeneous components in $R$. Grouping together components of the same degree shows that every homogeneous component of $x$ lies in $N$. 

    $(2)\implies(3)$ Clearly, the right hand side is contained in the left hand side. Conversely, if $x\in N$, then we can write $x = \sum_{i\in G} x_i$ where the $x_i$'s are the homogeneous components of $x$. According to (2), $x_i\in N\cap M_i$ for each $i\in G$, whence $x$ is contained in the right hand side, as desired. 

    $(3)\implies(1)$ Indeed, $N$ is generated as an $R$-module by the set 
    \begin{equation*}
        \bigcup_{i\in G} (N\cap M_i)
    \end{equation*}
    consisting only of homogeneous elements.
\end{proof}

\begin{proposition}
    Let $\displaystyle R = \bigoplus_{i\in G} R_i$ be a $G$-graded ring. Then $R_0$ is a subring of $R$ and for every graded $R$-module $\displaystyle M = \bigoplus_{i\in G} M_i$, each $M_i$ is naturally an $R_0$-module.
\end{proposition}
\begin{proof}
    $R_0$ is an additive subgroup of $R$ and $R_0 R_0\subseteq R_0$. Thus it suffices to show that $1\in R_0$. We can decompose $x$ into its homogeneous components as 
    \begin{equation*}
        1 = \sum_{i\in G} x_i.
    \end{equation*}
    For any $j\in G$, we then have 
    \begin{equation*}
        x_j = \sum_{i\in G} x_ix_j,
    \end{equation*}
    where $x_ix_j$ is homogeneous of degree $i + j$. Therefore, 
    \begin{equation*}
        x_ix_j = 
        \begin{cases}
            x_j & i = 0\\
            0 & i\ne 0.
        \end{cases}
    \end{equation*}
    Summing over all $j\in G$, we get 
    \begin{equation*}
        x_0 = \sum_{j\in G} x_0x_j = \sum_{j\in G} x_j = 1.
    \end{equation*}
    That is, $1\in R_0$, and hence $R_0$ is a subring of $R$. Finally, since $R_0 M_i\subseteq M_i$, it follows that each $M_i$ is naturally an $R_0$-module.
\end{proof}

\begin{remark}[Quotient of graded modules]
    Let $M = \bigoplus_{i\in G} M_i$ be a graded module over a graded ring $R = \bigoplus_{i\in G} R_i$, and let $N\subseteq M$ be a graded submodule of $M$. We can endow the quotient module $M/N$ with a natural grading 
    \begin{equation*}
        \bigoplus_{i\in G} M_i/N_i,
    \end{equation*}
    where the $R$-module structure is the obvious one. To see that this is indeed isomorphic to $M/N$ as an $R$-module, consider the graded projection $\pi\colon M\to M/N$ given by 
    \begin{equation*}
        \pi\left(\sum_{i\in G} x_i\right) = \sum_{i\in G} x_i\mod N_i.
    \end{equation*}
    One can check that this is an $R$-linear surjective homomorphism with $\ker\pi = N$, which implies the desired conclusion.

    Analogously, if $I\noreq R$ is a graded ideal, then $R/I$ is naturally a graded $R$-module as above, and has a ring structure given by 
    \begin{equation*}
        \left(r_i\mod I_i\right)\left(r_j\mod I_j\right) = r_ir_j \mod I_{i + j}.
    \end{equation*}
    Henceforth, $R/I$ shall always be thought of a graded ring with the above grading.
\end{remark}

Throughout this article, we shall mainly concern ourselves with the case $G = (\N, +, 0)$, and henceforth, a graded ring/module shall refer to an $\N$-graded ring/module. For an $\N$-graded ring $\displaystyle R = \bigoplus_{n\ge 0} R_n$, we set 
\begin{equation*}
    R_+ = \bigoplus_{n\ge 1} R_n,
\end{equation*}
which is clearly a homogeneous ideal in $R$ and is called the \define{irrelevant ideal} of $R$. Often when $R_0$ is a field, then the irrelevant ideal turns out to be the unique \emph{graded} maximal ideal and is denoted by $\frakm_+$ for empasis.

\begin{theorem}[Graded Nakayama]\thlabel{graded-nak}
    Let $\displaystyle R = \bigoplus_{n\ge 0} R_n$ be a graded ring, and $\displaystyle M = \bigoplus_{n\ge 0} M_n$ a graded $R$-module. If $R_+ M = M$, then $M = 0$.
\end{theorem}
\begin{proof}
    Let $n\ge 0$ be the smallest non-negative integer such that $M_n\ne 0$. Let $0\ne x_n\in M_n$. Using the fact that $R_+ M = M$, we can write 
    \begin{equation*}
        x_n = \sum_\lambda r^\lambda y^\lambda,
    \end{equation*}
    for some finite set of $r^\lambda\in R_+$ and $y^\lambda\in M$. Writing out each $r^\lambda$ and $y^\lambda$ in its homogeneous components and isolating terms of degree $n$, we get 
    \begin{equation*}
        x_n = \sum_{\lambda}\left(\sum_{i + j = n} r^\lambda_i y^\lambda_j\right).
    \end{equation*}
    But since $r^\lambda\in R_+$, if $r^\lambda_i\ne 0$ then $i\ge 1$, so that $j\le n - 1$, and hence $y^\lambda_j = 0$. Thus $x_n = 0$, a contradiction. This completes the proof.
\end{proof}

\begin{definition}
    A \define{filtered ring} is a ring $R$ together with a descending chain of additive subgroups 
    \begin{equation*}
        R = R_0\supseteq R_1\supseteq R_2\supseteq\cdots,
    \end{equation*}
    such that $R_nR_m\subseteq R_{n + m}$ for all $n, m\ge 0$. In particular, each $R_n$ is an ideal in the ring $R$.

    Let $R$ be a filtered ring as above. A \define{filtered module} over $R$ is an $R$-module $M$ together with a descending chain of $R$-submodules of $M$ 
    \begin{equation*}
        M = M_0\supseteq M_1\supseteq M_2\supseteq\cdots
    \end{equation*}
    such that $R_m M_n\subseteq M_{m + n}$ for all $m,n\ge 0$.

    A map $f\colon M\to N$ between filtered $R$-modules is said to be a filtered homomorphism if $f(M_n)\subseteq N_n$ for all $n\ge 0$.
\end{definition}

\begin{remark}
    Let $M$ be a filtered module over a filtered ring $R$ as above. For an $R$-submodule $N\subseteq M$, we define the \define{induced filtration} on $N$ as $N_n = N\cap M_n$ for all $n\ge 0$. 
    Similarly, we define the induced filtration on $M/N$ as 
    \begin{equation*}
        \left(\frac{M}{N}\right)_n = \frac{N + M_n}{N}.
    \end{equation*}

    Equipped with these filtrations, every map in the short exact sequence
    \begin{equation*}
        0\to N\into M\onto M/N\to 0
    \end{equation*}
    is a filtered homomorphism.
\end{remark}

\begin{definition}
    Let $R$ be a filtered ring and $M$ a filtered $R$-module as above. Define the \define{associated graded ring}
    \begin{equation*}
        \gr(R) = \bigoplus_{n\ge 0} R_n/R_{n + 1}
    \end{equation*}
    with product structure given by 
    \begin{equation*}
        (x + R_{n + 1})(y + R_{m + 1}) = xy + R_{n + m + 1}
    \end{equation*}
    for all $n, m\ge 0$. It is easy to check that $\gr(R)$ is a graded ring.

    We further define the \define{associated graded module}
    \begin{equation*}
        \gr(M) = \bigoplus_{n\ge 0} M_n/M_{n + 1}
    \end{equation*}
    which is a graded $\gr(R)$-module with the module structure given by 
    \begin{equation*}
        (a + R_{m + 1})\cdot(x + M_{n + 1}) = a\cdot x + M_{m + n + 1}.
    \end{equation*}

    If $N$ is another filtered $R$-module and $f\colon M\to N$ a filtered homomorphism, then there is an induced graded $\gr(R)$-homomorphism $\gr(f)\colon \gr(M)\to\gr(N)$ given by 
    \begin{equation*}
        \gr(f)\left(x + M_{n + 1}\right) = f(x) + N_{n + 1}.
    \end{equation*}
    We note that $\gr$, as defined above, is a functor from the category of filtered $R$-modules to the category of graded $\gr(R)$-modules. Indeed, it is trivial to check that $\gr(\id_M) = \id_{\gr(M)}$ and that $\gr(g\circ f) = \gr(g)\circ\gr(f)$.
\end{definition}

% TODO: Add stuff about gr preserving short exact sequences with induced filtrations

\begin{proposition}
    Let $R$ be a filtered ring, $M$ and $N$ filtered $R$-modules, and $f\colon M\to N$ a filtered homomorphism. If 
    \begin{enumerate}[label=(\roman*)]
        \item $\gr(f)\colon\gr(M)\to\gr(N)$ is injective, and 
        \item $\displaystyle\bigcap_{n\ge 0} M_n = 0$, 
    \end{enumerate}
    then $f$ is injective.
\end{proposition}
\begin{proof}
    Since $\gr(f)$ is injective, the map $\gr_n(f)\colon M_n/M_{n + 1}\to N_n/N_{n + 1}$ is injective for every $n\ge 0$. We shall first show by induction on $n\ge 0$ that $f^{-1}(N_n)\subseteq M_n$. Clearly $f^{-1}(N_0)\subseteq M_0$. As for the inductive step, note that 
    \begin{equation*}
        f^{-1}(N_{n + 1})\subseteq f^{-1}(N_n)\subseteq M_n.
    \end{equation*}
    Hence, 
    \begin{equation*}
        f^{-1}(N_{n + 1})\subseteq f^{-1}(N_{n + 1})\cap M_n\subseteq M_{n + 1},
    \end{equation*}
    where the last containment follows from the fact that $\gr_n(f)$ is injective. As a result, 
    \begin{equation*}
        f^{-1}(0)\subseteq f^{-1}\left(\bigcap_{n\ge 0} N_n\right)\subseteq\bigcap_{n\ge 0} M_n = 0,
    \end{equation*}
    thereby completing the proof.
\end{proof}

\begin{theorem}
    Let $\displaystyle R = \bigoplus_{n\ge 0} R_n$ be a graded ring. The following are equivalent: 
    \begin{enumerate}[label=(\arabic*)]
        \item $R$ is Noetherian. 
        \item $R_0$ is Noetherian and $R$ is a finitely generated $R_0$-algebra.
    \end{enumerate}
\end{theorem}
\begin{proof}
    Clearly $(2)\implies(1)$ due to the Hilbert Basis Theorem. Suppose now that $R$ is Noetherian. Since 
    \begin{equation*}
        R_0\cong R/R_+,
    \end{equation*}
    so is $R_0$. Now, $R_+$ is a finitely generated ideal in $R$ and we may assume that it is generated by homogeneous elements $x_1,\dots,x_r\in R_+$ of degrees $n_1,\dots,n_r$ respectively. Let $R' = R_0[x_1,\dots,x_n]\subseteq R$. We shall show by induction on $n\ge 0$ that $R_n\subseteq R'$. Trivially $R_0\subseteq R'$. Suppose it is known that $R_k\subseteq R'$ for all $k < n$. An element $x\in R_n$ can be written as an $R$-linear combination of $x_1,\dots,x_r$ as 
    \begin{equation*}
        x = a_1x_1 + \dots + a_r x_r.
    \end{equation*}
    Since $x$ is a homogeneous element, breaking each $a_i$ into its homogeneous components and grouping terms of the same degree, we may suppose that each $a_i$ is graded of degree $n - n_i$, with the convention that $R_d = 0$ for $d < 0$. Since $n_i\ge 1$ for all $1\le i\le r$, it follows that $a_i\in R'$ for all $1\le i\le r$. Hence $x\in R'$, thereby completing the proof.
\end{proof}

\begin{definition}
    Let $M = (M_n)_{n\ge 0}$ be a filtered $R$-module and $I$ an ideal in $R$. The filtration is said to be an \define{$I$-filtration} if $IM_n\subseteq M_{n + 1}$ for all $n\ge 0$. Further, an $I$-filtration is said to be \define{$I$-stable} if $IM_{n} = M_{n + 1}$ for all $n\gg 0$.
\end{definition}

Let $M$ be a filtered $R$-module with an $I$-filtration. We define the \define{Rees algebra} of $R$ as a subring of the polynomial algebra
\begin{equation*}
    R^\ast = \bigoplus_{n\ge 0} I^n T^n\subseteq R[T].
\end{equation*}
That is, $R^\ast$ consists of all polynomials $a_0 + a_1T + \dots + a_nT^n\in R[T]$ such that $a_i\in I^i$ for all $i\ge 0$. Note that if $R$ is Noetherian, then $I$ is a finitely generated ideal, say $I = (a_1,\dots,a_r)$. Then $R^\ast$ is precisely the ring 
\begin{equation*}
    R^\ast = R[a_1T,\dots, a_rT]\subseteq R[T],
\end{equation*}
and in particular, is a Noetherian ring.

Similarly, we define an $R^\ast$-module 
\begin{equation*}
    M^\ast = \bigoplus_{n\ge 0} M_nT^n
\end{equation*}
whose elements are formal sums 
\begin{equation*}
    \sum_{\substack{n\ge 0\\\text{finite}}} x_n T^n,
\end{equation*}
with the obvious module structure over $R^\ast$.

\begin{theorem}\thlabel{equivalent-conditions-for-I-stable}
    Let $R$ be a filtered Noetherian ring, $I$ an ideal of $R$, and $M$ a finitely generated filtered $R$-module equipped with an $I$-filtration. The following are equivalent: 
    \begin{enumerate}[label=(\arabic*)]
        \item The filtration on $M$ is $I$-stable. 
        \item $M^\ast$ is a finitely generated $R^\ast$-module.
    \end{enumerate}
\end{theorem}
\begin{proof}
    Set 
    \begin{equation*}
        M_n^\ast = M_0\oplus M_1T\oplus\dots\oplus M_nT^n\oplus IM_nT^{n + 1}\oplus I^2 M_n T^{n + 2}\oplus\cdots,
    \end{equation*}
    which is clearly an $R^\ast$-module. Further, since each $M_n$ is a finite $R$-module, we can choose a finite $R$-generating set for the module $M_0\oplus \cdots\oplus M_n$, which would then be an $R^\ast$-generating set for $M_n^\ast$. That is, each $M_n^\ast$ is a finite $R^\ast$-module.

    Note that the filtration on $M$ being $I$-stable is equivalent to the ascending chain $(M_n^\ast)_{n\ge 0}$. We also have 
    \begin{equation*}
        M^\ast = \bigcup_{n\ge 0} M_n^\ast.
    \end{equation*}
    Thus, if the chain stabilizes, then $M^\ast$ is a finite $R^\ast$-module. Conversely, if $M^\ast$ is a finite $R^\ast$-module, then $M^\ast$ is Noetherian, and hence the chain must stabilize. This completes the proof.
\end{proof}

\begin{lemma}[Artin-Rees]\thlabel{artin-rees}
    Let $R$ be a filtered Noetherian ring, and $M$ a finitely generated filtered $R$-module equipped with an $I$-stabie flitration. If $N$ is a submodule of $M$, then the induced filtration on $N$ is $I$-stable.
\end{lemma}
\begin{proof}
    Let $N_n = N\cap M_n$, which is the induced filtration on $N$. Clearly this filtration is $I$-stable. We shall treat $N^\ast$ as a natural $R^\ast$-submodule of $M^\ast$. Since the filtration on $M$ is $I$-stable, due to \thref{equivalent-conditions-for-I-stable}, $M^\ast$ is a finite $R^\ast$-module, but since $R^\ast$ is Noetherian, $N^\ast$ is also a finite $R^\ast$-module, so that by \thref{equivalent-conditions-for-I-stable}, the filtration on $N$ is $I$-stable.
\end{proof}

\begin{theorem}[Krull Intersection Theorem]
    Let $R$ be a Noetherian ring, $I$ an ideal of $R$, and $M$ a finite $R$-module. Then the module 
    \begin{equation*}
        \bigcap_{n\ge 0} I^n M
    \end{equation*}
    consists of precisely those elements that are annihilated by some element in $1 + I$.
\end{theorem}
\begin{proof}
    Let 
    \begin{equation*}
        N = \bigcap_{n\ge 0} I^n M,
    \end{equation*}
    which is a submodule of $M$. In view of \thref{artin-rees}, this filtration is $I$-stable. That is, for $n\gg 0$, 
    \begin{equation*}
        IN = I\left(N\cap I^n M\right) = N\cap I^{n + 1} M = N.
    \end{equation*}
    By Nakayama's lemma, $N$ is annihilated by an element of the form $1 + a$, where $a\in I$.

    Conversely, suppose $x\in M$ is such that $(1 + a)x = 0$ for some $a\in I$. Then 
    \begin{equation*}
        x = -ax = a^2 x = -a^3 x = \cdots,
    \end{equation*}
    and hence $x\in\bigcap_{n\ge 0} I^n M = N$.
\end{proof}

\begin{corollary}
    If $R$ is a Noetherian ring, $I$ an ideal of $R$ contained in the Jacobson radical, and $M$ a finite $R$-module, then 
    \begin{equation*}
        \bigcap_{n\ge 0} I^n M = 0.
    \end{equation*}
\end{corollary}
\begin{proof}
    This follows from the fact that every element in $1 + I$ is invertible.
\end{proof}

\begin{corollary}
    Let $R$ be a Noetherian domain, and $I$ an ideal in $R$. Then 
    \begin{equation*}
        \bigcap_{n\ge 0} I^n = 0.
    \end{equation*}
\end{corollary}
\begin{proof}
    This follows from the fact that every element in $1 + I$ is a non-zerodivisor.
\end{proof}
\end{document}