\documentclass[12pt]{article}

% \usepackage{./arxiv}

\title{The Universal Enveloping Algebra and The Poincar\'e-Birkhoff-Witt Theorem}
\author{Swayam Chube}
\date{\today}

\usepackage[utf8]{inputenc} % allow utf-8 input
\usepackage[T1]{fontenc}    % use 8-bit T1 fonts
\usepackage{hyperref}       % hyperlinks
\usepackage{url}            % simple URL typesetting
\usepackage{booktabs}       % professional-quality tables
\usepackage{amsfonts}       % blackboard math symbols
\usepackage{nicefrac}       % compact symbols for 1/2, etc.
\usepackage{microtype}      % microtypography
\usepackage{graphicx}
\usepackage{natbib}
\usepackage{doi}
\usepackage{amssymb}
\usepackage{bbm}
\usepackage{amsthm}
\usepackage{amsmath}
\usepackage{xcolor}
\usepackage{theoremref}
\usepackage{enumitem}
\usepackage{mathpazo}
% \usepackage{euler}
\usepackage{mathrsfs}
\usepackage{todonotes}
\usepackage{stmaryrd}
\usepackage[all,cmtip]{xy} % For diagrams, praise the Freyd–Mitchell theorem 
\usepackage{marvosym}
\usepackage{geometry}
\usepackage{titlesec}

\renewcommand{\qedsymbol}{$\blacksquare$}

% Uncomment to override  the `A preprint' in the header
% \renewcommand{\headeright}{}
% \renewcommand{\undertitle}{}
% \renewcommand{\shorttitle}{}

\hypersetup{
    pdfauthor={Lots of People},
    colorlinks=true,
}

\newtheoremstyle{thmstyle}%               % Name
  {}%                                     % Space above
  {}%                                     % Space below
  {}%                             % Body font
  {}%                                     % Indent amount
  {\bfseries\scshape}%                            % Theorem head font
  {.}%                                    % Punctuation after theorem head
  { }%                                    % Space after theorem head, ' ', or \newline
  {\thmname{#1}\thmnumber{ #2}\thmnote{ (#3)}}%                                     % Theorem head spec (can be left empty, meaning `normal')

\newtheoremstyle{defstyle}%               % Name
  {}%                                     % Space above
  {}%                                     % Space below
  {}%                                     % Body font
  {}%                                     % Indent amount
  {\bfseries\scshape}%                            % Theorem head font
  {.}%                                    % Punctuation after theorem head
  { }%                                    % Space after theorem head, ' ', or \newline
  {\thmname{#1}\thmnumber{ #2}\thmnote{ (#3)}}%                                     % Theorem head spec (can be left empty, meaning `normal')

\theoremstyle{thmstyle}
\newtheorem{theorem}{Theorem}[section]
\newtheorem{lemma}[theorem]{Lemma}
\newtheorem{proposition}[theorem]{Proposition}

\theoremstyle{defstyle}
\newtheorem{definition}[theorem]{Definition}
\newtheorem*{corollary}{Corollary}
\newtheorem{remark}[theorem]{Remark}
\newtheorem{example}[theorem]{Example}
\newtheorem*{notation}{Notation}

% Common Algebraic Structures
\newcommand{\R}{\mathbb{R}}
\newcommand{\Q}{\mathbb{Q}}
\newcommand{\Z}{\mathbb{Z}}
\newcommand{\N}{\mathbb{N}}
\newcommand{\bbC}{\mathbb{C}} 
\newcommand{\K}{\mathbb{K}} % Base field which is either \R or \bbC
\newcommand{\calA}{\mathcal{A}} % Banach Algebras
\newcommand{\calB}{\mathcal{B}} % Banach Algebras
\newcommand{\calI}{\mathcal{I}} % ideal in a Banach algebra
\newcommand{\calJ}{\mathcal{J}} % ideal in a Banach algebra
\newcommand{\frakM}{\mathfrak{M}} % sigma-algebra
\newcommand{\calO}{\mathcal{O}} % Ring of integers
\newcommand{\bbA}{\mathbb{A}} % Adele (or ring thereof)
\newcommand{\bbI}{\mathbb{I}} % Idele (or group thereof)
\newcommand{\frakT}{\mathfrak{T}}
\newcommand{\frakF}{\mathfrak{F}}

% Categories
\newcommand{\catTopp}{\mathbf{Top}_*}
\newcommand{\catGrp}{\mathbf{Grp}}
\newcommand{\catTopGrp}{\mathbf{TopGrp}}
\newcommand{\catSet}{\mathbf{Set}}
\newcommand{\catTop}{\mathbf{Top}}
\newcommand{\catRing}{\mathbf{Ring}}
\newcommand{\catCRing}{\mathbf{CRing}} % comm. rings
\newcommand{\catMod}{\mathbf{Mod}}
\newcommand{\catMon}{\mathbf{Mon}}
\newcommand{\catMan}{\mathbf{Man}} % manifolds
\newcommand{\catDiff}{\mathbf{Diff}} % smooth manifolds
\newcommand{\catAlg}{\mathbf{Alg}}
\newcommand{\catRep}{\mathbf{Rep}} % representations 
\newcommand{\catVec}{\mathbf{Vec}}

% Group and Representation Theory
\newcommand{\chr}{\operatorname{char}}
\newcommand{\Aut}{\operatorname{Aut}}
\newcommand{\GL}{\operatorname{GL}}
\newcommand{\im}{\operatorname{im}}
\newcommand{\tr}{\operatorname{tr}}
\newcommand{\id}{\mathbf{id}}
\newcommand{\cl}{\mathbf{cl}}
\newcommand{\Gal}{\operatorname{Gal}}
\newcommand{\Tr}{\operatorname{Tr}}
\newcommand{\sgn}{\operatorname{sgn}}
\newcommand{\Sym}{\operatorname{Sym}}
\newcommand{\Alt}{\operatorname{Alt}}

% Commutative and Homological Algebra
\newcommand{\spec}{\operatorname{spec}}
\newcommand{\mspec}{\operatorname{m-spec}}
\newcommand{\Tor}{\operatorname{Tor}}
\newcommand{\tor}{\operatorname{tor}}
\newcommand{\Ann}{\operatorname{Ann}}
\newcommand{\Supp}{\operatorname{Supp}}
\newcommand{\Hom}{\operatorname{Hom}}
\newcommand{\End}{\operatorname{End}}
\newcommand{\coker}{\operatorname{coker}}
\newcommand{\limit}{\varprojlim}
\newcommand{\colimit}{%
  \mathop{\mathpalette\colimit@{\rightarrowfill@\textstyle}}\nmlimits@
}
\makeatother


\newcommand{\fraka}{\mathfrak{a}} % ideal
\newcommand{\frakb}{\mathfrak{b}} % ideal
\newcommand{\frakc}{\mathfrak{c}} % ideal
\newcommand{\frakf}{\mathfrak{f}} % face map
\newcommand{\frakg}{\mathfrak{g}}
\newcommand{\frakh}{\mathfrak{h}}
\newcommand{\frakm}{\mathfrak{m}} % maximal ideal
\newcommand{\frakn}{\mathfrak{n}} % naximal ideal
\newcommand{\frakp}{\mathfrak{p}} % prime ideal
\newcommand{\frakq}{\mathfrak{q}} % qrime ideal
\newcommand{\fraks}{\mathfrak{s}}
\newcommand{\frakt}{\mathfrak{t}}
\newcommand{\frakz}{\mathfrak{z}}
\newcommand{\frakA}{\mathfrak{A}}
\newcommand{\frakI}{\mathfrak{I}}
\newcommand{\frakJ}{\mathfrak{J}}
\newcommand{\frakK}{\mathfrak{K}}
\newcommand{\frakL}{\mathfrak{L}}
\newcommand{\frakN}{\mathfrak{N}} % nilradical 
\newcommand{\frakO}{\mathfrak{O}} % dedekind domain
\newcommand{\frakP}{\mathfrak{P}} % Prime ideal above
\newcommand{\frakQ}{\mathfrak{Q}} % Qrime ideal above 
\newcommand{\frakR}{\mathfrak{R}} % jacobson radical
\newcommand{\frakU}{\mathfrak{U}}
\newcommand{\frakX}{\mathfrak{X}}

% General/Differential/Algebraic Topology 
\newcommand{\scrA}{\mathscr A}
\newcommand{\scrB}{\mathscr B}
\newcommand{\scrF}{\mathscr F}
\newcommand{\scrN}{\mathscr N}
\newcommand{\scrP}{\mathscr P}
\newcommand{\scrR}{\mathscr R}
\newcommand{\scrS}{\mathscr S}
\newcommand{\bbH}{\mathbb H}
\newcommand{\Int}{\operatorname{Int}}
\newcommand{\psimeq}{\simeq_p}
\newcommand{\wt}[1]{\widetilde{#1}}
\newcommand{\RP}{\mathbb{R}\text{P}}
\newcommand{\CP}{\mathbb{C}\text{P}}

% Miscellaneous
\newcommand{\wh}[1]{\widehat{#1}}
\newcommand{\calM}{\mathcal{M}}
\newcommand{\calP}{\mathcal{P}}
\newcommand{\onto}{\twoheadrightarrow}
\newcommand{\into}{\hookrightarrow}
\newcommand{\Gr}{\operatorname{Gr}}
\newcommand{\Span}{\operatorname{Span}}
\newcommand{\ev}{\operatorname{ev}}
\newcommand{\weakto}{\stackrel{w}{\longrightarrow}}

\newcommand{\define}[1]{\textcolor{blue}{\textit{#1}}}
\newcommand{\caution}[1]{\textcolor{red}{\textit{#1}}}
\renewcommand{\mod}{~\mathrm{mod}~}
\renewcommand{\le}{\leqslant}
\renewcommand{\leq}{\leqslant}
\renewcommand{\ge}{\geqslant}
\renewcommand{\geq}{\geqslant}
\newcommand{\Res}{\operatorname{Res}}
\newcommand{\floor}[1]{\left\lfloor #1\right\rfloor}
\newcommand{\ceil}[1]{\left\lceil #1\right\rceil}
\newcommand{\gl}{\mathfrak{gl}}
\newcommand{\ad}{\operatorname{ad}}
\newcommand{\Stab}{\operatorname{Stab}}
\newcommand{\bfX}{\mathbf{X}}
\newcommand{\Ind}{\operatorname{Ind}}
\newcommand{\bfG}{\mathbf{G}}
\newcommand{\rank}{\operatorname{rank}}
\newcommand{\calo}{\mathcal{o}}
\newcommand{\frako}{\mathfrak{o}}
\newcommand{\Cl}{\operatorname{Cl}}

\newcommand{\idim}{\operatorname{idim}}
\newcommand{\pdim}{\operatorname{pdim}}
\newcommand{\Ext}{\operatorname{Ext}}
\newcommand{\co}{\operatorname{co}}
\newcommand{\ind}{\operatorname{ind}}
\newcommand{\Der}{\operatorname{Der}}

\geometry {
    margin = 1in
}

\titleformat
{\section}
[block]
{\Large\bfseries\scshape}
{\S\thesection}
{0.5em}
{\centering}
[]


\titleformat
{\subsection}
[block]
{\normalfont\bfseries\sffamily}
{\S\S}
{0.5em}
{\centering}
[]


\begin{document}
\maketitle

\begin{abstract}
    In this article, we define and construct the universal enveloping algebra of a Lie algebra, then, state and prove the Poincar\'e-Birkhoff-Witt Theorem.
\end{abstract}

\section{The Universal Enveloping Algebra}

\begin{definition}
    Let $\frakg$ be a Lie algebra over $k$. A \define{universal enveloping algebra} is a pair $(\frakU, i)$ where $\frakU$ is an associative algebra (over $k$, with identity) and $i: \frakg\to\frakU$ is a homomorphism of Lie algebras such that for any associative algebra $\frakA$ (over $k$, with identity) and any Lie algebra homomorphism $\varphi:\frakg\to\frakA$, there is a unique $k$-algebra homomorphism $\wt\varphi:\frakU\to\frakA$ making the following diagram commute.
    \begin{equation*}
        \xymatrix {
            \frakg\ar[r]\ar[d]_{i}\ar[r]^{\varphi} & \frakA\\
            \frakU\ar@{-->}[ru]_{\exists!~\wt\varphi}
        }
    \end{equation*}
\end{definition}

\subsection{Construction}

Let $\frakT$ denote the \define{tensor algebra} over $\frakg$, that is, 
\begin{equation*}
    \frakT = \bigoplus_{n\ge 0}\frakg^{\otimes n}.
\end{equation*}

There is a map $\mu:\frakg^{\otimes n}\times\frakg^{\otimes m}\to\frakg^{\otimes m + n}$ given by 
\begin{equation*}
    \mu(x_1\otimes\dots\otimes x_n, y_1\otimes\dots\otimes y_m) = x_1\otimes\dots x_n\otimes y_1\otimes\dots\otimes y_m
\end{equation*}
and extending linearly. This gives $\frakT$ the structure of a $k$-algebra. 

Let $\frakK$ denote the ideal in $\frakT$ generated by all elements of the form 
\begin{equation*}
    [x, y] - x\otimes y + y\otimes x
\end{equation*}
for $x,y\in\frakg$. Set $\frakU = \frakT/\frakK$ and let $\iota: \frakg\to\frakU$ be the composition
\begin{equation*}
    \frakg\longrightarrow\frakT\stackrel{\pi}{\longrightarrow}\frakU.
\end{equation*}

\begin{theorem}
    $(\frakU, \iota)$ is a universal enveloping algebra for $\frakg$.
\end{theorem}
\begin{proof}
    Let $\varphi:\frakg\to\frakA$ be a homomorphism of Lie algebras where $\frakA$ is an associative $k$-algebra. The universal property of the tensor algebra extends this to a $k$-algebra homomorphism $\wt\varphi:\frakT\to\frakA$. 

    Note that 
    \begin{equation*}
        \wt\varphi(x\otimes y - y\otimes x) = \wt\varphi(x\otimes y) - \wt\varphi(y\otimes x) = \varphi(x)\varphi(y) - \varphi(y)\varphi(x) = \varphi([x, y]),
    \end{equation*}
    whence $\wt\varphi$ vanishes on $\frakK$, thereby inducing a unique (due to the universal property of the kernel) map $\wt{\wt\varphi}:\frakU\to\frakA$, thereby completing the proof.
\end{proof}


\section{The Poincar\'e-Birkhoff-Witt Theorem}

Let $u_1,\dots,u_n$ be a $k$-basis of $\frakg$. A \define{monomial} in $\frakT$ is an element of the form 
\begin{equation*}
    u_{i_1}\otimes\dots\otimes u_{i_n}
\end{equation*}
for $n\ge 1$. The number $n$ is said to be the \define{degree} of the monomial. The \define{index} of the monomial is given by 
\begin{equation*}
    \ind(u_{i_1}\otimes\dots\otimes u_{i_n}) = \sum_{j < k}\eta_{jk}
\end{equation*}
where 
\begin{equation*}
    \eta_{jk} = 
    \begin{cases}
        0 & i_j \le i_k\\
        1 & i_j > i_k
    \end{cases}
\end{equation*}
A monomial is said to be \define{standard} if its index is $0$. Let $\frakg_n$ denote the vector space spanned by monomials of degree $n$ and let $\frakg_{n, i}$ denote the subspace of $\frakg_n$ spanned by monomials of degree $n$ and index $\le i$.

\begin{lemma}
    Every element of $\frakT$ is congruent modulo $\frakK$ to a $k$-linear combination of $1$ and standard monomials.
\end{lemma}
\begin{proof}
    Straightforward induction on the index and degree of standard monomials.
\end{proof}

Let $\frakP$ denote the vector space spanned by $u_{i_1}\dots u_{i_n}$ where $i_1\le \dots \le i_n$. These are to be interpreted as formal symbols without meaning.

\begin{lemma}
    There is a $k$-linear map $\sigma:\frakT\to\frakP$ such that 
    \begin{equation*}
        \sigma(1) = 1\quad\text{ and }\quad\sigma(u_{i_1}\otimes\dots\otimes u_{i_n}) = u_{i_1}\dots u_{i_n}.
    \end{equation*}
    if $i_1\le \dots \le i_n$. Further, 
    \begin{equation*}
        \sigma(u_{j_1}\otimes\dots\otimes[u_{j_k}, u_{j_{k + 1}}]\otimes\dots\otimes u_{j_n}) = \sigma(u_{j_1}\otimes\dots\otimes u_{j_n} - u_{j_1}\otimes\dots u_{j_{k + 1}}\otimes u_{j_k}\otimes\dots u_{j_n}).
    \end{equation*}
\end{lemma}
\begin{proof}
    We induct on degree and index, in that order. Suppose a linear map $\sigma$ has been defined on $k\oplus\frakg_1\oplus\dots\oplus\frakg_{n - 1}$. It is easy to extend this to $k\oplus\dots\oplus\frakg_{n, 0}$ by setting 
    \begin{equation*}
        \sigma(u_{i_1}\otimes\dots u_{i_n}) = u_{i_1}\dots u_{i_n}.
    \end{equation*}
    Now, suppose $\sigma$ has already been defined for $k\oplus\dots\frakg_{n, i - 1}$. Suppose $j_k > j_{k + 1}$. Then, define 
    \begin{equation*}
        \sigma(u_{j_1}\otimes\dots\otimes u_{j_n}) = \sigma(u_{j_1}\otimes\dots\otimes u_{j_{k + 1}}\otimes u_{j_k}\otimes\dots\otimes u_{j_n}) + \sigma(u_{j_1}\otimes\dots\otimes[u_{j_k}, u_{j_{k + 1}}]\otimes\dots\otimes u_{j_n}).
    \end{equation*}
    The right hand side is well-defined because the first term on the right has index at most $i - 1$ and the second term on the right is a linear combination of monomials of smaller degree.

    We must show that this is a well-defined assignment of $\sigma$, that is the right hand choice is independent of the pair of inversion chosen. To this end, let $j_l > j_{l + 1}$. We must consider two cases. 

    \begin{description}
    \item[Case 1: $l > k + 1$.] Set $u_{j_k} = u$, $u_{j_{k + 1}} = v$, $u_{j_l} = w$, $u_{j_{l + 1}} = x$.

    We would like to show 
    \begin{align*}
        &\sigma(\dots v\otimes u\otimes\dots\otimes w\otimes x\dots) + \sigma(\dots\otimes[u,v]\otimes\dots\otimes w\otimes x\dots)\\
        &= \\
        &\sigma(\dots u\otimes v\otimes\dots x\otimes w\dots) + \sigma(\dots u\otimes v\otimes\dots\otimes [x, w]\otimes\dots).
    \end{align*}
    We can expand the left hand side of the above equality using the induction hypothesis as 
    \begin{align*}
        &\sigma(\dots v\otimes u\otimes \dots \otimes x\otimes w\dots) + \sigma(v\otimes u\otimes \dots \otimes [x, w]\otimes\dots)\\
        &+ \sigma(\dots\otimes [u,v]\otimes\dots\otimes x\otimes w\dots) + \sigma(\dots\otimes [u, v]\otimes\dots\otimes [w,x]\otimes\dots).
    \end{align*}

    The right hand side can be written as 
    \begin{align*}
        &\sigma(\dots v\otimes u\otimes\dots\otimes x\otimes w\dots) + \sigma(\dots\otimes[u,v]\otimes\dots\otimes x\otimes w\dots)\\
        &+ \sigma(\dots v\otimes u\otimes\dots\otimes [x, w]\otimes\dots) + \sigma(\dots\otimes [u,v]\otimes\dots\otimes[x, w]\otimes\dots).
    \end{align*}
    This completes the proof in this case. 

    \item[Case 2: $l = k + 1$.] We write $u_{j_k} = u$, $u_{j_{k + 1}} = v = u_{j_l}$ and $u_{j_{l + 1}} = w$. We want to show the equality
    \begin{equation*}
        \sigma(\dots v\otimes u\otimes w\dots) + \sigma(\dots[u, v]\otimes w\dots) = \sigma(\dots u\otimes w\otimes v\dots) + \sigma(\dots u\otimes[v, w]\dots).
    \end{equation*}

    The left hand side can be expanded further as 
    \begin{align*}
        &\sigma(\dots v\otimes w\otimes u\dots) + \sigma(\dots v\otimes [u, w]\dots) + \sigma(\dots [u, v]\otimes w\dots)\\
        &= \sigma(\dots w\otimes v\otimes u\dots) + \sigma(\dots[v, w]\otimes u\dots) + \sigma(\dots v\otimes [u, w]\dots) + \sigma(\dots [u, v]\otimes w\dots).
    \end{align*}

    Similarly, the right hand side can be expanded as 
    \begin{align*}
        &\sigma(\dots w\otimes u\otimes v\dots) + \sigma(\dots[u, w]\otimes v\dots) + \sigma(\dots u\otimes[v, w]\dots)\\
        &= \sigma(\dots w\otimes v\otimes u\dots) + \sigma(\dots w\otimes [u, w]\dots) + \sigma(\dots[u, w]\otimes v\dots) + \sigma(u\otimes [v,w]\dots).
    \end{align*}

    It remains to show the equality: 
    \begin{align*}
        &\sigma(\dots[v, w]\otimes u\dots) + \sigma(\dots v\otimes [u, w]\dots) + \sigma(\dots [u, v]\otimes w\dots)\\
        &= \sigma(\dots w\otimes [u, w]\dots) + \sigma(\dots[u, w]\otimes v\dots) + \sigma(u\otimes [v,w]\dots),
    \end{align*}
    which reduces to 
    \begin{equation*}
        \sigma(\dots[[v, w], u]\dots) + \sigma(\dots[v,[u,w]]\dots) + \sigma(\dots[[u,v],w]\dots) = 0,
    \end{equation*}
    which follows from Jacobi's Identity. This completes the proof in this case.
    \end{description}

    Now that $\sigma$ is well-defined for monomials, we can extend it linearly to $\frakg_{n, i}$, thereby completing the induction.
\end{proof}

\begin{theorem}[Poincar\'e-Birkhoff-Witt Theorem]
    The cosets of $1$ and the standard monomials form a basis for $\frakU = \frakT/\frakK$.
\end{theorem}
\begin{proof}
    We have shown that the standard monomials and $1$ span $\frakU$. It remains to show linear independence. This follows from the preceding lemma, since the $u_{i_1}\dots u_{i_n}$'s are linearly independent in $\frakP$.
\end{proof}

\section{Properties of the Universal Enveloping Algebra}

\begin{definition}
    A ring $R$ is said to be \define{filtered} if it is equipped with an increasing sequence $\mathscr R = \{R_i\}_{i\ge 0}$ of abelian subgroups such that 
    \begin{enumerate}[label=(\alph*)]
        \item $\displaystyle\bigcup_{i\ge 0} R_i = R$. 
        \item For all $i,j\ge 0$, $R_iR_j\subseteq R_{i + j}$.
    \end{enumerate}

    Each filtration of a ring gives rise to an \define{associated graded ring}, 
    \begin{equation*}
        \Gr_{\mathscr R}(R) = \bigoplus_{i\ge 0} R_i/R_{i - 1},
    \end{equation*}
    with the convention that $R_{i - 1} = 0.$

    For any $a\in R$, there is a non-negative integer $n$ such that $a\in R_n$ but $a\notin R_{n - 1}$. The homogeneous element $\overline b = b + R_{n - 1}\in\Gr_{\mathscr R}(R)$ is called the \define{leading term} of $b$. If $b = 0$, we take its leading term to be $0$.
\end{definition}

\begin{lemma}\thlabel{lem:domain-left-noetherian}
    Let $R$ be a filtered ring with an increasing filtration $\{R_i\}_{i\ge 0}$ and let $G$ denote the corresponding associated graded.
    \begin{enumerate}[label=(\alph*)]
        \item If $G$ is a domain, then so is $R$.
        \item If $G$ is left (resp. right) noetherian, then so is $R$.
    \end{enumerate}
\end{lemma}
\begin{proof}
\begin{enumerate}[label=(\alph*)]
    \item Suppose $a,b\in R\setminus\{0\}$ such that $ab = 0$ in $R$. Let $\overline a, \overline b$ denote the leading terms of $a$ and $b$ respectively. Then, $\overline a\overline b = 0$, a contradiction.

    \item Let $I$ be a left-ideal in $R$. We shall show that $I$ is finitely generated. Let $\overline I$ denote the abelian group generated by the leading terms of elements of $I$. It is easy to see that $\overline I$ is a left ideal in $G$ (whence, is a homogeneous left ideal). Since $G$ is left noetherian, there are $b_1,\dots,b_n\in I$ such that $\overline b_1,\dots,\overline b_n$ generate $\overline I$ as a left ideal in $G$ where $\overline b_i$ is the leading term of $b_i$.

    We contend that the $b_i$'s generate $I$. Let $b\in I$. Then, $\overline b$, the leading term of $b$ is a linear combination of the form 
    \begin{equation*}
        \overline b = \sum_{i} \overline a_i\overline b_i
    \end{equation*}
    where $\overline a_i\in G$. Since the left hand side is homogeneous, we may choose the $a_i$'s to be homogeneous in $G$, consequently, the $\overline a_i$'s are leading terms of some $a_i\in R$. 

    From the above equality, we deduce that $\overline b$ is the leading term of $\sum_i a_ib_i$ whence, $b - \sum_{i} a_ib_i$ has leading term of homogeneous degree smaller than that of $\overline b$. An induction argument finishes the proof.\qedhere
\end{enumerate}
\end{proof}

If $\frakg$ is a finite-dimensional Lie algebra over $k$ (no restriction), then its universal enveloping algebra $\frakU$ is equipped with a canonical filtration: 
\begin{equation*}
    \frakU^{(n)} = k\oplus\frakg\oplus\frakg^2\oplus\dots\oplus\frakg^{n}.
\end{equation*}
Using the Poincar\'e-Birkhoff-Witt theorem, it is not hard to see that the associated graded corresponding to the above filtration is isomorphic to $k[X_1,\dots,X_n]$ where $n = \dim_k\frakg$.

\begin{theorem}
    The universal enveloping algebra of a finite-dimensional Lie algebra over $k$ is a left (and right) Noetherian domain.
\end{theorem}
\begin{proof}
    Follows from \thref{lem:domain-left-noetherian} and the discussion above.
\end{proof}

\section{Free Lie Algebras}

\begin{definition}
    Let $X$ be a set. A \define{free algebra over $k$} on $X$ is a pair $(\frakL(X), \iota)$ where $\iota: X\to\frakL(X)$ is such that for any map of sets $\varphi: X\to\frakg$ where $\frakg$ is a Lie algebra over $k$, there is a unique Lie algebra homomorphism $\wt\varphi:\frakL(X)\to\frakg$ satisfying 
    \begin{equation*}
        \xymatrix {
            X\ar[r]^{\varphi}\ar[d] & \frakg\\
            \frakL(X)\ar@{-->}[ru]_{\exists!\wt\varphi}
        }
    \end{equation*}
\end{definition}

\subsection{Construction}

Let $\mathfrak F(X)$ denote the free $k$-algebra generated by $X$. Then, $\frakF(X)$ has the structure of a Lie algebra. Let $\frakL(X)$ denote the Lie subalgebra of $\frakF(X)$ generated by $X$. We contend that $X\into\frakL(X)$ is the free algebra on $X$.

Let $\varphi: X\to\frakg$ be a map of sets where $\frakg$ is a Lie algebra over $k$. Then, we have the following commutative diagram. 
\begin{equation*}
    \xymatrix {
        X\ar[r]^\varphi\ar[d] & \frakg\ar[d]\\
        \frakF(X)\ar@{-->}[r]_{\exists!\wt\varphi} & \frakU(\frakg)
    }
\end{equation*}
Where $\wt\varphi$ restricts to $\varphi$ on $X\subseteq\frakF(X)$. Note that $\varphi$ is also a Lie algebra homomorphism. Therefore, $\wt\varphi^{-1}\frakg$ is a Lie subalgebra of $\frakF(X)$ containing $X$ and hence, $\frakL(X)\subseteq\wt\varphi^{-1}\frakg$. It follows that $\wt\varphi$ restricts to a Lie algebra homomorphism $\wt\varphi: \frakL(X)\to\frakg$. The uniqueness follows since $X$ generates $\frakL(X)$. This completes the proof of existence. 

The above discussion also shows: 
\begin{proposition}
    $\frakF(X)$ is the universal enveloping algebra of $\frakL(X)$.
\end{proposition}

\section{Epimorphisms of Lie Algebras}

\begin{definition}
    Let $\frakg$ and $\frakh$ be Lie algebras with a Lie algebra homomorphism $\psi:\frakg\to\operatorname{Der}(\frakh)$. Then, there is a Lie algebra structure on $\frakt = \frakh\oplus\frakg$ given by 
    \begin{equation*}
        [(h, g), (h', g')] = \left([h, h'] + \psi_g(h') - \psi_{g'}(h), [g, g']\right).
    \end{equation*}
    This is the \define{semidirect product} and is often denoted by $\frakh\rtimes_{\psi}\frakg$
\end{definition}

\begin{lemma}\thlabel{lem:not-epimorphic}
    Suppose $\frakh\subsetneq\frakg$ is an inclusion of Lie algebras and $V$ a $\frakg$-module with a $0\ne\psi\in V$ that is annihilated by $\frakh$ but not by $\frakg$. Then the inclusion $\frakh\into\frakg$ is not an epimorphism.
\end{lemma}
\begin{proof}
    We can treat $V$ as an abelian Lie algebra and consider the semidirect product $\frakt = V\rtimes\frakg$. Note that since $V$ is abeilan, every $k$-linear map $V\to V$ is a derivation of $V$. 

    Define the map $\theta: \frakg\to V\rtimes\frakg$ by $\theta(x) = (x\cdot\psi, x)$. We contend that this is a Lie algebra homomorphism. Indeed, for $x,y\in\frakg$, we have 
    \begin{align*}
        [\theta(x), \theta(y)] &= \left[(x\cdot\psi, x), (y\cdot\psi, y)\right]\\
        &= \left(x\cdot(y\cdot\psi) - y\cdot(x\cdot\psi), [x, y]\right)\\
        &= [x, y]\cdot\psi.
    \end{align*}
     Further, for $x\in\frakh$, $\theta(\frakh) = (0, x)$. Consequently, the two maps $\theta:\frakg\to V\rtimes\frakg$ and $\iota:\frakg\into V\rtimes\frakg$ agree on $\frakh$ but not on $\frakg$. Thus, the inclusion $\frakh\into\frakg$ is not an epimorphism.
\end{proof}

\begin{theorem}\thlabel{thm:epi-surj-lie-alg}
    Epimorphisms in the category of Lie algebras (including the infinite-dimensional ones) are precisely the surjective Lie algebra homomorphisms.
\end{theorem}
\begin{proof}
    Obviously, surjecive Lie algebra homomorphisms $\frakh\to\frakg$ are epimorphisms. Therefore, it suffices to show that a proper inclusion $\frakh\subsetneq\frakg$ of Lie algebras is not epimorphic. We prove this in the case $\frakg$ is finite-dimensional but an analogous proof works in the infinite-dimensional case.

    Choose a $k$-basis $x_1,\dots,x_n$ of $\frakg$ such that $x_{p + 1},\dots,x_{n}$ is a $k$-basis of $\frakh$. Let $\frakU$ denote the universal enveloping algebra of $\frakg$. Recall that the Poincar\'e-Birkhoff-Witt theorem guarantees a $k$-basis of $\frakU$ in the form $x_1^{m_1}\cdots x_n^{m_n}$ where $m_i\ge 0$ for $1\le i\le n$. We use the notation $\mathbf{x}^m$ to denote products of the aforementioned kind.

    Let $V$ be the subspace of $\frakU$ spanned by $\mathbf{x}^k$ where $k_{p + 1} = \cdots = k_n = 0$, and let $\pi: \frakU\onto V$ denote the projection. Define a $\frakg$-action on $V$ as follows: For $v\in V\subseteq\frakU$ and $x\in\frakg$, set $x\cdot v = \pi(x v)$, where $xv$ is the standard product in the associative algebra $\frakU$. 

    First, we must show that this is gives $V$ the structure of a $\frakg$-module. To this end, we must show that for $x,x'\in\frakg$ and $v\in V$, 
    \begin{equation*}
        \pi\left([x, x']\cdot v\right) = \pi(x\cdot\pi(x'\cdot v)) - \pi(x'\cdot\pi(x\cdot v)).
    \end{equation*}
    But by definition, $\pi\left([x, x']\cdot v\right) = \pi(xx'v) - \pi(x'xv)$. Hence, it suffices to show that 
    \begin{equation*}
        \pi\left(x\cdot\pi(x'\cdot v)\right) = \pi(xx'v).
    \end{equation*}
    We can write $x'v = \pi(x'v) + v'$, where $v'$ is a linear combination of basis elements $\mathbf{x}^k$ with $k_i > 0$ for some $p < i\le n$. So it suffices to show that $\pi(x\mathbf{x}^k) = 0$ for some such $k$ and every $x\in\frakg$. For if this is shown, then 
    \begin{equation*}
        \pi(xx'v) = \pi\left(x\pi(x'v) + xv'\right) = \pi(x\pi(x'v)) + \pi(xv') = \pi(x\pi(x'v)),
    \end{equation*}
    as desired.

    Therefore, let $x\in\frakg$ and $k$ a multivector of nonnegative integers such that $k_i > 0$ for some $p < i\le n$. We can write $\mathbf{x}^k$ as $\mathbf{x}^h\mathbf{x}^l$ where $h_i = 0$ for $p < i\le n$ and $l_i = 0$ for $1\le i\le p$. Note that $l\ne 0$ as a vector. Then, we may write $x\mathbf{x}^h$ as a linear combination of elements $\mathbf{x}^{h'}\mathbf{x}^{l'}$ with $h_i' = 0$ for $i < p\le n$ and $l_i' = 0$ for $1\le i\le p$. In conclusion, $x\mathbf{x}^k$ is a linear combination of elements of the form $\mathbf{x}^{h'}\mathbf{x}^{l'}\mathbf{x}^{l}$.

    Since $\frakh$ is a subalgebra, note that the vector space spanned by $\mathbf{x}^{l}$ where $l_i = 0$ for $1\le i\le p$ is precisely the universal enveloping algebra of $\frakh$. In particular, it is an associative $k$-subalgebra of $\frakU$. This shows that $\mathbf{x}^{l'}\mathbf{x}^{l}$ can be written as a linear combination of elements of the form $\mathbf{x}^{l''}$ where $l_i'' = 0$ for $1\le i\le p$. Since the terms of degree greater than $0$ form an ideal in $\frakU(\frakh)\subseteq\frakU$, $l\ne 0$ will imply $l''\ne 0$. 

    The above paragraph shows that $x\mathbf{x}^k$ can be written as a linear combination of elements of the form $\mathbf{x}^{h'}\mathbf{x}^{l''}$ where $h_i' = 0$ for $p < i\le n$ and $l_i'' = 0$ for $1\le i\le p$ and $l''\ne 0$ as a vector. But these are all basis elements and are not contained in $V$, therefore, map to $0$ under $\pi$. Hence, $\pi(x\mathbf{x}^k) = 0$ whenever $k$ is such that $k_i > 0$ for some $p < i\le n$.

    We have established that $V$ is indeed a $\frakg$-module. The element $1\in V$ is not annihilated by $\frakg$, but is annihilated by $\frakh$ since for any $y\in\frakh$, $y\cdot 1 = \pi(y) = 0$, since $y\notin V\subseteq\frakU$. Invoking \thref{lem:not-epimorphic}, we have that $\frakh\into\frakg$ is not an epimorphism, thereby completing the proof.
\end{proof}

\begin{remark}
    In the case that $\frakg$ is infinite-dimensional, choose first a $k$-basis $\{x_\beta\colon\beta\in B\}$ of $\frakh$ and extend it to a $k$-basis 
    \begin{equation*}
        \{x_\alpha\colon\alpha\in A\}\sqcup\{x_\beta\colon \beta\in B\}
    \end{equation*}
    of $\frakg$. Well order $A$ and $B$ separately and define a well order on $A\sqcup B$ by setting $a < b$ whenever $a\in A$ and $b\in B$. That this is indeed a well-order is easy to check. The proof then remains unchanged by replacing each instance of ``$1\le i\le p$'' with ``$i\in A$'' and ``$p < i\le n$ with $i\in B$''.
\end{remark}


The analogue of \thref{thm:epi-surj-lie-alg} is not true in the category of finite-dimensional Lie algebras: 

\begin{theorem}
    Let $k$ be algebraically closed and $\chr k = 0$, $\frakg = \mathfrak{sl}_2(k)$, and $\frakh\subsetneq\frakg$ be the subalgebra of upper triangular matrices in $\frakg$. Then, $\frakh\into\frakg$ is an epimorphism.
\end{theorem}
\begin{proof}
    Recall that there is the standard basis $\{h, x, y\}$ of $\frakg$ over $k$ and $\frakh$ is generated by $\{h, x\}$. For the sake of this proof, make the replacements $h\mapsto\frac{1}{2}h$, $x\mapsto\frac{1}{\sqrt 2}x$, and $y\mapsto\frac{1}{\sqrt 2}y$, so that 
    \begin{equation*}
        [h, x] = x,\quad [h, y] = -y,\quad\text{and}\quad[x, y] = h.
    \end{equation*}
    Suppose now that there are two morphisms $\frakg\to\frakt$ that agree on $\frakh$. We shall show that both the morphisms are equal. Since $\frakg$ is simple, both morphisms must be injective, unless they are both $0$ in which case there's nothing to prove. Denote the images of $h$ and $x$ in $\frakt$ by $h$ and $x$ since both morphisms agree here and let $y, y'$ denote the images of $y$ in $\frakt$ under the two morphsims and suppose that $y\ne y'$.

    Set $u_0 = y - y'$, and define $u_n = [y, u_{n - 1}]$ for $n\ge 1$ with the convention that $u_{-1} = 0$. Note that $[x, u_0] = [x, y] - [x, y'] = 0$ and $[h, u_0] = -u_0$.

    \noindent\textbf{Claim 1.} $[x, u_n] = -\frac{1}{2}n(n + 1)u_{n - 1}$ and $[h, u_n] = -(n + 1)u_{n}$ for $n\ge 0$.

    \noindent We induct on $n$. The base case of $n = 0$ is clear. For $n\ge 1$, we have 
    \begin{align*}
        &[x, [y, u_{n - 1}]] + [y, [u_{n - 1}, x]] + [u_{n - 1}, [x, y]] = 0\\
        \implies&[x, u_n] + \frac{1}{2}n(n - 1)[y, u_{n - 2}] + [u_{n - 1}, h]\\
        \implies &[x, u_n] = -\frac{1}{2}n(n - 1)u_{n - 1} - nu_{n - 1} = -\frac{1}{2}n(n + 1)u_{n - 1}.
    \end{align*}
    Similarly, we have 
    \begin{align*}
        &[h, [y, u_{n - 1}]] + [y, [u_{n - 1}, h]] + [u_{n - 1}, [h, y]] = 0\\
        \implies&[h, u_n] + n[y, u_{n - 1}] - [u_{n - 1}, y] = 0\\
        \implies&[h, u_n] = -(n + 1)u_n.
    \end{align*}
    This proves Claim 1.

    \noindent\textbf{Claim 2.} $\{h, x, y\}\cup\{u_n\colon n\ge 0\}$ is linearly independent.

    \noindent Suppose not. Then there is a linear combination 
    \begin{equation*}
        \alpha_n u_n + \dots + \alpha_0 u_0 + \alpha_h h + \alpha_x x + \alpha_y y = 0.
    \end{equation*}
    with $\alpha_n\ne 0$. If $n\ge 1$, then apply $[x,\cdot]$ until you are left with a linear combination of the form 
    \begin{equation*}
        \beta_0 u_0 + \beta_h h + \beta_x x + \beta_y y = 0
    \end{equation*}
    where $\beta_0\ne 0$. Next, applying $[h,\cdot]$, we have 
    \begin{equation*}
        -\beta_0 u_0 + \beta_x x - \beta_y y = 0.
    \end{equation*}
    Adding the above two equations, we have $2\beta_x x + \beta_h h = 0$, whence $\beta_x = \beta_h = 0$. This gives $\beta_0 u_0 + \beta_y y = 0$. Applying $[x, \cdot]$, we get $\beta_y = 0$, which leaves us with $\beta_0u_0 = 0$, which is absurd, since $u_0\ne 0$ and $\beta_0\ne 0$. This proves Claim 2.

    Finally, we have our desired contradiction, since $\frakt$ is a finite-dimensional Lie algebra. It follows that the inclusion $\frakh\into\frakg$ is epimorphic.
\end{proof}
\end{document}