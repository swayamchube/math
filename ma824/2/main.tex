\documentclass[10pt]{amsart}

\title{MA 824: Assignment 2}
\author{Swayam Chube (200050141)}
\date{\today}

\usepackage[utf8]{inputenc} % allow utf-8 input
\usepackage[T1]{fontenc}    % use 8-bit T1 fonts
\usepackage{hyperref}       % hyperlinks
\usepackage{url}            % simple URL typesetting
\usepackage{booktabs}       % professional-quality tables
\usepackage{amsfonts}       % blackboard math symbols
\usepackage{nicefrac}       % compact symbols for 1/2, etc.
\usepackage{microtype}      % microtypography
\usepackage{lmodern}
\usepackage{graphicx}
\usepackage{natbib}
\usepackage{doi}
\usepackage{amssymb}
\usepackage{bbm}
\usepackage{amsthm}
\usepackage{amsmath}
\usepackage{xcolor}
\usepackage{theoremref}
\usepackage{enumitem}
% \usepackage{mathpazo}
% \usepackage{mlmodern}
\usepackage{mathrsfs}
\usepackage{todonotes}
\usepackage{stmaryrd}
\usepackage[all,cmtip]{xy} % For diagrams, praise the Freyd–Mitchell theorem 
\usepackage{marvosym}
\usepackage{geometry}
\usepackage{bbm}
\usepackage{mathtools}
\usepackage{fouriernc}

\renewcommand{\qedsymbol}{$\blacksquare$}

% Uncomment to override  the `A preprint' in the header
% \renewcommand{\headeright}{}
% \renewcommand{\undertitle}{}
% \renewcommand{\shorttitle}{}

\hypersetup{
    pdfauthor={Lots of People},
    colorlinks=true,
	citecolor=blue
}

\newtheoremstyle{thmstyle}%               % Name
  {}%                                     % Space above
  {}%                                     % Space below
  {}%                             % Body font
  {}%                                     % Indent amount
  {\bfseries\scshape}%                            % Theorem head font
  {.}%                                    % Punctuation after theorem head
  { }%                                    % Space after theorem head, ' ', or \newline
  {\thmname{#1}\thmnumber{ #2}\thmnote{ (#3)}}%                                     % Theorem head spec (can be left empty, meaning `normal')

\newtheoremstyle{defstyle}%               % Name
  {}%                                     % Space above
  {}%                                     % Space below
  {}%                                     % Body font
  {}%                                     % Indent amount
  {\bfseries\scshape}%                            % Theorem head font
  {.}%                                    % Punctuation after theorem head
  { }%                                    % Space after theorem head, ' ', or \newline
  {\thmname{#1}\thmnumber{ #2}\thmnote{ (#3)}}%                                     % Theorem head spec (can be left empty, meaning `normal')

\theoremstyle{thmstyle}
\newtheorem{theorem}{Theorem}[section]
\newtheorem{lemma}[theorem]{Lemma}
\newtheorem{proposition}[theorem]{Proposition}
\newtheorem{porism}[theorem]{Porism}
\newtheorem*{claim}{Claim}

\theoremstyle{defstyle}
\newtheorem{definition}[theorem]{Definition}
\newtheorem*{notation}{Notation}
\newtheorem*{corollary}{Corollary}
\newtheorem{remark}[theorem]{Remark}
\newtheorem{example}[theorem]{Example}

% Common Algebraic Structures
\newcommand{\R}{\mathbb{R}}
\newcommand{\Q}{\mathbb{Q}}
\newcommand{\Z}{\mathbb{Z}}
\newcommand{\N}{\mathbb{N}}
\newcommand{\bbC}{\mathbb{C}}
\newcommand{\K}{\mathbb{K}}
\newcommand{\calA}{\mathcal{A}}
\newcommand{\calH}{\mathcal{H}} % Hilbert space
\newcommand{\frakM}{\mathfrak{M}}
\newcommand{\calO}{\mathcal{O}}
\newcommand{\bbA}{\mathbb{A}}
\newcommand{\bbI}{\mathbb{I}}

% Categories
\newcommand{\catTopp}{\mathbf{Top}_*}
\newcommand{\catGrp}{\mathbf{Grp}}
\newcommand{\catTopGrp}{\mathbf{TopGrp}}
\newcommand{\catSet}{\mathbf{Set}}
\newcommand{\catTop}{\mathbf{Top}}
\newcommand{\catRing}{\mathbf{Ring}}
\newcommand{\catCRing}{\mathbf{CRing}} % comm. rings
\newcommand{\catMod}{\mathbf{Mod}}
\newcommand{\catMon}{\mathbf{Mon}}
\newcommand{\catMan}{\mathbf{Man}} % manifolds
\newcommand{\catDiff}{\mathbf{Diff}} % smooth manifolds
\newcommand{\catAlg}{\mathbf{Alg}}
\newcommand{\catRep}{\mathbf{Rep}} % representations 
\newcommand{\catVec}{\mathbf{Vec}}

% Group and Representation Theory
\newcommand{\chr}{\operatorname{char}}
\newcommand{\Aut}{\operatorname{Aut}}
\newcommand{\GL}{\operatorname{GL}}
\newcommand{\im}{\operatorname{im}}
\newcommand{\tr}{\operatorname{tr}}
\newcommand{\id}{\mathbf{id}}
\newcommand{\cl}{\mathbf{cl}}
\newcommand{\Gal}{\operatorname{Gal}}
\newcommand{\Tr}{\operatorname{Tr}}
\newcommand{\sgn}{\operatorname{sgn}}
\newcommand{\Sym}{\operatorname{Sym}}
\newcommand{\Alt}{\operatorname{Alt}}

% Commutative and Homological Algebra
\newcommand{\spec}{\operatorname{spec}}
\newcommand{\mspec}{\operatorname{m-spec}}
\newcommand{\Tor}{\operatorname{Tor}}
\newcommand{\tor}{\operatorname{tor}}
\newcommand{\Ann}{\operatorname{Ann}}
\newcommand{\Supp}{\operatorname{Supp}}
\newcommand{\Hom}{\operatorname{Hom}}
\newcommand{\End}{\operatorname{End}}
\newcommand{\coker}{\operatorname{coker}}
\newcommand{\limit}{\varprojlim}
\newcommand{\colimit}{%
  \mathop{\mathpalette\colimit@{\rightarrowfill@\textstyle}}\nmlimits@
}
\makeatother


\newcommand{\fraka}{\mathfrak{a}} % ideal
\newcommand{\frakb}{\mathfrak{b}} % ideal
\newcommand{\frakc}{\mathfrak{c}} % ideal
\newcommand{\frakf}{\mathfrak{f}} % face map
\newcommand{\frakg}{\mathfrak{g}}
\newcommand{\frakh}{\mathfrak{h}}
\newcommand{\frakm}{\mathfrak{m}} % maximal ideal
\newcommand{\frakn}{\mathfrak{n}} % naximal ideal
\newcommand{\frakp}{\mathfrak{p}} % prime ideal
\newcommand{\frakq}{\mathfrak{q}} % qrime ideal
\newcommand{\fraks}{\mathfrak{s}}
\newcommand{\frakt}{\mathfrak{t}}
\newcommand{\frakz}{\mathfrak{z}}
\newcommand{\frakA}{\mathfrak{A}}
\newcommand{\frakF}{\mathfrak{F}}
\newcommand{\frakI}{\mathfrak{I}}
\newcommand{\frakK}{\mathfrak{K}}
\newcommand{\frakL}{\mathfrak{L}}
\newcommand{\frakN}{\mathfrak{N}} % nilradical 
\newcommand{\frakP}{\mathfrak{P}} % nilradical 
\newcommand{\frakR}{\mathfrak{R}} % jacobson radical
\newcommand{\frakT}{\mathfrak{T}} % tensor algebra
\newcommand{\frakU}{\mathfrak{U}}
\newcommand{\frakX}{\mathfrak{X}}

% General/Differential/Algebraic Topology 
\newcommand{\scrA}{\mathscr A}
\newcommand{\scrB}{\mathscr B}
\newcommand{\scrF}{\mathscr F}
\newcommand{\scrP}{\mathscr P}
\newcommand{\scrS}{\mathscr S}
\newcommand{\bbH}{\mathbb H}
\newcommand{\Int}{\operatorname{Int}}
\newcommand{\psimeq}{\simeq_p}
\newcommand{\wt}[1]{\widetilde{#1}}
\newcommand{\RP}{\mathbb{R}\text{P}}
\newcommand{\CP}{\mathbb{C}\text{P}}

% Miscellaneous
\newcommand{\wh}[1]{\widehat{#1}}
\newcommand{\calM}{\mathcal{M}}
\newcommand{\calP}{\mathcal{P}}
\newcommand{\onto}{\twoheadrightarrow}
\newcommand{\into}{\hookrightarrow}
\newcommand{\Gr}{\operatorname{Gr}}
\newcommand{\Span}{\operatorname{Span}}
\newcommand{\ev}{\operatorname{ev}}
\newcommand{\weakto}{\stackrel{w}{\longrightarrow}}

\newcommand{\define}[1]{\textcolor{blue}{\textit{#1}}}
% \newcommand{\caution}[1]{\textcolor{red}{\textit{#1}}}
\newcommand{\important}[1]{\textcolor{red}{#1}}
\renewcommand{\mod}{~\mathrm{mod}~}
\renewcommand{\le}{\leqslant}
\renewcommand{\leq}{\leqslant}
\renewcommand{\ge}{\geqslant}
\renewcommand{\geq}{\geqslant}
\newcommand{\Res}{\operatorname{Res}}
\newcommand{\floor}[1]{\left\lfloor #1\right\rfloor}
\newcommand{\ceil}[1]{\left\lceil #1\right\rceil}
\newcommand{\gl}{\mathfrak{gl}}
\newcommand{\ad}{\operatorname{ad}}
\newcommand{\ind}{\operatorname{ind}}
\newcommand{\sminus}{\setminus}
\newcommand{\Sd}{\operatorname{Sd}}
\newcommand{\mesh}{\operatorname{mesh}}
\newcommand{\diam}{\operatorname{diam}}
\newcommand{\co}{\operatorname{co}}
\newcommand{\Lip}{\operatorname{Lip}}
\newcommand{\lip}{\operatorname{lip}}
\newcommand{\dist}{\operatorname{dist}}
\newcommand{\pv}{\operatorname{p.v.}}
\newcommand{\Graph}{\operatorname{Graph}}

\geometry {
    margin = 1in
}

\begin{document}
\maketitle 

Throughout this document, $\K$ denotes either $\R$ or $\bbC$. Most proofs should work over either of the two fields unless specified otherwise.

\section{Problem 1}
\begin{enumerate}[label=(\alph*)]
\item Recall the following result from complex analysis: 
\begin{theorem}\thlabel{thm:radius-of-convergence}
    Let $\displaystyle\sum_{n\ge 0} a_n(z - a)^n$ be a power series with $a_n\in\bbC$. The radius of convergence of the above power series is the unique real number $R\ge 0$ such that
    \begin{enumerate}[label=(\roman*)]
        \item if $|z - a| < R$, the series converges absolutely. 
        \item if $|z - a| > R$, the terms of the series become unbounded and so the series diverges.
    \end{enumerate}
\end{theorem}
\begin{proof}
    See \cite[Theorem III.1.3]{conway-complex}
\end{proof}
Returning back to our problem, let $\lambda\in\bbC$ with $|\lambda| < 1$. Let $x\in\calH$ denote the element $(1,|\lambda|,|\lambda|^2,\dots)$. That this is indeed an element of $\calH$ is clear from the fact that the sum
\begin{equation*}
    \sum_{n\ge 0} |\lambda|^{2n} = \frac{1}{1 - |\lambda|^2}
\end{equation*}
converges. Further, let $y\in\calH$ denote the element $(|\alpha_0|,|\alpha_1|,\dots)$. That this is an element of $\calH$ follows from the fact that the sum 
\begin{equation*}
    \sum_{n\ge 0} |\alpha_n|^2
\end{equation*}
converges, since the element $(\alpha_0,\alpha_1,\dots)$ is an element of $\calH$. The Cauchy-Schwarz inequality gives us: 
\begin{equation*}
    \sum_{n\ge 0}|\alpha_n||\lambda|^n = \langle x, y\rangle\le\|x\|\|y\| < \infty.
\end{equation*}
Thus, the sum $\sum_{n\ge 0}\alpha_n\lambda^n$ converges absolutely for $|\lambda| < 1$. Due to \thref{thm:radius-of-convergence}, it follows that the radius of convergence of the power series $\sum_{n\ge 0}a_nz^n$ is $\ge 1$.

\item 
\end{enumerate}

\section{Problem 4}

Clearly the bounded linear functional 
\begin{equation*}
    \Lambda\colon y\longmapsto\left\langle y,\frac{x}{\|x\|}\right\rangle.
\end{equation*}
is such that $\Lambda x = \|x\|$ and $\|\Lambda\| = \left\|\frac{x}{\|x\|}\right\| = 1$. Suppose $\Phi: X\to\bbC$ is another bounded linear functional such that $\Phi x = \|x\|$ and $\|\Phi\| = 1$. By the Riesz Representation Theorem, there is some $z\in X$ such that $\Phi y = \langle y,z \rangle$ for all $y\in X$. Then, $\|z\| = \|\Phi\| = 1$. It follows that 
\begin{equation*}
    \langle x, z\rangle = \|x\|\implies|\langle x, z\rangle| = \|x\| = \|x\|\|z\|.
\end{equation*}
Recall that the equality holds in the Cauchy-Schwarz inequality if and only if the vectors $x$ and $z$ are linearly dependent. Since both $x$ and $z$ are non-zero, this is tantamount to saying that there is some $\alpha\in\bbC$ such that $x = \alpha z$. In conclusion, 
\begin{equation*}
    \alpha = \langle x, z\rangle = \|x\|\implies z = \frac{x}{\|x\|},
\end{equation*}
i.e., $\Phi = \Lambda$, as desired.


\section{Problem 5}

Let $x,y\in\calH$ and $\alpha\in\bbC$. According to our hypothesis, we have 
\begin{align*}
    0 &= \langle A(x + \alpha y), x + \alpha y\rangle\\
    &= \langle Ax, x\rangle + \alpha\langle Ay, x\rangle + \overline\alpha\langle Ax, y\rangle + |\alpha|^2\langle Ay, y\rangle\\
    &= \alpha\langle Ay, x\rangle + \overline\alpha\langle Ax, y\rangle.
\end{align*}
Set $\alpha = 1$ to get 
\begin{equation*}
    \langle Ay, x\rangle + \langle Ax, y\rangle = 0
\end{equation*}
and set $\alpha = \iota$ to get 
\begin{equation*}
    \iota\left(\langle Ay, x\rangle - \langle Ax, y\rangle\right) = 0\implies\langle Ay, x\rangle - \langle Ax, y\rangle = 0.
\end{equation*}
Hence, $\langle Ax, y\rangle = 0 = \langle Ay, x\rangle$ for all $x, y\in\calH$. In particular, for any $x\in\calH$, setting $y = Ax$, we get 
\begin{equation*}
    \langle Ax, Ax\rangle = \langle Ax, y\rangle = 0,
\end{equation*}
and hence $Ax = 0$, that is, $A = 0$.

\section{Problem 9}

\subsection*{Exercise 12-1 (b) of \texorpdfstring{\cite{limaye-functional}}{[Lim14]}}

Suppose first that $\K$ is either $\R$ or $\bbC$. Let $\lambda\in\sigma(A)$, we shall show that $p(\lambda)\in\sigma\left(p(A)\right)$. The polynomial $p(X) - p(\lambda)$ has a root at $\lambda$, and hence, there is a polynomial $q(X)\in\bbC[X]$ such that $p(X) - p(\lambda) = (X - \lambda)q(X)$. Now, since $A$ commutes with the identity operator $I$, we get \begin{equation*}
    p(A) - p(\lambda)I = (A - \lambda I)q(A).
\end{equation*}
Since $A - \lambda I$ is not invertible, the left hand side, $p(A) - p(\lambda)I$ is not invertible, i.e., $p(\lambda)\in\sigma\left(p(A)\right)$. This shows that 
\begin{equation*}
    p\left(\sigma(A)\right)\coloneq\left\{p(\lambda)\colon\lambda\in\sigma(A)\right\}\subseteq \sigma\left(p(A)\right)
\end{equation*}

Suppose now that $\K = \bbC$. Further suppose that $p(X)$ is a non-constant polynomial. Let $\lambda\in\sigma(p(A))$. Then, due to the fundamental theorem of algebra, the polynomial $p(X) - \lambda$ factors as 
\begin{equation*}
    p(X) - \lambda = a_n\prod_{i = 1}^n (X - \alpha_i),
\end{equation*}
for some $\alpha_i\in\bbC$. Again, since $A$ commutes with the identity operator $I$, we have 
\begin{equation*}
    p(A) - \lambda = \prod_{i = 1}^n (A - \alpha_i I).
\end{equation*}
Since the left hand side is not invertible, at least one term on the right hand side must not be invertible, that is, there is an index $1\le j\le n$ with $A - \alpha_j I$ not invertible. Thus $\alpha_j\in \sigma(A)$. It follows that $\sigma(p(A))\subseteq p(\sigma(A))$. Thus $\sigma(p(A)) = p(\sigma(A))$.

Finally, if $p$ is a constant polynomial, say $p(X) \equiv c\in\bbC$, then $\sigma(p(A)) = \{c\}$, since $(c - \lambda)I$ is invertible if and only if $c\ne\lambda$. On the other hand, since $X$ is a Banach space over $\bbC$, due to Gelfand-Mazur, the spectrum $\sigma(A)$ is non-empty, whence $p(\sigma(A)) = \{c\}$, as desired.


\subsection*{Exercise 17-5 of \texorpdfstring{\cite{limaye-functional}}{[Lim14]}}

Suppose $p(X)\in\K[X]$ is a polynomial such that $p(A)$ is a compact opeartor. If $p(X)\equiv c$ is a constant polynomial, then $p(A) = cI$, which is given to be compact. Thus, the closure of the image of the unit ball under $p(A)$ is compact, that is,
\begin{equation*}
    \overline{cB_X(0, 1)} = \overline{B_X(0, c)} 
\end{equation*}
is compact. If $c\ne 0$, using the fact that $\overline{B_X(0, c)}$ is homeomorphic to $\overline{B_X(0, 1)}$ through the homeomorphism $x\mapsto c^{-1}x$ (as we have seen in class), it follows that the closed unit ball in $X$ is compact, which is absurd, since $X$ is infinite-dimensional. Thus $c = 0$. Clearly, if $c = 0$, then the operator $p(A)\equiv 0$ is compact.

Suppose now that $p(X) = a_nX^n + \dots + a_0\in\K[X]$ is a non-constant polynomial of degree $n > 0$, whence $a_n\ne 0$. According to the hypothesis, $p(A) = a_nA^n + \dots + a_0I$ is a compact operator. Recall that the compact operators form an ideal in the $\K$-algebra $\scrB(X)$ and $A$ is a compact operator, thus the operator 
\begin{equation*}
    a_nA^n + \dots + a_1A
\end{equation*}
is a compact operator. It follows that 
\begin{equation*}
    a_0I = p(A) - \left(a_nA^n + \dots + a_1A\right) 
\end{equation*}
is a compact operator. Because of what we just proved, we must have that $a_0 = 0$, that is, $p(0) = 0$.

Conversely, suppose $p(X)\in\K[X]$ is a polynomial such that $p(0) = 0$. If $p(X)$ is a constant polynomial, then it is identically zero, whence is trivially compact. Suppose now that $p(X)$ is non-constant. Then, we can write 
\begin{equation*}
    p(X) = a_nX^n + \dots + a_1X
\end{equation*}
for $n > 0$ and $a_i\in\K$ with $a_n\ne 0$ where $n$ is the degree of the polynomial $p(X)$. Thus $p(A) = a_nA^n + \dots + a_1A$. As we argued earlier, since the compact operators form an ideal in $\scrB(X)$ and $A$ is a compact operator, it is clear that $p(A)$ is a compact operator, as desired.

\bibliographystyle{alpha}
\bibliography{references}
\end{document}