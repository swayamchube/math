\documentclass[10pt]{amsart}

\title{MA 824: Assignment 2}
\author{Swayam Chube (200050141)}
\date{\today}

\usepackage[utf8]{inputenc} % allow utf-8 input
\usepackage[T1]{fontenc}    % use 8-bit T1 fonts
\usepackage{hyperref}       % hyperlinks
\usepackage{url}            % simple URL typesetting
\usepackage{booktabs}       % professional-quality tables
\usepackage{amsfonts}       % blackboard math symbols
\usepackage{nicefrac}       % compact symbols for 1/2, etc.
\usepackage{microtype}      % microtypography
\usepackage{lmodern}
\usepackage{graphicx}
\usepackage{natbib}
\usepackage{doi}
\usepackage{amssymb}
\usepackage{bbm}
\usepackage{amsthm}
\usepackage{amsmath}
\usepackage{xcolor}
\usepackage{theoremref}
\usepackage{enumitem}
% \usepackage{mathpazo}
% \usepackage{mlmodern}
\usepackage{mathrsfs}
\usepackage{todonotes}
\usepackage{stmaryrd}
\usepackage[all,cmtip]{xy} % For diagrams, praise the Freyd–Mitchell theorem 
\usepackage{marvosym}
\usepackage{geometry}
\usepackage{bbm}
\usepackage{mathtools}
\usepackage{fouriernc}

\renewcommand{\qedsymbol}{$\blacksquare$}

% Uncomment to override  the `A preprint' in the header
% \renewcommand{\headeright}{}
% \renewcommand{\undertitle}{}
% \renewcommand{\shorttitle}{}

\hypersetup{
    pdfauthor={Lots of People},
    colorlinks=true,
	citecolor=blue
}

\newtheoremstyle{thmstyle}%               % Name
  {}%                                     % Space above
  {}%                                     % Space below
  {}%                             % Body font
  {}%                                     % Indent amount
  {\bfseries\scshape}%                            % Theorem head font
  {.}%                                    % Punctuation after theorem head
  { }%                                    % Space after theorem head, ' ', or \newline
  {\thmname{#1}\thmnumber{ #2}\thmnote{ (#3)}}%                                     % Theorem head spec (can be left empty, meaning `normal')

\newtheoremstyle{defstyle}%               % Name
  {}%                                     % Space above
  {}%                                     % Space below
  {}%                                     % Body font
  {}%                                     % Indent amount
  {\bfseries\scshape}%                            % Theorem head font
  {.}%                                    % Punctuation after theorem head
  { }%                                    % Space after theorem head, ' ', or \newline
  {\thmname{#1}\thmnumber{ #2}\thmnote{ (#3)}}%                                     % Theorem head spec (can be left empty, meaning `normal')

\theoremstyle{thmstyle}
\newtheorem{theorem}{Theorem}[section]
\newtheorem{lemma}[theorem]{Lemma}
\newtheorem{proposition}[theorem]{Proposition}
\newtheorem{porism}[theorem]{Porism}
\newtheorem*{claim}{Claim}

\theoremstyle{defstyle}
\newtheorem{definition}[theorem]{Definition}
\newtheorem*{notation}{Notation}
\newtheorem*{corollary}{Corollary}
\newtheorem{remark}[theorem]{Remark}
\newtheorem{example}[theorem]{Example}

% Common Algebraic Structures
\newcommand{\R}{\mathbb{R}}
\newcommand{\Q}{\mathbb{Q}}
\newcommand{\Z}{\mathbb{Z}}
\newcommand{\N}{\mathbb{N}}
\newcommand{\bbC}{\mathbb{C}}
\newcommand{\K}{\mathbb{K}}
\newcommand{\calA}{\mathcal{A}}
\newcommand{\calH}{\mathcal{H}} % Hilbert space
\newcommand{\frakM}{\mathfrak{M}}
\newcommand{\calO}{\mathcal{O}}
\newcommand{\bbA}{\mathbb{A}}
\newcommand{\bbI}{\mathbb{I}}

% Categories
\newcommand{\catTopp}{\mathbf{Top}_*}
\newcommand{\catGrp}{\mathbf{Grp}}
\newcommand{\catTopGrp}{\mathbf{TopGrp}}
\newcommand{\catSet}{\mathbf{Set}}
\newcommand{\catTop}{\mathbf{Top}}
\newcommand{\catRing}{\mathbf{Ring}}
\newcommand{\catCRing}{\mathbf{CRing}} % comm. rings
\newcommand{\catMod}{\mathbf{Mod}}
\newcommand{\catMon}{\mathbf{Mon}}
\newcommand{\catMan}{\mathbf{Man}} % manifolds
\newcommand{\catDiff}{\mathbf{Diff}} % smooth manifolds
\newcommand{\catAlg}{\mathbf{Alg}}
\newcommand{\catRep}{\mathbf{Rep}} % representations 
\newcommand{\catVec}{\mathbf{Vec}}

% Group and Representation Theory
\newcommand{\chr}{\operatorname{char}}
\newcommand{\Aut}{\operatorname{Aut}}
\newcommand{\GL}{\operatorname{GL}}
\newcommand{\im}{\operatorname{im}}
\newcommand{\tr}{\operatorname{tr}}
\newcommand{\id}{\mathbf{id}}
\newcommand{\cl}{\mathbf{cl}}
\newcommand{\Gal}{\operatorname{Gal}}
\newcommand{\Tr}{\operatorname{Tr}}
\newcommand{\sgn}{\operatorname{sgn}}
\newcommand{\Sym}{\operatorname{Sym}}
\newcommand{\Alt}{\operatorname{Alt}}

% Commutative and Homological Algebra
\newcommand{\spec}{\operatorname{spec}}
\newcommand{\mspec}{\operatorname{m-spec}}
\newcommand{\Tor}{\operatorname{Tor}}
\newcommand{\tor}{\operatorname{tor}}
\newcommand{\Ann}{\operatorname{Ann}}
\newcommand{\Supp}{\operatorname{Supp}}
\newcommand{\Hom}{\operatorname{Hom}}
\newcommand{\End}{\operatorname{End}}
\newcommand{\coker}{\operatorname{coker}}
\newcommand{\limit}{\varprojlim}
\newcommand{\colimit}{%
  \mathop{\mathpalette\colimit@{\rightarrowfill@\textstyle}}\nmlimits@
}
\makeatother


\newcommand{\fraka}{\mathfrak{a}} % ideal
\newcommand{\frakb}{\mathfrak{b}} % ideal
\newcommand{\frakc}{\mathfrak{c}} % ideal
\newcommand{\frakf}{\mathfrak{f}} % face map
\newcommand{\frakg}{\mathfrak{g}}
\newcommand{\frakh}{\mathfrak{h}}
\newcommand{\frakm}{\mathfrak{m}} % maximal ideal
\newcommand{\frakn}{\mathfrak{n}} % naximal ideal
\newcommand{\frakp}{\mathfrak{p}} % prime ideal
\newcommand{\frakq}{\mathfrak{q}} % qrime ideal
\newcommand{\fraks}{\mathfrak{s}}
\newcommand{\frakt}{\mathfrak{t}}
\newcommand{\frakz}{\mathfrak{z}}
\newcommand{\frakA}{\mathfrak{A}}
\newcommand{\frakF}{\mathfrak{F}}
\newcommand{\frakI}{\mathfrak{I}}
\newcommand{\frakK}{\mathfrak{K}}
\newcommand{\frakL}{\mathfrak{L}}
\newcommand{\frakN}{\mathfrak{N}} % nilradical 
\newcommand{\frakP}{\mathfrak{P}} % nilradical 
\newcommand{\frakR}{\mathfrak{R}} % jacobson radical
\newcommand{\frakT}{\mathfrak{T}} % tensor algebra
\newcommand{\frakU}{\mathfrak{U}}
\newcommand{\frakX}{\mathfrak{X}}

% General/Differential/Algebraic Topology 
\newcommand{\scrA}{\mathscr A}
\newcommand{\scrB}{\mathscr B}
\newcommand{\scrF}{\mathscr F}
\newcommand{\scrP}{\mathscr P}
\newcommand{\scrS}{\mathscr S}
\newcommand{\bbH}{\mathbb H}
\newcommand{\Int}{\operatorname{Int}}
\newcommand{\psimeq}{\simeq_p}
\newcommand{\wt}[1]{\widetilde{#1}}
\newcommand{\RP}{\mathbb{R}\text{P}}
\newcommand{\CP}{\mathbb{C}\text{P}}

% Miscellaneous
\newcommand{\wh}[1]{\widehat{#1}}
\newcommand{\calM}{\mathcal{M}}
\newcommand{\calN}{\mathcal{N}}
\newcommand{\calP}{\mathcal{P}}
\newcommand{\onto}{\twoheadrightarrow}
\newcommand{\into}{\hookrightarrow}
\newcommand{\Gr}{\operatorname{Gr}}
\newcommand{\Span}{\operatorname{Span}}
\newcommand{\ev}{\operatorname{ev}}
\newcommand{\weakto}{\stackrel{w}{\longrightarrow}}

\newcommand{\define}[1]{\textcolor{blue}{\textit{#1}}}
% \newcommand{\caution}[1]{\textcolor{red}{\textit{#1}}}
\newcommand{\important}[1]{\textcolor{red}{#1}}
\renewcommand{\mod}{~\mathrm{mod}~}
\renewcommand{\le}{\leqslant}
\renewcommand{\leq}{\leqslant}
\renewcommand{\ge}{\geqslant}
\renewcommand{\geq}{\geqslant}
\newcommand{\Res}{\operatorname{Res}}
\newcommand{\floor}[1]{\left\lfloor #1\right\rfloor}
\newcommand{\ceil}[1]{\left\lceil #1\right\rceil}
\newcommand{\gl}{\mathfrak{gl}}
\newcommand{\ad}{\operatorname{ad}}
\newcommand{\ind}{\operatorname{ind}}
\newcommand{\sminus}{\setminus}
\newcommand{\Sd}{\operatorname{Sd}}
\newcommand{\mesh}{\operatorname{mesh}}
\newcommand{\diam}{\operatorname{diam}}
\newcommand{\co}{\operatorname{co}}
\newcommand{\Lip}{\operatorname{Lip}}
\newcommand{\lip}{\operatorname{lip}}
\newcommand{\dist}{\operatorname{dist}}
\newcommand{\pv}{\operatorname{p.v.}}
\newcommand{\Graph}{\operatorname{Graph}}

\geometry {
    margin = 1in
}

\begin{document}
\maketitle 

Throughout this document, $\K$ denotes either $\R$ or $\bbC$. Most proofs should work over either of the two fields unless specified otherwise.

\section{Problem 1}
\begin{enumerate}[label=(\alph*)]
\item Recall the following result from complex analysis: 
\begin{theorem}\thlabel{thm:radius-of-convergence}
    Let $\displaystyle\sum_{n\ge 0} a_n(z - a)^n$ be a power series with $a_n\in\bbC$. The radius of convergence of the above power series is the unique real number $R\ge 0$ such that
    \begin{enumerate}[label=(\roman*)]
        \item if $|z - a| < R$, the series converges absolutely. 
        \item if $|z - a| > R$, the terms of the series become unbounded and so the series diverges.
    \end{enumerate}
\end{theorem}
\begin{proof}
    See \cite[Theorem III.1.3]{conway-complex}
\end{proof}
Returning back to our problem, let $\lambda\in\bbC$ with $|\lambda| < 1$. Let $x\in\calH$ denote the element $(1,|\lambda|,|\lambda|^2,\dots)$. That this is indeed an element of $\calH$ is clear from the fact that the sum
\begin{equation*}
    \sum_{n\ge 0} |\lambda|^{2n} = \frac{1}{1 - |\lambda|^2}
\end{equation*}
converges. Further, let $y\in\calH$ denote the element $(|\alpha_0|,|\alpha_1|,\dots)$. That this is an element of $\calH$ follows from the fact that the sum 
\begin{equation*}
    \sum_{n\ge 0} |\alpha_n|^2
\end{equation*}
converges, since the element $(\alpha_0,\alpha_1,\dots)$ is an element of $\calH$. The Cauchy-Schwarz inequality gives us: 
\begin{equation*}
    \sum_{n\ge 0}|\alpha_n||\lambda|^n = \langle x, y\rangle\le\|x\|\|y\| < \infty.
\end{equation*}
Thus, the sum $\sum_{n\ge 0}\alpha_n\lambda^n$ converges absolutely for $|\lambda| < 1$. Due to \thref{thm:radius-of-convergence}, it follows that the radius of convergence of the power series $\sum_{n\ge 0}a_nz^n$ is $\ge 1$.

\item Let $h_0$ denote the element $(1, \overline\lambda,\overline\lambda^2,\dots)\in\calH$. To see that this is indeed an element of $\calH$, note that 
\begin{equation*}
    \|h\|^2 = \sum_{n\ge 0}|\overline\lambda|^{2n} = \sum_{n\ge 0}|\lambda|^{2n} = \frac{1}{1 - |\lambda|^2} < \infty.
\end{equation*}
Then, for any $h = (\alpha_0,\alpha_1,\dots)\in\calH$, we have, by definition, 
\begin{equation*}
    \langle h, h_0\rangle = \sum_{n\ge 0}\alpha_n\overline{\overline\lambda^n} = \sum_{n\ge 0}\alpha_n\lambda^n = L(h),
\end{equation*}
as desired. 

\item The following fact is well-known and has been proved in class: 
\begin{lemma}
    Let $\calH$ be a complex Hilbert space and $h_0\in\calH$. Let $\Lambda:\calH\to\bbC$ be the bounded linear functional given by $x\mapsto\langle x, h_0\rangle$. Then $\|\Lambda\| = \|h_0\|$.
\end{lemma}

Thus, 
\begin{equation*}
    \|L\| = \|h_0\| = \left(\sum_{n\ge 0}|\overline\lambda^n|^2\right)^{\frac{1}{2}} = \left(\sum_{n\ge 0}|\lambda|^{2n}\right)^{\frac{1}{2}} = \frac{1}{\sqrt{1 - |\lambda|^2}}.
\end{equation*}
\end{enumerate}

\section{Problem 2}

Let $\calH = H^2(\mathbb D)$. First, we argue that $\langle\cdot,\cdot\rangle$ is an inner product on $\calH$. Indeed, for $\alpha,\beta\in\bbC$ and $f,g, h\in\calH$, we have that 
\begin{align*}
    \langle\alpha f + \beta g, h\rangle &= \lim_{r\to 1}\int_{0}^{2\pi}(\alpha f + \beta g)(re^{it})\overline{h(re^{it})}~\frac{dt}{2\pi}\\
    &= \lim_{r\to 1}\alpha\int_{0}^{2\pi}f(re^{it})\overline{h(re^{it})}~\frac{dt}{2\pi} + \beta\int_{0}^{2\pi}f(re^{it})\overline{h(re^{it})}~\frac{dt}{2\pi}\\
    &=\alpha\langle f, h\rangle + \beta\langle g, h\rangle,
\end{align*}
and 
\begin{equation*}
    \langle g, f\rangle = \lim_{r\to 1}\int_{0}^{2\pi}\overline{f(re^{it})}g(re^{it})~\frac{dt}{2\pi} = \lim_{r\to 1}\overline{\int_{0}^{2\pi}f(re^{it})\overline{g(re^{it})}~\frac{dt}{2\pi}} = \overline{\langle f, g\rangle}.
\end{equation*}
Suppose $f\in\calH$ is such that $\langle f, f\rangle = 0$. That is,
\begin{equation*}
    \lim_{r\to 1}\int_0^{2\pi}|f(re^{it})|^2~\frac{dt}{2\pi} = 0.
\end{equation*}
Recall that $f(z)$ has an absolutely convergent power series expansion given by 
\begin{equation*}
    f(z) = \sum_{n\ge 0}a_n(f)z^n.
\end{equation*}
As a result, 
\begin{equation*}
    |f(re^{it})|^2 = f(re^{it})\overline{f(re^{it})} = \sum_{m\ge 0} a_m(f)r^me^{imt}\sum_{n\ge 0}\overline{a_n(f)}r^ne^{-int}.
\end{equation*}
Consequently, 
\begin{equation*}
    \int_{0}^{2\pi}|f(re^{it})|^2~\frac{dt}{2\pi} = \sum_{m,n\ge 0}\int_0^{2\pi}a_m(f)\overline{a_n(f)}r^{m + n}e^{i(m - n)t}~\frac{dt}{2\pi}.
\end{equation*}
Using the fact that for $n\in\Z$, 
\begin{equation*}
    \int_0^{2\pi}e^{int}~\frac{dt}{2\pi} = 
    \begin{cases}
        1 & n = 0\\
        0 & \text{otherwise},
    \end{cases}
\end{equation*}
we have 
\begin{equation*}
    \int_0^{2\pi}|f(re^{it})|^2~\frac{dt}{2\pi} = \sum_{n\ge 0}|a_n(f)|^2r^{2n},
\end{equation*}
whence 
\begin{equation*}
    0 = \langle f, f\rangle = \lim_{r\to 1}\sum_{n\ge 0}|a_n(f)|^2r^{2n} = \sum_{n\ge 0}|a_n(f)|^2.
\end{equation*}
This is possible if and only if $a_n(f) = 0$ for all $n\ge 0$, that is, $f\equiv 0$ on $\mathbb D$. Thus, $\langle\cdot,\cdot\rangle$ is an inner-product on $\calH$. Let $\|\cdot\|$ denote the induced norm. We also make an important observation, that because of the expression obtained above, we can write 
\begin{equation}
    \|f\|^2 = \lim_{r\to 1}\sum_{n\ge 0}|a_n(f)|^2r^{2n} = \sup_{0 < r < 1}\sum_{n\ge 0} |a_n(f)|^2r^{2n} = \sup_{0 < r < 1}\int_{0}^{2\pi}|f(re^{it})|^2~\frac{dt}{2\pi}.\tag{$\dagger$}\label{alternate-definition}
\end{equation}

Next, we show that $\calH$ is a Hilbert space. Let $(f_n)_{n\ge 1}$ be a Cauchy sequence. We claim that $f_n$ converges uniformly on compacta to a holomorphic function $f$ on $\mathbb D$. To this end, it suffices to show that $(f_n)_{n\ge 1}$ is uniformly Cauchy on compact subsets of $\mathbb D$. Since every compact subset is contained in some closed ball $\overline{B(0, r)}$ for some $r > 0$, it suffices to show that $(f_n)$ is uniformly Cauchy over these closed balls. 

Let $0 < r < R < 1$ and $z\in\overline{B(0, r)}$. For $m,n\in\N$, using Cauchy's integral formula, 
\begin{equation*}
    |f_m(z) - f_n(z)| = \left|\frac{1}{2\pi i}\int_0^{2\pi}\frac{(f_m - f_n)(Re^{it})}{Re^{it} - z}~dt\right|\le\int_0^{2\pi}\frac{|f_m(Re^{it}) - f_n(Re^{it})|}{R - r}~\frac{dt}{2\pi}.
\end{equation*}
Using the Cauchy-Schwarz inequality, we have 
\begin{equation*}
    \left(\int_0^{2\pi}|(f_m - f_n)(Re^{it})|~\frac{dt}{2\pi}\right)^{2}\le\int_0^{2\pi}|(f_m - f_n)(Re^{it})|^2~\frac{dt}{2\pi}.
\end{equation*}
Thus 
\begin{equation*}
    (R - r)|f_m(z) - f_n(z)|\le\left(\int_0^{2\pi}|(f_m - f_n)(Re^{it})|^2~\frac{dt}{2\pi}\right)^{\frac{1}{2}}.
\end{equation*}
Taking the limit $R\to 1^-$, we have 
\begin{equation*}
    (1 - r)|f_m(z) - f_n(z)|\le\|f_m - f_n\|\implies\|f_m(z) - f_n(z)|\le\frac{1}{1 - r}\|f_m - f_n\|.
\end{equation*}
If $\varepsilon > 0$, then there is a positive integer $N > 0$ such that for all $m,n\ge N$, $\|f_m - f_n\| < (1 - r)\varepsilon$, and hence, $|f_m(z) - f_n(z)| < \varepsilon$ for all $z\in\overline{B(0, r)}$ and $m,n\ge N$. This shows that $(f_n)$ is uniformly Cauchy on compact subsets of $\mathbb D$, and as such, it converges uniformly on compact subsets to a holomorphic function $f:\mathbb D\to\mathbb C$.

Let $0 < r < 1$. Since $f_n\to f$ uniformly on the compact subset $|z| = r$, we have 
\begin{equation*}
    \int_0^{2\pi} |f(re^{it}) - f_n(re^{it})|^2~\frac{dt}{2\pi} = \lim_{m\to\infty}\int_0^{2\pi}|f_m(re^{it}) - f_n(re^{it})|^2~\frac{dt}{2\pi}.
\end{equation*}
Let $\varepsilon > 0$, then there is a positive integer $N > 0$ such that for all $m,n\ge 0$, $\|f_n - f_m\| < \varepsilon$. Then, for every $0 < r < 1$, we have 
\begin{equation*}
    \int_0^{2\pi} |f(re^{it}) - f_n(re^{it})|^2~\frac{dt}{2\pi} = \lim_{m\to\infty}\int_0^{2\pi}|f_m(re^{it}) - f_n(re^{it})|^2~\frac{dt}{2\pi}\le\varepsilon^2,
\end{equation*}
where the inequality follows from \eqref{alternate-definition}. Taking the limit $r\to 1$ (or even taking the supremum), we have 
\begin{equation*}
    \|f - f_n\|^2\le\varepsilon^2\implies\|f - f_n\|\le\varepsilon,
\end{equation*}
for all $n\ge N$. This shows that $\calH$ is a Hilbert space. Finally, using the polarization identity, we have 
\begin{align*}
    \langle f, g\rangle &= \frac{1}{4}\sum_{k = 0}^3\iota^k\|f + \iota^k g\|^2 \\
    &= \frac{1}{4}\sum_{k = 0}^3\iota^k\sum_{n\ge 0}|a_n(f) + \iota^k a_n(g)|^2 \\
    &= \sum_{n\ge 0}\frac{1}{4}\sum_{k = 0}^3|a_n(f) + \iota^k a_n(g)|^2 \\
    &= \sum_{n\ge 0} a_n(f)\overline{a_n(g)} \\
    &= \left\langle\left(a_n(f)\right)_{n\ge 0}, \left(a_n(g)\right)_{n\ge 0}\right\rangle,
\end{align*}
as desired.

As a final comment, it is worth noting that we must also argue the ``existence'' of the expression for the inner product, that is, the fact that the limit in the definition of the inner product indeed exists. This is clear by first showing that the expression for $\langle f, f\rangle$ exists because is it is precisely equal to 
\begin{equation*}
    \lim_{r\to 1}\sum_{n\ge 0} |a_n(f)|^2r^{2n} = \sum_{n\ge 0}|a_n(f)|^2,
\end{equation*}
and then using the polarization identity to show that the corresponding limit in the definition of $\langle f, g\rangle$ exists.

\section{Problem 3}

Using the given definition, we have 
\begin{equation*}
    \|(u_1, u_2)\|^2 = \langle (u_1, u_2), (u_1, u_2)\rangle = \|u_1\|^2 + \|u_2\|^2.
\end{equation*}
Clearly $A: \calH\to\calH$ is $\bbC$-linear, since for $\alpha,\beta\in\bbC$,
\begin{equation*}
    A\left(\alpha (u_1, u_2) + \beta(v_1, v_2)\right) = A\left(\alpha u_1 + \beta v_1, \alpha u_2 + \beta v_2\right) = A_1(\alpha u_1 + \beta v_1) + A_2(\alpha u_2 + \beta v_2) = \alpha A(u_1, u_2) + \beta A(v_1, v_2).
\end{equation*}
Now, for $(u_1, u_2)\in\calH$, we have 
\begin{align*}
    \|A(u_1, u_2)\| &= \|A_1u_1 + A_2u_2\|\\ 
    &\le\left(\|A_1u_1\|^2 + \|A_2u_2\|^2\right)^{\frac{1}{2}}\\
    &\le\left(\|A_1\|^2\|u_1\|^2 + \|A_2\|^2\|u_2\|^2\right)^{\frac{1}{2}}\\
    &\le\max\left\{\|A_1\|, \|A_2\|\right\}\left(\|u_1\|^2 + \|u_2\|^2\right)^{\frac{1}{2}}\\
    &= \max\left\{\|A_1\|, \|A_2\|\right\}\|(u_1, u_2)\|.
\end{align*}
As a result, $A$ is a bounded linear operator from $\calH$ to $\calH$ with 
\begin{equation*}
    \|A\|\le\max\left\{\|A_1\|, \|A_2\|\right\}.
\end{equation*}
We shall show that equality holds. Suppose, without loss of generality, that $\|A_1\|\ge\|A_2\|$. Then 
\begin{equation*}
    \|A\|\coloneq\sup_{\|(u_1, u_2)\| = 1}{\|A(u_1, u_2)\|}\ge\sup_{\|u_1\| = 1}\|A(u_1, 0)\| = \sup_{\|u_1\| = 1} \|A_1u_1\| = \|A_1\| = \max\left\{\|A_1\|, \|A_2\|\right\}.
\end{equation*}
This shows that $\|A\| = \max\left\{\|A_1\|, \|A_2\|\right\}$, as desired.


\section{Problem 4}

Clearly the bounded linear functional 
\begin{equation*}
    \Lambda\colon y\longmapsto\left\langle y,\frac{x}{\|x\|}\right\rangle.
\end{equation*}
is such that $\Lambda x = \|x\|$ and $\|\Lambda\| = \left\|\frac{x}{\|x\|}\right\| = 1$. Suppose $\Phi: X\to\bbC$ is another bounded linear functional such that $\Phi x = \|x\|$ and $\|\Phi\| = 1$. By the Riesz Representation Theorem, there is some $z\in X$ such that $\Phi y = \langle y,z \rangle$ for all $y\in X$. Then, $\|z\| = \|\Phi\| = 1$. It follows that 
\begin{equation*}
    \langle x, z\rangle = \|x\|\implies|\langle x, z\rangle| = \|x\| = \|x\|\|z\|.
\end{equation*}
Recall that the equality holds in the Cauchy-Schwarz inequality if and only if the vectors $x$ and $z$ are linearly dependent. Since both $x$ and $z$ are non-zero, this is tantamount to saying that there is some $\alpha\in\bbC$ such that $x = \alpha z$. In conclusion, 
\begin{equation*}
    \alpha = \langle x, z\rangle = \|x\|\implies z = \frac{x}{\|x\|},
\end{equation*}
i.e., $\Phi = \Lambda$, as desired.


\section{Problem 5}

Let $x,y\in\calH$ and $\alpha\in\bbC$. According to our hypothesis, we have 
\begin{align*}
    0 &= \langle A(x + \alpha y), x + \alpha y\rangle\\
    &= \langle Ax, x\rangle + \alpha\langle Ay, x\rangle + \overline\alpha\langle Ax, y\rangle + |\alpha|^2\langle Ay, y\rangle\\
    &= \alpha\langle Ay, x\rangle + \overline\alpha\langle Ax, y\rangle.
\end{align*}
Set $\alpha = 1$ to get 
\begin{equation*}
    \langle Ay, x\rangle + \langle Ax, y\rangle = 0
\end{equation*}
and set $\alpha = \iota$ to get 
\begin{equation*}
    \iota\left(\langle Ay, x\rangle - \langle Ax, y\rangle\right) = 0\implies\langle Ay, x\rangle - \langle Ax, y\rangle = 0.
\end{equation*}
Hence, $\langle Ax, y\rangle = 0 = \langle Ay, x\rangle$ for all $x, y\in\calH$. In particular, for any $x\in\calH$, setting $y = Ax$, we get 
\begin{equation*}
    \langle Ax, Ax\rangle = \langle Ax, y\rangle = 0,
\end{equation*}
and hence $Ax = 0$, that is, $A = 0$.

\section{Problem 6}

I assume the problem wanted to say $\|I - T\| < 1$. Set $S = 1 - T$, so that $\|S\| < 1$. 

First, we show that the sum $1 + S + S^2 + \dots$ converges in $\scrB(\calH)$. Indeed, let $S_n\coloneq 1 + S + \dots + S^n$. Then, for $m < n$, 
\begin{equation*}
    \|S_n - S_m\| = \left\|\sum_{k = n + 1}^m S^k\right\|\le\sum_{k = n + 1}^m \|S\|^k\le\sum_{k = n + 1}^\infty\|S\|^k = \frac{\|S\|^{n + 1}}{1 - \|S\|}.
\end{equation*}
Since $\|S\| < 1$, it follows that if $m,n\ge N$, then 
\begin{equation*}
    \|S_m - S_n\|\le\frac{\|S\|^{N + 1}}{1 - \|S\|}\to 0
\end{equation*}
as $N\to\infty$, i.e., the sequence $(S_n)_{n\ge 1}$ is Cauchy in $\scrB(\calH)$. But since $\scrB(\calH)$ is Cauchy, there is an operator $U\in\scrB(\calH)$ such that $S_n\to U$ as $n\to\infty$.


Then, for $n\in\N$, we have 
\begin{equation*}
    (I - S)S_n = (I - S)(I + S + S^2 + \dots + S^n) = I - S^{n + 1},
\end{equation*}
and hence, 
\begin{equation*}
    \|I - (I - S)S_n\| = \|I - (I - S)(I + S + S^2 + \dots + S^n)\| = \|S^{n + 1}\|\le \|S\|^{n + 1}\to 0
\end{equation*}
as $n\to\infty$. Thus $(I - S)S_n\to I$ as $n\to\infty$. Finally, since 
\begin{equation*}
    \|(I - S)(U - S_n)\|\le\|I - S\|\|U - S_n\|\to 0
\end{equation*}
as $n\to\infty$, we see that $(I - S)S_n\to (I - S)U$ as $n\to\infty$. This shows that $(I - S)U = I$. Using the same argument with $S_n(I - S)$ in place of $(I - S)S_n$, one can show that $U(I - S) = I$. 

In conclusion, we have shown that $T = I - S$ is invertible with inverse 
\begin{equation*}
    T^{-1} = (I - S)^{-1} = \sum_{n\ge 0} S^n = \sum_{n\ge 0}(I - T)^n,
\end{equation*}
as desired.

\section{Problem 7}

\subsection*{Part (a)} 

$(a)\implies(b)$: For any $x\in\calH$, $Qx\in\calN\subseteq\calM$, and since $P$ acts as the identity operator on $\calM$, it follows that $P(Qx) = Qx$ for all $x\in\calH$, i.e., $PQ = Q$.

$(b)\implies(c)$: Let $x\in\calM^\perp$. We shall show that $x\in\calN^\perp$. Indeed, for any $y\in\calN$, we have $Qy = y$, and hence, 
\begin{equation*}
    \langle x, y\rangle = \langle x, Qy\rangle = \langle x, PQy\rangle = \langle Px, y\rangle.
\end{equation*}
But since $x\in\calM^\perp$ and $P = P_{\calM}$, we have $Px = 0$, so that $\langle x, y\rangle = 0$. Thus $x\in\calN^\perp$.

$(c)\implies(a)$: Clearly, 
\begin{equation*}
    \calN = \calN^{\perp\perp} = \left\{x\in\calH\colon \langle x, y\rangle = 0~\forall~y\in\calN^\perp\right\}\subseteq\left\{x\in\calH\colon \langle x, y\rangle = 0~\forall~y\in\calM^\perp\right\} = \calM^{\perp\perp} = \calM.
\end{equation*}
This establishes the equivalence of the first three assertions.

$(c)\implies(d)$: Note that 
\begin{equation*}
    1 - P = 1 - P_{\calM} = P_{\calM^\perp}\quad\text{ and }\quad 1 - Q = 1 - P_{\calN} = P_{\calN^\perp}.
\end{equation*}
Since $\calM^\perp\subseteq\calN^\perp$, using the same argument as in the proof of $(a)\implies(b)$, it follows that 
\begin{equation*}
    (1 - Q)(1 - P) = P_{\calN^\perp}P_{\calM^\perp} = P_{\calM^\perp} = 1 - P.
\end{equation*}

$(d)\iff(e)$: We clearly have 
\begin{equation*}
    1 - P = (1 - Q)(1 - P) = 1 - Q - P + QP \iff Q = QP.
\end{equation*}

$(d)\implies(a)$: Note that (d) is the same as writing 
\begin{equation*}
    P_{\calN^\perp}P_{\calM^\perp} = P_{\calM^\perp}.
\end{equation*}
Using the same method of proof as $(b)\implies(c)$, it follows that 
\begin{equation*}
    \calN = \left(\calN^\perp\right)^\perp\subseteq\left(\calM^\perp\right)^\perp = \calM,
\end{equation*}
as desired. This establishes the equivalence of all five assertions. 

\subsection{Part (b)}

Due to part (a), we have the equivalences
\begin{equation*}
    \calN\subseteq\calM^\perp\iff P_{\calM^\perp}P_{\calN} = P_{\calN}\iff P_{\calN}P_{\calM^\perp} = P_{\calN}.
\end{equation*}
Next, using the fact that $P_{\calM^\perp} = 1 - P$ and $P_{\calN} = Q$, one sees that 
\begin{equation*}
    P_{\calM^\perp}P_{\calN} = P_{\calN}\iff (1 - P)Q = Q\iff PQ = 0
\end{equation*}
and 
\begin{equation*}
    P_{\calN}P_{\calM^\perp} = P_{\calN}\iff Q(1 - P) = Q\iff QP = 0.
\end{equation*}
This gives the desired equivalence of the three assertions.

\section{Problem 8}

Let $R = P + Q$. Then 
\begin{equation*}
    R^2 = (P + Q)(P + Q) = P^2 + PQ + QP + Q^2 = P + Q = R,
\end{equation*}
since $P^2 = P$, $Q^2 = Q$, $PQ = 0$, and $QP = 0$. Further, 
\begin{equation*}
    R^\ast = P^\ast + Q^\ast = P + Q,
\end{equation*}
since $P^\ast = P$ and $Q^\ast = Q$. We have seen in class that these two conditions are equivalent to stating that $R$ is a projection operator. We contend that the range of $R$ is precisely $\calM + \calN$. Indeed, for any $x\in\calH$, 
\begin{equation*}
    Rx = Px + Qx\in\calM + \calN,
\end{equation*}
since $Px\in\calM$ and $Qx\in\calN$. On the other hand, any element of $\calM + \calN$ looks like $z = x + y$ where $x\in\calM$ and $y\in\calN$. So 
\begin{equation*}
    Rz = (P + Q)(x + y) = Px + Py + Qx + Qy.
\end{equation*}
Now, since $\calM\perp\calN$, $y\in\calM^\perp$ and $x\in\calN^\perp$, so that $Py = 0 = Qx$, and thus 
\begin{equation*}
    Rz = Px + Qy = x + y = z
\end{equation*}
It follows that $\calM + \calN$ is precisely equal to the range of $R$. That is, $R = P_{\calM + \calN}$. This proves (b), and in turn gives that $\calM + \calN$ is a closed subspace of $\calH$.

Finally, note that $\calM,\calN\subseteq\calM + \calN$, so that 
\begin{equation*}
    \Span\left(\calM\cup\calN\right)\subseteq\calM + \calN\implies\overline{\Span}\left(\calM\cup\calN\right)\subseteq\calM + \calN.
\end{equation*}
where the last containment follows from the fact that $\calM + \calN$ is a closed subspace of $\calH$. Conversely, it is clear from elementary linear algebra that $\calM + \calN\subseteq\Span\left(\calM\cup\calN\right)$, whence it follows that 
\begin{equation*}
    \overline{\Span}\left(\calM\cup\calN\right) = \calM + \calN,
\end{equation*}
thereby completing the proof.

\section{Problem 9}

\subsection*{Exercise 12-1 (b) of \texorpdfstring{\cite{limaye-functional}}{[Lim14]}}

Suppose first that $\K$ is either $\R$ or $\bbC$. Let $\lambda\in\sigma(A)$, we shall show that $p(\lambda)\in\sigma\left(p(A)\right)$. The polynomial $p(X) - p(\lambda)$ has a root at $\lambda$, and hence, there is a polynomial $q(X)\in\bbC[X]$ such that $p(X) - p(\lambda) = (X - \lambda)q(X)$. Now, since $A$ commutes with the identity operator $I$, we get \begin{equation*}
    p(A) - p(\lambda)I = (A - \lambda I)q(A).
\end{equation*}
Since $A - \lambda I$ is not invertible, the left hand side, $p(A) - p(\lambda)I$ is not invertible, i.e., $p(\lambda)\in\sigma\left(p(A)\right)$. This shows that 
\begin{equation*}
    p\left(\sigma(A)\right)\coloneq\left\{p(\lambda)\colon\lambda\in\sigma(A)\right\}\subseteq \sigma\left(p(A)\right)
\end{equation*}

Suppose now that $\K = \bbC$. Further suppose that $p(X)$ is a non-constant polynomial. Let $\lambda\in\sigma(p(A))$. Then, due to the fundamental theorem of algebra, the polynomial $p(X) - \lambda$ factors as 
\begin{equation*}
    p(X) - \lambda = a_n\prod_{i = 1}^n (X - \alpha_i),
\end{equation*}
for some $\alpha_i\in\bbC$. Again, since $A$ commutes with the identity operator $I$, we have 
\begin{equation*}
    p(A) - \lambda = \prod_{i = 1}^n (A - \alpha_i I).
\end{equation*}
Since the left hand side is not invertible, at least one term on the right hand side must not be invertible, that is, there is an index $1\le j\le n$ with $A - \alpha_j I$ not invertible. Thus $\alpha_j\in \sigma(A)$. It follows that $\sigma(p(A))\subseteq p(\sigma(A))$. Thus $\sigma(p(A)) = p(\sigma(A))$.

Finally, if $p$ is a constant polynomial, say $p(X) \equiv c\in\bbC$, then $\sigma(p(A)) = \{c\}$, since $(c - \lambda)I$ is invertible if and only if $c\ne\lambda$. On the other hand, since $X$ is a Banach space over $\bbC$, due to Gelfand-Mazur, the spectrum $\sigma(A)$ is non-empty, whence $p(\sigma(A)) = \{c\}$, as desired.


\subsection*{Exercise 17-5 of \texorpdfstring{\cite{limaye-functional}}{[Lim14]}}

Suppose $p(X)\in\K[X]$ is a polynomial such that $p(A)$ is a compact opeartor. If $p(X)\equiv c$ is a constant polynomial, then $p(A) = cI$, which is given to be compact. Thus, the closure of the image of the unit ball under $p(A)$ is compact, that is,
\begin{equation*}
    \overline{cB_X(0, 1)} = \overline{B_X(0, c)} 
\end{equation*}
is compact. If $c\ne 0$, using the fact that $\overline{B_X(0, c)}$ is homeomorphic to $\overline{B_X(0, 1)}$ through the homeomorphism $x\mapsto c^{-1}x$ (as we have seen in class), it follows that the closed unit ball in $X$ is compact, which is absurd, since $X$ is infinite-dimensional. Thus $c = 0$. Clearly, if $c = 0$, then the operator $p(A)\equiv 0$ is compact.

Suppose now that $p(X) = a_nX^n + \dots + a_0\in\K[X]$ is a non-constant polynomial of degree $n > 0$, whence $a_n\ne 0$. According to the hypothesis, $p(A) = a_nA^n + \dots + a_0I$ is a compact operator. Recall that the compact operators form an ideal in the $\K$-algebra $\scrB(X)$ and $A$ is a compact operator, thus the operator 
\begin{equation*}
    a_nA^n + \dots + a_1A
\end{equation*}
is a compact operator. It follows that 
\begin{equation*}
    a_0I = p(A) - \left(a_nA^n + \dots + a_1A\right) 
\end{equation*}
is a compact operator. Because of what we just proved, we must have that $a_0 = 0$, that is, $p(0) = 0$.

Conversely, suppose $p(X)\in\K[X]$ is a polynomial such that $p(0) = 0$. If $p(X)$ is a constant polynomial, then it is identically zero, whence is trivially compact. Suppose now that $p(X)$ is non-constant. Then, we can write 
\begin{equation*}
    p(X) = a_nX^n + \dots + a_1X
\end{equation*}
for $n > 0$ and $a_i\in\K$ with $a_n\ne 0$ where $n$ is the degree of the polynomial $p(X)$. Thus $p(A) = a_nA^n + \dots + a_1A$. As we argued earlier, since the compact operators form an ideal in $\scrB(X)$ and $A$ is a compact operator, it is clear that $p(A)$ is a compact operator, as desired.

\bibliographystyle{alpha}
\bibliography{references}
\end{document}