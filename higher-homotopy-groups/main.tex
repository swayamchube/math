\documentclass[11pt]{article}

\usepackage[utf8]{inputenc} % allow utf-8 input
\usepackage[T1]{fontenc}    % use 8-bit T1 fonts
\usepackage{hyperref}       % hyperlinks
\usepackage{url}            % simple URL typesetting
\usepackage{booktabs}       % professional-quality tables
\usepackage{amsfonts}       % blackboard math symbols
\usepackage{nicefrac}       % compact symbols for 1/2, etc.
\usepackage{microtype}      % microtypography
\usepackage{graphicx}
\usepackage{natbib}
\usepackage{doi}
\usepackage{amssymb}
\usepackage{bbm}
\usepackage{amsthm}
\usepackage{amsmath}
\usepackage{xcolor}
\usepackage{theoremref}
\usepackage{enumitem}
% \usepackage{lmodern}
\usepackage{mathpazo}
\usepackage{fouriernc}
% \usepackage{euler}
% \usepackage{sansmath}
% \usepackage{sfmath}
\usepackage{mathrsfs}
\setlength{\marginparwidth}{2cm}
\usepackage{todonotes}
\usepackage{stmaryrd}
\usepackage[all,cmtip]{xy} % For diagrams, praise the Freyd-Mitchell theorem 
\usepackage{marvosym}
\usepackage{geometry}
\usepackage{mdframed}
\usepackage{titlesec}
\usepackage{mathtools}
\usepackage{tikz}
\usetikzlibrary{cd}
\usepackage{epigraph}

\renewcommand{\qedsymbol}{$\blacksquare$}
% \renewcommand{\familydefault}{\sfdefault} % Do you want this font? 

% Uncomment to override  the `A preprint' in the header
% \renewcommand{\headeright}{}
% \renewcommand{\undertitle}{}
% \renewcommand{\shorttitle}{}

\hypersetup{
    pdfauthor={Lots of People},
    colorlinks=true,
	citecolor=blue
}

\newtheoremstyle{thmstyle}%               % Name
  {}%                                     % Space above
  {}%                                     % Space below
  {}%                             % Body font
  {}%                                     % Indent amount
  {\bfseries\scshape}%                            % Theorem head font
  {.}%                                    % Punctuation after theorem head
  { }%                                    % Space after theorem head, ' ', or \newline
  {\thmname{#1}\thmnumber{ #2}\thmnote{ (#3)}}%                                     % Theorem head spec (can be left empty, meaning `normal')

\newtheoremstyle{defstyle}%               % Name
  {}%                                     % Space above
  {}%                                     % Space below
  {}%                                     % Body font
  {}%                                     % Indent amount
  {\bfseries\scshape}%                            % Theorem head font
  {.}%                                    % Punctuation after theorem head
  { }%                                    % Space after theorem head, ' ', or \newline
  {\thmname{#1}\thmnumber{ #2}\thmnote{ (#3)}}%                                     % Theorem head spec (can be left empty, meaning `normal')

\theoremstyle{thmstyle}
\newtheorem{theorem}{Theorem}
\newtheorem{lemma}[theorem]{Lemma}
\newtheorem{proposition}[theorem]{Proposition}

\theoremstyle{defstyle}
\newtheorem{definition}[theorem]{Definition}
\newtheorem{corollary}[theorem]{Corollary}
\newtheorem{porism}[theorem]{Porism}
\newtheorem{remark}[theorem]{Remark}
\newtheorem{interlude}[theorem]{Interlude}
\newtheorem{example}[theorem]{Example}
\newtheorem*{notation}{Notation}
\newtheorem*{claim}{Claim}

% Common Algebraic Structures
\newcommand{\R}{\mathbb{R}}
\newcommand{\Q}{\mathbb{Q}}
\newcommand{\Z}{\mathbb{Z}}
\newcommand{\N}{\mathbb{N}}
\newcommand{\bbC}{\mathbb{C}} 
\newcommand{\K}{\mathbb{K}} % Base field which is either \R or \bbC
\newcommand{\calA}{\mathcal{A}} % Banach Algebras
\newcommand{\calB}{\mathcal{B}} % Banach Algebras
\newcommand{\calI}{\mathcal{I}} % ideal in a Banach algebra
\newcommand{\calJ}{\mathcal{J}} % ideal in a Banach algebra
\newcommand{\frakM}{\mathfrak{M}} % sigma-algebra
\newcommand{\calO}{\mathcal{O}} % Ring of integers
\newcommand{\bbA}{\mathbb{A}} % Adele (or ring thereof)
\newcommand{\bbI}{\mathbb{I}} % Idele (or group thereof)
\newcommand{\bbD}{\mathbb{D}} % Unit disk

% Categories
\newcommand{\catTopp}{\mathbf{Top}_*}
\newcommand{\catGrp}{\mathbf{Grp}}
\newcommand{\catTopGrp}{\mathbf{TopGrp}}
\newcommand{\catSet}{\mathbf{Set}}
\newcommand{\catTop}{\mathbf{Top}}
\newcommand{\catRing}{\mathbf{Ring}}
\newcommand{\catCRing}{\mathbf{CRing}} % comm. rings
\newcommand{\catMod}{\mathbf{Mod}}
\newcommand{\catMon}{\mathbf{Mon}}
\newcommand{\catMan}{\mathbf{Man}} % manifolds
\newcommand{\catDiff}{\mathbf{Diff}} % smooth manifolds
\newcommand{\catAlg}{\mathbf{Alg}}
\newcommand{\catRep}{\mathbf{Rep}} % representations 
\newcommand{\catVec}{\mathbf{Vec}}

% Group and Representation Theory
\newcommand{\chr}{\operatorname{char}}
\newcommand{\Aut}{\operatorname{Aut}}
\newcommand{\GL}{\operatorname{GL}}
\newcommand{\im}{\operatorname{im}}
\newcommand{\tr}{\operatorname{tr}}
\newcommand{\id}{\mathbf{id}}
\newcommand{\cl}{\mathbf{cl}}
\newcommand{\Gal}{\operatorname{Gal}}
\newcommand{\Tr}{\operatorname{Tr}}
\newcommand{\sgn}{\operatorname{sgn}}
\newcommand{\Sym}{\operatorname{Sym}}
\newcommand{\Alt}{\operatorname{Alt}}

% Commutative and Homological Algebra
\newcommand{\spec}{\operatorname{spec}}
\newcommand{\mspec}{\operatorname{m-spec}}
\newcommand{\Spec}{\operatorname{Spec}}
\newcommand{\MaxSpec}{\operatorname{MaxSpec}}
\newcommand{\Tor}{\operatorname{Tor}}
\newcommand{\tor}{\operatorname{tor}}
\newcommand{\Ann}{\operatorname{Ann}}
\newcommand{\Supp}{\operatorname{Supp}}
\newcommand{\Hom}{\operatorname{Hom}}
\newcommand{\End}{\operatorname{End}}
\newcommand{\coker}{\operatorname{coker}}
\newcommand{\limit}{\varprojlim}
\newcommand{\colimit}{%
  \mathop{\mathpalette\colimit@{\rightarrowfill@\textstyle}}\nmlimits@
}
\makeatother


\newcommand{\fraka}{\mathfrak{a}} % ideal
\newcommand{\frakb}{\mathfrak{b}} % ideal
\newcommand{\frakc}{\mathfrak{c}} % ideal
\newcommand{\frakf}{\mathfrak{f}} % face map
\newcommand{\frakg}{\mathfrak{g}}
\newcommand{\frakh}{\mathfrak{h}}
\newcommand{\frakm}{\mathfrak{m}} % maximal ideal
\newcommand{\frakn}{\mathfrak{n}} % naximal ideal
\newcommand{\frakp}{\mathfrak{p}} % prime ideal
\newcommand{\frakq}{\mathfrak{q}} % qrime ideal
\newcommand{\fraks}{\mathfrak{s}}
\newcommand{\frakt}{\mathfrak{t}}
\newcommand{\frakz}{\mathfrak{z}}
\newcommand{\frakA}{\mathfrak{A}}
\newcommand{\frakI}{\mathfrak{I}}
\newcommand{\frakJ}{\mathfrak{J}}
\newcommand{\frakK}{\mathfrak{K}}
\newcommand{\frakL}{\mathfrak{L}}
\newcommand{\frakN}{\mathfrak{N}} % nilradical 
\newcommand{\frakO}{\mathfrak{O}} % dedekind domain
\newcommand{\frakP}{\mathfrak{P}} % Prime ideal above
\newcommand{\frakQ}{\mathfrak{Q}} % Qrime ideal above 
\newcommand{\frakR}{\mathfrak{R}} % jacobson radical
\newcommand{\frakU}{\mathfrak{U}}
\newcommand{\frakV}{\mathfrak{V}}
\newcommand{\frakW}{\mathfrak{W}}
\newcommand{\frakX}{\mathfrak{X}}

% General/Differential/Algebraic Topology 
\newcommand{\scrA}{\mathscr{A}}
\newcommand{\scrB}{\mathscr{B}}
\newcommand{\scrF}{\mathscr{F}}
\newcommand{\scrM}{\mathscr{M}}
\newcommand{\scrN}{\mathscr{N}}
\newcommand{\scrP}{\mathscr{P}}
\newcommand{\scrO}{\mathscr{O}} % sheaf
\newcommand{\scrR}{\mathscr{R}}
\newcommand{\scrS}{\mathscr{S}}
\newcommand{\scrU}{\mathscr{U}}
\newcommand{\bbH}{\mathbb H}
\newcommand{\Int}{\operatorname{Int}}
\newcommand{\psimeq}{\simeq_p}
\newcommand{\wt}[1]{\widetilde{#1}}
\newcommand{\RP}{\mathbb{R}\text{P}}
\newcommand{\CP}{\mathbb{C}\text{P}}

% Miscellaneous
\newcommand{\wh}[1]{\widehat{#1}}
\newcommand{\calE}{\mathcal{E}}
\newcommand{\calM}{\mathcal{M}}
\newcommand{\calN}{\mathcal{N}}
\newcommand{\calK}{\mathcal{K}}
\newcommand{\calP}{\mathcal{P}}
\newcommand{\calU}{\mathcal{U}}
\newcommand{\onto}{\twoheadrightarrow}
\newcommand{\into}{\hookrightarrow}
\newcommand{\Gr}{\operatorname{Gr}}
\newcommand{\Span}{\operatorname{Span}}
\newcommand{\ev}{\operatorname{ev}}
\newcommand{\weakto}{\stackrel{w}{\longrightarrow}}

\newcommand{\define}[1]{\textcolor{blue}{\textit{#1}}}
% \newcommand{\caution}[1]{\textcolor{red}{\textit{#1}}}
\newcommand{\important}[1]{\textcolor{red}{\textit{#1}}}
\renewcommand{\mod}{~\mathrm{mod}~}
\renewcommand{\le}{\leqslant}
\renewcommand{\leq}{\leqslant}
\renewcommand{\ge}{\geqslant}
\renewcommand{\geq}{\geqslant}
\newcommand{\Res}{\operatorname{Res}}
\newcommand{\floor}[1]{\left\lfloor #1\right\rfloor}
\newcommand{\ceil}[1]{\left\lceil #1\right\rceil}
\newcommand{\gl}{\mathfrak{gl}}
\newcommand{\ad}{\operatorname{ad}}
\newcommand{\Stab}{\operatorname{Stab}}
\newcommand{\bfX}{\mathbf{X}}
\newcommand{\Ind}{\operatorname{Ind}}
\newcommand{\bfG}{\mathbf{G}}
\newcommand{\rank}{\operatorname{rank}}
\newcommand{\calo}{\mathcal{o}}
\newcommand{\frako}{\mathfrak{o}}
\newcommand{\Cl}{\operatorname{Cl}}

\newcommand{\idim}{\operatorname{idim}}
\newcommand{\pdim}{\operatorname{pdim}}
\newcommand{\Ext}{\operatorname{Ext}}
\newcommand{\co}{\operatorname{co}}
\newcommand{\bfO}{\mathbf{O}}
\newcommand{\bfF}{\mathbf{F}} % Fitting Subgroup
\newcommand{\Syl}{\operatorname{Syl}}
\newcommand{\nor}{\vartriangleleft}
\newcommand{\noreq}{\trianglelefteqslant}
\newcommand{\subnor}{\nor\!\nor}
\newcommand{\Soc}{\operatorname{Soc}}
\newcommand{\core}{\operatorname{core}}
\newcommand{\Sd}{\operatorname{Sd}}
\newcommand{\mesh}{\operatorname{mesh}}
\newcommand{\sminus}{\setminus}
\newcommand{\diam}{\operatorname{diam}}
\newcommand{\Ass}{\operatorname{Ass}}
\newcommand{\projdim}{\operatorname{proj~dim}}
\newcommand{\injdim}{\operatorname{inj~dim}}
\newcommand{\gldim}{\operatorname{gl~dim}}
\newcommand{\embdim}{\operatorname{emb~dim}}
\newcommand{\hght}{\operatorname{ht}}
\newcommand{\depth}{\operatorname{depth}}
\newcommand{\ul}[1]{\underline{#1}}
\newcommand{\type}{\operatorname{type}}
\newcommand{\CM}{\operatorname{CM}}
\newcommand{\Irr}{\operatorname{Irr}}
\newcommand{\scrC}{\mathscr{C}}
\newcommand{\calL}{\mathcal{L}}
\newcommand{\calF}{\mathcal{F}}
\newcommand{\calC}{\mathcal{C}}
\newcommand{\calR}{\mathcal{R}}
\newcommand{\FV}{\operatorname{FV}}
\newcommand{\Th}{\operatorname{Th}}
\newcommand{\bbone}{\mathbbm{1}}
\renewcommand{\Re}{\operatorname{Re}}
\renewcommand{\Im}{\operatorname{Im}}
\newcommand{\pr}{\operatorname{pr}}

\geometry {
    margin = 1in
}

\titleformat
{\section}
[block]
{\Large\bfseries\sffamily}
{\S\thesection}
{0.5em}
{\centering}
[]


\titleformat
{\subsection}
[block]
{\normalfont\bfseries\sffamily}
{\S\S}
{0.5em}
{\centering}
[]

\setlength\epigraphwidth{0.8\textwidth}

\begin{document}
\title{Higher Homotopy Groups}
\author{Swayam Chube}
\date{Last Updated: \today}
\maketitle

\section{Function Spaces}

We begin with some preliminaries about function spaces. 
\begin{definition}
	Let $X$ and $Y$ be topological spaces. We use the shorthand $X^Y$ to denote the set of continuous functions from $Y$ to $X$. Endow this set with the \define{compact open topology}, that is, the topology generated by the subbasis elements 
	\begin{equation*}
		(K; U)\coloneq \left\{f\in X^Y\colon f(K)\subseteq U\right\},
	\end{equation*}
	for all compact sets $K\subseteq Y$ and open sets $U\subseteq X$.
\end{definition}

For each continuous function $F\colon Z\times Y\to X$, there is the \define{associate}  $F^\sharp\colon Z\to X^Y$ defined by 
\begin{equation*}
	F^\sharp(z) = \left(y\longmapsto F(z, y)\right).
\end{equation*}
Further, there is the \define{evaluation map} $\ev\colon X^Y\times Y\to X$ given by $\ev(f, y) = f(y)$.

\begin{theorem}
	Let $X$ and $Z$ be topological spaces, $Y$ a locally compact Hausdorff space, and equip $X^Y$ with the compact-open topology.
	\begin{enumerate}[label=(\arabic*)]
		\item The evaluation map $\ev\colon X^Y\times Y\to X$ is continuous. 
		\item A function $F\colon Z\times Y\to X$ is continuous if and only if its associate $F^\sharp\colon Z\to X^Y$ is continuous.
	\end{enumerate}
\end{theorem}

\section{Group and Cogroup Objects}

In a category $\scrC$, if the product $X\times Y$ exists, then given any object $Z$ and morphisms $f\colon Z\to X$ and $g\colon Z\to Y$, there is a unique map $(f, g)\colon Z\to X\times Y$ making the diagram 
\begin{equation*}
	\xymatrix {
		X\times Y\ar[r]^-{\pr_1}\ar[d]_-{\pr_2} & X\\
		Y & Z\ar[l]^g\ar[u]_f\ar@{.>}[lu]|-{(f, g)}
	}
\end{equation*}
commute. 

Similarly, if the coproduct $X\coprod Y$ exists, then given any object $Z$ and morphisms $f\colon X\to Z$ and $g\colon Y\to Z$, there is a unique map $(f, g)\colon X\coprod Y\to Z$ making 
\begin{equation*}
	\xymatrix {
		Z & X\ar[d]^{\iota_1}\ar[l]_{f}\\
		Y_{\iota_2}\ar[r]\ar[u]^{g} & X\coprod Y\ar@{.>}[lu]|-{(f, g)}
	}
\end{equation*}

\begin{definition}
	Let $\scrC$ be a category admitting finite products and a terminal object $Z$. A \define{group object} in $\scrC$ is an object $G$ together with morphisms 
	\begin{equation*}
		\mu\colon G\times G\to G,\quad\eta\colon G\to G,\quad\text{ and }\quad\varepsilon\colon Z\to G
	\end{equation*}
	such that the following diagrams commute: 
	\begin{description}
		\item[Associativity]
		\begin{equation*}
			\xymatrix{
				G\times G\times G\ar[r]^-{\bbone\times\mu}\ar[d]_{\mu\times\bbone} & G\times G\ar[d]^\mu\\
				G\times G\ar[r]_-\mu & G
			}
		\end{equation*}
		\item[Identity]
		\begin{equation*}
			\xymatrix {
				G\times Z\ar[r]^-{\bbone\times\varepsilon}\ar[rd]_-{\pr_1} & G\times G\ar[d]_-{\mu} & Z\times G\ar[l]_-{\varepsilon\times\bbone}\ar[ld]^-{\pr_2}\\
				& G & 
			}
		\end{equation*}
		Note that the projections $\pr_1$ and $\pr_2$ are isomorphisms in the category.
		\item[Inverse] 
		\begin{equation*}
			\xymatrix {
				G\ar[r]^-{(\bbone, \eta)}\ar[d] & G\times G\ar[d]_{\mu} & G\ar[l]_-{(\eta, \bbone)}\ar[d]\\
				Z\ar[r]_\varepsilon & G & Z\ar[l]^\varepsilon
			}
		\end{equation*}
	\end{description}
	The maps $\mu$, $\eta$, and $\mu$ are called \define{multiplication}, \define{inversion}, and \define{unit} respectively.
\end{definition}

\begin{definition}
	Let $\scrC$ be a category admitting finite coproducts and an initial object $A$. A \define{cogroup object} in $\scrC$ is an object $C$ together with morphisms 
	\begin{equation*}
		m\colon C\to C\coprod C,\quad h\colon C\to C,\quad\text{ and }\quad e\colon C\to A
	\end{equation*}
	such that the following diagrams commute 
	\begin{description}
		\item[Co-associativity]  
		\begin{equation*}
			\xymatrix {
				C\ar[r]^-m\ar[d]_m & C\coprod C\ar[d]^-{\bbone\coprod m}\\
				C\coprod C\ar[r]_-{m\coprod\bbone} & C\coprod C\coprod C
			}
		\end{equation*}
		\item[Co-identity] 
		\begin{equation*}
			\xymatrix {
				C\coprod A & C\coprod C\ar[l]_{\bbone\coprod e}\ar[r]^{e\coprod\bbone} & A\coprod C\\
				& C\ar[u]^m\ar[lu]^{\iota_1}\ar[ru]_{\iota_2} & 
			}
		\end{equation*}
		\item[Co-inverse]
		\begin{equation*}
			\xymatrix {
				C & C\coprod C\ar[r]^-{(h, 1)}\ar[l]_-{(1, h)} & C\\
				A\ar[u] & C\ar[r]_e\ar[l]^e\ar[u]^m & A\ar[u]
			}
		\end{equation*}
	\end{description}
\end{definition}

\begin{theorem}
	Let $\scrC$ be a category admitting finite coproducts and a terminal object. An object $G$ in $\scrC$ is a group object if and only if $\Hom_\scrC(X, G)$ has the structure of a group for every object $X$ of $G$.
\end{theorem}
\begin{proof}
	Omitted.
\end{proof}

\begin{corollary}
	Every abelian group is a group object in $\catGrp$ and every topological group (with identity as basepoint) is a group object in $\catTop_\ast$.
\end{corollary}

\begin{proposition}\thlabel{group-object-in-Grp}
	A group object in $\catGrp$ is an abelian group.
\end{proposition}
\begin{proof}
	Suppose $(G, \mu, \eta, \varepsilon)$ is a group object in $\catGrp$. Clearly, this gives an alternate group structure on $G$, which we denote by $(G, \otimes, a\mapsto\eta(a))$. Since $\mu\colon G\times G\to G$ is a group homomorphism with respect to the original group structure of $G$, we have 
	\begin{equation*}
		(ac)\otimes(bd) = (a\otimes b)(c\otimes d).
	\end{equation*}
	Due to \href{https://en.wikipedia.org/wiki/Eckmann%E2%80%93Hilton_argument}{Eckmann-Hilton}, both group structures on $G$ must agree and must be commutative, as desired.
\end{proof}
\begin{corollary}
	The fundamental group of a topological group (with identity as basepoint) is abelian.
\end{corollary}
\begin{proof}
	$\pi_1\colon\catTop_\ast\to\catGrp$ is a functor preserving finite products, and sending terminal objects to terminal objects, therefore, $\pi_1$ sends group objects to group objects. Since a topological group with identity as its basepoint is a group object in $\catTop_\ast$, it follows that $\pi_1$ sends it to an abelian group due to \thref{group-object-in-Grp}.
\end{proof}

\end{document}