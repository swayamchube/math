\documentclass[11pt]{article}

\usepackage[utf8]{inputenc} % allow utf-8 input
\usepackage[T1]{fontenc}    % use 8-bit T1 fonts
\usepackage{hyperref}       % hyperlinks
\usepackage{url}            % simple URL typesetting
\usepackage{booktabs}       % professional-quality tables
\usepackage{amsfonts}       % blackboard math symbols
\usepackage{nicefrac}       % compact symbols for 1/2, etc.
\usepackage{microtype}      % microtypography
\usepackage{graphicx}
\usepackage{natbib}
\usepackage{doi}
\usepackage{amssymb}
\usepackage{bbm}
\usepackage{amsthm}
\usepackage{amsmath}
\usepackage{xcolor}
\usepackage{theoremref}
\usepackage{enumitem}
% \usepackage{lmodern}
\usepackage{mathpazo}
\usepackage{fouriernc}
% \usepackage{euler}
% \usepackage{sansmath}
% \usepackage{sfmath}
\usepackage{mathrsfs}
\setlength{\marginparwidth}{2cm}
\usepackage{todonotes}
\usepackage{stmaryrd}
\usepackage[all,cmtip]{xy} % For diagrams, praise the Freyd-Mitchell theorem 
\usepackage{marvosym}
\usepackage{geometry}
\usepackage{titlesec}
\usepackage{mathtools}
\usepackage{tikz}
\usetikzlibrary{cd}

\renewcommand{\qedsymbol}{$\blacksquare$}
% \renewcommand{\familydefault}{\sfdefault} % Do you want this font? 

% Uncomment to override  the `A preprint' in the header
% \renewcommand{\headeright}{}
% \renewcommand{\undertitle}{}
% \renewcommand{\shorttitle}{}

\hypersetup{
    pdfauthor={Lots of People},
    colorlinks=true,
	citecolor=blue
}

\newtheoremstyle{thmstyle}%               % Name
  {}%                                     % Space above
  {}%                                     % Space below
  {}%                             % Body font
  {}%                                     % Indent amount
  {\bfseries\scshape}%                            % Theorem head font
  {.}%                                    % Punctuation after theorem head
  { }%                                    % Space after theorem head, ' ', or \newline
  {\thmname{#1}\thmnumber{ #2}\thmnote{ (#3)}}%                                     % Theorem head spec (can be left empty, meaning `normal')

\newtheoremstyle{defstyle}%               % Name
  {}%                                     % Space above
  {}%                                     % Space below
  {}%                                     % Body font
  {}%                                     % Indent amount
  {\bfseries\scshape}%                            % Theorem head font
  {.}%                                    % Punctuation after theorem head
  { }%                                    % Space after theorem head, ' ', or \newline
  {\thmname{#1}\thmnumber{ #2}\thmnote{ (#3)}}%                                     % Theorem head spec (can be left empty, meaning `normal')

\theoremstyle{thmstyle}
\newtheorem{theorem}{Theorem}[section]
\newtheorem{lemma}[theorem]{Lemma}
\newtheorem{proposition}[theorem]{Proposition}

\theoremstyle{defstyle}
\newtheorem{definition}[theorem]{Definition}
\newtheorem{corollary}[theorem]{Corollary}
\newtheorem{porism}[theorem]{Porism}
\newtheorem{remark}[theorem]{Remark}
\newtheorem{interlude}[theorem]{Interlude}
\newtheorem{example}[theorem]{Example}
\newtheorem*{notation}{Notation}
\newtheorem*{claim}{Claim}

% Common Algebraic Structures
\newcommand{\R}{\mathbb{R}}
\newcommand{\Q}{\mathbb{Q}}
\newcommand{\Z}{\mathbb{Z}}
\newcommand{\N}{\mathbb{N}}
\newcommand{\bbC}{\mathbb{C}} 
\newcommand{\K}{\mathbb{K}} % Base field which is either \R or \bbC
\newcommand{\calA}{\mathcal{A}} % Banach Algebras
\newcommand{\calB}{\mathcal{B}} % Banach Algebras
\newcommand{\calI}{\mathcal{I}} % ideal in a Banach algebra
\newcommand{\calJ}{\mathcal{J}} % ideal in a Banach algebra
\newcommand{\frakM}{\mathfrak{M}} % sigma-algebra
\newcommand{\calO}{\mathcal{O}} % Ring of integers
\newcommand{\bbA}{\mathbb{A}} % Adele (or ring thereof)
\newcommand{\bbI}{\mathbb{I}} % Idele (or group thereof)
\newcommand{\bbD}{\mathbb{D}} % Unit disk

% Categories
\newcommand{\catTopp}{\mathbf{Top}_*}
\newcommand{\catGrp}{\mathbf{Grp}}
\newcommand{\catTopGrp}{\mathbf{TopGrp}}
\newcommand{\catSet}{\mathbf{Set}}
\newcommand{\catTop}{\mathbf{Top}}
\newcommand{\catRing}{\mathbf{Ring}}
\newcommand{\catCRing}{\mathbf{CRing}} % comm. rings
\newcommand{\catMod}{\mathbf{Mod}}
\newcommand{\catMon}{\mathbf{Mon}}
\newcommand{\catMan}{\mathbf{Man}} % manifolds
\newcommand{\catDiff}{\mathbf{Diff}} % smooth manifolds
\newcommand{\catAlg}{\mathbf{Alg}}
\newcommand{\catRep}{\mathbf{Rep}} % representations 
\newcommand{\catVec}{\mathbf{Vec}}

% Group and Representation Theory
\newcommand{\chr}{\operatorname{char}}
\newcommand{\Aut}{\operatorname{Aut}}
\newcommand{\GL}{\operatorname{GL}}
\newcommand{\im}{\operatorname{im}}
\newcommand{\tr}{\operatorname{tr}}
\newcommand{\id}{\mathbf{id}}
\newcommand{\cl}{\mathbf{cl}}
\newcommand{\Gal}{\operatorname{Gal}}
\newcommand{\Tr}{\operatorname{Tr}}
\newcommand{\sgn}{\operatorname{sgn}}
\newcommand{\Sym}{\operatorname{Sym}}
\newcommand{\Alt}{\operatorname{Alt}}

% Commutative and Homological Algebra
\newcommand{\spec}{\operatorname{spec}}
\newcommand{\mspec}{\operatorname{m-spec}}
\newcommand{\Spec}{\operatorname{Spec}}
\newcommand{\MaxSpec}{\operatorname{MaxSpec}}
\newcommand{\Tor}{\operatorname{Tor}}
\newcommand{\tor}{\operatorname{tor}}
\newcommand{\Ann}{\operatorname{Ann}}
\newcommand{\Supp}{\operatorname{Supp}}
\newcommand{\Hom}{\operatorname{Hom}}
\newcommand{\End}{\operatorname{End}}
\newcommand{\coker}{\operatorname{coker}}
\newcommand{\limit}{\varprojlim}
\newcommand{\colimit}{%
  \mathop{\mathpalette\colimit@{\rightarrowfill@\textstyle}}\nmlimits@
}
\makeatother


\newcommand{\fraka}{\mathfrak{a}} % ideal
\newcommand{\frakb}{\mathfrak{b}} % ideal
\newcommand{\frakc}{\mathfrak{c}} % ideal
\newcommand{\frakf}{\mathfrak{f}} % face map
\newcommand{\frakg}{\mathfrak{g}}
\newcommand{\frakh}{\mathfrak{h}}
\newcommand{\frakm}{\mathfrak{m}} % maximal ideal
\newcommand{\frakn}{\mathfrak{n}} % naximal ideal
\newcommand{\frakp}{\mathfrak{p}} % prime ideal
\newcommand{\frakq}{\mathfrak{q}} % qrime ideal
\newcommand{\fraks}{\mathfrak{s}}
\newcommand{\frakt}{\mathfrak{t}}
\newcommand{\frakz}{\mathfrak{z}}
\newcommand{\frakA}{\mathfrak{A}}
\newcommand{\frakI}{\mathfrak{I}}
\newcommand{\frakJ}{\mathfrak{J}}
\newcommand{\frakK}{\mathfrak{K}}
\newcommand{\frakL}{\mathfrak{L}}
\newcommand{\frakN}{\mathfrak{N}} % nilradical 
\newcommand{\frakO}{\mathfrak{O}} % dedekind domain
\newcommand{\frakP}{\mathfrak{P}} % Prime ideal above
\newcommand{\frakQ}{\mathfrak{Q}} % Qrime ideal above 
\newcommand{\frakR}{\mathfrak{R}} % jacobson radical
\newcommand{\frakU}{\mathfrak{U}}
\newcommand{\frakV}{\mathfrak{V}}
\newcommand{\frakW}{\mathfrak{W}}
\newcommand{\frakX}{\mathfrak{X}}

% General/Differential/Algebraic Topology 
\newcommand{\scrA}{\mathscr{A}}
\newcommand{\scrB}{\mathscr{B}}
\newcommand{\scrF}{\mathscr{F}}
\newcommand{\scrM}{\mathscr{M}}
\newcommand{\scrN}{\mathscr{N}}
\newcommand{\scrP}{\mathscr{P}}
\newcommand{\scrO}{\mathscr{O}} % sheaf
\newcommand{\scrR}{\mathscr{R}}
\newcommand{\scrS}{\mathscr{S}}
\newcommand{\scrU}{\mathscr{U}}
\newcommand{\bbH}{\mathbb H}
\newcommand{\Int}{\operatorname{Int}}
\newcommand{\psimeq}{\simeq_p}
\newcommand{\wt}[1]{\widetilde{#1}}
\newcommand{\RP}{\mathbb{R}\text{P}}
\newcommand{\CP}{\mathbb{C}\text{P}}

% Miscellaneous
\newcommand{\wh}[1]{\widehat{#1}}
\newcommand{\calE}{\mathcal{E}}
\newcommand{\calM}{\mathcal{M}}
\newcommand{\calN}{\mathcal{N}}
\newcommand{\calK}{\mathcal{K}}
\newcommand{\calP}{\mathcal{P}}
\newcommand{\calU}{\mathcal{U}}
\newcommand{\onto}{\twoheadrightarrow}
\newcommand{\into}{\hookrightarrow}
\newcommand{\Gr}{\operatorname{Gr}}
\newcommand{\Span}{\operatorname{Span}}
\newcommand{\ev}{\operatorname{ev}}
\newcommand{\weakto}{\stackrel{w}{\longrightarrow}}

\newcommand{\define}[1]{\textcolor{blue}{\textit{#1}}}
% \newcommand{\caution}[1]{\textcolor{red}{\textit{#1}}}
\newcommand{\important}[1]{\textcolor{red}{\textit{#1}}}
\renewcommand{\mod}{~\mathrm{mod}~}
\renewcommand{\le}{\leqslant}
\renewcommand{\leq}{\leqslant}
\renewcommand{\ge}{\geqslant}
\renewcommand{\geq}{\geqslant}
\newcommand{\Res}{\operatorname{Res}}
\newcommand{\floor}[1]{\left\lfloor #1\right\rfloor}
\newcommand{\ceil}[1]{\left\lceil #1\right\rceil}
\newcommand{\gl}{\mathfrak{gl}}
\newcommand{\ad}{\operatorname{ad}}
\newcommand{\Stab}{\operatorname{Stab}}
\newcommand{\bfX}{\mathbf{X}}
\newcommand{\Ind}{\operatorname{Ind}}
\newcommand{\bfG}{\mathbf{G}}
\newcommand{\rank}{\operatorname{rank}}
\newcommand{\calo}{\mathcal{o}}
\newcommand{\frako}{\mathfrak{o}}
\newcommand{\Cl}{\operatorname{Cl}}

\newcommand{\idim}{\operatorname{idim}}
\newcommand{\pdim}{\operatorname{pdim}}
\newcommand{\Ext}{\operatorname{Ext}}
\newcommand{\co}{\operatorname{co}}
\newcommand{\bfO}{\mathbf{O}}
\newcommand{\bfF}{\mathbf{F}} % Fitting Subgroup
\newcommand{\Syl}{\operatorname{Syl}}
\newcommand{\nor}{\vartriangleleft}
\newcommand{\noreq}{\trianglelefteqslant}
\newcommand{\subnor}{\nor\!\nor}
\newcommand{\Soc}{\operatorname{Soc}}
\newcommand{\core}{\operatorname{core}}
\newcommand{\Sd}{\operatorname{Sd}}
\newcommand{\mesh}{\operatorname{mesh}}
\newcommand{\sminus}{\setminus}
\newcommand{\diam}{\operatorname{diam}}
\newcommand{\Ass}{\operatorname{Ass}}
\newcommand{\projdim}{\operatorname{proj~dim}}
\newcommand{\injdim}{\operatorname{inj~dim}}
\newcommand{\gldim}{\operatorname{gl~dim}}
\newcommand{\embdim}{\operatorname{emb~dim}}
\newcommand{\hght}{\operatorname{ht}}
\newcommand{\depth}{\operatorname{depth}}
\newcommand{\ul}[1]{\underline{#1}}
\newcommand{\type}{\operatorname{type}}
\newcommand{\CM}{\operatorname{CM}}
\newcommand{\Irr}{\operatorname{Irr}}
\newcommand{\scrC}{\mathscr{C}}
\newcommand{\calL}{\mathcal{L}}
\newcommand{\calF}{\mathcal{F}}
\newcommand{\calC}{\mathcal{C}}
\newcommand{\calR}{\mathcal{R}}
\newcommand{\FV}{\operatorname{FV}}
\newcommand{\Th}{\operatorname{Th}}
\renewcommand{\Re}{\operatorname{Re}}
\renewcommand{\Im}{\operatorname{Im}}
\newcommand{\bbM}{\mathbb{M}}

\geometry {
    margin = 1in
}

\titleformat
{\section}
[block]
{\Large\bfseries\sffamily}
{\S\thesection}
{0.5em}
{\centering}
[]


\titleformat
{\subsection}
[block]
{\normalfont\bfseries\sffamily}
{\S\S}
{0.5em}
{\centering}
[]


\begin{document}
\title{Determinants}
\author{Swayam Chube}
\date{Last Updated: \today}
\maketitle
\tableofcontents

% \section{Determinants}

% Throughout this section fix a commutative ring $R$ (with unity). Let $\bbM_n(R)$ denote the set (or monoid, or $R$-module) of $n\times n$ matrices with entries in $R$. Note that there is a natural bijection between the set $\bbM_n(R)$ and the set $M\coloneq \underbrace{R^n\times\cdots\times R^n}_{n\text{ times}}$. We shall tacitly make this identification for defining the determinant. Explicitly, the identification is: 
% \begin{equation*}
%     \begin{pmatrix}
%         a_{11}  & \dots & a_{1n}\\
%         \vdots  & \ddots & \vdots\\
%         a_{n1}  & \cdots & a_{nn}
%     \end{pmatrix}
%     \longleftrightarrow
%     \left(
%         \begin{pmatrix}
%             a_{11}\\\vdots\\ a_{n1}
%         \end{pmatrix},
%         \cdots,
%         \begin{pmatrix}
%             a_{1n}\\ \vdots \\ a_{nn}
%         \end{pmatrix}
%     \right).
% \end{equation*}


% \begin{definition}
%     Let $n\ge 1$ be a positive integer. A map $T\colon M\to R$ is said to be \define{multilinear} if for each $1\le i\le n$ and $$v_1,\dots, v_{i - 1}, v_{i + 1}, \dots, v_n\in R^n,$$ the map 
%     \begin{equation*}
%         v\mapsto T(v_1,\dots, v_{i - 1}, v, v_{i + 1},\dots, v_n)
%     \end{equation*}
%     is $R$-linear.
% \end{definition}

% \begin{definition}
%     Let $n\ge 1$ be a positive integer. Let $\omega\colon M\to R$ be a multilinear map. It is said to be \define{alternating} if $\omega(v_1,\dots, v_n) = 0$ whenever $v_i = v_j$ for some $1\le i < j \le n$.
% \end{definition}

% \begin{definition}
%     Let $n\ge 1$ be a positive integer. A function $\omega\colon M\to R$ is said to be a \define{determinant function} if $\omega$ is multilinear, alternating, and $\omega(I_n) = 1$, where $I_n$ denotes the identity matrix.
% \end{definition}

% \begin{lemma}
%     Let $\omega\colon M\to R$ be an alternating map. Then 
%     \begin{equation*}
%         \omega(v_1,\dots, v_i,\dots, v_j,\dots, v_n) = -\omega(v_1,\dots,v_j,\dots,v_i,\dots,v_n),
%     \end{equation*}
%     where on the right hand side, $v_j$ is in the $i$-th place, $v_i$ is in the $j$-th place, and all other entries remain unchanged.
% \end{lemma}

% \begin{lemma}
%     Let $\omega\colon M\to R$ be a multilinear map. If $\omega(A) = 0$ whenever two adjacent entries (i.e. columns) of $A$ are equal, then $\omega$ is alternating.
% \end{lemma}
% \begin{proof}
    
% \end{proof}

\section{Triangulation, Diagonalization, and Primary Decomposition}

\subsection{Eigenvalues and Eigenvectors}

\begin{definition}
    Let $V$ be a vector space over the field $F$ and let $T\colon V\to V$ be a linear map. A \define{eigenvalue} of $T$ is a scalar $\lambda\in F$ such that there is a non-zero vector $\alpha\in V$ with $T\alpha = \lambda\alpha$. 

    If $\lambda$ is an eigenvalue of $T$, then 
    \begin{enumerate}[label=(\roman*)]
        \item any $\alpha\in V$ such that $T\alpha = \lambda\alpha$ is called an eigenvector of $T$ associated to the eigenvalue $\lambda$.
        \item the collection of all $\alpha\in V$ such that $T\alpha = \lambda\alpha$ is called the \define{eigenspace} of $T$ associated to the eigenvalue $\lambda$.
    \end{enumerate}
\end{definition}

\begin{theorem}\thlabel{eigenvalues}
    Let $T\colon V\to V$ be a linear map on a finite-dimensional space $V$ and let $\lambda\in F$. The following are equivalent: 
    \begin{enumerate}[label=(\arabic*)]
        \item $\lambda$ is an eigenvalue of $T$. 
        \item The operator $T - \lambda I$ is not invertible. 
        \item $\det(T - \lambda I) = 0$.
    \end{enumerate}
\end{theorem}
\begin{proof}
    Trivial. 
\end{proof}

\begin{definition}
    Let $n$ be a positive integer and $A$ an $n\times n$ matrix with entries in $F$. The \define{characteristic polynomial} of $A$ is defined to be $\chi_A(X) = \det(X\cdot I - A)\in F[X]$.

    Given a linear map $T\colon V\to V$ where $V$ is a finite-dimensional vector space over $F$, define the characteristic polynomial of $T$ to be the characteristic polynomial of its matrix representation with respect to any basis of $V$.
\end{definition}

\begin{remark}
    The definition and \thref{eigenvalues} immediately imply that $\lambda\in F$ is an eigenvalue if and only if $\chi_T(\lambda) = 0$.
\end{remark}

\begin{definition}
    Let $T\colon V\to V$ be a linear map on a finite-dimensional vector space $V$. We say that $T$ is \define{diagonalizable} if there is a basis for $V$, each vector of which is an eigenvector of $T$.
\end{definition}

\begin{remark}
    It is clear from the definition that $T$ is diagonalizable if and only if there is a basis of $V$ with respect to which $T$ is given by a diagonal matrix.
\end{remark}

\begin{lemma}\thlabel{eigenspaces-are-independent}
    Let $T\colon V\to V$ be a linear map on a finite-dimensional vector space $V$ over $F$. Let $\lambda_1,\dots,\lambda_k\in F$ be the distinct eigenvalues of $T$ and let $W_i$ denote the eigenspace of $T$ associated with $\lambda_i$ for $1\le i\le k$. If $W = W_1 + \dots + W_k$, then 
    \begin{equation*}
        \dim W = \dim W_1 + \dots + \dim W_k.
    \end{equation*}
\end{lemma}
\begin{proof}
    It suffices to show that the given sum is direct. Indeed, suppose $\beta_i\in W_i$ for $1\le i\le k$ are such that $\beta_1 + \dots + \beta_k = 0$. Using Lagrange's method of interpolation, choose a polynomial $h_i(X)\in F[X]$ such that 
    \begin{equation*}
        h_i(\lambda_j) = 
        \begin{cases}
            1 & i = j\\
            0 & \text{otherwise}.
        \end{cases}
    \end{equation*}
    Then 
    \begin{equation*}
        0 = h_i(T)\left(\beta_1 + \dots + \beta_k\right) = \beta_i
    \end{equation*}
    for $1\le i\le n$.
\end{proof}

As a result, we obtain: 

\begin{theorem}
    Let $T\colon V\to V$ be a linear operator on a finite-dimensional vector space $V$ over a field $F$. Let $\lambda_1,\dots,\lambda_k\in F$ be the distinct eigenvalues of $T$ and let $W_i$ be the eigenspace of $T$ associated with $\lambda_i$ for $1\le i\le k$. Then the following are equivalent: 
    \begin{enumerate}[label=(\arabic*)]
        \item $T$ is diagonalizable.
        \item The characteristic polynomial for $T$ is 
        \begin{equation*}
            \chi(X) = (X - \lambda_1)^{d_1}\cdots(X - \lambda_k)^{d_k},
        \end{equation*}
        where $\dim W_i = d_i$ for $1\le i\le k$. 
        \item $\dim W_1 + \dots + \dim W_k = \dim V$.
    \end{enumerate}
\end{theorem}
\begin{proof}
    The implication $(1)\implies(2)$ is clear by considering the matrix representation of $T$ with respect to a suitable basis. Further, the implication $(2)\implies(3)$ is clear from the the fact that the degree of the characteristic polynomial is equal to the dimension of $V$. Finally, the implication $(3)\implies(1)$ follows from \thref{eigenspaces-are-independent}, since that would imply $V = W_1 + \dots + W_k$, that is, $V$ has a basis consisting of eigenvectors of $T$.
\end{proof}

\subsection{The Minimal Polynomial}

\begin{definition}
    Let $T\colon V\to V$ be a linear operator on a finite-dimensional vector space $V$ over a field $F$. Let $\frakA$ denote the set of all polynomials $f(X)\in F[X]$ such that $f(T) = 0$ as a linear operator. Clearly $\frakA$ is an ideal in $F[X]$. The unique\footnote{Because $(F[X])^\times = F^\times$.} monic generator of $\frakA$ is called the \define{minimal polynomial} for $T$.
\end{definition}

\begin{remark}
    Since $F[X]$ is a Euclidean domain with the Euclidean function given by the degree map, the minimal polynomial is the unique monic polynomial in $\frakA$ having the smallest degree.
\end{remark}

\begin{proposition}
    Let $T\colon V\to V$ be a linear operator on a finite-dimensional vector space $V$ over a field $F$. Then $\lambda\in F$ is a root of the characteristic polynomial of $T$ if and only if it is a root of the minimal polynomial of $T$.
\end{proposition}
\begin{proof}
    Let $p(X)\in F[X]$ be the minimal polynomial for $T$ and let $\chi(X)\in F[X]$ denote the characteristic polynomial. Suppose first that $p(\lambda) = 0$. Then $p(X) = (X - \lambda)q(X)$ for some polynomial $q(X)\in F[X]$. Since $\deg q < \deg p$, we must have $q(T)\ne 0$. Choose a vector $\beta\in V$ such that $\alpha\coloneq q(T)\beta\ne 0$. Then 
    \begin{equation*}
        0 = p(T)\beta = (T - \lambda I)q(T)\beta = (T - \lambda I)\alpha,
    \end{equation*}
    so that $\lambda$ is an eigenvalue of $T$, whence $\chi(\lambda) = 0$.

    Conversely, suppose $\chi(\lambda) = 0$, that is, $\lambda$ is an eigenvalue of $T$, so there exists a non-zero vector $\alpha\in V$ with $T\alpha = \lambda\alpha$. Then 
    \begin{equation*}
        0 = p(T)\alpha = p(\lambda)\alpha\implies p(\lambda) = 0,
    \end{equation*}
    thereby completing the proof.
\end{proof}

\begin{theorem}[Cayley-Hamilton]\thlabel{cayley-hamilton}
    Let $T\colon V\to V$ be a linear operator on a finite-dimensional vector space $V$ over a field $F$. If $\chi(X)\in F[X]$ denotes the characteristic polynomial of $T$, then $\chi(T) = 0$.
\end{theorem}
\begin{proof}
    % TODO: Add in later.
\end{proof}

 
\end{document}