\documentclass[11pt]{article}

\usepackage[utf8]{inputenc} % allow utf-8 input
\usepackage[T1]{fontenc}    % use 8-bit T1 fonts
\usepackage{hyperref}       % hyperlinks
\usepackage{url}            % simple URL typesetting
\usepackage{booktabs}       % professional-quality tables
\usepackage{amsfonts}       % blackboard math symbols
\usepackage{nicefrac}       % compact symbols for 1/2, etc.
\usepackage{microtype}      % microtypography
\usepackage{graphicx}
\usepackage{natbib}
\usepackage{doi}
\usepackage{amssymb}
\usepackage{bbm}
\usepackage{amsthm}
\usepackage{amsmath}
\usepackage{xcolor}
\usepackage{theoremref}
\usepackage{enumitem}
\usepackage{fouriernc}
\usepackage{mdframed}
\usepackage{mathrsfs}
\setlength{\marginparwidth}{2cm}
\usepackage{todonotes}
\usepackage{stmaryrd}
\usepackage[all,cmtip]{xy} % For diagrams, praise the Freyd-Mitchell theorem 
\usepackage{marvosym}
\usepackage{geometry}
\usepackage{titlesec}
\usepackage{mathtools}
\usepackage{tikz}
\usetikzlibrary{cd}
\usepackage{epigraph}
\setlength\epigraphwidth{0.4\textwidth}

\renewcommand{\qedsymbol}{$\blacksquare$}
% \renewcommand{\familydefault}{\sfdefault} % Do you want this font? 

% Uncomment to override  the `A preprint' in the header
% \renewcommand{\headeright}{}
% \renewcommand{\undertitle}{}
% \renewcommand{\shorttitle}{}

\hypersetup{
    pdfauthor={Swayam Chube},
    colorlinks=true,
	citecolor=blue,
}

\newtheoremstyle{thmstyle}%               % Name
  {}%                                     % Space above
  {}%                                     % Space below
  {}%                             % Body font
  {}%                                     % Indent amount
  {\bfseries\scshape}%                            % Theorem head font
  {.}%                                    % Punctuation after theorem head
  { }%                                    % Space after theorem head, ' ', or \newline
  {\thmname{#1}\thmnumber{ #2}\thmnote{ (#3)}}%                                     % Theorem head spec (can be left empty, meaning `normal')

\newtheoremstyle{defstyle}%               % Name
  {}%                                     % Space above
  {}%                                     % Space below
  {}%                                     % Body font
  {}%                                     % Indent amount
  {\bfseries\scshape}%                            % Theorem head font
  {.}%                                    % Punctuation after theorem head
  { }%                                    % Space after theorem head, ' ', or \newline
  {\thmname{#1}\thmnumber{ #2}\thmnote{ (#3)}}%                                     % Theorem head spec (can be left empty, meaning `normal')

\theoremstyle{thmstyle}
\newtheorem{theorem}{Theorem}[section]
\newtheorem{lemma}[theorem]{Lemma}
\newtheorem{proposition}[theorem]{Proposition}

\theoremstyle{defstyle}
\newtheorem{definition}[theorem]{Definition}
\newtheorem{corollary}[theorem]{Corollary}
\newtheorem{porism}[theorem]{Porism}
\newtheorem{remark}[theorem]{Remark}
\newtheorem{interlude}[theorem]{Interlude}
\newtheorem{example}[theorem]{Example}
\newtheorem*{notation}{Notation}
\newtheorem*{claim}{Claim}

% Common Algebraic Structures
\newcommand{\R}{\mathbb{R}}
\newcommand{\Q}{\mathbb{Q}}
\newcommand{\Z}{\mathbb{Z}}
\newcommand{\N}{\mathbb{N}}
\newcommand{\bbC}{\mathbb{C}} 
\newcommand{\K}{\mathbb{K}} % Base field which is either \R or \bbC
\newcommand{\F}{\mathbb{F}} % Base field which is either \R or \bbC
\newcommand{\calA}{\mathcal{A}} % Banach Algebras
\newcommand{\calB}{\mathcal{B}} % Banach Algebras
\newcommand{\calI}{\mathcal{I}} % ideal in a Banach algebra
\newcommand{\calJ}{\mathcal{J}} % ideal in a Banach algebra
\newcommand{\frakM}{\mathfrak{M}} % sigma-algebra
\newcommand{\calO}{\mathcal{O}} % Ring of integers
\newcommand{\bbA}{\mathbb{A}} % Adele (or ring thereof)
\newcommand{\bbI}{\mathbb{I}} % Idele (or group thereof)

% Categories
\newcommand{\catTopp}{\mathbf{Top}_*}
\newcommand{\catGrp}{\mathbf{Grp}}
\newcommand{\catTopGrp}{\mathbf{TopGrp}}
\newcommand{\catSet}{\mathbf{Set}}
\newcommand{\catTop}{\mathbf{Top}}
\newcommand{\catRing}{\mathbf{Ring}}
\newcommand{\catCRing}{\mathbf{CRing}} % comm. rings
\newcommand{\catMod}{\mathbf{Mod}}
\newcommand{\catMon}{\mathbf{Mon}}
\newcommand{\catMan}{\mathbf{Man}} % manifolds
\newcommand{\catDiff}{\mathbf{Diff}} % smooth manifolds
\newcommand{\catAlg}{\mathbf{Alg}}
\newcommand{\catRep}{\mathbf{Rep}} % representations 
\newcommand{\catVec}{\mathbf{Vec}}

% Group and Representation Theory
\newcommand{\chr}{\operatorname{char}}
\newcommand{\Aut}{\operatorname{Aut}}
\newcommand{\GL}{\operatorname{GL}}
\newcommand{\im}{\operatorname{im}}
\newcommand{\tr}{\operatorname{tr}}
\newcommand{\id}{\mathbf{id}}
\newcommand{\cl}{\mathbf{cl}}
\newcommand{\Gal}{\operatorname{Gal}}
\newcommand{\Tr}{\operatorname{Tr}}
\newcommand{\sgn}{\operatorname{sgn}}
\newcommand{\Sym}{\operatorname{Sym}}
\newcommand{\Alt}{\operatorname{Alt}}

% Commutative and Homological Algebra
\newcommand{\spec}{\operatorname{spec}}
\newcommand{\mspec}{\operatorname{m-spec}}
\newcommand{\Spec}{\operatorname{Spec}}
\newcommand{\MaxSpec}{\operatorname{MaxSpec}}
\newcommand{\Tor}{\operatorname{Tor}}
\newcommand{\tor}{\operatorname{tor}}
\newcommand{\Ann}{\operatorname{Ann}}
\newcommand{\Supp}{\operatorname{Supp}}
\newcommand{\Hom}{\operatorname{Hom}}
\newcommand{\End}{\operatorname{End}}
\newcommand{\coker}{\operatorname{coker}}
\newcommand{\limit}{\varprojlim}
\newcommand{\colimit}{%
  \mathop{\mathpalette\colimit@{\rightarrowfill@\textstyle}}\nmlimits@
}
\makeatother


\newcommand{\fraka}{\mathfrak{a}} % ideal
\newcommand{\frakb}{\mathfrak{b}} % ideal
\newcommand{\frakc}{\mathfrak{c}} % ideal
\newcommand{\frakf}{\mathfrak{f}} % face map
\newcommand{\frakg}{\mathfrak{g}}
\newcommand{\frakh}{\mathfrak{h}}
\newcommand{\frakm}{\mathfrak{m}} % maximal ideal
\newcommand{\frakn}{\mathfrak{n}} % naximal ideal
\newcommand{\frakp}{\mathfrak{p}} % prime ideal
\newcommand{\frakq}{\mathfrak{q}} % qrime ideal
\newcommand{\fraks}{\mathfrak{s}}
\newcommand{\frakt}{\mathfrak{t}}
\newcommand{\frakz}{\mathfrak{z}}
\newcommand{\frakA}{\mathfrak{A}}
\newcommand{\frakB}{\mathfrak{B}}
\newcommand{\frakI}{\mathfrak{I}}
\newcommand{\frakJ}{\mathfrak{J}}
\newcommand{\frakK}{\mathfrak{K}}
\newcommand{\frakL}{\mathfrak{L}}
\newcommand{\frakN}{\mathfrak{N}} % nilradical 
\newcommand{\frakO}{\mathfrak{O}} % dedekind domain
\newcommand{\frakP}{\mathfrak{P}} % Prime ideal above
\newcommand{\frakQ}{\mathfrak{Q}} % Qrime ideal above 
\newcommand{\frakR}{\mathfrak{R}} % jacobson radical
\newcommand{\frakU}{\mathfrak{U}}
\newcommand{\frakV}{\mathfrak{V}}
\newcommand{\frakW}{\mathfrak{W}}
\newcommand{\frakX}{\mathfrak{X}}

% General/Differential/Algebraic Topology 
\newcommand{\scrA}{\mathscr{A}}
\newcommand{\scrB}{\mathscr{B}}
\newcommand{\scrC}{\mathscr{C}}
\newcommand{\scrD}{\mathscr{D}}
\newcommand{\scrF}{\mathscr{F}}
\newcommand{\scrM}{\mathscr{M}}
\newcommand{\scrN}{\mathscr{N}}
\newcommand{\scrP}{\mathscr{P}}
\newcommand{\scrO}{\mathscr{O}} % sheaf
\newcommand{\scrR}{\mathscr{R}}
\newcommand{\scrS}{\mathscr{S}}
\newcommand{\bbH}{\mathbb H}
\newcommand{\Int}{\operatorname{Int}}
\newcommand{\psimeq}{\simeq_p}
\newcommand{\wt}[1]{\widetilde{#1}}
\newcommand{\RP}{\mathbb{R}\text{P}}
\newcommand{\CP}{\mathbb{C}\text{P}}

% Miscellaneous
\newcommand{\wh}[1]{\widehat{#1}}
\newcommand{\calM}{\mathcal{M}}
\newcommand{\calP}{\mathcal{P}}
\newcommand{\onto}{\twoheadrightarrow}
\newcommand{\into}{\hookrightarrow}
\newcommand{\Gr}{\operatorname{Gr}}
\newcommand{\Span}{\operatorname{Span}}
\newcommand{\ev}{\operatorname{ev}}
\newcommand{\weakto}{\stackrel{w}{\longrightarrow}}

\newcommand{\define}[1]{\textcolor{blue}{\textit{#1}}}
% \newcommand{\caution}[1]{\textcolor{red}{\textit{#1}}}
\newcommand{\important}[1]{\textcolor{red}{\textit{#1}}}
\renewcommand{\mod}{~\mathrm{mod}~}
\renewcommand{\le}{\leqslant}
\renewcommand{\leq}{\leqslant}
\renewcommand{\ge}{\geqslant}
\renewcommand{\geq}{\geqslant}
\newcommand{\Res}{\operatorname{Res}}
\newcommand{\floor}[1]{\left\lfloor #1\right\rfloor}
\newcommand{\ceil}[1]{\left\lceil #1\right\rceil}
\newcommand{\gl}{\mathfrak{gl}}
\newcommand{\ad}{\operatorname{ad}}
\newcommand{\Stab}{\operatorname{Stab}}
\newcommand{\bfX}{\mathbf{X}}
\newcommand{\Ind}{\operatorname{Ind}}
\newcommand{\bfG}{\mathbf{G}}
\newcommand{\rank}{\operatorname{rank}}
\newcommand{\calo}{\mathcal{o}}
\newcommand{\frako}{\mathfrak{o}}
\newcommand{\Cl}{\operatorname{Cl}}

\newcommand{\idim}{\operatorname{idim}}
\newcommand{\pdim}{\operatorname{pdim}}
\newcommand{\Ext}{\operatorname{Ext}}
\newcommand{\co}{\operatorname{co}}
\newcommand{\bfO}{\mathbf{O}}
\newcommand{\bfF}{\mathbf{F}} % Fitting Subgroup
\newcommand{\Syl}{\operatorname{Syl}}
\newcommand{\nor}{\vartriangleleft}
\newcommand{\noreq}{\trianglelefteqslant}
\newcommand{\subnor}{\nor\!\nor}
\newcommand{\Soc}{\operatorname{Soc}}
\newcommand{\core}{\operatorname{core}}
\newcommand{\Sd}{\operatorname{Sd}}
\newcommand{\mesh}{\operatorname{mesh}}
\newcommand{\sminus}{\setminus}
\newcommand{\diam}{\operatorname{diam}}
\newcommand{\Ass}{\operatorname{Ass}}
\newcommand{\projdim}{\operatorname{proj~dim}}
\newcommand{\injdim}{\operatorname{inj~dim}}
\newcommand{\gldim}{\operatorname{gl~dim}}
\newcommand{\embdim}{\operatorname{emb~dim}}
\newcommand{\hght}{\operatorname{ht}}
\newcommand{\depth}{\operatorname{depth}}
\newcommand{\ul}[1]{\underline{#1}}
\newcommand{\type}{\operatorname{type}}
\newcommand{\CM}{\operatorname{CM}}
\newcommand{\cech}[1]{\mathbin{\check{#1}}}
\newcommand{\cdim}{\operatorname{cdim}}
\newcommand{\Der}{\operatorname{Der}}
\newcommand{\trdeg}{\operatorname{trdeg}}
\newcommand{\Kom}{\operatorname{Kom}}
\newcommand{\scrK}{\mathscr{K}}
\newcommand{\scrI}{\mathscr{I}}
\newcommand{\Mor}{\operatorname{Mor}}
\newcommand{\op}{\mathrm{op}}
\newcommand{\cone}{\operatorname{cone}}
\newcommand{\cyl}{\operatorname{cyl}}
\newcommand{\bbone}{\mathbbm{1}}

\geometry {
    margin = 1in
}

\titleformat
{\section}
[block]
{\Large\bfseries\sffamily}
{\S\thesection}
{0.5em}
{\centering}
[]


\titleformat
{\subsection}
[block]
{\normalfont\bfseries\sffamily}
{\S\S}
{0.5em}
{\centering}
[]


\begin{document}
\title{Derived and Triangulated Categories}
\author{Swayam Chube}
\date{Last Updated: \today}
\maketitle
\tableofcontents

\hrulefill 

We fix some notation before proceeding. Categories will usually denoted by calligraphic symbols such as $\scrA,\scrB,\scrC,\scrD$. The opposite category of a category $\scrA$ is denoted by $\scrA^{\op}$. Corresponding to each object $A\in\scrA$, there is an object $A^{\op}\in\scrA^{\op}$ and corresponding to each morphism $f\colon A\to B$ in $\scrA$, there is a morphism $f^{\op}\in\scrA^{\op}$. If $A\xrightarrow{f} B\xrightarrow{g} C$ are morphisms in $\scrA$, then $C^{\op}\xrightarrow{g^{\op}} B^{\op}\xrightarrow{f^{\op}} A^{\op}$ with $f^\op\circ g^\op = (g\circ f)^\op$. 

Throughout this article, all chain complexes will be assumed to be cochain complexes, that is, the boundary maps have positive degree: 
\begin{equation*}
	\cdots\to X^n\xrightarrow{d_n} X^{n + 1}\xrightarrow{d_{n + 1}} X^{n + 2}\to\cdots,
\end{equation*}
and its cohomology is denoted by 
\begin{equation*}
	H^i\left(X^\bullet\right) = \frac{\ker d_i}{\im d_{i - 1}}.
\end{equation*}

\section{Localization of Categories}
\begin{theorem}
	Let $\scrA$ be a category, and $S$ be a class of morphisms in $\scrA$. Then there is a category $\scrA[S^{-1}]$ and a functor $Q\colon\scrA\to\scrA[S^{-1}]$ such that for every functor $F\colon\scrA\to\scrB$ such that $F(s)$ is an isomorphism in $\scrB$ for each $s\in S$, there is a unique functor $G\colon\scrA[S^{-1}]\to\scrB$ making 
	\begin{equation*}
		\xymatrix {
			\scrA\ar[r]^F\ar[d]_Q & \scrB\\
			\scrA[S^{-1}]\ar@{.>}[ru]_-{\exists!~G}
		}
	\end{equation*}
	commute. Further, the pair $(\scrA[S^{-1}], Q)$ is unique up to a unique isomorphism of categories and is called the \define{localization} of $\scrA$ by the class of morphisms $S$.
\end{theorem}

\subsection{Localizing Classes}

Quite generally, the category $\scrA[S^{-1}]$ is quite ugly and difficult to work with. Therefore, we restrict ourselves to a more managable class $S$ of localizing morphisms.

\begin{definition}
	Let $\scrA$ be a category. A class of morphisms $S$ in $\scrA$ is said to be a \define{localizing class} if 
	\begin{enumerate}[label=(\textbf{LC}\arabic*)]
		\item For any object $M\in\scrA$, $\id_A\in S$.\label{lc1}
		\item If $s, t$ are composable morphisms in $S$, then so is their composition.\label{lc2}
		\item 
		\begin{enumerate}
			\item Every diagram of the form 
			\begin{equation*}
				\xymatrix {
					& L\ar[d]^s\\
					M\ar[r]_f & N
				}
			\end{equation*}
			with $f\in\Mor(\scrA)$ and $s\in S$ can be enlarged to a commutative square 
			\begin{equation*}
				\xymatrix {
					K\ar@{.>}[r]^g\ar@{.>}[d]_t & L\ar[d]^s\\
					M\ar[r]_f & N
				}
			\end{equation*}
			for some $K\in\scrA$, $g\in\Mor(\scrA)$, and $s\in S$. \label{pullback-type}
			\item Every diagram of the form 
			\begin{equation*}
				\xymatrix {
					N\ar[r]^f\ar[d]_s & M\\
					L
				}
			\end{equation*}
			with $f\in\Mor(\scrA)$ and $s\in S$ can be enlarged to a commutative square 
			\begin{equation*}
				\xymatrix {
					N\ar[r]^f\ar[d]_s & M\ar@{.>}[d]^t\\
					L\ar@{.>}[r]_g & K
				}
			\end{equation*}
			with $K\in\scrA$, $g\in\Mor(\scrA)$, and $t\in S$.\label{pushout-type}
		\end{enumerate}
		\item Let $f,g\colon M\to N$ be two morphisms in $\scrA$. Then \label{annihilation-condition}
		\begin{equation*}
			\exists~s\in S\text{ such that }s\circ f = s\circ g\iff \exists~t\in S\text{ such that }f\circ t = g\circ t.
		\end{equation*}
	\end{enumerate}
\end{definition}

\begin{center}
Clearly, if $S$ is a localizing class in $\scrA$, then $S^\op = \{s^\op\colon s\in S\}$ is a localizing class in $\scrA^\op$.
\end{center}

Our goal will now be to describe $\scrA[S^{-1}]$ given that $S$ is a localizing class of morphisms in $\scrA$. Define a \define{left roof} between two objects $M$ and $N$ in $\scrA$ to be a diagram of the form 
\begin{equation*}
	\xymatrix {
		& L\ar[ld]_s\ar[rd]^f & \\
		M & & N
	}
\end{equation*}
where $s\in S$ and $f\in\Mor(\scrA)$. Two left roofs 
\begin{equation*}
	\xymatrix {
		& L\ar[ld]_s\ar[rd]^f & \\
		M & & N
	}
	\quad\text{and}\quad 
	\xymatrix {
		& K\ar[ld]_t\ar[rd]^g & \\
		M & & N
	}
\end{equation*}
are said to be \define{equivalent} if there exists an object $H\in\scrA$ and morphisms $p\colon H\to L$ and $q\colon H\to K$ making the diagram
\begin{equation*}
	\xymatrix {
		& L\ar[ld]_s\ar[rd]^f & \\
		M & H\ar[u]_{p}\ar[d]^{q} & N\\
		& K\ar[lu]^t\ar[ru]_g & 
	}
\end{equation*}
commute and $s\circ p = q\circ t\in S$. It can be checked that this is indeed an equivalence relation.

Analogously a \define{right roof} between two objects $M$ and $N$ in $\scrA$ is defined to ba diagram of the form 
\begin{equation*}
	\xymatrix {
		& L &\\
		M\ar[ru]^f & & N\ar[lu]_s
	}
\end{equation*}
where $s\in S$ and $f\in\Mor(\scrA)$. Clearly there is a natural bijection between the left roofs in $\scrA$ and the right roofs in $\scrA^\op$ with respect to $S$ and $S^\op$ respectively. Two right roofs 
\begin{equation*}
	\xymatrix {
		& L &\\
		M\ar[ru]^f & & N\ar[lu]_s
	}\quad\text{and}\quad
	\xymatrix {
		& K &\\
		M\ar[ru]^g & & N\ar[lu]_t
	}
\end{equation*}
are said to be equivalent if there exists an object $H\in\scrA$ and morphisms $p\colon L\to H$ and $q\colon K\to H$ such that 
\begin{equation*}
	\xymatrix {
		& L\ar[d]^p & \\
		M\ar[ru]^f\ar[rd]_g & H & N\ar[lu]_s\ar[ld]^t\\
		& K\ar[u]_q & 
	}
\end{equation*}
commutes. Again, it can be checked that this is indeed an equivalence relation. In fact, a shorter way to conclude this is to move from $\scrA$ to $\scrA^\op$ since left roofs are mapped to right roofs. It is clear that two left roofs in $\scrA$ are equivalent if and only if the corresponding right roofs in $\scrA^\op$ are equivalent.

Next, we show how to ``compose'' two equivalence classes of left roofs. Begin by considering two representatives 
\begin{equation*}
	\xymatrix {
		& L\ar[ld]_s\ar[rd]^f & \\
		M & & N
	}
	\quad\text{and}\quad 
	\xymatrix {
		& K\ar[ld]_t\ar[rd]^g & \\
		N & & P.
	}
\end{equation*}
According to \ref{pullback-type}, we obtain a commutative diagram 
\begin{equation*}
	\xymatrix {
		& & U\ar[ld]_u\ar[rd]^h & & \\
		& L\ar[ld]_s\ar[rd]^f & & K\ar[ld]_t\ar[rd]^g & \\
		M & & N & & P
	}
\end{equation*}
with $u\in S$. Define the composition of the aforementioned equivalence classes to be the left roof 
\begin{equation*}
	\xymatrix {
		& U\ar[ld]_{s\circ u}\ar[rd]^{g\circ h} & \\
		M & & N
	}
\end{equation*}
since $s\circ u\in S$. It is a bit tedious, but it can be checked that this ``composition'' is well-defined. Once this is done, it is clear that the ``composition'' must be associative. That is, given three representatives 
\begin{equation*}
	\xymatrix {
		& L\ar[ld]_s\ar[rd]^f & \\
		M & & N
	}
	\quad
	\xymatrix {
		& K\ar[ld]_t\ar[rd]^g & \\
		N & & P.
	}
	\quad\text{and}
	\xymatrix { 
		& U\ar[ld]_u\ar[rd]^h &\\
		P & & R
	}
\end{equation*}
using \thref{pullback-type} repeatedly, we can complete this to a commutative diagram
\begin{equation*}
	\xymatrix {
		& & & C\ar[ld]_{u'}\ar[rd]^c & & &\\
		& & A\ar[ld]_{s'}\ar[rd]^{a} & & B\ar[ld]_{t'}\ar[rd]^b & & \\
		& L\ar[ld]_s\ar[rd]^f & & K\ar[ld]_t\ar[rd]^g & & U\ar[ld]_u\ar[rd]^h & \\
		M & & N & & P & & R
	}
\end{equation*}
and so it is clear that either composition 
\begin{equation*}
	\left(
	\xymatrix {
		& L\ar[ld]_s\ar[rd]^f & \\
		M & & N
	}
	\circ
	\xymatrix {
		& K\ar[ld]_t\ar[rd]^g & \\
		N & & P.
	}
	\right)
	\circ
	\xymatrix { 
		& U\ar[ld]_u\ar[rd]^h &\\
		P & & R
	}
\end{equation*}
or 
\begin{equation*}
	\xymatrix {
		& L\ar[ld]_s\ar[rd]^f & \\
		M & & N
	}
	\circ
	\left(
	\xymatrix {
		& K\ar[ld]_t\ar[rd]^g & \\
		N & & P.
	}
	\circ
	\xymatrix { 
		& U\ar[ld]_u\ar[rd]^h &\\
		P & & R
	}
	\right)
\end{equation*}
is equal to the equivalence class of the left roof 
\begin{equation*}
	\xymatrix {
		& C\ar[ld]_{s\circ s'\circ u'}\ar[rd]^{h\circ b\circ c} & \\
		M & & R.
	}
\end{equation*}
Finally, for each $M\in\scrA$, consider the left roof 
\begin{equation*}
	\xymatrix {
		& M\ar[ld]_{\id_M}\ar[rd]^{\id_M} & \\
		M & & M.
	}
\end{equation*}
For any left roof $\xymatrix { & L\ar[ld]_s\ar[rd]^f & \\ M & & N }$, one can compute their composition using the diagram 
\begin{equation*}
	\xymatrix {
		& & L\ar[ld]_s\ar[rd]^f & & \\
		& M\ar[ld]_{\id_M}\ar[rd]^{\id_M} & & L\ar[ld]_s\ar[rd]^f & \\
		M & & M & & N
	}
\end{equation*}
which yields the latter left roof.

\begin{mdframed}
Thus, we can define a category $\scrA_S$ where $\operatorname{ob}(\scrA_S) = \operatorname{ob}(\scrA)$, and $\Mor_{\scrA_S}(M, N)$ is the set of equivalence classes of left roofs equipped with composition and identity maps as defined above. 
\end{mdframed}

There is a natural functor $Q\colon\scrA\to\scrA_S$ which is the identity on objects and sends a morphism $f\colon M\to N$ in $\scrA$ to the equivalence class of the left roof 
\begin{equation*}
	\xymatrix {
		& M\ar[ld]_{\id_M}\ar[rd]^f & \\
		M & & N
	}
\end{equation*}
in $\scrA_S$. Indeed, it clearly takes $\id_M$ to the roof representing the identity at $M$ in $\scrA_S$; further, if $M\xrightarrow{f} N\xrightarrow{g} P$ are two composable morphisms, then we have a commutative diagram 
\begin{equation*}
	\xymatrix {
		&& M\ar[ld]_{\id_M}\ar[rd]^f && \\
		& M\ar[ld]_{\id_M}\ar[rd]^f & & N\ar[ld]_{\id_N}\ar[rd]^g & \\
		M & & N & & P
	}
\end{equation*}
so that the composition of the bottom two left roofs is 
\begin{equation*}
	\xymatrix {
		& M\ar[ld]_{\id_M}\ar[rd]^{g\circ f} & \\
		M & & P
	} = Q(g\circ f).
\end{equation*}
Thus $Q$ is indeed a functor. Finally, we claim that the pair $(Q,\scrA_S)$ has the universal property of localization. Indeed, let $F\colon\scrA\to\scrB$ be a functor sending every $s\in S$ to an isomorphism $F(s)$ in $\scrB$. Define a functor $G\colon\scrA_S\to\scrB$ such that 
\begin{equation*}
	G(A) = F(A) \quad\text{for every object }A\in\scrA,
\end{equation*}
and 
\begin{equation*}
	G\left(
		\xymatrix{
			& L\ar[ld]_{s}\ar[rd]^f & \\
			M & & N
		}
	\right) = F(f)\circ F(s)^{-1}.
\end{equation*}
It is easily checked that this is well-defined on the equivalence class of left roofs. That $G$ is a functor is also a trivial verification, and by construction, $F = G\circ Q$.

Finally, we must show that such a $G$ is unique. Indeed, if $F = G\circ Q$, for each object $A\in\scrA$, we must have $F(A) = G(Q(A)) = G(A)$. Now, a left roof 
\begin{equation*}
	\xymatrix { 
		& L\ar[ld]_s\ar[rd]^f & \\
		M & & N
	}
\end{equation*}
can be decomposed as the composition 
\begin{equation*}
	\xymatrix {
		& L\ar[ld]_s\ar[rd]^{\id_L} & & L\ar[ld]_{\id_L}\ar[rd]^f & \\
		M & & L & & N
	}
\end{equation*}
which is easy to see by completing the diagram above by putting an $L$ at the peak and identity morphisms from it to both the $L$'s below it. But note that 
\begin{equation*}
	\xymatrix {
		& L\ar[ld]_{s}\ar[rd]^{\id_L} & \\
		M & & L
	}\text{ is the inverse of }
	\xymatrix {
		& L\ar[ld]_{\id_L}\ar[rd]^s & \\
		L && M
	}
\end{equation*}
so that 
\begin{equation*}
	G\left(
		\xymatrix {
			& L\ar[ld]_{s}\ar[rd]^{\id_L} & \\
			M & & L
		}
	\right) = 
	G\left(
		\xymatrix {
			& L\ar[ld]_{\id_L}\ar[rd]^s & \\
			L && M
		}
	\right)^{-1} = F(s)^{-1},
\end{equation*}
and hence 
\begin{equation*}
	G\left(\xymatrix{
		& L\ar[ld]_s\ar[rd]^f & \\
		M & & N
	}\right) = F(f)\circ F(s)^{-1},
\end{equation*}
which completes the proof of uniqueness. We have therefore shown: 
\begin{theorem}
	Let $\scrA$ be a category and $S$ be a localizing class of morphisms in $\scrA$. Then the functor $Q\colon\scrA\to\scrA_S$ as described above is the localization of the category $\scrA$ at $S$.
\end{theorem}

\subsection{Localization and Subcategories}

\begin{theorem}\thlabel{localizing-subcategories}
	Let $\scrA$ be a category, $\scrB\subseteq\scrA$ a full subcategory, and $S$ a localizing class of morphisms in $\scrA$. Suppose 
	\begin{enumerate}[label=(\textbf{LS}\arabic*)]
		\item $S_{\scrB} = S\cap\Mor(\scrB)$ is a localizing class in $\scrB$, and \label{ls1}
		\item for each morphism $s\colon N\to M$ in $S$ with $M\in\scrB$, there exists a morphism $u\colon P\to N$ with $P\in\scrB$ such that $s\circ u\in S$.\label{ls2}
	\end{enumerate}
	Then the induced functor $\scrB[S_{\scrB}^{-1}]\to\scrA[S^{-1}]$ is fully faithful.
	\begin{equation*}
		\xymatrix {
			\scrB\ar@{^{(}->}[r]\ar[d]_{Q_B} & \scrA\ar[d]^{Q_A}\\
			\scrB[S_\scrB^{-1}]\ar[r] & \scrA[S^{-1}]
		}
	\end{equation*}
\end{theorem}
\begin{proof}
	Since $S_\scrB$ is a localizing class in $\scrB$, by tracing the arrows in the commutative diagram of functors above, the map $\scrB[S_\scrB^{-1}]\to\scrA[S^{-1}]$ explicitly sends a roof in $\scrB$ to the equivalence class of the same roof in $\scrA[S^{-1}]$. 

	First, we show that the map is full. Let 
	\begin{equation*}
		\xymatrix {
			& L\ar[ld]_s\ar[rd]^f & \\
			M & & N
		}
	\end{equation*}
	be a left roof in $\scrA[S^{-1}]$ with $M, N\in\scrB$. Then due to \ref{ls2}, there exists $U\in\scrB$ and a morphism $u\colon U\to L$ such that $s\circ u\in S$, and hence in $S_{\scrB}$.

	To see that the map is faithful, suppose two left roofs 
	\begin{equation*}
		\xymatrix {
			& L\ar[ld]_s\ar[rd]^f & \\
			M & & N
		}
		\quad\text{and}\quad 
		\xymatrix {
			& K\ar[ld]_t\ar[rd]^g & \\
			M && N
		}
	\end{equation*}
	in $\scrB[S_{\scrB}^{-1}]$ are equivalent in $\scrA[S^{-1}]$, that is, there exists an object $H\in\scrA$, and morphisms $p\colon H\to L$ and $q\colon H\to K$ in $\scrA$ such that 
	\begin{equation*}
		\xymatrix {
			& L\ar[ld]_s\ar[rd]^f & \\
			M & H\ar[u]_p\ar[d]^q & N\\
			& K\ar[lu]^t\ar[ru]_g & 
		}
	\end{equation*}
	commutes and $s\circ p = t\circ q\in S$. Hence, there exists an object $U\in\scrB$ and a morphism $u\colon U\to H$ such that $s\circ p\circ u = t\circ q \circ u\in S$, and hence in $S_{\scrB}$. Thus, the diagram 
	\begin{equation*}
		\xymatrix {
			& L\ar[ld]_s\ar[rd]^f & \\
			M & U\ar[u]_{p\circ u}\ar[d]^{q\circ u} & N\\
			& K\ar[lu]^t\ar[ru]_g & 
		}
	\end{equation*}
	commutes and consists of morphisms in $\scrB$. Thus, the two roofs are equivalent in $\scrB[S_{\scrB}^{-1}]$.
\end{proof}

\subsection{Localizing Additive Categories}

We begin by showing that one can ``take common demonimators'' for morphisms in $\scrA[S^{-1}]$.
\begin{lemma}\thlabel{common-denominator}
	Let $\scrA$ be a category (not necessarily additive) and $S$ a localizing class of morphisms in $\scrA$. Let 
	\begin{equation*}
		\xymatrix {
			& L_i\ar[ld]_{s_i}\ar[rd]^{f_i} &\\
			M && N
		}
	\end{equation*}
	be left roofs in $\scrA$ representing morphisms $\varphi_i\colon M\to N$ in $\scrA[S^{-1}]$ for $1\le i\le n$ respectively. Then there exists an object $L\in\scrA$ and morphisms $L\xrightarrow{s} M\in S$, and $g_i\colon L\to N$ for $1\le i\le n$ such that 
	\begin{equation*}
		\xymatrix {
			& L\ar[ld]_s\ar[rd]^{g_i} & \\
			M && N
		}
	\end{equation*}
	represents $\varphi_i$ for $1\le i\le n$.
\end{lemma}
\begin{proof}
	We prove this by induction on $n$. The base case $n = 1$ is trivial. Suppose now that $n > 1$ and that the statement has been proven for $n - 1$. Hence, there exists an object $K$ and a morphism $K\xrightarrow{t} M\in S$ such that 
	\begin{equation*}
		\xymatrix {
			& K\ar[rd]^{h_i}\ar[ld]_t & \\
			M & & N
		}
	\end{equation*}
	represents $\varphi_i$ for $1\le i\le n - 1$. Using \thref{pullback-type}, there exists a commutative diagram 
	\begin{equation*}
		\xymatrix {
			U\ar[r]^{v}\ar[d]_{u} & L_n\ar[d]^{s_n}\\
			K\ar[r]_{t} & M
		}
	\end{equation*}
	with $u\in S$. Set $s = s_n\circ v = t\circ u\in S$. Then the diagram 
	\begin{equation*}
		\xymatrix {
			& K\ar[ld]_t\ar[rd]^{h_i} & \\
			M & U\ar[u]_u\ar@{=}[d] & N\\
			& U\ar[lu]^s\ar[ru]_{h_i\circ u} & 
		}
	\end{equation*}
	commutes for $1\le i\le n - 1$, and 
	\begin{equation*}
		\xymatrix {
			& L_n\ar[rd]^{f_n}\ar[ld]_{s_n} & \\
			M & U\ar[u]_{v}\ar@{=}[d] & N\\
			& U\ar[lu]^{s_n\circ v}\ar[ru]_{f_n\circ v} & 
		}
	\end{equation*}
	commutes with $s_n\circ v = s\in S$. Set $g_i = h_i\circ u$ for $1\le i\le n - 1$ and $g_n = f_n\circ v$; then 
	\begin{equation*}
		\xymatrix {
			& U\ar[ld]_s\ar[rd]^{g_i} & \\
			M && N
		}
	\end{equation*}
	represents $\varphi_i$ for $1\le i\le n$, thereby completing the proof.
\end{proof}

Now let $\scrA$ be an \emph{additive category} and $S$ a localizing class of morphisms in $\scrA$. We shall show that $\scrA[S^{-1}]$ is naturally an additive category. For objects $M, N\in\scrA[S^{-1}]$ and morphisms 
\begin{equation*}
	\varphi = \xymatrix { 
		& L\ar[ld]_s\ar[rd]^f & \\
		M & & N
	}
	\quad\text{and}\quad 
	\psi = \xymatrix {
		& K\ar[ld]_t\ar[rd]^g & \\
		M & & N
	}
\end{equation*}
in $\scrA[S^{-1}]$; using \thref{common-denominator}, we can find an object $U$ and morphisms $U\xrightarrow{u} M\in S$ and $f', g'\colon U\to N$ such that 
\begin{equation*}
	\varphi = \xymatrix {
		& U\ar[ld]_u\ar[rd]^{f'} & \\
		M & & N
	}\quad\text{and}\quad 
	\psi = \xymatrix {
		& U\ar[ld]_{u}\ar[rd]^{g'} & \\
		M & & N.
	}
\end{equation*}
Define 
\begin{equation*}
	\varphi + \psi = \xymatrix {
		& U\ar[ld]_u\ar[rd]^{f' + g'} & \\
		M & & N.
	}
\end{equation*}
Note that there are three choices being made here: the choice of the representatives for $\varphi$ and $\psi$, and choice of ``common denominator'' for both morphisms. It follows that $\Mor_{\scrA[S^{-1}]}(M, N)$ has the structure of an abelian group. Further, it must be checked that 
\begin{equation*}
	\chi\circ(\varphi + \psi) = \chi\circ\varphi + \chi\circ\psi\quad\text{ and }\quad(\varphi + \psi)\circ\chi = \varphi\circ\chi + \psi\circ\chi
\end{equation*}
for suitably composable morphisms $\chi,\varphi,\psi$ in $\scrA[S^{-1}]$. The zero object in $\Mor_{\scrA[S^{-1}]}(M, N)$ is given by the morphism
\begin{equation*}
	\xymatrix {
		& M\ar[ld]_{\id_M}\ar[rd]^0 & \\
		M  && N
	}.
\end{equation*}
Finally, given objects $M, N\in\scrA[S^{-1}]$, define their direct sum/direct product to be the object $M\oplus N$ where the direct sum is taken in $\scrA$, and the canonical projections and injections are the images of those in $\scrA$. Again, it is straightforward, but must be checked, that these have the desired universal properties. In this way, $\scrA[S^{-1}]$ has been given a natural additive structure.

Finally, note that the localization functor $Q\colon\scrA\to\scrA[S^{-1}]$ is an additive functor. Indeed, if $f, g\colon M\to N$ are morphisms, then 
\begin{equation*}
	Q(f) = 
	\xymatrix {
		& M\ar[ld]_{\id_M}\ar[rd]^f & \\
		M & & N
	}\quad\text{ and }\quad 
	Q(g) = 
	\xymatrix {
		& M\ar[ld]_{\id_M}\ar[rd]^g & \\
		M & & N
	}
\end{equation*}
so that by definition,
\begin{equation*}
	Q(f) + Q(g) = 
	\xymatrix {
		& M\ar[rd]^{f + g}\ar[ld]_{\id_M} & \\
		M & & N
	} = Q(f + g).
\end{equation*}

Finally, we have 
\begin{theorem}
	Let $\scrA$ be an additive category and $S$ a localizing class of morphisms in $\scrA$. Then the category $\scrA[S^{-1}]$ is naturally an additive category and the localizing functor $Q\colon\scrA\to\scrA[S^{-1}]$ is additive. 

	Further, given any additive functor $F\colon\scrA\to\scrB$ such that $F(s)$ is an isomorphism in $\scrB$ for each $s\in S$, there exists a unique additive functor $G\colon\scrA[S^{-1}]\to B$ making 
	\begin{equation*}
	\xymatrix {
		\scrA\ar[r]^F\ar[d]_Q & \scrB\\
		\scrA[S^{-1}]\ar[ru]_G
	}
	\end{equation*}
	commute.
\end{theorem}
\begin{proof}
	We have already proved the first part of the theorem. As for the second part, suppose $\varphi,\psi\colon M\to N$ are two morphisms in $\scrA[S^{-1}]$. Using \thref{common-denominator}, we may suppose that they are represented by 
	\begin{equation*}
		\xymatrix {
			& L\ar[ld]_s\ar[rd]^f & \\
			M &  & N
		}\quad\text{ and }\quad 
		\xymatrix {
			& L\ar[ld]_s\ar[rd]^g & \\
			M & & N
		}
	\end{equation*}
	respectively. As a result, 
	\begin{equation*}
		G(\varphi + \psi) = F(f + g)F(s)^{-1} = F(f)F(s)^{-1} + F(g)F(s)^{-1} = G(\varphi) + G(\psi),
	\end{equation*}
	so that $G$ is an additive functor. That $G$ is unique has already been argued.
\end{proof}

\begin{lemma}\thlabel{morphism-being-zero}
	Let $\varphi\colon M\to N$ be a morphism in $\scrA[S^{-1}]$ represented by a left roof 
	\begin{equation*}
		\xymatrix {
			& L\ar[ld]_s\ar[rd]^f & \\
			M & & N
		}.
	\end{equation*}
	Then the following are equivalent: 
	\begin{enumerate}[label=(\arabic*)]
		\item $\varphi = 0$.
		\item There exists $t\in S$ such that $t\circ f = 0$.
		\item There exists $t\in S$ such that $f\circ t = 0$.
	\end{enumerate}
\end{lemma}
\begin{proof}
	Clearly (2) and (3) are equivalent due to \ref{annihilation-condition}. Now if $\varphi = 0$, then $Q(f)\circ Q(s)^{-1} = 0$, so that $Q(f) = 0$, i.e., 
	\begin{equation*}
		\xymatrix {
			& L\ar[ld]_{\id_L}\ar[rd]^f &\\
			L & & N
		}
	\end{equation*}
	represents $0$. Hence, there exists an object $H$ and and morphisms $p, q\colon H\to L$ such that 
	\begin{equation*}
		\xymatrix {
			& L\ar[ld]_{\id_L}\ar[rd]^f &\\
			L & H\ar[u]_p\ar[d]^q & N\\
			& L\ar[lu]^{\id_L}\ar[ru]_0 & 
		}
	\end{equation*}
	commutes and $p = q\in S$. The commutativity implies $f\circ p = 0$, so that $(1)\implies(2)$.

	Conversely, suppose $f\circ t = 0$ for some $t\colon H\to L\in S$. Then the diagram 
	\begin{equation*}
		\xymatrix {
			& L\ar[ld]_s\ar[rd]^f & \\
			M & H\ar[u]_t\ar[d]^{s\circ t} & N\\
			& M\ar[lu]^{\id_M}\ar[ru]_0 & 
		}
	\end{equation*}
	commutes with $s\circ t\in S$. This shows that $\varphi = 0$, thereby completing the proof.
\end{proof}

\begin{corollary}
	Let $M$ be an object in $\scrA$. Then the following are equivalent: 
	\begin{enumerate}[label=(\arabic*)]
		\item $Q(M) = 0$. 
		\item There exists an object $N\in\scrA$ such that the zero morphism $N\xrightarrow{0} M$ is in $S$.
		\item There exists an object $N\in\scrA$ such that the zero morphism $M\xrightarrow{0} N$ is in $S$.
	\end{enumerate}
\end{corollary}
\begin{proof}
	The equivalence of (2) and (3) follows from an immediate application of \thref{pullback-type} and \thref{pushout-type}. Now if $Q(M) = 0$, then $Q(\id_M) = 0$, so that by \thref{morphism-being-zero} there exists $s\in S$ with $\id_M\circ s = 0$, and hence $s = 0$. This proves (2).
	
	Conversely, if there is an object $N\in\scrA$ with $N\xrightarrow{0} M \in S$, then the image of this map, which is the zero map $Q(N)\xrightarrow{0} Q(M)$ must be an isomorphism. Thus $Q(N) = Q(M) = 0$.
\end{proof}

\begin{lemma}
	Let $f\colon M\to N$ be a morphism in $\scrA$. Then 
	\begin{enumerate}[label=(\arabic*)]
		\item If $f$ is monic, then so is $Q(f)$. 
		\item If $f$ is epic, then so is $Q(f)$.
	\end{enumerate}
\end{lemma}
\begin{proof}
	% TODO: Add in later
\end{proof}

\section{Derived Categories}

Let $\scrA$ be an abelian category. Let $\Kom(\scrA)$, $\Kom^+(\scrA)$, $\Kom^-(\scrA)$, and $\Kom^b(\scrA)$ denote the categories of chain complexes, bounded below chain complexes, bounded above chain complexes, and bounded chain complexes respectively. The morphisms in each category are precisely the chain maps. Clearly, all these categories are abelian. 

Identifying morphisms in each category up to homotopy, we obtain categories $\scrK(\scrA)$, $\scrK^+(\scrA)$, $\scrK^-(\scrA)$, and $\scrK^b(\scrA)$ respectively. We remark that these categories are generally \emph{not} abelian. 

\begin{definition}
	Let $\ast\in\{+, -, b, \emptyset\}$. Define the category $\wt\scrD^\ast(\scrA)$ to be the localization of $\Kom^\ast(\scrA)$ at the class of quasi-isomorphisms. For the time being, call this the \define{dirty derived category} with $\wt Q\colon\Kom^\ast(\scrA)\to\wt\scrD(\scrA)$ being the localizing functor.
	
	Similarly, define the category $\scrD^\ast(\scrA)$ to be the localization of $\scrK^\ast(\scrA)$ at the class of quasi-isomorphisms. For the time being, call this the \define{clean derived category} with $Q\colon\Kom^\ast(\scrA)\to\scrD(\scrA)$ being the localizing functor.
\end{definition}

\begin{mdframed}
	Our first goal will be to show that the dirty and clean derived categories are infact isomorphic. The reason for making this distinction is that the quasi-isomorphisms in $\scrK^\ast(\scrA)$ form a localizing class, while in $\Kom^\ast(\scrA)$, they usually don't.
\end{mdframed}

\subsection{The Cone and The Cylinder}

\begin{definition}
	Let $(K^\bullet, d_K)$ be a chain complex. The \define{translation} is defined to be the complex $\left(K[1]^\bullet, d_{K[1]}\right)$, where $K[1]^i = K^{i + 1}$ and $d^i_{K[1]}\colon K^{i + 1}\to K^{i + 2}$ is equal to $-d^{i + 1}_K$ for all $i\in\Z$.

	If $f\colon K^\bullet\to L^\bullet$ is a morphism of complexes, then define $f[1]\colon K[1]^\bullet\to L[1]^\bullet$ is given by $f[1]^i = f^{i + 1}$ for all $i\in\Z$. This is clearly a morphism of complexes.
\end{definition}

\begin{definition}
	Let $f\colon K^\bullet\to L^\bullet$ be a morphism of complexes. The \define{cone} of $f$ is defined to be the complex 
	\begin{equation*}
		\cone(f)^i = K^{i + 1}\oplus L^i = K[1]^i\oplus L^i\quad\text{ and }\quad d_{\cone(f)}\left(k^{i + 1}, l^i\right) = \left(-d_K^{i + 1}k^{i + 1}, f\left(k^{i + 1}\right) + d^i_Ll^i\right).
	\end{equation*}
	Treating elements of $\cone(f)^i$ as column vectors, we can denote the boundary map as 
	\begin{equation*}
		d_{\cone(f)}^i = 
		\begin{pmatrix}
			d_{K[1]} & 0\\
			f[1] & d_L
		\end{pmatrix}.
	\end{equation*}
\end{definition}

\begin{definition}
	Let $f\colon K^\bullet\to L^\bullet$ be a morphism of complexes. The \define{cylinder} of $f$ is defined to be the complex 
	\begin{equation*}
		\cyl(f)^i = K^i\oplus K^{i + 1}\oplus L^i = K^i\oplus K[1]^i\oplus L^i
	\end{equation*}
	and 
	\begin{equation*}
		d_{\cyl(f)}^i\left(k^i, k^{i + 1}, l^i\right) = \left(d_K^i k^i - k^{i + 1}, -d_K^{i + 1}k^{i + 1}, f(k^{i + 1}) + d_L^i l^i\right).
	\end{equation*}
	Treating elements of $\cyl(f)^i$ fas column vectors, we can denote the boundary map as 
	\begin{equation*}
		d_{\cyl(f)} = 
		\begin{pmatrix}
			d_K & -\bbone & 0\\
			& d_{K[1]} & 0\\
			0 & f & d_L
		\end{pmatrix}.
	\end{equation*}
\end{definition}

There is a natural short exact sequence of complexes:
\begin{equation*}
	0\to L^\bullet\xrightarrow{\iota} \cone(f)\xrightarrow{\delta} K[1]^\bullet\to 0
\end{equation*}
where 
\begin{equation*}
	\iota^i(l^i) = \left(0, l^i\right)\quad\text{ and }\quad\delta^i\left(k^{i + 1}, l^i\right) = k^{i + 1}.
\end{equation*}
Clearly both $\iota$ and $\delta$ are morphisms of complexes and the above sequence is exact. Next, there is another short exact sequence of complexes: 
\begin{equation*}
	0\to K^\bullet\xrightarrow{\gamma}\cyl(f)\xrightarrow{\pi}\cone(f)\to 0
\end{equation*}
where 
\begin{equation*}
	\gamma(k^i) = \left(k^i, 0, 0\right)\quad\text{ and }\quad \pi(k^i, k^{i + 1}, l^i) = \left(k^{i + 1}, l^i\right).
\end{equation*}
Again, it is easily checked that both $\gamma$ and $\pi$ are morphisms of complexes, and that the above sequence is exact.

Finally, define maps 
\begin{equation*}
	L^\bullet\xrightarrow{\alpha}\cyl(f)\xrightarrow{\beta} L^\bullet
\end{equation*}
as 
\begin{equation*}
	\alpha\left(l^i\right) = \left(0, 0, l^i\right)\quad\text{ and }\quad \beta\left(k^i, k^{i + 1}, l^i\right) = f(k^i) + l^i,
\end{equation*}
which are easily checked to be chain maps.

\begin{theorem}\thlabel{diagram-cone-cylinder}
	The diagram 
	\begin{equation*}
		\xymatrix {
			& 0\ar[r] & L^\bullet\ar[r]^\iota\ar[d]_\alpha & \cone(f)\ar[r]^\delta\ar@{=}[d] & K[1]^\bullet\ar[r] & 0\\
			0\ar[r] & K^\bullet\ar[r]^\gamma\ar@{=}[d] & \cyl(f)\ar[r]^\pi\ar[d]_\beta & \cone(f)\ar[r] & 0\\
			& K^\bullet\ar[r]_f & L^\bullet
		}
	\end{equation*}
	commutes in $\Kom(\scrA)$. Further, the maps $\alpha$ and $\beta$ are quasi-isomorphisms, $\beta\circ\alpha = \id_{L^\bullet}$, and $\alpha\circ\beta$ is homotopic to $\id_{\cyl(f)}$. As a result, $\wt Q(\alpha)$ and $\wt Q(\beta)$ are inverses in the dirty derived category.
\end{theorem}
\begin{proof}
	From the above constructions, it is clear that the diagram commutes, and $\beta\circ\alpha = \id_{L^\bullet}$. Define $h^i\colon\cyl(f)^i\to\cyl(f)^{i - 1}$ by 
	\begin{equation*}
		h^i\left(k^i, k^{i + 1}, l^i\right) = \left(0, k^i, 0\right).
	\end{equation*}
	Then 
	\begin{equation*}
		d_{\cyl(f)}h^i\left(k^i, k^{i + 1}, l^i\right) = d_{\cyl(f)}\left(0, k^i, 0\right) =  \left(-k^i, -d_Kk^i, f(k^i)\right),
	\end{equation*}
	and 
	\begin{equation*}
		h^id_{\cyl(f)}\left(k^i, k^{i + 1}, l^i\right) = h^i\left(d_Kk^i - k^{i + 1}, -d_Kk^{i + 1}, f(k^{i + 1}) + d_Ll^i\right) = \left(0, d_Kk^i - k^{i + 1}, 0\right)
	\end{equation*}
	so that 
	\begin{equation*}
		\left(d_{\cyl(f)}\circ h^i + h^i\circ d_{\cyl(f)}\right)\left(k^i, k^{i + 1}, l^i\right) = \left(-k^i, -k^{i + 1}, f(k^i)\right) = \left(\alpha\circ\beta - \id_{\cyl(f)}\right)\left(k^i, k^{i + 1}, l^i\right).
	\end{equation*}
	Thus $h$ is the desired homotopy, whence $\alpha\circ\beta$ is homotopic to the identity. As a ressult, $\alpha$ and $\beta$ are quasi-isomorphisms so that they are isomorphisms in $\wt\scrD^\ast(\scrA)$. But since $\wt Q(\beta)\circ\wt Q(\alpha) = \id$, it follows that they are inverses in the dirty derived category.
\end{proof}

\begin{lemma}\thlabel{homotopic-maps-dirty-derived-category}
	Let $f,g\colon K^\bullet\to L^\bullet$ be chain homotopic maps in $\Kom^\ast(\scrA)$. Then $\wt Q(f) = \wt Q(g)$.
\end{lemma}
\begin{proof}
	Suppose $h$ is a homotopy between $f$ and $g$, that is, 
	\begin{equation*}
		f - g = d\circ h + h\circ d.
	\end{equation*}
	To avoid unnecessary clutter, we shall play fast and loose with indexing henceforth. Define a map $\wt h\colon\cone(f)\to\cone(g)$ as 
	\begin{equation*}
		\wt h\left(k^{i + 1}, l^i\right) = \left(k^{i + 1}, l^i + h(k^{i + 1})\right).
	\end{equation*}
	First, we verify that this is indeed a chain map. We have 
	\begin{equation*}
		\xymatrix {
			\left(k^{i + 1}, l^i\right)\ar@{|->}[r]^-{d_{\cone(f)}}\ar@{|->}[dd]_{\wt h} & \left(-d_K k^{i + 1}, f(k^{i + 1}) + d_L l^i\right)\ar@{|->}[d]^{\wt h}\\
			& \left(-d_K k^{i + 1}, f(k^{i + 1}) + d_L l^i - h(d_K k^{i + 1})\right)\ar@{=}[d]\\
			\left(k^{i + 1}, l^i + h(k^{i + 1})\right)\ar@{|->}[r]_-{d_{\cone(g)}} & \left(-d_K k^{i + 1}, g(k^{i + 1}) + d_Ll^i + d_L h(k^{i + 1})\right)
		}
	\end{equation*}
	where the bottom right is an equality since $d_L\circ h + h\circ d_K = f - g$. Next, it is clear that the diagram 
	\begin{equation*}
		\xymatrix {
			0\ar[r] & L^\bullet\ar[r]^-\iota\ar@{=}[d] & \cone(f)\ar[d]^{\wt h}\ar[r]^\delta & K[1]^\bullet\ar[r]\ar@{=}[d] & 0\\
			0\ar[r] & L^\bullet\ar[r]_-\iota\ar[r] & \cone(g)\ar[r]_\delta & K[1]^\bullet\ar[r] & 0
		}
	\end{equation*}
	commutes in $\Kom^\ast(\scrA)$. Taking cohomologies and invoking the five lemma through the commutative diagram of long exact sequences: 
	\begin{equation*}
		\xymatrix {
			\cdots\ar[r] & H^i(L^\bullet)\ar[r]\ar@{=}[d] & H^i(\cone(f))\ar[r]\ar[d]^{H^i(\wt h)} & H^i(K[1]^\bullet)\ar[r]\ar@{=}[d] & \cdots\\
			\cdots\ar[r] & H^i(L^\bullet)\ar[r] & H^i(\cone(g))\ar[r] & H^i(K[1]^\bullet)\ar[r] & \cdots
		}
	\end{equation*}
	it follows that $\wt h$ is a quasi-isomorphism.

	Similarly, define the map $\wh h\colon\cyl(f)\to\cyl(g)$ by 
	\begin{equation*}
		\wh h\left(k^i, k^{i + 1}, l^i\right) = \left(k^i, k^{i + 1}, l^i + h(k^{i + 1})\right).
	\end{equation*}
	We first verify that this is indeed a chain map. 
	\begin{equation*}
		\xymatrix {
			\left(k^i, k^{i + 1}, l^i\right)\ar@{|->}[r]^-{d_{\cyl(f)}}\ar@{|->}[dd]_{\wh h} & \left(d_Kk^i - k^{i + 1}, -d_Kk^{i + 1}, f(k^{i + 1}) + d_Ll^i\right)\ar@{|->}[d]^{\wh h}\\
			& \left(d_Kk^i - k^{i + 1}, -d_Kk^{i + 1}, f(k^{i + 1}) + d_Ll^i - h(d_Kk^{i + 1})\right)\ar@{=}[d]\\ 
			\left(k^i, k^{i + 1}, l^i + h(k^{i + 1})\right)\ar@{|->}[r]_-{d_{\cyl(g)}} & \left(d_Kk^i- k^{i + 1}, -d_Kk^{i + 1}, g(k^{i + 1}) + d_L l^i + d_Lh(k^{i + 1})\right)
		}
	\end{equation*}
	where as last time, the bottom right equality follows from the fact that $f - g = d_L\circ h + h\circ d_K$. Further, note that we have a commutative diagram 
	\begin{equation*}
		\xymatrix {
			0\ar[r] & K^\bullet\ar[r]^-\gamma\ar@{=}[d] & \cyl(f)\ar[r]^-\pi\ar[d]^{\wh h} & \cone(f)\ar[r]\ar[d]^{\wt h} & 0\\
			0\ar[r] & K^\bullet\ar[r]_-\gamma & \cyl(g)\ar[r]_-\pi & \cone(g)\ar[r] & 0
		}
	\end{equation*}
	so that moving to cohomology, using the five lemma and the fact that $\wt h$ is a quasi-isomorphism, it would follow that $\wh h$ is a quasi-isomorphism too.

	Finally, consider the following diagram with notation as in \thref{diagram-cone-cylinder}:
	\begin{equation*}
		\xymatrix {
			& L^\bullet\ar[d]^{\alpha_f}\\
			K^\bullet\ar[r]^-\gamma\ar@{=}[d]\ar[ru]^-f & \cyl(f)\ar[d]^{\wh h}\\
			K^\bullet\ar[r]_\gamma\ar[rd]_-g & \cyl(g)\ar[d]^{\beta_g}\\
			& L^\bullet
		}
	\end{equation*}
	where the middle square and the bottom triangle commute in $\Kom^\ast(\scrA)$. We contend that the diagram commutes in $\wt\scrD(\scrA)$. Indeed, as seen in \thref{diagram-cone-cylinder}, $\wt Q(\alpha_f)$ and $\wt Q(\beta_f)$ are inverses to one another in $\wt\scrD(\scrA)$, and by construction, $f = \beta_f\circ \gamma$, so that in $\wt\scrD(\scrA)$, 
	\begin{equation*}
		\wt Q(\alpha_f)\circ\wt Q(f) = \wt Q(\alpha_f)\circ\wt Q(\beta_f)\circ\wt Q(\gamma) = \wt Q(\gamma).
	\end{equation*}
	Thus the top triangle commutes in $\wt D(\scrA)$. Finally, since $\beta_g\circ\wh h\circ\alpha_f = \id_{L^\bullet}$, we obtain
	\begin{equation*}
		\wt Q(g) = \wt Q(\beta_g)\circ\wt Q(\wh h)\circ\wt Q(\alpha_f)\circ\wt Q(f) = \wt Q(f),
	\end{equation*}
	thereby completing the proof.
\end{proof}

\begin{theorem}
	The clean and dirty derived categories are canonically isomorphic.
\end{theorem}
\begin{proof}
	Let $\Pi\colon\Kom^\ast(\scrA)\to\scrK^\ast(\scrA)$ denote the natural full functor between the two categories. The universal property of localization furnishes a functor $G\colon\wt\scrD(\scrA)\to\scrD(\scrA)$ making the diagram of functors
	\begin{equation*}
		\xymatrix {
			\Kom^\ast(\scrA)\ar[r]^-\Pi\ar[d]_{\wt Q} & \scrK^\ast(\scrA)\ar[d]^{Q}\\
			\wt\scrD(\scrA)\ar[r]_{G} & \scrD(\scrA)
		}
	\end{equation*}
	commute. Note that $G$ is the identity on objects, and since every morphism in $\scrK^\ast(\scrA)$ lifts to a morphism in $\Kom^\ast(\scrA)$, and quasi-isomorphisms lift to quasi-isomorphisms, it follows from the explicit construction of morphisms in the localization that $G$ is full. It remains to show that $G$ is faithful. 

	In view of \thref{homotopic-maps-dirty-derived-category}, there exists a functor $\Phi\colon\scrK^\ast(\scrA)\to\wt\scrD(\scrA)$ such that $\Phi\circ\Pi = \wt Q$. Again, $\Phi$ sends quasi-isomorphisms in $\wt K^\ast(\scrA)$ to isomorphisms in $\wt\scrD(\scrA)$, and hence, $\Phi$ induces $H\colon\scrD(\scrA)\to\wt\scrD(\scrA)$ such that $\Phi = H\circ Q$. Consequently, 
	\begin{equation*}
		\wt Q = \Phi\circ\Pi = H\circ Q\circ\Pi = H\circ G\circ\wt Q.
	\end{equation*}
	The universal property would then force $H\circ G = \id_{\wt\scrD(\scrA)}$, and hence $G$ is faithful, thereby completing the proof.
\end{proof}

\subsection{Distinguished Triangles}

\begin{definition}
	For this definition, we work in a category of complexes, such as $\Kom^\ast(\scrA)$, $\scrK^\ast(\scrA)$, $\wt\scrD^\ast(\scrA)$, or $\scrD^\ast(\scrA)$ for $\ast\in\{+, -, b, \emptyset\}$.
	\begin{enumerate}[label=(\arabic*)]
		\item A \define{triangle} in the category is a diagram of the form 
		\begin{equation*}
			K^\bullet\to L^\bullet\to M^\bullet\to K[1]^\bullet.
		\end{equation*}
		\item A \define{morphism} of triangles is commutative diagram (in that category) of the form 
		\begin{equation*}
			\xymatrix {
				0\ar[r] & K^\bullet\ar[r]\ar[d]_u & L^\bullet\ar[r]\ar[d]_v & M^\bullet\ar[r]\ar[d]_w & K[1]^\bullet\ar[d]^{u[1]}\\
				0\ar[r] & K_1^\bullet\ar[r] & L_1^\bullet\ar[r] & M_1^\bullet\ar[r] & K_1[1]^\bullet\\
			}
		\end{equation*}
		and further, such a morphism is said to be an \define{isomorphism} if $u$, $v$, and $w$ are isomorphisms in that category. 
		\item A triangle is said to be \define{distinguished} if it is isomorphic to the \define{standard} triangle 
		\begin{equation*}
			K^\bullet\xrightarrow{\gamma}\cyl(f)\xrightarrow{\pi}\cone(f)\xrightarrow{\delta} K[1]^\bullet,
		\end{equation*}
		where the morphisms are as in \thref{diagram-cone-cylinder}.
	\end{enumerate}
\end{definition}

\begin{proposition}
	The triangle $K^\bullet\xrightarrow{f} L^\bullet\xrightarrow{\iota}\cone(f)\xrightarrow{\delta} K[1]^\bullet$ is isomorphic to the standard triangle $K^\bullet\xrightarrow{\gamma}\cyl(f)\xrightarrow{\pi}\cone(f)\xrightarrow{\delta} K[1]^\bullet$ in $\scrK^\ast(\scrA)$.
\end{proposition}
\begin{proof}
	Consider the diagram 
	\begin{equation*}
		\xymatrix {
			K^\bullet\ar[r]^-{\gamma}\ar@{=}[d] & \cyl(f)\ar[r]^-{\pi}\ar@<-0.5ex>[d]_\beta & \cone(f)\ar[r]^-{\delta}\ar@{=}[d] & K[1]^\bullet\ar@{=}[d]\\
			K^\bullet\ar[r]_-{f} & L^\bullet\ar[r]_-{\iota}\ar@<-0.5ex>[u]_\alpha & \cone(f)\ar[r]_-{\delta} & K[1]^\bullet
		}
	\end{equation*}
	where the top row is a distinguished triangle in $\scrK^\ast(\scrA)$. Further $\alpha$ and $\beta$ are isomorphisms and inverses to one another in $\scrK^\ast(\scrA)$. 

	Now, $\beta\circ\gamma = f$ and $\pi\circ\alpha = \iota$ in $\Kom^\ast(\scrA)$. Using the fact that $\alpha$ and $\beta$ are inverses, it is immediate that the above diagram commutes in $\scrK^\ast(\scrA)$, thereby completing the proof.
\end{proof}

\begin{corollary}\thlabel{dist-triangles-homotopy-category}
	A triangle in $\scrK^\ast(\scrA)$ is distinguished if and only if it is isomorphic to a triangle of the form $K^\bullet\xrightarrow{f} L^\bullet\xrightarrow{\iota}\cone(f)\xrightarrow{\delta} K[1]^\bullet$.
\end{corollary}

\begin{lemma}\thlabel{tr2-homotopy-category}
	If $K^\bullet\xrightarrow{u} L^\bullet\xrightarrow{v} M^\bullet\xrightarrow{w} K[1]^\bullet$ is a distinguished triangle in $\scrK^\ast(\scrA)$, then so is 
	\begin{equation*}
		L^\bullet\xrightarrow{v} M^\bullet\xrightarrow{w} K[1]^\bullet\xrightarrow{-u[1]} L[1]^\bullet.
	\end{equation*}
\end{lemma}
\begin{proof}
	Since the triangle is distinguished, due to \thref{dist-triangles-homotopy-category}, there is an isomorphism of triangles 
	\begin{equation*}
		\xymatrix {
			K^\bullet\ar[r]^-u\ar[d]_\wr & L^\bullet\ar[r]^-v\ar[d]_\wr & M^\bullet\ar[r]^-w\ar[d]_\wr & K[1]^\bullet\ar[d]_\wr\\
			X^\bullet\ar[r]_f & Y^\bullet\ar[r]_-\iota & \cone(f)\ar[r]_-\delta & X[1]^\bullet
		}
	\end{equation*}
	Using the commutativity of the left square, it is easy to argue that the bottom triangle (and hence the top triangle) is isomorphic to 
	\begin{equation*}
		K^\bullet\xrightarrow{u} L^\bullet\xrightarrow{\iota}\cone(u)\xrightarrow{\delta} K[1]^\bullet.
	\end{equation*}
	Therefore, it suffices to show that the triangle 
	\begin{equation*}
		L^\bullet\xrightarrow{\iota}\cone(u)\xrightarrow{\delta} K[1]^\bullet\xrightarrow{-u[1]} L[1]^\bullet
	\end{equation*}
	is distinguished. We shall do so by defining an explicit isomorphism with the distinguished triangle 
	\begin{equation*}
		L^\bullet\xrightarrow{\iota}\cone(u)\xrightarrow{j}\cone(\iota)\xrightarrow{\eta} L[1]^\bullet.
	\end{equation*}
	Let $\theta\colon K[1]^\bullet\to\cone(\iota)$ be given by 
	\begin{equation*}
		\theta^i\left(k^{i + 1}\right) = \left(-u^{i + 1}k^{i + 1}, k^{i + 1}, 0\right)\in\cone(\iota)^i = L^{i + 1}\oplus K^{i + 1}\oplus L^i.
	\end{equation*}
	First, we must verify that this is a chain map. Indeed, for $k^{i + 1}\in k[1]^i = k^{i + 1}$, 
	\begin{equation*}
		\xymatrix {
			k^{i + 1}\ar@{|->}[r]^-{d_{K[1]}}\ar@{|->}[dd]_{\theta^i} & -d_K k^{i + 1}\ar@{|->}[d]^{\theta^{i + 1}}\\
			& \left(u^{i + 2}d_K k^{i + 1}, -d_K k^{i + 1}, 0\right)\ar@{=}[d]\\
			\left(-u^{i + 1}k^{i + 1}, k^{i + 1}, 0\right)\ar@{|->}[r]_-{d_{\cone(\iota)}} & \left(d_Ku^{i + 1}k^{i + 1}, -d_K k^{i + 1}, 0\right)
		}
	\end{equation*}
	where the bottom right equality follows from the fact that $u$ is a chain map. Now, we shall show that the diagram 
	\begin{equation*}
		\xymatrix {
			L^\bullet\ar[r]^-{\iota}\ar@{=}[d] & \cone(u)\ar[r]^-\delta\ar@{=}[d] & K[1]^\bullet\ar[r]^-{-u[1]}\ar[d]^\theta & L[1]^\bullet\ar@{=}[d]\\
			L^\bullet\ar[r]_-{\iota} & \cone(u)\ar[r]_-j & \cone(\iota)\ar[r]_-{\eta} & L[1]^\bullet
		}
	\end{equation*}
	is an isomorphism of triangles in $\scrK^\ast(\scrA)$. Before we proceed, note that 
	\begin{equation*}
		d_{\cone(\iota)} = 
		\begin{pmatrix}
			-d_L & 0 & 0\\
			0 & -d_K & 0\\
			\id_L & u[1] & d_L
		\end{pmatrix}.
	\end{equation*}
	Clearly the first square commutes in $\Kom^\ast(\scrA)$. The square on the right commutes too: 
	\begin{equation*}
		\xymatrix {
			k^{i + 1}\ar@{|->}[r]^-{-u[1]^i}\ar@{|->}[d]_{\theta^i} & -u^{i + 1}k^{i + 1}\ar@{|->}[d]^{\id}\\
			\left(-u^{i + 1}k^{i + 1}, k^{i + 1}, 0\right)\ar[r]_-{\eta^i} & -u^{i + 1}k^{i + 1}.
		}
	\end{equation*}
	We claim that $\theta\circ\delta$ and $j\colon\cone(u)\to\cone(\iota)$ are homotopic. Define $h^i\colon\cone(u)\to\cone(\iota)$ by 
	\begin{equation*}
		h^i\left(k^{i + 1}, l^i\right) = \left(l^i, 0, 0\right)\in \cone(\iota)^{i - 1} = L^i\oplus K^i\oplus L^{i - 1}.
	\end{equation*}
	Then 
	\begin{equation*}
		h^{i + 1}\circ d_{\cone(u)}\left(k^{i + 1}, l^i\right) = h^{i + 1}\left(-d_K k^{i + 1}, f(k^{i + 1}) + d_L l^i\right) = \left(f(k^{i + 1}) + d_Ll^i, 0, 0\right)\in\cone(\iota)^i,
	\end{equation*}
	and 
	\begin{equation*}
		d_{\cone(\iota)}\circ h^i\left(k^{i + 1}, l^i\right) = d_{\cone(\iota)}\left(l^i, 0, 0\right) = \left(-d_Ll^i, 0, l^i\right)\in\cone(\iota)^i = L^{i + 1}\oplus K^{i + 1}\oplus L^i.
	\end{equation*}
	So that 
	\begin{equation*}
		\left(h^{i + 1}\circ d_{\cone(u)} + d_{\cone(\iota)}\circ h^i\right)\left(k^{i + 1}, l^i\right) = \left(u(k^{i + 1}), 0, l^i\right).
	\end{equation*}
	Now note that 
	\begin{equation*}
		\theta\circ\delta\left(k^{i + 1}, l^i\right) = \theta\left(k^{i + 1}\right) = \left(-u^{i + 1}k^{i + 1}, k^{i + 1}, 0\right)
	\end{equation*}
	and 
	\begin{equation*}
		j\left(k^{i + 1}, l^i\right) = \left(0, k^{i + 1}, l^i\right).
	\end{equation*}
	Hence, $h\circ d + d\circ h = j - \theta\circ\delta$, that is, the diagram commutes in $\scrK^\ast(\scrA)$. It remains to show that $\theta$ is an isomorphism in $\scrK^\ast(\scrA)$. 

	Let $\psi\colon\cone(\iota)\to K[1]^\bullet$ be the map 
	\begin{equation*}
		\psi^i\left(l^{i + 1}, k^{i + 1}, l^i\right) = k^{i + 1}\in K[1]^i.
	\end{equation*}
	Clearly $\psi\circ\theta = \id_{K[1]}$. On the other hand, 
	\begin{equation*}
		\left(\theta\circ\psi\right)^i\left(l^{i + 1}, k^{i + 1}, k^i\right) = \left(-u(k^{i + 1}), k^{i + 1}, 0\right).
	\end{equation*}
	Set $h^i\colon\cone(\iota)^i\to\cone(\iota)^{i - 1}$ to be 
	\begin{equation*}
		h^i\left(l^{i + 1}, k^{i + 1}, l^i\right) = \left(l^i, 0, 0\right).
	\end{equation*}
	Then 
	\begin{equation*}
		h\circ d_{\cone(\iota)}\left(l^{i + 1}, k^{i + 1}, l^i\right) = \left(l^{i + 1} + u(k^{i + 1}) + d_L l^i, 0, 0\right)
	\end{equation*}
	and 
	\begin{equation*}
		d_{\cone(\iota)}\circ h\left(l^{i + 1}, k^{i + 1}, l^i\right) = d_{\cone(\iota)}\left(l^i, 0, 0\right) = \left(-d_L l^i, 0, l^i\right),
	\end{equation*}
	so that 
	\begin{equation*}
		\left(h\circ d_{\cone(\iota)} + d_{\cone(\iota)}\circ h\right)\left(l^{i + 1}, k^{i + 1}, l^i\right) = \left(l^{i + 1} + u(k^{i + 1}), 0, l^i\right) = \left(\id_{\cone(\iota)} - \theta\circ\psi\right)\left(l^{i + 1}, k^{i + 1}, l^i\right).
	\end{equation*}
	This shows that $\theta$ is an isomorphism in $\scrK^\ast(\scrA)$, thereby completing the proof.
\end{proof}

\begin{corollary}
	If $K^\bullet\xrightarrow{u} L^\bullet\xrightarrow{v} M^\bullet\xrightarrow{w} K[1]^\bullet$ is a distinguished triangle in $\scrK^\ast(\scrA)$, then so is
	\begin{equation*}
		M^\bullet[-1]\xrightarrow{-w[-1]} K^\bullet\xrightarrow{u} L^\bullet\xrightarrow{v} M^\bullet.
	\end{equation*}
\end{corollary}
\begin{proof}
	Apply \thref{tr2-homotopy-category} twice and translate by $X^\bullet\mapsto X[-1]^\bullet$.
\end{proof}

\begin{theorem}\thlabel{quasi-isomorphism-localizing-class}
	The class of quasi-isomorphisms in $\scrK^\ast(\scrA)$ forms a localizing class of morphisms.
\end{theorem}
\begin{proof}
	Clearly \ref{lc1} and \ref{lc2} are satisfied. Let us first verify \ref{pullback-type}. Consider a diagram of the form 
	\begin{equation*}
		\xymatrix {
			& M^\bullet\ar[d]^g\\
			K^\bullet\ar[r]^f_-{\rm quiso} & L^\bullet
		}
	\end{equation*}
	where $f$ is a quasi-isomorphism. Then using the notation of \thref{diagram-cone-cylinder}, let $\iota\colon L^\bullet\to\cone(f)$ be the natural map, and consider the distinguished triangle 
	\begin{equation*}
		M^\bullet\xrightarrow{\iota g}\cone(f)\xrightarrow{\alpha}\cone(\iota g)\xrightarrow{\eta} M[1]^\bullet
	\end{equation*}
	in $\scrK^\ast(\scrA)$. Due to \thref{tr2-homotopy-category}, it follows that 
	\begin{equation*}
		\cone(\iota g)[-1]\xrightarrow{-\eta[-1]} M^\bullet\xrightarrow{\iota g}\cone(f)\xrightarrow{\alpha}\cone(\iota g)
	\end{equation*}
	is a distinguished triangle too. As we have seen earlier, this implies that 
	\begin{equation*}
		\cone\left(-\eta[-1]\right)\cong\cone(f)
	\end{equation*}
	as complexes. But since $f$ is a quasi-isomorphism, $\cone(f)$ is acyclic, therefore, so is $\cone\left(-\eta[-1]\right)$, and hence $-\eta[-1]$ is a quasi-isomorphism too. Let $s = -\eta[-1]$ for brevity. Note that explicitly, the map is given by 
	\begin{equation*}
		s\colon \left(m^i, k^i, l^{i - 1}\right)\longmapsto -m^i,
	\end{equation*}
	where $\left(m^i, k^i, l^{i - 1}\right)\in \cone(\iota g)[-1]^i = \cone(\iota g)^{i - 1} = M^i\oplus K^i\oplus L^{i - 1}$. Define $h\colon\cone(\iota g)[-1]\to K^\bullet$ by 
	\begin{equation*}
		h\left(m^i, k^i, l^{i - 1}\right) = -k^i.
	\end{equation*}
	We claim that the diagram 
	\begin{equation*}
		\xymatrix {
			\cone(\iota g)[-1]\ar[r]^-{-s}\ar[d]_h & M^\bullet\ar[d]^g\\
			K^\bullet\ar[r]_-f & L^\bullet
		}
	\end{equation*}
	commutes in $\scrK^\ast(\scrA)$. Indeed, 
	\begin{equation*}
		\left(g\circ s\right)\left(m^i, k^i, l^{i - 1}\right) = -g\left(m^i\right)\quad\text{ and }\quad \left(f\circ h\right)\left(m^i, k^i, l^{i - 1}\right) = -f\left(k^i\right).
	\end{equation*}
	Define $\chi^i\colon\cone(\iota g)[-1]^i = \cone(\iota g)^{i - 1}\to L^{i - 1}$ by 
	\begin{equation*}
		\chi^i\left(m^i, k^i, l^{i - 1}\right) = -l^{i - 1}.
	\end{equation*}
	Note that 
	\begin{equation*}
		d_{\cone(\iota g)} = 
		\begin{pmatrix}
			-d_M & 0 & 0\\
			0 & -d_K & 0\\
			g & f & d_L
		\end{pmatrix}
	\end{equation*}
	so that 
	\begin{equation*}
		d_{\cone(\iota g)[-1]}\left(m^i, k^i, l^{i - 1}\right) = \left(d_M m^i, d_K k^i, -d_Ll^{i - 1} - f(k^i) - g(m^i)\right).
	\end{equation*}
	Thus 
	\begin{equation*}
		\chi\circ d_{\cone(\iota g)[-1]}\left(m^i, k^i, l^{i - 1}\right) = d_Ll^{i - 1} + f(k^i) + g(m^i)\quad\text{ and }\quad d_L\circ\chi\left(m^i, k^i, l^{i - 1}\right) = -d_L l^{i - 1}.
	\end{equation*}
	It follows that 
	\begin{equation*}
		\left(\chi\circ d_{\cone(\iota g)} + d_L\circ\chi\right)\left(m^i, k^i, l^{i - 1}\right) = f(k^i) + g(m^i) = \left(g\circ(-s) - f\circ h\right)\left(m^i, k^i, l^{i - 1}\right),
	\end{equation*}
	which proves our claim, thereby verifying \ref{pullback-type}. The condition \ref{pushout-type} is proved in a similar manner; we omit the details.

	Finally, we must verify \ref{annihilation-condition}. Let $f\colon K^\bullet\to L^\bullet$ be a morphism in $\scrK^\ast(\scrA)$ such that there exists a quasi-isomorphism $s\colon L^\bullet\to M^\bullet$ such that $s\circ f \simeq 0$ in $\scrK^\ast(\scrA)$\footnote{Since $\scrK^\ast(\scrA)$ is an additive category, it suffices to show that whenever $s\circ f = 0$, there exists a quasi-isomorphism $t$ such that $f\circ t = 0$.}. Let $h^i\colon K^i\to M^{i - 1}$ be a homotopy between $s\circ f$ and $0$, that is, $s\circ f = h\circ d + d\circ h$. Consider the following diagram 
	\begin{equation*}
		\xymatrix {
			\cone(s)[-1]\ar[r]^-{\delta(s)[-1]}\ar@{=}[d] & L^\bullet\ar[r]^-s & M^\bullet\ar[r] & \cone(s)\\
			\cone(s)[-1] & K^\bullet\ar[l]^-g\ar[u]_f & \cone(g)[-1]\ar[l]^-{\delta(g)[-1]}
		}
	\end{equation*}
	Define $g^i\colon K^i\to\cone(s)[-1]^i = \cone(s)^{i - 1} = L^i\oplus M^{i - 1}$ by 
	\begin{equation*}
		g^i\left(k^i\right) = \left(f(k^i), -h(k^i)\right).
	\end{equation*}
	First, we must show that $g$ is a morphism of complexes. Indeed, 
	\begin{equation*}
		\xymatrix {
			k^i\ar[r]\ar[dd] & d_K k^i\ar[d]\\
			& \left(f d_k(k^i), -h d_K(k^i)\right)\ar@{=}[d]\\
			\left(f(k^i), -h(k^i)\right)\ar[r]_-{d_{\cone(s)[-1]}} & \left(d_L f(k^i), -sf(k^i) + d_M h(k^i)\right)
		}
	\end{equation*}
	where the bottom right equality follows from the fact that $h$ is a homotopy. Next, it is clear that the square commutes in $\Kom^\ast(\scrA)$. Let 
	\begin{equation*}
		t = \delta(g)[-1]\colon \cone(g)[-1]\to K^\bullet.
	\end{equation*}
	Then 
	\begin{equation*}
		f\circ t = \delta(s)[-1]\circ g\circ \delta(g)[-1] = 0.
	\end{equation*}
	Finally, we must argue that $t$ is a quasi-isomorphism. The triangle 
	\begin{equation*}
		K^\bullet\xrightarrow{g}\cone(s)[-1]\into\cone(g)\xrightarrow{\delta(g)} K[1]^\bullet
	\end{equation*}
	is distinguished, and due to \thref{tr2-homotopy-category}, so is 
	\begin{equation*}
		\cone(g)[-1]\xrightarrow{-\delta(g)[-1]} K^\bullet\xrightarrow{g}\cone(s)[-1]\into\cone(g).
	\end{equation*}
	But as we have argued time and again, this means 
	\begin{equation*}
		\cone(s)[-1]\cong\cone\left(-\delta(g)[-1]\right).
	\end{equation*}
	Now since $s$ is a quasi-isomorphism, $\cone(s)$ is acyclic, and hence so is $\cone(-t)$, that is, $-t$ is a quasi-isomorphism, consequently, so is $t$. This completes the proof.
\end{proof}

\subsection{The Subcategory of Injective Complexes}

\begin{lemma}\thlabel{acyclic-to-injective-nulhomotopic}
	Let $f\colon A^\bullet\to I^\bullet$ be a quasi-isomorphism of bounded below complexes with $A^\bullet$ an acyclic complex and $I^\bullet$ a complex of injectives. Then $f$ is homotopic to the zero map.
\end{lemma}
\begin{proof}
	We may suppose without loss of generality that $A^i = I^i = 0$ for $i < 0$. We shall construct a homotopy between $f$ and $0$ by inductively defining $h^i\colon A^i\to I^{i - 1}$.
	\begin{equation*}
		\xymatrix {
			\cdots\ar[r] & 0\ar[r] & A^0\ar[r]\ar[d]^{f^0}\ar@{.>}[ld]|-{h^0} & A^1\ar[r]\ar[d]^{f^1}\ar@{.>}[ld]|-{h^1} & A^2\ar[r]\ar[d]^{f^2}\ar@{.>}[ld]|-{h^2} & \cdots\\
			\cdots\ar[r] & 0\ar[r] & I^0\ar[r] & I^1\ar[r] & I^2\ar[r] & \cdots\\
		}
	\end{equation*}
	Begin by defining $h^i = 0$ for $i\le 0$. Since $d_A^0$ is injective, we can lift $f^0$ to a map $h^1\colon A^1\to I^0$ satisfying 
	\begin{equation*}
		f^0 = h^1\circ d_A^0.
	\end{equation*}
	Next, consider the diagram 
	\begin{equation*}
		\xymatrix {
			A^0\ar[r]^-{d^0_A} & A^1\ar[d]^{f^1 - d^0_Ih^1}\ar[r]^-{d^1_A} & A^2\\
			& I^1
		}
	\end{equation*}
	where 
	\begin{equation*}
		\left(f^1 - d^0_I h^1\right)\circ d^0_A = f^1 d^0_A - d^0_I h^1 d^0_A = f^1d^0_A - d^0_If^1 = 0.
	\end{equation*}
	Therefore, the above diagram factors as 
	\begin{equation*}
		\xymatrix {
			A^1\ar[r]\ar[d] & \coker d^0_A\ar[r]\ar[ld] & A^2\ar@{.>}[lld]^{h^2}\\
			I^1
		}
	\end{equation*}
	where, since $A^\bullet$ is acyclic, the map $\coker d^0_A\to A^2$ is injective, and hence, the map $\coker d^0_A\to I^1$ lifts to a map $h^2\colon A^2\to I^1$. Then clearly, 
	\begin{equation*}
		h^2\circ d^1_A = f^1 - d_I^0\circ h^1,
	\end{equation*}
	as desired. Continuing in this fashion and arguing similarly, we would have constructed the desired homotopy between $f$ and $0$.
\end{proof}

\begin{lemma}\thlabel{quasi-isomorphism-splits}
	Let $s\colon I^\bullet\to A^\bullet$ be a quasi-isomorphism between bounded below complexes with $I^\bullet$ being a complex of injectives. Then there is a morphism of complexes $t\colon A^\bullet\to I^\bullet$ such that $t\circ s$ is homotopic to $\id_{I^\bullet}$.
\end{lemma}
\begin{proof}
	Since $s$ is a quasi-isomorphism, $\cone(s)$ is a bounded below acyclic complex. Following in the notation of \thref{diagram-cone-cylinder}, consider the map $\delta\colon\cone(s)\to I[1]^\bullet$, which, in view of \thref{acyclic-to-injective-nulhomotopic} is homotopic to the zero morphism. Let 
	\begin{equation*}
		h^i\colon\cone(s)^i = I^{i + 1}\oplus A^i\to I[1]^{i - 1} = I^i
	\end{equation*}
	be the homotopy. This is given in components as $(\chi^i, \varphi^i)$, where 
	\begin{equation*}
		\chi^i\colon I^{i + 1}\to I^i\quad\text{ and }\quad \varphi^i\colon A^i\to I^i
	\end{equation*}
	are morphisms in $\scrA$. We claim that $\varphi\colon A^\bullet\to I^\bullet$ is the desired splitting in $\scrK^+(\scrA)$. Indeed, we have 
	\begin{equation*}
		\delta = d_{I[1]}\circ h + h\circ d_{\cone(s)}.
	\end{equation*}
	We compute: 
	\begin{equation*}
		d_{I[1]}\circ h^i\left(x^{i + 1}, k^i\right) = -d_I\left(\chi^i(x^{i + 1}) + \varphi^i(k^i)\right) = -d_I\chi^i(x^{i + 1}) - d_I\varphi^i(k^i),
	\end{equation*}
	and 
	\begin{equation*}
		h^{i + 1}\circ d_{\cone(s)}\left(x^{i + 1}, k^i\right) = h^{i + 1}\left(-d_I(x^{i + 1}), s(x^{i + 1}) + d_K k^i\right) = -\chi^{i + 1}d_I(x^{i + 1}) + \varphi^{i + 1} s(x^{i + 1}) + \varphi^{i + 1} d_K (k^i).
	\end{equation*}
	Putting these together, we get 
	\begin{equation*}
		x^{i + 1} = \delta(x^{i + 1}, k^i) = -d_I\chi^i(x^{i + 1}) - d_I\varphi^i(k^i) - \chi^{i + 1}d_I(x^{i + 1}) + \varphi^{i + 1} s(x^{i + 1}) + \varphi^{i + 1} d_K (k^i).
	\end{equation*}
	Since the left hand side is independent of $k^i\in K^i$, it follows that 
	\begin{equation*}
		\varphi^{i + 1}d_K (k^i) = d_I\varphi^i(k^i),
	\end{equation*}
	and hence, 
	\begin{equation*}
		\left(\varphi^{i + 1}\circ s^{i + 1} - \id_{I^\bullet}\right)(x^{i + 1}) = \left(d_I\circ\chi^i + \chi^{i + 1}\circ d_I\right)(x^{i + 1}).
	\end{equation*}
	Therefore, $\varphi\colon A^\bullet\to I^\bullet$ is a morphism of complexes and $\varphi\circ s$ is homotopic to $\id_{I^\bullet}$, thereby completing the proof.
\end{proof}

\begin{corollary}\thlabel{quasi-isomorphism-injectives}
	A quasi-isomorphism between bounded below complexes of injectives is a homotopy equivalence.
\end{corollary}
\begin{proof}
	Let $s\colon I^\bullet\to J^\bullet$ be a quasi-isomorphism between bounded below complexes of injectives. Due to \thref{quasi-isomorphism-splits}, there exists a morphism $t\colon J^\bullet\to I^\bullet$ such that $t\circ s$ is homotopic to $\id_{I^\bullet}$. Since $s$ is a quasi-isomorphism, so is $t$, and hence, there exists $u\colon I^\bullet\to J^\bullet$ such that $u\circ t$ is homotopic to $\id_{J^\bullet}$. Therefore, 
	\begin{equation*}
		s\simeq (u\circ t)\circ s \simeq u\circ(t\circ s)\simeq u.
	\end{equation*}
	Hence, $s$ and $t$ are homotopy equivalences, thereby completing the proof.
\end{proof}

The following theorem is a generalization of the fact that every object admits an injective resolution in an abelian category with enough injectives.

\begin{theorem}\thlabel{injective-resolution-of-complexes}
	Suppose $\scrA$ has enough injectives. Then for each bounded below complex $A^\bullet$, there exists a bounded below complex of injectives $I^\bullet$ and a quasi-isomorphism $s\colon A^\bullet\to I^\bullet$.
\end{theorem}
% \begin{proof}
% 	% TODO: Add in later
% \end{proof}

Now let $\scrI$ denote the full subcategory of $\scrA$ consisting only of injective objects. Similarly, let $\scrK^+(\scrI)$ be the full subcategory of $\scrK^+(\scrA)$ consisting only of bounded below complexes of injectives. Consider the composition of functors 
\begin{equation*}
	\scrK^+(\scrI)\longrightarrow\scrK^+(\scrA)\longrightarrow\scrD^+(\scrA).
\end{equation*}
Let $S$ denote the class of quasi-isomorphisms in $\scrK^+(\scrA)$, which is a localizing class due to \thref{quasi-isomorphism-localizing-class}. Let $S_{\scrI} = S\cap\Mor\left(\scrK^+(\scrI)\right)$. Recall that in the proof of \thref{quasi-isomorphism-localizing-class} we only used the cone construction, and clearly the cone of a morphism between bounded below complexes of injectives is another bounded below complex of injectives. Therefore, $S_{\scrI}$ is a localizing class in $\scrK^+(\scrI)$. Finally, if $s\colon I^\bullet\to A^\bullet$ is a quasi-isomorphism in $\scrK^+(\scrA)$ with $I^\bullet\in\scrK^+(\scrI)$, due to \thref{quasi-isomorphism-splits}, there exists a morphism $t\colon A^\bullet\to I^\bullet$ such that $t\circ s\simeq \id_{I^\bullet}\in S_{\scrI}$. Hence, we have verified the conditions of \thref{localizing-subcategories}, and hence, the induced map 
\begin{equation*}
	\xymatrix {
		\scrK^+(\scrI)\ar[r]\ar[d] & \scrK^+(\scrA)\ar[d]\\
		\scrK^+(\scrI)\left[S_\scrI^{-1}\right]\ar[r] & \scrD^+(\scrA)
	}
\end{equation*}
is fully faithful. Furthermore, due to \thref{injective-resolution-of-complexes}, the map is also essentially surjective on objects, consequently, it is an equivalence of categories. Finally, in view of \thref{quasi-isomorphism-injectives}, the functor $\id\colon\scrK^+(\scrI)\to\scrK^+(\scrI)$ is the localizing functor, and hence, the composition 
\begin{equation*}
	\scrK^+(\scrI)\to \scrK^+(\scrA)\to\scrD^+(\scrA)
\end{equation*}
is an equivalence of categories.

We summarize our conclusions: 
\begin{theorem}
	The composition of functors $\scrK^+(\scrI)\to\scrK^+(\scrA)\to\scrD^+(\scrA)$ is fully faithful. Further, if $\scrA$ has enough injectives, then it is an equivalence of categories.
\end{theorem}
\end{document}