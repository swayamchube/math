\documentclass[11pt]{article}

\usepackage[utf8]{inputenc} % allow utf-8 input
\usepackage[T1]{fontenc}    % use 8-bit T1 fonts
\usepackage{hyperref}       % hyperlinks
\usepackage{url}            % simple URL typesetting
\usepackage{booktabs}       % professional-quality tables
\usepackage{amsfonts}       % blackboard math symbols
\usepackage{nicefrac}       % compact symbols for 1/2, etc.
\usepackage{microtype}      % microtypography
\usepackage{graphicx}
\usepackage{natbib}
\usepackage{doi}
\usepackage{amssymb}
\usepackage{bbm}
\usepackage{amsthm}
\usepackage{amsmath}
\usepackage{xcolor}
\usepackage{theoremref}
\usepackage{enumitem}
\usepackage{fouriernc}
\usepackage{mdframed}
\usepackage{mathrsfs}
\setlength{\marginparwidth}{2cm}
\usepackage{todonotes}
\usepackage{stmaryrd}
\usepackage[all,cmtip]{xy} % For diagrams, praise the Freyd-Mitchell theorem 
\usepackage{marvosym}
\usepackage{geometry}
\usepackage{titlesec}
\usepackage{mathtools}
\usepackage{tikz}
\usetikzlibrary{cd}
\usepackage{epigraph}
\setlength\epigraphwidth{0.4\textwidth}

\renewcommand{\qedsymbol}{$\blacksquare$}
% \renewcommand{\familydefault}{\sfdefault} % Do you want this font? 

% Uncomment to override  the `A preprint' in the header
% \renewcommand{\headeright}{}
% \renewcommand{\undertitle}{}
% \renewcommand{\shorttitle}{}

\hypersetup{
    pdfauthor={Swayam Chube},
    colorlinks=true,
	citecolor=blue,
}

\newtheoremstyle{thmstyle}%               % Name
  {}%                                     % Space above
  {}%                                     % Space below
  {}%                             % Body font
  {}%                                     % Indent amount
  {\bfseries\scshape}%                            % Theorem head font
  {.}%                                    % Punctuation after theorem head
  { }%                                    % Space after theorem head, ' ', or \newline
  {\thmname{#1}\thmnumber{ #2}\thmnote{ (#3)}}%                                     % Theorem head spec (can be left empty, meaning `normal')

\newtheoremstyle{defstyle}%               % Name
  {}%                                     % Space above
  {}%                                     % Space below
  {}%                                     % Body font
  {}%                                     % Indent amount
  {\bfseries\scshape}%                            % Theorem head font
  {.}%                                    % Punctuation after theorem head
  { }%                                    % Space after theorem head, ' ', or \newline
  {\thmname{#1}\thmnumber{ #2}\thmnote{ (#3)}}%                                     % Theorem head spec (can be left empty, meaning `normal')

\theoremstyle{thmstyle}
\newtheorem{theorem}{Theorem}[section]
\newtheorem{lemma}[theorem]{Lemma}
\newtheorem{proposition}[theorem]{Proposition}

\theoremstyle{defstyle}
\newtheorem{definition}[theorem]{Definition}
\newtheorem{corollary}[theorem]{Corollary}
\newtheorem{porism}[theorem]{Porism}
\newtheorem{remark}[theorem]{Remark}
\newtheorem{interlude}[theorem]{Interlude}
\newtheorem{example}[theorem]{Example}
\newtheorem*{notation}{Notation}
\newtheorem*{claim}{Claim}

% Common Algebraic Structures
\newcommand{\R}{\mathbb{R}}
\newcommand{\Q}{\mathbb{Q}}
\newcommand{\Z}{\mathbb{Z}}
\newcommand{\N}{\mathbb{N}}
\newcommand{\bbC}{\mathbb{C}} 
\newcommand{\K}{\mathbb{K}} % Base field which is either \R or \bbC
\newcommand{\F}{\mathbb{F}} % Base field which is either \R or \bbC
\newcommand{\calA}{\mathcal{A}} % Banach Algebras
\newcommand{\calB}{\mathcal{B}} % Banach Algebras
\newcommand{\calI}{\mathcal{I}} % ideal in a Banach algebra
\newcommand{\calJ}{\mathcal{J}} % ideal in a Banach algebra
\newcommand{\frakM}{\mathfrak{M}} % sigma-algebra
\newcommand{\calO}{\mathcal{O}} % Ring of integers
\newcommand{\bbA}{\mathbb{A}} % Adele (or ring thereof)
\newcommand{\bbI}{\mathbb{I}} % Idele (or group thereof)

% Categories
\newcommand{\catTopp}{\mathbf{Top}_*}
\newcommand{\catGrp}{\mathbf{Grp}}
\newcommand{\catTopGrp}{\mathbf{TopGrp}}
\newcommand{\catSet}{\mathbf{Set}}
\newcommand{\catTop}{\mathbf{Top}}
\newcommand{\catRing}{\mathbf{Ring}}
\newcommand{\catCRing}{\mathbf{CRing}} % comm. rings
\newcommand{\catMod}{\mathbf{Mod}}
\newcommand{\catMon}{\mathbf{Mon}}
\newcommand{\catMan}{\mathbf{Man}} % manifolds
\newcommand{\catDiff}{\mathbf{Diff}} % smooth manifolds
\newcommand{\catAlg}{\mathbf{Alg}}
\newcommand{\catRep}{\mathbf{Rep}} % representations 
\newcommand{\catVec}{\mathbf{Vec}}

% Group and Representation Theory
\newcommand{\chr}{\operatorname{char}}
\newcommand{\Aut}{\operatorname{Aut}}
\newcommand{\GL}{\operatorname{GL}}
\newcommand{\im}{\operatorname{im}}
\newcommand{\tr}{\operatorname{tr}}
\newcommand{\id}{\mathbf{id}}
\newcommand{\cl}{\mathbf{cl}}
\newcommand{\Gal}{\operatorname{Gal}}
\newcommand{\Tr}{\operatorname{Tr}}
\newcommand{\sgn}{\operatorname{sgn}}
\newcommand{\Sym}{\operatorname{Sym}}
\newcommand{\Alt}{\operatorname{Alt}}

% Commutative and Homological Algebra
\newcommand{\spec}{\operatorname{spec}}
\newcommand{\mspec}{\operatorname{m-spec}}
\newcommand{\Spec}{\operatorname{Spec}}
\newcommand{\MaxSpec}{\operatorname{MaxSpec}}
\newcommand{\Tor}{\operatorname{Tor}}
\newcommand{\tor}{\operatorname{tor}}
\newcommand{\Ann}{\operatorname{Ann}}
\newcommand{\Supp}{\operatorname{Supp}}
\newcommand{\Hom}{\operatorname{Hom}}
\newcommand{\End}{\operatorname{End}}
\newcommand{\coker}{\operatorname{coker}}
\newcommand{\limit}{\varprojlim}
\newcommand{\colimit}{%
  \mathop{\mathpalette\colimit@{\rightarrowfill@\textstyle}}\nmlimits@
}
\makeatother


\newcommand{\fraka}{\mathfrak{a}} % ideal
\newcommand{\frakb}{\mathfrak{b}} % ideal
\newcommand{\frakc}{\mathfrak{c}} % ideal
\newcommand{\frakf}{\mathfrak{f}} % face map
\newcommand{\frakg}{\mathfrak{g}}
\newcommand{\frakh}{\mathfrak{h}}
\newcommand{\frakm}{\mathfrak{m}} % maximal ideal
\newcommand{\frakn}{\mathfrak{n}} % naximal ideal
\newcommand{\frakp}{\mathfrak{p}} % prime ideal
\newcommand{\frakq}{\mathfrak{q}} % qrime ideal
\newcommand{\fraks}{\mathfrak{s}}
\newcommand{\frakt}{\mathfrak{t}}
\newcommand{\frakz}{\mathfrak{z}}
\newcommand{\frakA}{\mathfrak{A}}
\newcommand{\frakB}{\mathfrak{B}}
\newcommand{\frakI}{\mathfrak{I}}
\newcommand{\frakJ}{\mathfrak{J}}
\newcommand{\frakK}{\mathfrak{K}}
\newcommand{\frakL}{\mathfrak{L}}
\newcommand{\frakN}{\mathfrak{N}} % nilradical 
\newcommand{\frakO}{\mathfrak{O}} % dedekind domain
\newcommand{\frakP}{\mathfrak{P}} % Prime ideal above
\newcommand{\frakQ}{\mathfrak{Q}} % Qrime ideal above 
\newcommand{\frakR}{\mathfrak{R}} % jacobson radical
\newcommand{\frakU}{\mathfrak{U}}
\newcommand{\frakV}{\mathfrak{V}}
\newcommand{\frakW}{\mathfrak{W}}
\newcommand{\frakX}{\mathfrak{X}}

% General/Differential/Algebraic Topology 
\newcommand{\scrA}{\mathscr{A}}
\newcommand{\scrB}{\mathscr{B}}
\newcommand{\scrC}{\mathscr{C}}
\newcommand{\scrD}{\mathscr{D}}
\newcommand{\scrF}{\mathscr{F}}
\newcommand{\scrM}{\mathscr{M}}
\newcommand{\scrN}{\mathscr{N}}
\newcommand{\scrP}{\mathscr{P}}
\newcommand{\scrO}{\mathscr{O}} % sheaf
\newcommand{\scrR}{\mathscr{R}}
\newcommand{\scrS}{\mathscr{S}}
\newcommand{\bbH}{\mathbb H}
\newcommand{\Int}{\operatorname{Int}}
\newcommand{\psimeq}{\simeq_p}
\newcommand{\wt}[1]{\widetilde{#1}}
\newcommand{\RP}{\mathbb{R}\text{P}}
\newcommand{\CP}{\mathbb{C}\text{P}}

% Miscellaneous
\newcommand{\wh}[1]{\widehat{#1}}
\newcommand{\calM}{\mathcal{M}}
\newcommand{\calP}{\mathcal{P}}
\newcommand{\onto}{\twoheadrightarrow}
\newcommand{\into}{\hookrightarrow}
\newcommand{\Gr}{\operatorname{Gr}}
\newcommand{\Span}{\operatorname{Span}}
\newcommand{\ev}{\operatorname{ev}}
\newcommand{\weakto}{\stackrel{w}{\longrightarrow}}

\newcommand{\define}[1]{\textcolor{blue}{\textit{#1}}}
% \newcommand{\caution}[1]{\textcolor{red}{\textit{#1}}}
\newcommand{\important}[1]{\textcolor{red}{\textit{#1}}}
\renewcommand{\mod}{~\mathrm{mod}~}
\renewcommand{\le}{\leqslant}
\renewcommand{\leq}{\leqslant}
\renewcommand{\ge}{\geqslant}
\renewcommand{\geq}{\geqslant}
\newcommand{\Res}{\operatorname{Res}}
\newcommand{\floor}[1]{\left\lfloor #1\right\rfloor}
\newcommand{\ceil}[1]{\left\lceil #1\right\rceil}
\newcommand{\gl}{\mathfrak{gl}}
\newcommand{\ad}{\operatorname{ad}}
\newcommand{\Stab}{\operatorname{Stab}}
\newcommand{\bfX}{\mathbf{X}}
\newcommand{\Ind}{\operatorname{Ind}}
\newcommand{\bfG}{\mathbf{G}}
\newcommand{\rank}{\operatorname{rank}}
\newcommand{\calo}{\mathcal{o}}
\newcommand{\frako}{\mathfrak{o}}
\newcommand{\Cl}{\operatorname{Cl}}

\newcommand{\idim}{\operatorname{idim}}
\newcommand{\pdim}{\operatorname{pdim}}
\newcommand{\Ext}{\operatorname{Ext}}
\newcommand{\co}{\operatorname{co}}
\newcommand{\bfO}{\mathbf{O}}
\newcommand{\bfF}{\mathbf{F}} % Fitting Subgroup
\newcommand{\Syl}{\operatorname{Syl}}
\newcommand{\nor}{\vartriangleleft}
\newcommand{\noreq}{\trianglelefteqslant}
\newcommand{\subnor}{\nor\!\nor}
\newcommand{\Soc}{\operatorname{Soc}}
\newcommand{\core}{\operatorname{core}}
\newcommand{\Sd}{\operatorname{Sd}}
\newcommand{\mesh}{\operatorname{mesh}}
\newcommand{\sminus}{\setminus}
\newcommand{\diam}{\operatorname{diam}}
\newcommand{\Ass}{\operatorname{Ass}}
\newcommand{\projdim}{\operatorname{proj~dim}}
\newcommand{\injdim}{\operatorname{inj~dim}}
\newcommand{\gldim}{\operatorname{gl~dim}}
\newcommand{\embdim}{\operatorname{emb~dim}}
\newcommand{\hght}{\operatorname{ht}}
\newcommand{\depth}{\operatorname{depth}}
\newcommand{\ul}[1]{\underline{#1}}
\newcommand{\type}{\operatorname{type}}
\newcommand{\CM}{\operatorname{CM}}
\newcommand{\cech}[1]{\mathbin{\check{#1}}}
\newcommand{\cdim}{\operatorname{cdim}}
\newcommand{\Der}{\operatorname{Der}}
\newcommand{\trdeg}{\operatorname{trdeg}}
\newcommand{\Kom}{\operatorname{Kom}}
\newcommand{\scrK}{\mathscr{K}}
\newcommand{\Mor}{\operatorname{Mor}}
\newcommand{\op}{\mathrm{op}}

\geometry {
    margin = 1in
}

\titleformat
{\section}
[block]
{\Large\bfseries\sffamily}
{\S\thesection}
{0.5em}
{\centering}
[]


\titleformat
{\subsection}
[block]
{\normalfont\bfseries\sffamily}
{\S\S}
{0.5em}
{\centering}
[]


\begin{document}
\title{Derived and Triangulated Categories}
\author{Swayam Chube}
\date{Last Updated: \today}
\maketitle
\tableofcontents

\hrulefill 

We fix some notation before proceeding. Categories will usually denoted by calligraphic symbols such as $\scrA,\scrB,\scrC,\scrD$. The opposite category of a category $\scrA$ is denoted by $\scrA^{\op}$. Corresponding to each object $A\in\scrA$, there is an object $A^{\op}\in\scrA^{\op}$ and corresponding to each morphism $f\colon A\to B$ in $\scrA$, there is a morphism $f^{\op}\in\scrA^{\op}$. If $A\xrightarrow{f} B\xrightarrow{g} C$ are morphisms in $\scrA$, then $C^{\op}\xrightarrow{g^{\op}} B^{\op}\xrightarrow{f^{\op}} A^{\op}$ with $f^\op\circ g^\op = (g\circ f)^\op$.

\section{Localization of Categories}
\begin{theorem}
	Let $\scrA$ be a category, and $S$ be a class of morphisms in $\scrA$. Then there is a category $\scrA[S^{-1}]$ and a functor $Q\colon\scrA\to\scrA[S^{-1}]$ such that for every functor $F\colon\scrA\to\scrB$ such that $F(s)$ is an isomorphism in $\scrB$ for each $s\in S$, there is a unique functor $G\colon\scrA[S^{-1}]\to\scrB$ making 
	\begin{equation*}
		\xymatrix {
			\scrA\ar[r]^F\ar[d]_Q & \scrB\\
			\scrA[S^{-1}]\ar@{.>}[ru]_-{\exists!~G}
		}
	\end{equation*}
	commute. Further, the pair $(\scrA[S^{-1}], Q)$ is unique up to a unique isomorphism of categories and is called the \define{localization} of $\scrA$ by the class of morphisms $S$.
\end{theorem}

\subsection{Localizing Classes}

Quite generally, the category $\scrA[S^{-1}]$ is quite ugly and difficult to work with. Therefore, we restrict ourselves to a more managable class $S$ of localizing morphisms.

\begin{definition}
	Let $\scrA$ be a category. A class of morphisms $S$ in $\scrA$ is said to be a \define{localizing class} if 
	\begin{enumerate}[label=(\textbf{LC}\arabic*)]
		\item For any object $M\in\scrA$, $\id_A\in S$.
		\item If $s, t$ are composable morphisms in $S$, then so is their composition.
		\item 
		\begin{enumerate}
			\item Every diagram of the form 
			\begin{equation*}
				\xymatrix {
					& L\ar[d]^s\\
					M\ar[r]_f & N
				}
			\end{equation*}
			with $f\in\Mor(\scrA)$ and $s\in S$ can be enlarged to a commutative square 
			\begin{equation*}
				\xymatrix {
					K\ar@{.>}[r]^g\ar@{.>}[d]_t & L\ar[d]^s\\
					M\ar[r]_f & N
				}
			\end{equation*}
			for some $K\in\scrA$, $g\in\Mor(\scrA)$, and $s\in S$. \label{pullback-type}
			\item Every diagram of the form 
			\begin{equation*}
				\xymatrix {
					N\ar[r]^f\ar[d]_s & M\\
					L
				}
			\end{equation*}
			with $f\in\Mor(\scrA)$ and $s\in S$ can be enlarged to a commutative square 
			\begin{equation*}
				\xymatrix {
					N\ar[r]^f\ar[d]_s & M\ar@{.>}[d]^t\\
					L\ar@{.>}[r]_g & K
				}
			\end{equation*}
			with $K\in\scrA$, $g\in\Mor(\scrA)$, and $t\in S$.\label{pushout-type}
		\end{enumerate}
		\item Let $f,g\colon M\to N$ be two morphisms in $\scrA$. Then \label{annihilation-condition}
		\begin{equation*}
			\exists~s\in S\text{ such that }s\circ f = s\circ g\iff \exists~t\in S\text{ such that }f\circ t = g\circ t.
		\end{equation*}
	\end{enumerate}
\end{definition}

\begin{center}
Clearly, if $S$ is a localizing class in $\scrA$, then $S^\op = \{s^\op\colon s\in S\}$ is a localizing class in $\scrA^\op$.
\end{center}

Our goal will now be to describe $\scrA[S^{-1}]$ given that $S$ is a localizing class of morphisms in $\scrA$. Define a \define{left roof} between two objects $M$ and $N$ in $\scrA$ to be a diagram of the form 
\begin{equation*}
	\xymatrix {
		& L\ar[ld]_s\ar[rd]^f & \\
		M & & N
	}
\end{equation*}
where $s\in S$ and $f\in\Mor(\scrA)$. Two left roofs 
\begin{equation*}
	\xymatrix {
		& L\ar[ld]_s\ar[rd]^f & \\
		M & & N
	}
	\quad\text{and}\quad 
	\xymatrix {
		& K\ar[ld]_t\ar[rd]^g & \\
		M & & N
	}
\end{equation*}
are said to be \define{equivalent} if there exists an object $H\in\scrA$ and morphisms $p\colon H\to L$ and $q\colon H\to K$ making the diagram
\begin{equation*}
	\xymatrix {
		& L\ar[ld]_s\ar[rd]^f & \\
		M & H\ar[u]_{p}\ar[d]^{q} & N\\
		& K\ar[lu]^t\ar[ru]_g & 
	}
\end{equation*}
commute and $s\circ p = q\circ t\in S$. It can be checked that this is indeed an equivalence relation.

Analogously a \define{right roof} between two objects $M$ and $N$ in $\scrA$ is defined to ba diagram of the form 
\begin{equation*}
	\xymatrix {
		& L &\\
		M\ar[ru]^f & & N\ar[lu]_s
	}
\end{equation*}
where $s\in S$ and $f\in\Mor(\scrA)$. Clearly there is a natural bijection between the left roofs in $\scrA$ and the right roofs in $\scrA^\op$ with respect to $S$ and $S^\op$ respectively. Two right roofs 
\begin{equation*}
	\xymatrix {
		& L &\\
		M\ar[ru]^f & & N\ar[lu]_s
	}\quad\text{and}\quad
	\xymatrix {
		& K &\\
		M\ar[ru]^g & & N\ar[lu]_t
	}
\end{equation*}
are said to be equivalent if there exists an object $H\in\scrA$ and morphisms $p\colon L\to H$ and $q\colon K\to H$ such that 
\begin{equation*}
	\xymatrix {
		& L\ar[d]^p & \\
		M\ar[ru]^f\ar[rd]_g & H & N\ar[lu]_s\ar[ld]^t\\
		& K\ar[u]_q & 
	}
\end{equation*}
commutes. Again, it can be checked that this is indeed an equivalence relation. In fact, a shorter way to conclude this is to move from $\scrA$ to $\scrA^\op$ since left roofs are mapped to right roofs. It is clear that two left roofs in $\scrA$ are equivalent if and only if the corresponding right roofs in $\scrA^\op$ are equivalent.

Next, we show how to ``compose'' two equivalence classes of left roofs. Begin by considering two representatives 
\begin{equation*}
	\xymatrix {
		& L\ar[ld]_s\ar[rd]^f & \\
		M & & N
	}
	\quad\text{and}\quad 
	\xymatrix {
		& K\ar[ld]_t\ar[rd]^g & \\
		N & & P.
	}
\end{equation*}
According to \ref{pullback-type}, we obtain a commutative diagram 
\begin{equation*}
	\xymatrix {
		& & U\ar[ld]_u\ar[rd]^h & & \\
		& L\ar[ld]_s\ar[rd]^f & & K\ar[ld]_t\ar[rd]^g & \\
		M & & N & & P
	}
\end{equation*}
with $u\in S$. Define the composition of the aforementioned equivalence classes to be the left roof 
\begin{equation*}
	\xymatrix {
		& U\ar[ld]_{s\circ u}\ar[rd]^{g\circ h} & \\
		M & & N
	}
\end{equation*}
since $s\circ u\in S$. It is a bit tedious, but it can be checked that this ``composition'' is well-defined. Once this is done, it is clear that the ``composition'' must be associative. That is, given three representatives 
\begin{equation*}
	\xymatrix {
		& L\ar[ld]_s\ar[rd]^f & \\
		M & & N
	}
	\quad
	\xymatrix {
		& K\ar[ld]_t\ar[rd]^g & \\
		N & & P.
	}
	\quad\text{and}
	\xymatrix { 
		& U\ar[ld]_u\ar[rd]^h &\\
		P & & R
	}
\end{equation*}
using \thref{pullback-type} repeatedly, we can complete this to a commutative diagram
\begin{equation*}
	\xymatrix {
		& & & C\ar[ld]_{u'}\ar[rd]^c & & &\\
		& & A\ar[ld]_{s'}\ar[rd]^{a} & & B\ar[ld]_{t'}\ar[rd]^b & & \\
		& L\ar[ld]_s\ar[rd]^f & & K\ar[ld]_t\ar[rd]^g & & U\ar[ld]_u\ar[rd]^h & \\
		M & & N & & P & & R
	}
\end{equation*}
and so it is clear that either composition 
\begin{equation*}
	\left(
	\xymatrix {
		& L\ar[ld]_s\ar[rd]^f & \\
		M & & N
	}
	\circ
	\xymatrix {
		& K\ar[ld]_t\ar[rd]^g & \\
		N & & P.
	}
	\right)
	\circ
	\xymatrix { 
		& U\ar[ld]_u\ar[rd]^h &\\
		P & & R
	}
\end{equation*}
or 
\begin{equation*}
	\xymatrix {
		& L\ar[ld]_s\ar[rd]^f & \\
		M & & N
	}
	\circ
	\left(
	\xymatrix {
		& K\ar[ld]_t\ar[rd]^g & \\
		N & & P.
	}
	\circ
	\xymatrix { 
		& U\ar[ld]_u\ar[rd]^h &\\
		P & & R
	}
	\right)
\end{equation*}
is equal to the equivalence class of the left roof 
\begin{equation*}
	\xymatrix {
		& C\ar[ld]_{s\circ s'\circ u'}\ar[rd]^{h\circ b\circ c} & \\
		M & & R.
	}
\end{equation*}
Finally, for each $M\in\scrA$, consider the left roof 
\begin{equation*}
	\xymatrix {
		& M\ar[ld]_{\id_M}\ar[rd]^{\id_M} & \\
		M & & M.
	}
\end{equation*}
For any left roof $\xymatrix { & L\ar[ld]_s\ar[rd]^f & \\ M & & N }$, one can compute their composition using the diagram 
\begin{equation*}
	\xymatrix {
		& & L\ar[ld]_s\ar[rd]^f & & \\
		& M\ar[ld]_{\id_M}\ar[rd]^{\id_M} & & L\ar[ld]_s\ar[rd]^f & \\
		M & & M & & N
	}
\end{equation*}
which yields the latter left roof.

\begin{mdframed}
Thus, we can define a category $\scrA_S$ where $\operatorname{ob}(\scrA_S) = \operatorname{ob}(\scrA)$, and $\Mor_{\scrA_S}(M, N)$ is the set of equivalence classes of left roofs equipped with composition and identity maps as defined above. 
\end{mdframed}

There is a natural functor $Q\colon\scrA\to\scrA_S$ which is the identity on objects and sends a morphism $f\colon M\to N$ in $\scrA$ to the equivalence class of the left roof 
\begin{equation*}
	\xymatrix {
		& M\ar[ld]_{\id_M}\ar[rd]^f & \\
		M & & N
	}
\end{equation*}
in $\scrA_S$. Indeed, it clearly takes $\id_M$ to the roof representing the identity at $M$ in $\scrA_S$; further, if $M\xrightarrow{f} N\xrightarrow{g} P$ are two composable morphisms, then we have a commutative diagram 
\begin{equation*}
	\xymatrix {
		&& M\ar[ld]_{\id_M}\ar[rd]^f && \\
		& M\ar[ld]_{\id_M}\ar[rd]^f & & N\ar[ld]_{\id_N}\ar[rd]^g & \\
		M & & N & & P
	}
\end{equation*}
so that the composition of the bottom two left roofs is 
\begin{equation*}
	\xymatrix {
		& M\ar[ld]_{\id_M}\ar[rd]^{g\circ f} & \\
		M & & P
	} = Q(g\circ f).
\end{equation*}
Thus $Q$ is indeed a functor. Finally, we claim that the pair $(Q,\scrA_S)$ has the universal property of localization. Indeed, let $F\colon\scrA\to\scrB$ be a functor sending every $s\in S$ to an isomorphism $F(s)$ in $\scrB$. Define a functor $G\colon\scrA_S\to\scrB$ such that 
\begin{equation*}
	G(A) = F(A) \quad\text{for every object }A\in\scrA,
\end{equation*}
and 
\begin{equation*}
	G\left(
		\xymatrix{
			& L\ar[ld]_{s}\ar[rd]^f & \\
			M & & N
		}
	\right) = F(f)\circ F(s)^{-1}.
\end{equation*}
It is easily checked that this is well-defined on the equivalence class of left roofs. That $G$ is a functor is also a trivial verification, and by construction, $F = G\circ Q$.

Finally, we must show that such a $G$ is unique. Indeed, if $F = G\circ Q$, for each object $A\in\scrA$, we must have $F(A) = G(Q(A)) = G(A)$. Now, a left roof 
\begin{equation*}
	\xymatrix { 
		& L\ar[ld]_s\ar[rd]^f & \\
		M & & N
	}
\end{equation*}
can be decomposed as the composition 
\begin{equation*}
	\xymatrix {
		& L\ar[ld]_s\ar[rd]^{\id_L} & & L\ar[ld]_{\id_L}\ar[rd]^f & \\
		M & & L & & N
	}
\end{equation*}
which is easy to see by completing the diagram above by putting an $L$ at the peak and identity morphisms from it to both the $L$'s below it. But note that 
\begin{equation*}
	\xymatrix {
		& L\ar[ld]_{s}\ar[rd]^{\id_L} & \\
		M & & L
	}\text{ is the inverse of }
	\xymatrix {
		& L\ar[ld]_{\id_L}\ar[rd]^s & \\
		L && M
	}
\end{equation*}
so that 
\begin{equation*}
	G\left(
		\xymatrix {
			& L\ar[ld]_{s}\ar[rd]^{\id_L} & \\
			M & & L
		}
	\right) = 
	G\left(
		\xymatrix {
			& L\ar[ld]_{\id_L}\ar[rd]^s & \\
			L && M
		}
	\right)^{-1} = F(s)^{-1},
\end{equation*}
and hence 
\begin{equation*}
	G\left(\xymatrix{
		& L\ar[ld]_s\ar[rd]^f & \\
		M & & N
	}\right) = F(f)\circ F(s)^{-1},
\end{equation*}
which completes the proof of uniqueness. We have therefore shown: 
\begin{theorem}
	Let $\scrA$ be a category and $S$ be a localizing class of morphisms in $\scrA$. Then the functor $Q\colon\scrA\to\scrA_S$ as described above is the localization of the category $\scrA$ at $S$.
\end{theorem}

\subsection{Localization and Subcategories}

\begin{theorem}
	Let $\scrA$ be a category, $\scrB\subseteq\scrA$ a full subcategory, and $S$ a localizing class of morphisms in $\scrA$. Suppose 
	\begin{enumerate}[label=(\textbf{LS}\arabic*)]
		\item $S_{\scrB} = S\cap\Mor(\scrB)$ is a localizing class in $\scrB$, and \label{ls1}
		\item for each morphism $s\colon N\to M$ in $S$ with $M\in\scrB$, there exists a morphism $u\colon P\to N$ with $P\in\scrB$ such that $s\circ u\in S$.\label{ls2}
	\end{enumerate}
	Then the induced functor $\scrB[S_{\scrB}^{-1}]\to\scrA[S^{-1}]$ is fully faithful.
	\begin{equation*}
		\xymatrix {
			\scrB\ar@{^{(}->}[r]\ar[d]_{Q_B} & \scrA\ar[d]^{Q_A}\\
			\scrB[S_\scrB^{-1}]\ar[r] & \scrA[S^{-1}]
		}
	\end{equation*}
\end{theorem}
\begin{proof}
	Since $S_\scrB$ is a localizing class in $\scrB$, by tracing the arrows in the commutative diagram of functors above, the map $\scrB[S_\scrB^{-1}]\to\scrA[S^{-1}]$ explicitly sends a roof in $\scrB$ to the equivalence class of the same roof in $\scrA[S^{-1}]$. 

	First, we show that the map is full. Let 
	\begin{equation*}
		\xymatrix {
			& L\ar[ld]_s\ar[rd]^f & \\
			M & & N
		}
	\end{equation*}
	be a left roof in $\scrA[S^{-1}]$ with $M, N\in\scrB$. Then due to \ref{ls2}, there exists $U\in\scrB$ and a morphism $u\colon U\to L$ such that $s\circ u\in S$, and hence in $S_{\scrB}$.

	To see that the map is faithful, suppose two left roofs 
	\begin{equation*}
		\xymatrix {
			& L\ar[ld]_s\ar[rd]^f & \\
			M & & N
		}
		\quad\text{and}\quad 
		\xymatrix {
			& K\ar[ld]_t\ar[rd]^g & \\
			M && N
		}
	\end{equation*}
	in $\scrB[S_{\scrB}^{-1}]$ are equivalent in $\scrA[S^{-1}]$, that is, there exists an object $H\in\scrA$, and morphisms $p\colon H\to L$ and $q\colon H\to K$ in $\scrA$ such that 
	\begin{equation*}
		\xymatrix {
			& L\ar[ld]_s\ar[rd]^f & \\
			M & H\ar[u]_p\ar[d]^q & N\\
			& K\ar[lu]^t\ar[ru]_g & 
		}
	\end{equation*}
	commutes and $s\circ p = t\circ q\in S$. Hence, there exists an object $U\in\scrB$ and a morphism $u\colon U\to H$ such that $s\circ p\circ u = t\circ q \circ u\in S$, and hence in $S_{\scrB}$. Thus, the diagram 
	\begin{equation*}
		\xymatrix {
			& L\ar[ld]_s\ar[rd]^f & \\
			M & U\ar[u]_{p\circ u}\ar[d]^{q\circ u} & N\\
			& K\ar[lu]^t\ar[ru]_g & 
		}
	\end{equation*}
	commutes and consists of morphisms in $\scrB$. Thus, the two roofs are equivalent in $\scrB[S_{\scrB}^{-1}]$.
\end{proof}

\subsection{Localizing Additive Categories}

We begin by showing that one can ``take common demonimators'' for morphisms in $\scrA[S^{-1}]$.
\begin{lemma}\thlabel{common-denominator}
	Let $\scrA$ be a category (not necessarily additive) and $S$ a localizing class of morphisms in $\scrA$. Let 
	\begin{equation*}
		\xymatrix {
			& L_i\ar[ld]_{s_i}\ar[rd]^{f_i} &\\
			M && N
		}
	\end{equation*}
	be left roofs in $\scrA$ representing morphisms $\varphi_i\colon M\to N$ in $\scrA[S^{-1}]$ for $1\le i\le n$ respectively. Then there exists an object $L\in\scrA$ and morphisms $L\xrightarrow{s} M\in S$, and $g_i\colon L\to N$ for $1\le i\le n$ such that 
	\begin{equation*}
		\xymatrix {
			& L\ar[ld]_s\ar[rd]^{g_i} & \\
			M && N
		}
	\end{equation*}
	represents $\varphi_i$ for $1\le i\le n$.
\end{lemma}
\begin{proof}
	We prove this by induction on $n$. The base case $n = 1$ is trivial. Suppose now that $n > 1$ and that the statement has been proven for $n - 1$. Hence, there exists an object $K$ and a morphism $K\xrightarrow{t} M\in S$ such that 
	\begin{equation*}
		\xymatrix {
			& K\ar[rd]^{h_i}\ar[ld]_t & \\
			M & & N
		}
	\end{equation*}
	represents $\varphi_i$ for $1\le i\le n - 1$. Using \thref{pullback-type}, there exists a commutative diagram 
	\begin{equation*}
		\xymatrix {
			U\ar[r]^{v}\ar[d]_{u} & L_n\ar[d]^{s_n}\\
			K\ar[r]_{t} & M
		}
	\end{equation*}
	with $u\in S$. Set $s = s_n\circ v = t\circ u\in S$. Then the diagram 
	\begin{equation*}
		\xymatrix {
			& K\ar[ld]_t\ar[rd]^{h_i} & \\
			M & U\ar[u]_u\ar@{=}[d] & N\\
			& U\ar[lu]^s\ar[ru]_{h_i\circ u} & 
		}
	\end{equation*}
	commutes for $1\le i\le n - 1$, and 
	\begin{equation*}
		\xymatrix {
			& L_n\ar[rd]^{f_n}\ar[ld]_{s_n} & \\
			M & U\ar[u]_{v}\ar@{=}[d] & N\\
			& U\ar[lu]^{s_n\circ v}\ar[ru]_{f_n\circ v} & 
		}
	\end{equation*}
	commutes with $s_n\circ v = s\in S$. Set $g_i = h_i\circ u$ for $1\le i\le n - 1$ and $g_n = f_n\circ v$; then 
	\begin{equation*}
		\xymatrix {
			& U\ar[ld]_s\ar[rd]^{g_i} & \\
			M && N
		}
	\end{equation*}
	represents $\varphi_i$ for $1\le i\le n$, thereby completing the proof.
\end{proof}

Now let $\scrA$ be an \emph{additive category} and $S$ a localizing class of morphisms in $\scrA$. We shall show that $\scrA[S^{-1}]$ is naturally an additive category. For objects $M, N\in\scrA[S^{-1}]$ and morphisms 
\begin{equation*}
	\varphi = \xymatrix { 
		& L\ar[ld]_s\ar[rd]^f & \\
		M & & N
	}
	\quad\text{and}\quad 
	\psi = \xymatrix {
		& K\ar[ld]_t\ar[rd]^g & \\
		M & & N
	}
\end{equation*}
in $\scrA[S^{-1}]$; using \thref{common-denominator}, we can find an object $U$ and morphisms $U\xrightarrow{u} M\in S$ and $f', g'\colon U\to N$ such that 
\begin{equation*}
	\varphi = \xymatrix {
		& U\ar[ld]_u\ar[rd]^{f'} & \\
		M & & N
	}\quad\text{and}\quad 
	\psi = \xymatrix {
		& U\ar[ld]_{u}\ar[rd]^{g'} & \\
		M & & N.
	}
\end{equation*}
Define 
\begin{equation*}
	\varphi + \psi = \xymatrix {
		& U\ar[ld]_u\ar[rd]^{f' + g'} & \\
		M & & N.
	}
\end{equation*}
Note that there are three choices being made here: the choice of the representatives for $\varphi$ and $\psi$, and choice of ``common denominator'' for both morphisms. It follows that $\Mor_{\scrA[S^{-1}]}(M, N)$ has the structure of an abelian group. Further, it must be checked that 
\begin{equation*}
	\chi\circ(\varphi + \psi) = \chi\circ\varphi + \chi\circ\psi\quad\text{ and }\quad(\varphi + \psi)\circ\chi = \varphi\circ\chi + \psi\circ\chi
\end{equation*}
for suitably composable morphisms $\chi,\varphi,\psi$ in $\scrA[S^{-1}]$. The zero object in $\Mor_{\scrA[S^{-1}]}(M, N)$ is given by the morphism
\begin{equation*}
	\xymatrix {
		& M\ar[ld]_{\id_M}\ar[rd]^0 & \\
		M  && N
	}.
\end{equation*}
Finally, given objects $M, N\in\scrA[S^{-1}]$, define their direct sum/direct product to be the object $M\oplus N$ where the direct sum is taken in $\scrA$, and the canonical projections and injections are the images of those in $\scrA$. Again, it is straightforward, but must be checked, that these have the desired universal properties. In this way, $\scrA[S^{-1}]$ has been given a natural additive structure.

Finally, note that the localization functor $Q\colon\scrA\to\scrA[S^{-1}]$ is an additive functor. Indeed, if $f, g\colon M\to N$ are morphisms, then 
\begin{equation*}
	Q(f) = 
	\xymatrix {
		& M\ar[ld]_{\id_M}\ar[rd]^f & \\
		M & & N
	}\quad\text{ and }\quad 
	Q(g) = 
	\xymatrix {
		& M\ar[ld]_{\id_M}\ar[rd]^g & \\
		M & & N
	}
\end{equation*}
so that by definition,
\begin{equation*}
	Q(f) + Q(g) = 
	\xymatrix {
		& M\ar[rd]^{f + g}\ar[ld]_{\id_M} & \\
		M & & N
	} = Q(f + g).
\end{equation*}

Finally, we have 
\begin{theorem}
	Let $\scrA$ be an additive category and $S$ a localizing class of morphisms in $\scrA$. Then the category $\scrA[S^{-1}]$ is naturally an additive category and the localizing functor $Q\colon\scrA\to\scrA[S^{-1}]$ is additive. 

	Further, given any additive functor $F\colon\scrA\to\scrB$ such that $F(s)$ is an isomorphism in $\scrB$ for each $s\in S$, there exists a unique additive functor $G\colon\scrA[S^{-1}]\to B$ making 
	\begin{equation*}
	\xymatrix {
		\scrA\ar[r]^F\ar[d]_Q & \scrB\\
		\scrA[S^{-1}]\ar[ru]_G
	}
	\end{equation*}
	commute.
\end{theorem}
\begin{proof}
	We have already proved the first part of the theorem. As for the second part, suppose $\varphi,\psi\colon M\to N$ are two morphisms in $\scrA[S^{-1}]$. Using \thref{common-denominator}, we may suppose that they are represented by 
	\begin{equation*}
		\xymatrix {
			& L\ar[ld]_s\ar[rd]^f & \\
			M &  & N
		}\quad\text{ and }\quad 
		\xymatrix {
			& L\ar[ld]_s\ar[rd]^g & \\
			M & & N
		}
	\end{equation*}
	respectively. As a result, 
	\begin{equation*}
		G(\varphi + \psi) = F(f + g)F(s)^{-1} = F(f)F(s)^{-1} + F(g)F(s)^{-1} = G(\varphi) + G(\psi),
	\end{equation*}
	so that $G$ is an additive functor. That $G$ is unique has already been argued.
\end{proof}

\begin{lemma}\thlabel{morphism-being-zero}
	Let $\varphi\colon M\to N$ be a morphism in $\scrA[S^{-1}]$ represented by a left roof 
	\begin{equation*}
		\xymatrix {
			& L\ar[ld]_s\ar[rd]^f & \\
			M & & N
		}.
	\end{equation*}
	Then the following are equivalent: 
	\begin{enumerate}[label=(\arabic*)]
		\item $\varphi = 0$.
		\item There exists $t\in S$ such that $t\circ f = 0$.
		\item There exists $t\in S$ such that $f\circ t = 0$.
	\end{enumerate}
\end{lemma}
\begin{proof}
	Clearly (2) and (3) are equivalent due to \ref{annihilation-condition}. Now if $\varphi = 0$, then $Q(f)\circ Q(s)^{-1} = 0$, so that $Q(f) = 0$, i.e., 
	\begin{equation*}
		\xymatrix {
			& L\ar[ld]_{\id_L}\ar[rd]^f &\\
			L & & N
		}
	\end{equation*}
	represents $0$. Hence, there exists an object $H$ and and morphisms $p, q\colon H\to L$ such that 
	\begin{equation*}
		\xymatrix {
			& L\ar[ld]_{\id_L}\ar[rd]^f &\\
			L & H\ar[u]_p\ar[d]^q & N\\
			& L\ar[lu]^{\id_L}\ar[ru]_0 & 
		}
	\end{equation*}
	commutes and $p = q\in S$. The commutativity implies $f\circ p = 0$, so that $(1)\implies(2)$.

	Conversely, suppose $f\circ t = 0$ for some $t\colon H\to L\in S$. Then the diagram 
	\begin{equation*}
		\xymatrix {
			& L\ar[ld]_s\ar[rd]^f & \\
			M & H\ar[u]_t\ar[d]^{s\circ t} & N\\
			& M\ar[lu]^{\id_M}\ar[ru]_0 & 
		}
	\end{equation*}
	commutes with $s\circ t\in S$. This shows that $\varphi = 0$, thereby completing the proof.
\end{proof}

\begin{corollary}
	Let $M$ be an object in $\scrA$. Then the following are equivalent: 
	\begin{enumerate}[label=(\arabic*)]
		\item $Q(M) = 0$. 
		\item There exists an object $N\in\scrA$ such that the zero morphism $N\xrightarrow{0} M$ is in $S$.
		\item There exists an object $N\in\scrA$ such that the zero morphism $M\xrightarrow{0} N$ is in $S$.
	\end{enumerate}
\end{corollary}
\begin{proof}
	The equivalence of (2) and (3) follows from an immediate application of \thref{pullback-type} and \thref{pushout-type}. Now if $Q(M) = 0$, then $Q(\id_M) = 0$, so that by \thref{morphism-being-zero} there exists $s\in S$ with $\id_M\circ s = 0$, and hence $s = 0$. This proves (2).
	
	Conversely, if there is an object $N\in\scrA$ with $N\xrightarrow{0} M \in S$, then the image of this map, which is the zero map $Q(N)\xrightarrow{0} Q(M)$ must be an isomorphism. Thus $Q(N) = Q(M) = 0$.
\end{proof}

\begin{lemma}
	Let $f\colon M\to N$ be a morphism in $\scrA$. Then 
	\begin{enumerate}[label=(\arabic*)]
		\item If $f$ is monic, then so is $Q(f)$. 
		\item If $f$ is epic, then so is $Q(f)$.
	\end{enumerate}
\end{lemma}
\begin{proof}
	% TODO: Add in later
\end{proof}

\subsection{Localization of Abelian Categories}
\end{document}