\documentclass[12pt]{article}

% \usepackage{./arxiv}

\title{Homological methods in Commutative Algebra}
\author{Swayam Chube}
\date{\today}

\usepackage[utf8]{inputenc} % allow utf-8 input
\usepackage[T1]{fontenc}    % use 8-bit T1 fonts
\usepackage{hyperref}       % hyperlinks
\usepackage{url}            % simple URL typesetting
\usepackage{booktabs}       % professional-quality tables
\usepackage{amsfonts}       % blackboard math symbols
\usepackage{nicefrac}       % compact symbols for 1/2, etc.
\usepackage{microtype}      % microtypography
\usepackage{graphicx}
\usepackage{natbib}
\usepackage{doi}
\usepackage{amssymb}
\usepackage{bbm}
\usepackage{amsthm}
\usepackage{amsmath}
\usepackage{xcolor}
\usepackage{theoremref}
\usepackage{enumitem}
\usepackage{mathpazo}
% \usepackage{euler}
\usepackage{mathrsfs}
\setlength{\marginparwidth}{2cm}
\usepackage{todonotes}
\usepackage{stmaryrd}
\usepackage[all,cmtip]{xy} % For diagrams, praise the Freyd–Mitchell theorem 
\usepackage{marvosym}
\usepackage{geometry}
\usepackage{titlesec}

\renewcommand{\qedsymbol}{$\blacksquare$}

% Uncomment to override  the `A preprint' in the header
% \renewcommand{\headeright}{}
% \renewcommand{\undertitle}{}
% \renewcommand{\shorttitle}{}

\hypersetup{
    pdfauthor={Lots of People},
    colorlinks=true,
}

\newtheoremstyle{thmstyle}%               % Name
  {}%                                     % Space above
  {}%                                     % Space below
  {}%                             % Body font
  {}%                                     % Indent amount
  {\bfseries\scshape}%                            % Theorem head font
  {.}%                                    % Punctuation after theorem head
  { }%                                    % Space after theorem head, ' ', or \newline
  {\thmname{#1}\thmnumber{ #2}\thmnote{ (#3)}}%                                     % Theorem head spec (can be left empty, meaning `normal')

\newtheoremstyle{defstyle}%               % Name
  {}%                                     % Space above
  {}%                                     % Space below
  {}%                                     % Body font
  {}%                                     % Indent amount
  {\bfseries\scshape}%                            % Theorem head font
  {.}%                                    % Punctuation after theorem head
  { }%                                    % Space after theorem head, ' ', or \newline
  {\thmname{#1}\thmnumber{ #2}\thmnote{ (#3)}}%                                     % Theorem head spec (can be left empty, meaning `normal')

\theoremstyle{thmstyle}
\newtheorem{theorem}{Theorem}[section]
\newtheorem{lemma}[theorem]{Lemma}
\newtheorem{proposition}[theorem]{Proposition}

\theoremstyle{defstyle}
\newtheorem{definition}[theorem]{Definition}
\newtheorem*{corollary}{Corollary}
\newtheorem{remark}[theorem]{Remark}
\newtheorem{example}[theorem]{Example}
\newtheorem*{notation}{Notation}

% Common Algebraic Structures
\newcommand{\R}{\mathbb{R}}
\newcommand{\Q}{\mathbb{Q}}
\newcommand{\Z}{\mathbb{Z}}
\newcommand{\N}{\mathbb{N}}
\newcommand{\bbC}{\mathbb{C}} 
\newcommand{\K}{\mathbb{K}} % Base field which is either \R or \bbC
\newcommand{\calA}{\mathcal{A}} % Banach Algebras
\newcommand{\calB}{\mathcal{B}} % Banach Algebras
\newcommand{\calI}{\mathcal{I}} % ideal in a Banach algebra
\newcommand{\calJ}{\mathcal{J}} % ideal in a Banach algebra
\newcommand{\frakM}{\mathfrak{M}} % sigma-algebra
\newcommand{\calO}{\mathcal{O}} % Ring of integers
\newcommand{\bbA}{\mathbb{A}} % Adele (or ring thereof)
\newcommand{\bbI}{\mathbb{I}} % Idele (or group thereof)

% Categories
\newcommand{\catTopp}{\mathbf{Top}_*}
\newcommand{\catGrp}{\mathbf{Grp}}
\newcommand{\catTopGrp}{\mathbf{TopGrp}}
\newcommand{\catSet}{\mathbf{Set}}
\newcommand{\catTop}{\mathbf{Top}}
\newcommand{\catRing}{\mathbf{Ring}}
\newcommand{\catCRing}{\mathbf{CRing}} % comm. rings
\newcommand{\catMod}{\mathbf{Mod}}
\newcommand{\catMon}{\mathbf{Mon}}
\newcommand{\catMan}{\mathbf{Man}} % manifolds
\newcommand{\catDiff}{\mathbf{Diff}} % smooth manifolds
\newcommand{\catAlg}{\mathbf{Alg}}
\newcommand{\catRep}{\mathbf{Rep}} % representations 
\newcommand{\catVec}{\mathbf{Vec}}

% Group and Representation Theory
\newcommand{\chr}{\operatorname{char}}
\newcommand{\Aut}{\operatorname{Aut}}
\newcommand{\GL}{\operatorname{GL}}
\newcommand{\im}{\operatorname{im}}
\newcommand{\tr}{\operatorname{tr}}
\newcommand{\id}{\mathbf{id}}
\newcommand{\cl}{\mathbf{cl}}
\newcommand{\Gal}{\operatorname{Gal}}
\newcommand{\Tr}{\operatorname{Tr}}
\newcommand{\sgn}{\operatorname{sgn}}
\newcommand{\Sym}{\operatorname{Sym}}
\newcommand{\Alt}{\operatorname{Alt}}

% Commutative and Homological Algebra
\newcommand{\spec}{\operatorname{spec}}
\newcommand{\mspec}{\operatorname{m-spec}}
\newcommand{\Tor}{\operatorname{Tor}}
\newcommand{\tor}{\operatorname{tor}}
\newcommand{\Ann}{\operatorname{Ann}}
\newcommand{\Supp}{\operatorname{Supp}}
\newcommand{\Hom}{\operatorname{Hom}}
\newcommand{\End}{\operatorname{End}}
\newcommand{\coker}{\operatorname{coker}}
\newcommand{\limit}{\varprojlim}
\newcommand{\colimit}{%
  \mathop{\mathpalette\colimit@{\rightarrowfill@\textstyle}}\nmlimits@
}
\makeatother


\newcommand{\fraka}{\mathfrak{a}} % ideal
\newcommand{\frakb}{\mathfrak{b}} % ideal
\newcommand{\frakc}{\mathfrak{c}} % ideal
\newcommand{\frakf}{\mathfrak{f}} % face map
\newcommand{\frakg}{\mathfrak{g}}
\newcommand{\frakh}{\mathfrak{h}}
\newcommand{\frakm}{\mathfrak{m}} % maximal ideal
\newcommand{\frakn}{\mathfrak{n}} % naximal ideal
\newcommand{\frakp}{\mathfrak{p}} % prime ideal
\newcommand{\frakq}{\mathfrak{q}} % qrime ideal
\newcommand{\fraks}{\mathfrak{s}}
\newcommand{\frakt}{\mathfrak{t}}
\newcommand{\frakz}{\mathfrak{z}}
\newcommand{\frakA}{\mathfrak{A}}
\newcommand{\frakI}{\mathfrak{I}}
\newcommand{\frakJ}{\mathfrak{J}}
\newcommand{\frakK}{\mathfrak{K}}
\newcommand{\frakL}{\mathfrak{L}}
\newcommand{\frakN}{\mathfrak{N}} % nilradical 
\newcommand{\frakO}{\mathfrak{O}} % dedekind domain
\newcommand{\frakP}{\mathfrak{P}} % Prime ideal above
\newcommand{\frakQ}{\mathfrak{Q}} % Qrime ideal above 
\newcommand{\frakR}{\mathfrak{R}} % jacobson radical
\newcommand{\frakU}{\mathfrak{U}}
\newcommand{\frakX}{\mathfrak{X}}

% General/Differential/Algebraic Topology 
\newcommand{\scrA}{\mathscr A}
\newcommand{\scrB}{\mathscr B}
\newcommand{\scrF}{\mathscr F}
\newcommand{\scrN}{\mathscr N}
\newcommand{\scrP}{\mathscr P}
\newcommand{\scrR}{\mathscr R}
\newcommand{\scrS}{\mathscr S}
\newcommand{\bbH}{\mathbb H}
\newcommand{\Int}{\operatorname{Int}}
\newcommand{\psimeq}{\simeq_p}
\newcommand{\wt}[1]{\widetilde{#1}}
\newcommand{\RP}{\mathbb{R}\text{P}}
\newcommand{\CP}{\mathbb{C}\text{P}}

% Miscellaneous
\newcommand{\wh}[1]{\widehat{#1}}
\newcommand{\calM}{\mathcal{M}}
\newcommand{\calP}{\mathcal{P}}
\newcommand{\onto}{\twoheadrightarrow}
\newcommand{\into}{\hookrightarrow}
\newcommand{\Gr}{\operatorname{Gr}}
\newcommand{\Span}{\operatorname{Span}}
\newcommand{\ev}{\operatorname{ev}}
\newcommand{\weakto}{\stackrel{w}{\longrightarrow}}

\newcommand{\define}[1]{\textcolor{blue}{\textit{#1}}}
\newcommand{\caution}[1]{\textcolor{red}{\textit{#1}}}
\renewcommand{\mod}{~\mathrm{mod}~}
\renewcommand{\le}{\leqslant}
\renewcommand{\leq}{\leqslant}
\renewcommand{\ge}{\geqslant}
\renewcommand{\geq}{\geqslant}
\newcommand{\Res}{\operatorname{Res}}
\newcommand{\floor}[1]{\left\lfloor #1\right\rfloor}
\newcommand{\ceil}[1]{\left\lceil #1\right\rceil}
\newcommand{\gl}{\mathfrak{gl}}
\newcommand{\ad}{\operatorname{ad}}
\newcommand{\Stab}{\operatorname{Stab}}
\newcommand{\bfX}{\mathbf{X}}
\newcommand{\Ind}{\operatorname{Ind}}
\newcommand{\bfG}{\mathbf{G}}
\newcommand{\rank}{\operatorname{rank}}
\newcommand{\calo}{\mathcal{o}}
\newcommand{\frako}{\mathfrak{o}}
\newcommand{\Cl}{\operatorname{Cl}}

\newcommand{\idim}{\operatorname{idim}}
\newcommand{\pdim}{\operatorname{pdim}}
\newcommand{\Ext}{\operatorname{Ext}}
\newcommand{\co}{\operatorname{co}}
\newcommand{\depth}{\operatorname{depth}}
\newcommand{\projdim}{\operatorname{proj~dim}}
\newcommand{\injdim}{\operatorname{inj~dim}}

\geometry {
    margin = 1in
}

\titleformat
{\section}
[block]
{\Large\bfseries\scshape}
{\S\thesection}
{0.5em}
{\centering}
[]


\titleformat
{\subsection}
[block]
{\normalfont\bfseries\sffamily}
{\S\S}
{0.5em}
{\centering}
[]


\begin{document}
\maketitle

\section{The Koszul Complex}

\begin{definition}
    Let $A$ be a ring and $x_1,\dots,x_n\in A$. Set $K_0 = A$ and for $1\le p\le n$, let 
    \begin{equation*}
        K_p = \bigoplus_{1\le i_1 < \cdots < i_p\le n} A e_{i_1}\wedge\dots\wedge e_{i_p},
    \end{equation*}
    which is a free module of rank $\binom{n}{p}$.

    Define $d: K_p\to K_{p - 1}$ as 
    \begin{equation*}
        d\left(e_{i_1}\wedge\dots\wedge e_{i_p}\right) = \sum_{r = 1}^p (-1)^{r - 1} x_{i_r} e_{i_1}\wedge\dots\wedge\wh{e_{i_r}}\wedge\dots\wedge e_{i_p},
    \end{equation*}
    and extending linearly. This is called the \define{Koszul complex}.  For an $A$-module $M$, we set $K_\bullet(\underline x, M) = K_\bullet(\underline x)\otimes_A M$. The homologies of this complex $H_p\left(K_\bullet(\underline x, M)\right)$ are denoted by $H_p(\underline x, M)$. For a complex $C_\bullet$ of $A$-modules, we set $C_\bullet(\underline x) = C_\bullet\otimes_A K_\bullet(\underline x)$.\footnote{Recall that the tensor product of two complexes is obtained by taking the total complex corresponding to the tensor double complex.} 
\end{definition}

\begin{proposition}
    For $p\ge 2$, $d\circ d = 0$ as a map $K_p\to K_{p - 2}$.
\end{proposition}
\begin{proof}
    In the expression for $(d\circ d)\left(e_{i_1}\wedge\dots\wedge e_{i_p}\right)$, we find the coefficient of 
    \begin{equation*}
        e_{i_1}\wedge\dots\wedge\wh{e_{i_a}}\wedge\dots\wedge\wh{e_{i_b}}\wedge\dots\wedge e_{i_p},
    \end{equation*}
    where $1\le a < b\le p$. The coefficient is equal to the coefficient in 
    \begin{equation*}
        (-1)^{a - 1}x_{i_a}d\left(e_{i_1}\wedge\dots\wh e_{i_a}\wedge\dots\wedge e_{i_p}\right) + (-1)^{b - 1} x_{i_b}d\left(e_{i_1}\wedge\dots\wedge e_{i_b}\wedge\dots\wedge e_{i_p}\right),
    \end{equation*}
    which is equal to 
    \begin{equation*}
        (-1)^{a - 1}x_{i_a}\cdot(-1)^{b - 2}x_{i_b} + (-1)^{b - 1}x_{i_b}\cdot(-1)^{a - 1}x_{i_a} = 0. \qedhere
    \end{equation*}
\end{proof}

\begin{theorem}
    Let $C_\bullet$ be a complex of $A$-modules and $x\in A$. There is an exact sequence of complexes 
    \begin{equation*}
        0\longrightarrow C_\bullet\longrightarrow C_\bullet(x)\longrightarrow C_\bullet[-1]\longrightarrow 0.
    \end{equation*}
    This furnishes an exact sequence 
    \begin{equation*}
        \cdots\to H_p(C_\bullet)\to H_p(C_\bullet(x))\to H_{p - 1}(C_\bullet)\xrightarrow{(-1)^{p - 1}x} H_{p - 1}(C_\bullet)\to\cdots.
    \end{equation*}
    Further, we have $x\cdot H_{p}(C_\bullet(x)) = 0$ for all $p$.
\end{theorem}

\begin{corollary}
    Let $M$ be an $A$-module and $x_1,\dots,x_n\in A$. Then, $(\underline x)$ annihilates $H_p(\underline x, M)$ for all $p$.
\end{corollary}
\begin{proof}
    Induct on $n$. The inductive step follows from the fact that 
    \begin{equation*}
        K_\bullet(x_1,\dots, x_n, M)\cong K_\bullet(x_n)\otimes_A K(x_1,\dots, x_{n - 1}, M).
    \end{equation*}
    We know that $(x_1,\dots, x_{n - 1})$ annihilates the homology groups of the latter and hence, they annihilate the homology groups of $K_\bullet(x_1,\dots, x_n, M)$. Further, due to the preceding theorem, $x_n$ annihilates the homologies of the above tensor product. This completes the proof.
\end{proof}

\begin{theorem}
    Let $M$ be an $A$-module and $x_1,\dots,x_n\in A$ an $M$-sequence. Then 
    \begin{equation*}
        H_p(\underline x, M) = 0\quad\text{for } p > 0\qquad H_0(\underline x, M) = M/(\underline x)M.
    \end{equation*}
\end{theorem}
\begin{proof}
    
\end{proof}

\begin{theorem}
    Let $A$ be a Noethering ring, $M$ a finite $A$-odule, and $I$ an ideal of $A$; suppose that $IM\ne M$. For a positive integer $n$, the following conditions are equivalent: 
    \begin{enumerate}[label=(\alph*)]
        \item $\Ext^i_A(N, M) = 0$ for $0\le i < n$ and for any finite $A$-module $N$ with $\Supp(N)\subseteq V(I)$. 
        \item $\Ext^i_A(A/I, M) = 0$ for $0\le i < n$. 
        \item $\Ext^i_A(N, M) = 0$ for $0\le i < n$ and \emph{some} finite $A$-module $N$ with $\Supp(N) = V(I)$. 
        \item there exists an $M$-sequence of length $n$ contained in $I$.
    \end{enumerate}
\end{theorem}
\begin{proof}
\end{proof}

\begin{corollary}
    Let $A$ be a Noetherian ring, $I$ an ideal of $A$, and $M$ a finite $A$-module such that $M\ne IM$; then the length of a maximal $M$-sequence in $I$ is determined by 
    \begin{equation*}
        \Ext^i_A(A/I, M) = 0\quad\text{for } i < n\quad\text{and}\quad\Ext^n_A(A/I, M)\ne 0.
    \end{equation*}
\end{corollary}

This integer is denoted by $\depth(I, M)$ and is called the \define{$I$-depth} of $M$. In other words, 
\begin{equation*}
    \depth(I, M) = \inf\left\{i\colon\Ext^i_A(A/I, M)\ne 0\right\}.
\end{equation*}
For a Noetherian local ring $(A,\frakm, k)$, we write $\depth_A M$ for $\depth(\frakm, M)$.

\begin{corollary}
    Let $A$ be a Noetherian ring, $I, I'$ ideals of $A$, and $M$ a finite $A$-module such that $M\ne IM$ and $M\ne I'M$. Then, $\depth(I, M) = \depth(I', M)$.
\end{corollary}

\section{Regular Rings}

\begin{definition}
    Let $(A,\frakm, k)$ be a local ring, $M$ a finite $A$-module. A \define{minimal (free) resolution} is an exact sequence $L_\bullet\to M\xrightarrow{\varepsilon} 0$ such that 
    \begin{enumerate}[label=(\alph*)]
        \item 
    \end{enumerate}
\end{definition}

\begin{lemma}
    Let $(A,\frakm, k)$ be a local ring, and $M$ a finite $A$-module. Suppose $L_\bullet\to M$ is a minimal resolution of $M$; then 
    \begin{enumerate}[label=(\alph*)]
        \item $\dim_k \Tor^A_i(M, k) = \rank L_i$ for all $i$, 
        \item $\projdim M = \sup\left\{i\colon\Tor^A_i(M, k)\ne 0\right\}\le\projdim_A k$,
        \item if $M\ne 0$ and $\projdim M = r < \infty$, then for any finite $A$-module $N\ne 0$ we have $\Ext^r_A(M, N)\ne 0$.
    \end{enumerate}
\end{lemma}
\end{document}