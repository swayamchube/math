\subsection{Regular Rings}

\begin{definition}
    Let $(A,\frakm, k)$ be a local ring and let $M$ be a finite $A$-module. An exact sequence 
    \begin{equation*}
        \cdots\to L_i\xrightarrow{d_i} L_{i - 1}\xrightarrow{d_{i - 1}}\cdots\to L_1\xrightarrow{d_1} L_0\xrightarrow\varepsilon M\to 0
    \end{equation*}
    is called a \define{minimal (free) resolution} of $M$ if 
    \begin{itemize}
        \item each $L_i$ is a finite free $A$-module
        \item $0 = \overline d_i: L_{i}\otimes_A k\to L_{i - 1}\otimes_A k$, or equivalently $d_i L_i\subseteq\frakm L_{i - 1}$ for all $i\ge 1$, and 
        \item $\overline\varepsilon: L_0\otimes_A k\to M\otimes_A k$ is an isomorphism.
    \end{itemize}
\end{definition}

It is easy to see that a minimal free resolution exists for every finite module over a Noetherian local ring; at each stage simply take a minimal generating set of the kernel and continue.

\begin{lemma}\thlabel{lem:proj-dim-char}
    Let $(A,\frakm, k)$ be a local ring, and $M$ a finite $A$-module. Suppose $L_\bullet$ is a minimal resolution of $M$; then 
    \begin{enumerate}[label=(\arabic*)]
        \item $\dim_k\Tor^A_i(M, k) = \rank L_i$ for all $i$.
        \item $\projdim_A M = \sup\left\{i\colon\Tor^A_i(M, k)\ne 0\right\}\le\projdim_A k$, 
        \item if $M\ne 0$ and $\projdim_A M = r < \infty$, then for any finite $A$-module $N\ne 0$, we have $\Ext^r_A(M, N)\ne 0$.
    \end{enumerate}
\end{lemma}
\begin{proof}
\begin{enumerate}[label=(\arabic*)]
    \item This follows immediately from the fact that $\overline d_i = 0$ for all $i\ge 1$.
    \item The second inequality is straightforward. For if $\projdim_A k = \infty$, then there is nothing to prove. If $\projdim_A k < \infty$, then take a projective resolution of this length and tensor with $A$ to conclude.
    
    From (1) it immediately follows that $\projdim_A M\le\sup\left\{i\colon \Tor^A_i(M, k)\ne 0\right\}$, since this quantity is precisely the length of the minimal free resolution of $M$. If $\projdim_A M = \infty$, then there is nothing to prove. If $\projdim_A M < \infty$, then take a projective resolution of $M$ achieving this length and tensor with $k$ whence it follows that $\sup\left\{i\colon\Tor^A_i(M, k)\ne 0\right\}\le\projdim_A M$, as desired.

    \item Applying $\Hom_A(-, N)$ to the resolution $L_\bullet\to M$, we obtain a complex which ends with 
    \begin{equation*}
        \Hom_A(L_{r - 1}, N)\xrightarrow{d_r^\ast}\Hom_A(L_r, N)\to 0,
    \end{equation*}
    where $\Ext^r_A(M, N)$ is the cokernel of the above map. Since each $L_i$ is free, we can write $\Hom_A(L_i, N)$ as a direct sum of some copies of $N$ and we can express every boundary map $d_i: L_i\to L_{i - 1}$ as a matrix with entries in $\frakm$. It follows that $d_i^\ast$ is given by the same matrix (with entries in $\frakm$). Hence, the image of $d_r^\ast$ is contained in $\frakm\Hom_A(L_r, N)$, which is properly contained in $\Hom_A(L_r, N)$ by Nakayama's lemma. This completes the proof.\qedhere
\end{enumerate}
\end{proof}

\begin{remark}
    The above proof also shows that the minimal resolution is indeed the one that achieves the projective dimension of a module.
\end{remark}

\begin{theorem}[Auslander-Buchsbaum]\thlabel{thm:auslander-buchsbaum-formula}
    Let $A$ be a Noetherian local ring and $M\ne 0$ a finite $A$-module. If $\projdim_A M < \infty$, then 
    \begin{equation*}
        \projdim_A M + \depth M = \depth A.
    \end{equation*}
\end{theorem}
\begin{proof}
    We shall induct on $h = \projdim_A M$. If $h = 0$, then $M$ is a free module, and there is nothing to prove. If $h = 1$, then the minimal resolution looks like 
    \begin{equation*}
        0\to A^m\xrightarrow{\varphi} A^n\to M\to 0,
    \end{equation*}
    where $\varphi$ is given by an $n\times m$ matrix with entries in $\frakm$.
\end{proof}

\begin{lemma}\thlabel{lem:trivial-lemma}
    Let $A$ be a ring and $n\ge 0$ an integer. Then the following are equivalent: 
    \begin{enumerate}[label=(\arabic*)]
        \item $\projdim_A M\le n$ for every $A$-module $M$, 
        \item $\projdim_A M\le n$ for every finite $A$-module $M$, 
        \item $\injdim_A N\le n$ for every $A$-module $N$, and 
        \item $\Ext^{n + 1}_A(M, N) = 0$ for all $A$-modules $M$ and $N$.
    \end{enumerate}
\end{lemma}
\begin{proof}
    All implications are straightforward.
\end{proof}

\begin{definition}
    The \define{global dimension} of a ring is defined as 
    \begin{equation*}
        \gldim A = \sup\left\{\projdim M\colon M\text{ is an $A$-module}\right\}.
    \end{equation*}
\end{definition}

Due to \thref{lem:trivial-lemma}, the above supremum can also be taken over all finite $A$-modules. Further, if $(A,\frakm, k)$ is a Noetherian local ring, due to \thref{lem:proj-dim-char} (2), we have 
\begin{equation*}
    \gldim A = \projdim_A k.
\end{equation*}

Recall that the \define{embedding dimension} of a Noetherian local ring $(A,\frakm, k)$ is defined to be 
\begin{equation*}
    \embdim A = \dim_k \frakm/\frakm^2.
\end{equation*}

\begin{theorem}[Serre]
    Let $(A,\frakm, k)$ be a Noetherian local ring. Then the following are equivalent 
    \begin{enumerate}[label=(\arabic*)]
        \item $A$ is regular;
        \item $\gldim A = \dim A$; 
        \item $\gldim A < \infty$.
    \end{enumerate}
\end{theorem}
\begin{proof}
    $(1)\implies(2)$ Choose a regular system of parameters $x_1,\dots,x_n\in\frakm$, so that $n = \dim A$. Since $\ub x = x_1,\dots,x_n$ is an $A$-sequence, it follows from \thref{thm:koszul-homology-vanishes} that $K_\bullet(\ub x)$ is exact, whence it is a free resolution of $k$. Note further that the transition matrices in the Koszul complex have entries lying in $\frakm$, whence the Koszul complex is a minimal free resolution of $\frakm$. Thus, 
    \begin{equation*}
        \gldim A = \projdim_A k = n = \dim A,
    \end{equation*}
    as desired. 

    $(2)\implies(3)$ is clear. We shall show that $(3)\implies(1)$. Let $\gldim A = r < \infty$, and set $\embdim A = s$. We shall show that $A$ is regular by induction on $s$. If $s = 0$, then $\frakm = 0$, and hence, $A$ is a field, so it is regular. 

    Suppose now that $s > 0$. We claim that $\frakm\notin\Ass_A(A)$. If not, then consider a minimal resolution of $k$, 
    \begin{equation*}
        0\to L_r\to L_{r - 1}\to\cdots\to L_0\to k\to 0,
    \end{equation*}
    where the maps are given by matrices with entries in $\frakm$. Now, there is some $0\ne a\in A$ such that $\frakm = \Ann_A(a)$. It follows that the element $(a,a,\dots,a)\in L_r$ lies in the kernel of the map $L_r\to L_{r - 1}$, a contradiction. 

    Thus $\frakm\notin\Ass_A(A)$. Choose 
    \begin{equation*}
        x\in\frakm\setminus\left(\frakm^2\cup\bigcup_{\frakp\in\Ass_A(A)}\frakp\right).
    \end{equation*}
    using prime avoidance\footnote{TODO: Add in the statement}. Then $x$ is $A$-regular, hence also $\frakm$-regular. Setting $B = A/xA$, and using \thref{lem:some-base-change-results} (2), we have $\Ext^i_A(\frakm, N) = \Ext^i_B(\frakm/x\frakm, N)$ for all $B$-modules $N$. Hence, $\Ext^{r + 1}_B(\frakm/x\frakm, N) = 0$ for every $B$-module $N$; that is, $\projdim_B \frakm/x\frakm\le r$.

    Next, we show that the natural surjection $\frakm/x\frakm\to \frakm/xA$ splits as $A$-modules (and hence as $B$-modules). First, choose a minimal generating set $x, x_2,\dots,x_s$ of $\frakm$ and set $\frakb = (x_2,\dots,x_s)$. Note that $\frakb\cap xA\subseteq x\frakm$. Indeed, if $y = a_2x_2 + \dots + a_nx_n = ax\in \frakb\cap xA$, then looking at the equality modulo $\frakm$, we see that $a\in\frakm$, whence $x\in\frakb\cap x\frakm\subseteq x\frakm$. Now, consider the chain of natural maps 
    \begin{equation*}
        \frac{\frakm}{xA} = \frac{\frakb + xA}{xA}\xrightarrow{\sim}\frac{\frakb}{\frakb\cap xA}\to\frac{\frakm}{x\frakm}\to\frac{\frakm}{xA}.
    \end{equation*}
    Their composition is the identity, and hence, the surjection $\frakm/x\frakm\to \frakm/xA$ splits. In particularly, this means that 
    \begin{equation*}
        \projdim_B \frakm/xA\le\projdim_B \frakm/x\frakm\le r.
    \end{equation*}
    Because of the exact sequence $0\to\frakm/xA\to B\to k\to 0$, we see that $\gldim B = \projdim_B k\le r + 1$. Since $\embdim B = r - 1$, the induction hypothesis gives that $B$ is a regular local ring. Now, since $x$ is $A$-regular, $\dim B = \dim A - 1$, and therefore, 
    \begin{equation*}
        \embdim A = \embdim B + 1 = \dim B + 1 = \dim A,
    \end{equation*}
    whence $A$ is a regular local ring, as desired.
\end{proof}

\begin{theorem}[Serre]
    Let $A$ be a regular local ring and $\frakp$ a prime ideal of $A$. Then $A_\frakp$ is a regular local ring.
\end{theorem}
\begin{proof}
    If $\projdim_A k < \infty$, then localizing a finite projective resolution of $k$ at $\frakp$, we obtain the desired conclusion.
\end{proof}

\begin{definition}
    A \define{regular ring} is a Noetherian ring such that the localization at every prime is a regular local ring.
\end{definition}

\subsection{Finite Free Resolutions}

\begin{lemma}[Schanuel]\thlabel{schanuel}
    Let $A$ be a ring and $M$ an $A$-module. Suppose that 
    \begin{equation*}
        0\to K\to P\to M\to 0\quad\text{ and }\quad 0\to K'\to P'\to M\to 0
    \end{equation*}
    are exact sequences with $P$ and $P'$ projective. Then $K\oplus P'\cong K'\oplus P$.
\end{lemma}
\begin{proof}
    Since $P$ and $P'$ are projective, there are maps 
    \begin{equation*}
        \xymatrix {
            0\ar[r] & K\ar[r] & P\ar[r]^\alpha\ar@<-.5ex>[d]_\lambda & M\ar[r]\ar@{=}[d] & 0\\
            0\ar[r] & K'\ar[r] & P'\ar[r]_{\alpha'}\ar@<-.5ex>[u]_{\lambda'} & M\ar[r] & 0
        }
    \end{equation*}
    $\lambda\colon P\to P'$ and $\lambda'\colon P'\to P$ making the square on the right commute. Adding in the summands $P'$ and $P$ to the respective rows, we obtain another commutative diagram 
    \begin{equation*}
        \xymatrix {
            0\ar[r] & K\oplus P'\ar[r]\ar@{.>}[d]_\theta & P\oplus P'\ar[r]^-{(\alpha, 0)}\ar@<-.5ex>[d]_\varphi & M\ar[r]\ar@{=}[d] & 0\\
            0\ar[r] & K'\oplus P\ar[r] & P\oplus P'\ar[r]_-{(0, \alpha')}\ar@<-.5ex>[u]_{\psi} & M\ar[r] & 0
        }
    \end{equation*}
    where $\varphi\colon P\oplus P'\to P\oplus P'$ is defined by 
    \begin{equation*}
        \varphi\begin{pmatrix}x\\ x'\end{pmatrix} = 
        \begin{pmatrix}
            \id_P & -\lambda'\\
            \lambda & \id_{P'} - \lambda\circ\lambda'
        \end{pmatrix}
        \begin{pmatrix}
            x\\ x'
        \end{pmatrix},
    \end{equation*}
    and 
    \begin{equation*}
        \psi\begin{pmatrix}
            x\\x'
        \end{pmatrix} = 
        \begin{pmatrix}
            \id_P - \lambda'\circ\lambda & \lambda'\\
            -\lambda & \id_{P'}
        \end{pmatrix}
        \begin{pmatrix}
            x\\x'
        \end{pmatrix}.
    \end{equation*}
    One can check that $\varphi\circ\psi = \psi\circ\varphi = \id_{P\oplus P'}$, so that $\varphi$ and $\psi$ are isomorphisms. There is a map $\theta\colon K\oplus P'\to K'\oplus P$ making the entire diagram above commute. Using the five-lemma or otherwise on this diagram, one concludes that $\theta$ is an isomorphism.
\end{proof}

\begin{lemma}[Generalized Schanuel]\thlabel{generalized-schanuel}
    Let $A$ be a ring and $M$ an $A$-module. Suppose 
    \begin{equation*}
        0\to P_n\to\cdots\to P_0\to M\to 0\quad\text{ and } 0\to Q_n\to\cdots\to Q_0\to M\to 0
    \end{equation*}
    are two exact sequences with $P_i$ and $Q_i$ projective for $0\le i\le n - 1$, then 
    \begin{equation*}
        P_0\oplus Q_1\oplus\cdots\cong Q_0\oplus P_1\oplus\cdots.
    \end{equation*}
\end{lemma}
\begin{proof}
    Induct on $n$. The base  case with $n = 0$ is precisely \thref{schanuel}. Let $K$ denote the kernel of $P_0\to M$ and $K'$ the kernel of $Q_0\to M$. Due to \thref{schanuel}, $K\oplus Q_0\cong K'\oplus P_0$. Add in the summands $Q_0$ and $P_0$ to the respective resolutions as follows: 
    \begin{equation*}
        \xymatrix {
            0\ar[r] & P_n\ar[r] & \cdots\ar[r] & P_2\ar[r] & P_1\oplus Q_0\ar[r] & K\oplus Q_0\ar[r]\ar[d]^\wr & 0\\
            0\ar[r] & Q_n\ar[r] & \cdots\ar[r] & Q_2\ar[r] & Q_1\oplus P_0\ar[r] & K'\oplus P_0\ar[r] & 0.
        }
    \end{equation*}
    Using the inductive hypothesis, we have the desired isomorphism.
\end{proof}

\begin{definition}
    A \define{finite free resolution} of an $A$-module $M$ is an exact sequence 
    \begin{equation*}
        0\to F_n\to\cdots \to F_1\to F_0\to M\to 0
    \end{equation*}
    such that each $F_i$ is a finite free $A$-module. If $M$ admits a finite free resolution as above, we define its \define{Euler number} to be 
    \begin{equation*}
        \chi_A(M) = \sum_{i = 0}^\infty (-1)^i \rank_A F_i.
    \end{equation*}
\end{definition}

Clearly, due to \thref{generalized-schanuel}, $\chi(M)$ is independent of the chosen finite free resolution. Further, if $M$ admits an FFR over $A$, then for any prime ideal $\frakp\subseteq A$, $M_\frakp$ admits an FFR over $A_\frakp$ and $\chi_A(M) = \chi_{A_\frakp}(M_\frakp)$.

\begin{proposition}\thlabel{maximal-ideal-associated}
    Let $(A,\frakm)$ be a local ring such that for each finite subset $E\subseteq\frakm$ there exists $0\ne y\in A$ with $y E = 0$. Then the only $A$-modules having an FFR over $A$ are the (finite rank) free modules.
\end{proposition}
\begin{proof}
    Let $0\to F_n\to F_{n - 1}\to\cdots\to F_0\to M\to 0$ be an FFR of $M$, and set $N = \coker\left(F_n\to F_{n - 1}\right)$. Our first goal will be to show that $N$ is a free module of finite rank. Clearly, $N$ is a finite $A$-module. If it were not free, then it would admit a minimal free resolution of the form $0\to L_1\to L_0\to N\to 0$, since it already admits a free resolution of length $1$. Using \thref{schanuel}, we have $L_1\oplus F_{n - 1} \cong L_0\oplus F_n$, so that $L_1$ is a finite rank free module. Treating $L_1$ as a submodule of $L_0$, we can write down a basis for $L_1$ in terms of a basis for $L_0$ with coefficients in $\frakm$ since the resolution is minimal. Thus, there would exist $0\ne y\in A$ annihilating all those coefficients, whence $y L_1 = 0$, a contradiction. Thus $N$ must be a finite rank free $A$-module.

    Coming back to the proof at hand, workin backwards from the given free resolution and replacing the map $F_n\to F_{n - 1}$ by $\coker\left(F_n\to F_{n - 1}\right)$ at each stage, we reduce to the case $0\to F_1\to F_0\to M\to 0$, which we have handled above. Hence, $M$ is a finite rank free module over $A$. Conversely, it is clear that every finite rank free $A$-module has an FFR.
\end{proof}

\begin{theorem}\thlabel{euler-number-non-negative}
    Let $A$ be a ring. If an $A$-module $M$ admits an FFR, then $\chi_A(M)\ge 0$.
\end{theorem}
\begin{proof}
    Let $\frakp$ be a minimal prime of $A$. Since $\chi_A(M) = \chi_{A_\frakp}(M_\frakp)$, we can replace $A$ by $A_\frakp$ and $M$ by $M_\frakp$ and assume that $(A,\frakm)$ is a local ring whose maximal ideal is equal to the nilradical. We claim that the hypothesis of \thref{maximal-ideal-associated} is satisfied. Indeed, let $x_1,\dots,x_r\in\frakm$. We shall induct on $r$ to show that there exists $0\ne y\in A$ such that $yx_i = 0$ for all $1\le i\le r$. If $r = 1$, then the nilpotence of $x_1$ implies the existence of such a $y$. Suppose $r > 1$, then using the inductive hypothesis, there exists $0\ne z\in A$ such that $zx_1 = \dots = zx_{r - 1} = 0$. Let $j\ge 1$ be the minimal integer such that $zx_r^j = 0$, which exists since $x_r$ is nilpotent. Choosing $y = zx_r^{j - 1}\ne 0$, we have that $yx_i = 0$ for $1\le i\le r$. As a consequence of \thref{maximal-ideal-associated}, we see that $M$ is finite free, so that $\chi_A(M)\ge 0$.
\end{proof}

\begin{corollary}
    Let $A$ be a ring and suppose there is an injective $A$-linear map $A^m\into A^n$, then $m\le n$.
\end{corollary}
\begin{proof}
    Let $M = \coker\left(A^m\into A^n\right)$. Then $M$ has a finite free resolution and $\chi_A(M) = n - m\ge 0$ due to \thref{euler-number-non-negative}, thereby completing the proof.
\end{proof}

\begin{theorem}[Auslander-Buchsbaum]\thlabel{equivalent-conditions-euler-number-zero}
    Let $A$ be a Noetherian ring and $M$ an $A$-module admitting an FFR. The following are equivalent: 
    \begin{enumerate}[label=(\arabic*)]
        \item $\Ann_A(M)\ne 0$. 
        \item $\chi_A(M) = 0$. 
        \item $\Ann_A(M)$ contains an $A$-regular element.
    \end{enumerate}
\end{theorem}
\begin{proof}
    $(1)\implies(2)$ Let $I = \Ann_A(M)\ne 0$ and set $J = \Ann_A(I)$. Suppose $\chi_A(M) > 0$. Choose any $\frakp\in\Ass_A(A)$. Then $\chi_{A_\frakp}(M_\frakp) > 0$, and hence $M_\frakp\ne 0$. Further, since $\frakp A_\frakp\in\Ass_{A_\frakp}(A_\frakp)$, it follows from \thref{maximal-ideal-associated} that $M_\frakp$ is a free $A_\frakp$-module. Now note that $I A_\frakp = \Ann_{A_\frakp}(M_\frakp) = 0$, so that $J\not\subseteq\frakp$. Since this holds for every $\frakp\in\Ass_A(A)$, it follows from Prime Avoidance that $J$ must contain an $A$-regular element. But since $J\cdot I = 0$, we would have $I = 0$, a contradiction. Thus $\chi_A(M) = 0$.

    $(2)\implies(3)$ If $\chi_A(M) = 0$, then as argued above, for every $\frakp\in\Ass_A(A)$, $M_\frakp$ is a free $A_\frakp$-module and $\chi_{A_\frakp}(M_\frakp) = 0$, whence $M_\frakp = 0$. Since $M$ is a finite $A$-module, this must imply that $\Ann_A(M)\not\subseteq\frakp$ for every $\frakp\in\Ass_A(A)$. This is equivalent to stating that $\Ann_A(M)$ contains an $A$-regular element. 

    $(3)\implies(1)$ is clear. This completes the proof.
\end{proof}
% TODO: Add in the proof of Ferrand-Vasconcelos whenever you get time.

\begin{definition}
    An $A$-module $M$ is said to be \define{stably free} if there exist finite free $A$-modules $F$ and $F'$ such that $M\oplus F\cong F'$ as $A$-modules.
\end{definition}

Clearly, every stably free module is projective and has an FFR, $0\to F\to F'\to M\to 0$. Conversely, we also have: 

\begin{lemma}
    A finite projective module having an FFR is stably free.
\end{lemma}
\begin{proof}
    We shall induct on the length of the FFR. The base cases of length $0$ and $1$ are trivial. Suppose now that 
    \begin{equation*}
        0\to F_n\to\cdots\to F_0\to P\to 0
    \end{equation*}
    is a finite free resolution of a projective $A$-module $M$ with $n\ge 2$. Let $K = \ker\left(F_0\to P\right)$. Then $0\to K\to F_0\to P\to 0$ splits, so that $K$ is a finite projective module admitting an FFR of length $n - 1$. Using the inductive hypothesis, $K$ is stably free, that is, there are finite free modules $F$ and $F'$ such that $K\oplus F\cong F'$. Hence, 
    \begin{equation*}
        P\oplus F'\cong P\oplus K\oplus F\cong F_0\oplus F,
    \end{equation*}
    so that $P$ is also stably free.
\end{proof}

\begin{lemma}\thlabel{ffr-implies-regular}
    Let $A$ be a Noetherian ring. If every finite $A$-module admits an FFR, then $A$ is a regular ring.
\end{lemma}
\begin{proof}
    Let $\frakp$ be a prime ideal in $A$. According to the hypothesis, the $A$-module $A/\frakp$ admits an FFR. Localizing this resolution at $\frakp$, one obtains an FFR of $\kappa(\frakp)$ over $A_\frakp$, whence $A_\frakp$ is a regular local ring. Thus $A$ is a regular ring.
\end{proof}

\subsection{Unique Factorization Domains}

\begin{theorem}
    Let $A$ be a Noetherian domain. Then $A$ is a UFD if and only if every height $1$ prime ideal in $A$ is principal.
\end{theorem}
\begin{proof}
    Suppose $A$ is a UFD and $\frakp$ a height $1$ prime ideal in $A$. Choose any $0\ne a\in\frakp$ and factorize $a = \pi_1\cdots\pi_n$ into irreducibles, which are the same things as primes in this case. Since $\frakp$ is a prime ideal, there exists a $\pi_i\in\frakp$. This gives a chain of prime ideals $(0)\subseteq(\pi_i)\subseteq\frakp$. Since $\frakp$ is height $1$, it follows that $\frakp = (\pi_i)$, i.e., is principal. 

    Conversely, suppose every height $1$ prime ideal in $A$ is principal. Every Noetherian domain is a factorization domain, therefore, it suffices to show that all irreducibles in $A$ are prime. Let $0\ne a\in A$ be an irreducible element and choose a prime ideal $\frakp$ minimal among those containing the ideal $(a)$. Due to the Hauptidealsatz, $\hght\frakp = 1$, so that $\frakp = (b)$ for some $0\ne b\in A$, whence there exists $0\ne c\in A$ such that $a = bc$. Since $A$ is irreducible, $c$ must be a unit, and hence $(a) = \frakp$, i.e., $a$ is a prime element in $A$, thereby completing the proof.
\end{proof}

\begin{theorem}
    Let $A$ be a Noetherian domain, $\Gamma$ a set of prime elements of $A$, and $S$ the multiplicative set generated by $\Gamma$. If $S^{-1}A$ is a UFD, then so is $A$.
\end{theorem}
\begin{proof}
    
\end{proof}

\begin{lemma}
    Let $A$ be an integral domain, and $\fraka$ an ideal of $A$ such that $\fraka\oplus A^n\cong A^{n + 1}$ for some $n\ge 0$. Then $\fraka$ is a principal ideal.
\end{lemma}
\begin{proof}
    % TODO: Add in later
\end{proof}