\subsection{Regular Rings}

\begin{definition}
    Let $(A,\frakm, k)$ be a local ring and let $M$ be a finite $A$-module. An exact sequence 
    \begin{equation*}
        \cdots\to L_i\xrightarrow{d_i} L_{i - 1}\xrightarrow{d_{i - 1}}\cdots\to L_1\xrightarrow{d_1} L_0\xrightarrow\varepsilon M\to 0
    \end{equation*}
    is called a \define{minimal (free) resolution} of $M$ if 
    \begin{itemize}
        \item each $L_i$ is a finite free $A$-module
        \item $0 = \overline d_i: L_{i}\otimes_A k\to L_{i - 1}\otimes_A k$, or equivalently $d_i L_i\subseteq\frakm L_{i - 1}$ for all $i\ge 1$, and 
        \item $\overline\varepsilon: L_0\otimes_A k\to M\otimes_A k$ is an isomorphism.
    \end{itemize}
\end{definition}

It is easy to see that a minimal free resolution exists for every finite module over a Noetherian local ring; at each stage simply take a minimal generating set of the kernel and continue.

\begin{lemma}\thlabel{lem:proj-dim-char}
    Let $(A,\frakm, k)$ be a local ring, and $M$ a finite $A$-module. Suppose $L_\bullet$ is a minimal resolution of $M$; then 
    \begin{enumerate}[label=(\arabic*)]
        \item $\dim_k\Tor^A_i(M, k) = \rank L_i$ for all $i$.
        \item $\projdim_A M = \sup\left\{i\colon\Tor^A_i(M, k)\ne 0\right\}\le\projdim_A k$, 
        \item if $M\ne 0$ and $\projdim_A M = r < \infty$, then for any finite $A$-module $N\ne 0$, we have $\Ext^r_A(M, N)\ne 0$.
    \end{enumerate}
\end{lemma}
\begin{proof}
\begin{enumerate}[label=(\arabic*)]
    \item This follows immediately from the fact that $\overline d_i = 0$ for all $i\ge 1$.
    \item The second inequality is straightforward. For if $\projdim_A k = \infty$, then there is nothing to prove. If $\projdim_A k < \infty$, then take a projective resolution of this length and tensor with $A$ to conclude.
    
    From (1) it immediately follows that $\projdim_A M\le\sup\left\{i\colon \Tor^A_i(M, k)\ne 0\right\}$, since this quantity is precisely the length of the minimal free resolution of $M$. If $\projdim_A M = \infty$, then there is nothing to prove. If $\projdim_A M < \infty$, then take a projective resolution of $M$ achieving this length and tensor with $k$ whence it follows that $\sup\left\{i\colon\Tor^A_i(M, k)\ne 0\right\}\le\projdim_A M$, as desired.

    \item Applying $\Hom_A(-, N)$ to the resolution $L_\bullet\to M$, we obtain a complex which ends with 
    \begin{equation*}
        \Hom_A(L_{r - 1}, N)\xrightarrow{d_r^\ast}\Hom_A(L_r, N)\to 0,
    \end{equation*}
    where $\Ext^r_A(M, N)$ is the cokernel of the above map. Since each $L_i$ is free, we can write $\Hom_A(L_i, N)$ as a direct sum of some copies of $N$ and we can express every boundary map $d_i: L_i\to L_{i - 1}$ as a matrix with entries in $\frakm$. It follows that $d_i^\ast$ is given by the same matrix (with entries in $\frakm$). Hence, the image of $d_r^\ast$ is contained in $\frakm\Hom_A(L_r, N)$, which is properly contained in $\Hom_A(L_r, N)$ by Nakayama's lemma. This completes the proof.\qedhere
\end{enumerate}
\end{proof}

\begin{remark}
    The above proof also shows that the minimal resolution is indeed the one that achieves the projective dimension of a module.
\end{remark}

\begin{theorem}[Auslander-Buchsbaum]\thlabel{thm:auslander-buchsbaum-formula}
    Let $A$ be a Noetherian local ring and $M\ne 0$ a finite $A$-module. If $\projdim_A M < \infty$, then 
    \begin{equation*}
        \projdim_A M + \depth M = \depth A.
    \end{equation*}
\end{theorem}
\begin{proof}
    We shall induct on $h = \projdim_A M$. If $h = 0$, then $M$ is a free module, and there is nothing to prove. If $h = 1$, then the minimal resolution looks like 
    \begin{equation*}
        0\to A^m\xrightarrow{\varphi} A^n\to M\to 0,
    \end{equation*}
    where $\varphi$ is given by an $n\times m$ matrix with entries in $\frakm$.
\end{proof}

\begin{lemma}\thlabel{lem:trivial-lemma}
    Let $A$ be a ring and $n\ge 0$ an integer. Then the following are equivalent: 
    \begin{enumerate}[label=(\arabic*)]
        \item $\projdim_A M\le n$ for every $A$-module $M$, 
        \item $\projdim_A M\le n$ for every finite $A$-module $M$, 
        \item $\injdim_A N\le n$ for every $A$-module $N$, and 
        \item $\Ext^{n + 1}_A(M, N) = 0$ for all $A$-modules $M$ and $N$.
    \end{enumerate}
\end{lemma}
\begin{proof}
    All implications are straightforward.
\end{proof}

\begin{definition}
    The \define{global dimension} of a ring is defined as 
    \begin{equation*}
        \gldim A = \sup\left\{\projdim M\colon M\text{ is an $A$-module}\right\}.
    \end{equation*}
\end{definition}

Due to \thref{lem:trivial-lemma}, the above supremum can also be taken over all finite $A$-modules. Further, if $(A,\frakm, k)$ is a Noetherian local ring, due to \thref{lem:proj-dim-char} (2), we have 
\begin{equation*}
    \gldim A = \projdim_A k.
\end{equation*}

Recall that the \define{embedding dimension} of a Noetherian local ring $(A,\frakm, k)$ is defined to be 
\begin{equation*}
    \embdim A = \dim_k \frakm/\frakm^2.
\end{equation*}

\begin{theorem}[Serre]
    Let $(A,\frakm, k)$ be a Noetherian local ring. Then the following are equivalent 
    \begin{enumerate}[label=(\arabic*)]
        \item $A$ is regular;
        \item $\gldim A = \dim A$; 
        \item $\gldim A < \infty$.
    \end{enumerate}
\end{theorem}
\begin{proof}
    $(1)\implies(2)$ Choose a regular system of parameters $x_1,\dots,x_n\in\frakm$, so that $n = \dim A$. Since $\ub x = x_1,\dots,x_n$ is an $A$-sequence, it follows from \thref{thm:koszul-homology-vanishes} that $K_\bullet(\ub x)$ is exact, whence it is a free resolution of $k$. Note further that the transition matrices in the Koszul complex have entries lying in $\frakm$, whence the Koszul complex is a minimal free resolution of $\frakm$. Thus, 
    \begin{equation*}
        \gldim A = \projdim_A k = n = \dim A,
    \end{equation*}
    as desired. 

    $(2)\implies(3)$ is clear. We shall show that $(3)\implies(1)$. Let $\gldim A = r < \infty$, and set $\embdim A = s$. We shall show that $A$ is regular by induction on $s$. If $s = 0$, then $\frakm = 0$, and hence, $A$ is a field, so it is regular. 

    Suppose now that $s > 0$. We claim that $\frakm\notin\Ass_A(A)$. If not, then consider a minimal resolution of $k$, 
    \begin{equation*}
        0\to L_r\to L_{r - 1}\to\cdots\to L_0\to k\to 0,
    \end{equation*}
    where the maps are given by matrices with entries in $\frakm$. Now, there is some $0\ne a\in A$ such that $\frakm = \Ann_A(a)$. It follows that the element $(a,a,\dots,a)\in L_r$ lies in the kernel of the map $L_r\to L_{r - 1}$, a contradiction. 

    Thus $\frakm\notin\Ass_A(A)$. Choose 
    \begin{equation*}
        x\in\frakm\setminus\left(\frakm^2\cup\bigcup_{\frakp\in\Ass_A(A)}\frakp\right).
    \end{equation*}
    using prime avoidance\footnote{TODO: Add in the statement}. Then $x$ is $A$-regular, hence also $\frakm$-regular. Setting $B = A/xA$, and using \thref{lem:some-base-change-results} (2), we have $\Ext^i_A(\frakm, N) = \Ext^i_B(\frakm/x\frakm, N)$ for all $B$-modules $N$. Hence, $\Ext^{r + 1}_B(\frakm/x\frakm, N) = 0$ for every $B$-module $N$; that is, $\projdim_B \frakm/x\frakm\le r$.

    Next, we show that the natural surjection $\frakm/x\frakm\to \frakm/xA$ splits as $A$-modules (and hence as $B$-modules). First, choose a minimal generating set $x, x_2,\dots,x_s$ of $\frakm$ and set $\frakb = (x_2,\dots,x_s)$. Note that $\frakb\cap xA\subseteq x\frakm$. Indeed, if $y = a_2x_2 + \dots + a_nx_n = ax\in \frakb\cap xA$, then looking at the equality modulo $\frakm$, we see that $a\in\frakm$, whence $x\in\frakb\cap x\frakm\subseteq x\frakm$. Now, consider the chain of natural maps 
    \begin{equation*}
        \frac{\frakm}{xA} = \frac{\frakb + xA}{xA}\xrightarrow{\sim}\frac{\frakb}{\frakb\cap xA}\to\frac{\frakm}{x\frakm}\to\frac{\frakm}{xA}.
    \end{equation*}
    Their composition is the identity, and hence, the surjection $\frakm/x\frakm\to \frakm/xA$ splits. In particularly, this means that 
    \begin{equation*}
        \projdim_B \frakm/xA\le\projdim_B \frakm/x\frakm\le r.
    \end{equation*}
    Because of the exact sequence $0\to\frakm/xA\to B\to k\to 0$, we see that $\gldim B = \projdim_B k\le r + 1$. Since $\embdim B = r - 1$, the induction hypothesis gives that $B$ is a regular local ring. Now, since $x$ is $A$-regular, $\dim B = \dim A - 1$, and therefore, 
    \begin{equation*}
        \embdim A = \embdim B + 1 = \dim B + 1 = \dim A,
    \end{equation*}
    whence $A$ is a regular local ring, as desired.
\end{proof}

\begin{theorem}[Serre]
    Let $A$ be a regular local ring and $\frakp$ a prime ideal of $A$. Then $A_\frakp$ is a regular local ring.
\end{theorem}
\begin{proof}

\end{proof}
