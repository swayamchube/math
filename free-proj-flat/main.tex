\documentclass[12pt]{article}

% \usepackage{./arxiv}

\title{Free, Projective, and Flat Modules}
\author{Swayam Chube}
\date{\today}

\usepackage[utf8]{inputenc} % allow utf-8 input
\usepackage[T1]{fontenc}    % use 8-bit T1 fonts
\usepackage{hyperref}       % hyperlinks
\usepackage{url}            % simple URL typesetting
\usepackage{booktabs}       % professional-quality tables
\usepackage{amsfonts}       % blackboard math symbols
\usepackage{nicefrac}       % compact symbols for 1/2, etc.
\usepackage{microtype}      % microtypography
\usepackage{graphicx}
\usepackage{natbib}
\usepackage{doi}
\usepackage{amssymb}
\usepackage{bbm}
\usepackage{amsthm}
\usepackage{amsmath}
\usepackage{xcolor}
\usepackage{theoremref}
\usepackage{enumitem}
\usepackage{mathpazo}
% \usepackage{euler}
\usepackage{mathrsfs}
\usepackage{todonotes}
\usepackage{stmaryrd}
\usepackage[all,cmtip]{xy} % For diagrams, praise the Freyd–Mitchell theorem 
\usepackage{marvosym}
\usepackage{geometry}
\usepackage{titlesec}

\renewcommand{\qedsymbol}{$\blacksquare$}

% Uncomment to override  the `A preprint' in the header
% \renewcommand{\headeright}{}
% \renewcommand{\undertitle}{}
% \renewcommand{\shorttitle}{}

\hypersetup{
    pdfauthor={Lots of People},
    colorlinks=true,
}

\newtheoremstyle{thmstyle}%               % Name
  {}%                                     % Space above
  {}%                                     % Space below
  {}%                             % Body font
  {}%                                     % Indent amount
  {\bfseries\scshape}%                            % Theorem head font
  {.}%                                    % Punctuation after theorem head
  { }%                                    % Space after theorem head, ' ', or \newline
  {\thmname{#1}\thmnumber{ #2}\thmnote{ (#3)}}%                                     % Theorem head spec (can be left empty, meaning `normal')

\newtheoremstyle{defstyle}%               % Name
  {}%                                     % Space above
  {}%                                     % Space below
  {}%                                     % Body font
  {}%                                     % Indent amount
  {\bfseries\scshape}%                            % Theorem head font
  {.}%                                    % Punctuation after theorem head
  { }%                                    % Space after theorem head, ' ', or \newline
  {\thmname{#1}\thmnumber{ #2}\thmnote{ (#3)}}%                                     % Theorem head spec (can be left empty, meaning `normal')

\theoremstyle{thmstyle}
\newtheorem{theorem}{Theorem}[section]
\newtheorem{lemma}[theorem]{Lemma}
\newtheorem{proposition}[theorem]{Proposition}

\theoremstyle{defstyle}
\newtheorem{definition}[theorem]{Definition}
\newtheorem*{corollary}{Corollary}
\newtheorem{remark}[theorem]{Remark}
\newtheorem{example}[theorem]{Example}
\newtheorem*{notation}{Notation}

% Common Algebraic Structures
\newcommand{\R}{\mathbb{R}}
\newcommand{\Q}{\mathbb{Q}}
\newcommand{\Z}{\mathbb{Z}}
\newcommand{\N}{\mathbb{N}}
\newcommand{\bbC}{\mathbb{C}} 
\newcommand{\K}{\mathbb{K}} % Base field which is either \R or \bbC
\newcommand{\calA}{\mathcal{A}} % Banach Algebras
\newcommand{\calB}{\mathcal{B}} % Banach Algebras
\newcommand{\calI}{\mathcal{I}} % ideal in a Banach algebra
\newcommand{\calJ}{\mathcal{J}} % ideal in a Banach algebra
\newcommand{\frakM}{\mathfrak{M}} % sigma-algebra
\newcommand{\calO}{\mathcal{O}} % Ring of integers
\newcommand{\bbA}{\mathbb{A}} % Adele (or ring thereof)
\newcommand{\bbI}{\mathbb{I}} % Idele (or group thereof)

% Categories
\newcommand{\catTopp}{\mathbf{Top}_*}
\newcommand{\catGrp}{\mathbf{Grp}}
\newcommand{\catTopGrp}{\mathbf{TopGrp}}
\newcommand{\catSet}{\mathbf{Set}}
\newcommand{\catTop}{\mathbf{Top}}
\newcommand{\catRing}{\mathbf{Ring}}
\newcommand{\catCRing}{\mathbf{CRing}} % comm. rings
\newcommand{\catMod}{\mathbf{Mod}}
\newcommand{\catMon}{\mathbf{Mon}}
\newcommand{\catMan}{\mathbf{Man}} % manifolds
\newcommand{\catDiff}{\mathbf{Diff}} % smooth manifolds
\newcommand{\catAlg}{\mathbf{Alg}}
\newcommand{\catRep}{\mathbf{Rep}} % representations 
\newcommand{\catVec}{\mathbf{Vec}}

% Group and Representation Theory
\newcommand{\chr}{\operatorname{char}}
\newcommand{\Aut}{\operatorname{Aut}}
\newcommand{\GL}{\operatorname{GL}}
\newcommand{\im}{\operatorname{im}}
\newcommand{\tr}{\operatorname{tr}}
\newcommand{\id}{\mathbf{id}}
\newcommand{\cl}{\mathbf{cl}}
\newcommand{\Gal}{\operatorname{Gal}}
\newcommand{\Tr}{\operatorname{Tr}}
\newcommand{\sgn}{\operatorname{sgn}}
\newcommand{\Sym}{\operatorname{Sym}}
\newcommand{\Alt}{\operatorname{Alt}}

% Commutative and Homological Algebra
\newcommand{\spec}{\operatorname{spec}}
\newcommand{\mspec}{\operatorname{m-spec}}
\newcommand{\Tor}{\operatorname{Tor}}
\newcommand{\tor}{\operatorname{tor}}
\newcommand{\Ann}{\operatorname{Ann}}
\newcommand{\Supp}{\operatorname{Supp}}
\newcommand{\Hom}{\operatorname{Hom}}
\newcommand{\End}{\operatorname{End}}
\newcommand{\coker}{\operatorname{coker}}
\newcommand{\limit}{\varprojlim}
\newcommand{\colimit}{%
  \mathop{\mathpalette\colimit@{\rightarrowfill@\textstyle}}\nmlimits@
}
\makeatother


\newcommand{\fraka}{\mathfrak{a}} % ideal
\newcommand{\frakb}{\mathfrak{b}} % ideal
\newcommand{\frakc}{\mathfrak{c}} % ideal
\newcommand{\frakf}{\mathfrak{f}} % face map
\newcommand{\frakg}{\mathfrak{g}}
\newcommand{\frakh}{\mathfrak{h}}
\newcommand{\frakm}{\mathfrak{m}} % maximal ideal
\newcommand{\frakn}{\mathfrak{n}} % naximal ideal
\newcommand{\frakp}{\mathfrak{p}} % prime ideal
\newcommand{\frakq}{\mathfrak{q}} % qrime ideal
\newcommand{\fraks}{\mathfrak{s}}
\newcommand{\frakt}{\mathfrak{t}}
\newcommand{\frakz}{\mathfrak{z}}
\newcommand{\frakA}{\mathfrak{A}}
\newcommand{\frakI}{\mathfrak{I}}
\newcommand{\frakJ}{\mathfrak{J}}
\newcommand{\frakK}{\mathfrak{K}}
\newcommand{\frakL}{\mathfrak{L}}
\newcommand{\frakN}{\mathfrak{N}} % nilradical 
\newcommand{\frakO}{\mathfrak{O}} % dedekind domain
\newcommand{\frakP}{\mathfrak{P}} % Prime ideal above
\newcommand{\frakQ}{\mathfrak{Q}} % Qrime ideal above 
\newcommand{\frakR}{\mathfrak{R}} % jacobson radical
\newcommand{\frakU}{\mathfrak{U}}
\newcommand{\frakX}{\mathfrak{X}}

% General/Differential/Algebraic Topology 
\newcommand{\scrA}{\mathscr A}
\newcommand{\scrB}{\mathscr B}
\newcommand{\scrF}{\mathscr F}
\newcommand{\scrN}{\mathscr N}
\newcommand{\scrP}{\mathscr P}
\newcommand{\scrR}{\mathscr R}
\newcommand{\scrS}{\mathscr S}
\newcommand{\bbH}{\mathbb H}
\newcommand{\Int}{\operatorname{Int}}
\newcommand{\psimeq}{\simeq_p}
\newcommand{\wt}[1]{\widetilde{#1}}
\newcommand{\RP}{\mathbb{R}\text{P}}
\newcommand{\CP}{\mathbb{C}\text{P}}

% Miscellaneous
\newcommand{\wh}[1]{\widehat{#1}}
\newcommand{\calM}{\mathcal{M}}
\newcommand{\calP}{\mathcal{P}}
\newcommand{\onto}{\twoheadrightarrow}
\newcommand{\into}{\hookrightarrow}
\newcommand{\Gr}{\operatorname{Gr}}
\newcommand{\Span}{\operatorname{Span}}
\newcommand{\ev}{\operatorname{ev}}
\newcommand{\weakto}{\stackrel{w}{\longrightarrow}}

\newcommand{\define}[1]{\textcolor{blue}{\textit{#1}}}
\newcommand{\caution}[1]{\textcolor{red}{\textit{#1}}}
\renewcommand{\mod}{~\mathrm{mod}~}
\renewcommand{\le}{\leqslant}
\renewcommand{\leq}{\leqslant}
\renewcommand{\ge}{\geqslant}
\renewcommand{\geq}{\geqslant}
\newcommand{\Res}{\operatorname{Res}}
\newcommand{\floor}[1]{\left\lfloor #1\right\rfloor}
\newcommand{\ceil}[1]{\left\lceil #1\right\rceil}
\newcommand{\gl}{\mathfrak{gl}}
\newcommand{\ad}{\operatorname{ad}}
\newcommand{\Stab}{\operatorname{Stab}}
\newcommand{\bfX}{\mathbf{X}}
\newcommand{\Ind}{\operatorname{Ind}}
\newcommand{\bfG}{\mathbf{G}}
\newcommand{\rank}{\operatorname{rank}}
\newcommand{\calo}{\mathcal{o}}
\newcommand{\frako}{\mathfrak{o}}
\newcommand{\Cl}{\operatorname{Cl}}

\newcommand{\idim}{\operatorname{idim}}
\newcommand{\pdim}{\operatorname{pdim}}
\newcommand{\Ext}{\operatorname{Ext}}
\newcommand{\co}{\operatorname{co}}
\newcommand{\Spec}{\operatorname{Spec}}
\newcommand{\MaxSpec}{\operatorname{MaxSpec}}

\geometry {
    margin = 1in
}

\titleformat
{\section}
[block]
{\Large\bfseries\scshape}
{\S\thesection}
{0.5em}
{\centering}
[]


\titleformat
{\subsection}
[block]
{\normalfont\bfseries\sffamily}
{\S\S}
{0.5em}
{\centering}
[]


\begin{document}
\maketitle

\subsection{Cartier}

\begin{theorem}
    Let $M$ be a finitely generated module over an integral domain $A$. If for every $\frakm\in\MaxSpec(A)$, $M_\frakm$ is free as an $A_\frakm$-module, then $M$ is a projective $A$-module.
\end{theorem}
\begin{proof}
    First show that $M$ is a torsion-free $A$-module. Suppose $am = 0$ for some $0\ne a\in A$ and $m\in M$. Let $\fraka$ be the annihilator of $m$ in $A$ and $\frakm$ a maximal ideal containing $A$. Note that $\frac{a}{1}\frac{m}{1} = 0$ in $M_\frakm$, which is free over $A_\frakm$, an integral domain, whence, is torsion free. That is, $\frac{m}{1} = 0$, whence, there is some $s\in A\setminus\frakm$ such that $sm = 0$, which is absurd, since $\fraka\subseteq\frakm$. This shows that $M$ is torsion-free.

    Now, choose a set of generators $\{m_i\colon 1\le i\le n\}$ for $M$ over $A$. Let $\scrP$ be the collection of $A$-endomorphisms of $M$ which are of the form 
    \begin{equation*}
        m\longmapsto\sum_{i = 1}^n f_i(m)m_i,
    \end{equation*}
    where $f_1,\dots, f_n: M\to A$ are $A$-module homomorphisms. Note that $\scrP$ is an $A$-submodule of $\End_A(M)$. We shall show that $\id_M\in\scrP$.

    Let $\frakm$ be a maximal ideal of $A$. We know that $M_\frakm$ is free as an $A_\frakm$-module and hence, there are $A_\frakm$-module homomorphisms $f_i: M_\frakm\to A_\frakm$ such that 
    \begin{equation*}
        m' = \sum_{i = 1}^n f_i'(m')\frac{m_i}{1}\quad\forall m'\in M_\frakm.
    \end{equation*}
    To see that this is possible, first consider an $A_\frakm$-basis $\{e_i\colon 1\le i\le N\}$ for $M_\frakm$. We can write 
    \begin{equation*}
        e_i = \sum_{j = 1}^n a_{ij}\frac{m_j}{1}\quad\forall~1\le i\le N.
    \end{equation*}
    Further, there are $A_\frakm$-linear maps $f_i: M_\frakm\to A_\frakm$ such that 
    \begin{equation*}
        m' = \sum_{j = 1}^N f_j(m')e_j.
    \end{equation*}
    Set 
    \begin{equation*}
        f_j'(m') = \sum_{i = 1}^N a_{ij}f_i(m')\quad\forall~m'\in M_\frakm.
    \end{equation*}
    Then, 
    \begin{equation*}
        \sum_{j = 1}^nf_j'(m')\frac{m_j}{1} = \sum_{i = 1}^N\sum_{j = 1}^n a_{ij}f_i(m')\frac{m_j}{1} = \sum_{i = 1}^N f_i(m')e_i = m'.
    \end{equation*}

    Coming back, since $M$ is torsion-free, the canonical map $M\to M_\frakm$ is an injective map of $A$-modules. Further, we can find an $s\in A\setminus\frakm$ such that $s f_i'\left(\frac{m_j}{1}\right)\in A$ for $1\le i,j\le n$. 

    Note that $m'\mapsto sf_i'(m')$ is $A_\frakm$-linear as a map $M_\frakm\to A_\frakm$, and hence, is $A$-linear. The restriction of this map to $M\subseteq M_\frakm$ takes values in $A$. Thus, we can identify $sf_i'$ with an $A$-linear map $M\to A$. Further, for every $m\in M$, we have 
    \begin{equation*}
        sm = \sum_{i = 1}^n sf_i'(m)m_i.
    \end{equation*}
    That is, $s\cdot\id_M\in\scrP$. Now, let $\fraka$ be the collection of all $a\in A$ such that $a\cdot\id_M\in\scrP$. Then $\fraka$ is an ideal of $A$. If $\fraka$ were a proper ideal, it would be contained in a maximal ideal $\frakm$. But from our preceding conclusion, there is some $s\in A\setminus\frakm$ such that $s\cdot\id_M\in\scrP$, a contradiction. Thus, $\fraka = A$, in particular, $\id_M\in\scrP$.

    Finally, we show that $M$ is projective. We have shown that there are $A$-linear maps $f_i: M\to A$ such that 
    \begin{equation*}
        m = \sum_{i = 1}^n f_i(m)m_i\quad\forall~m\in M.
    \end{equation*}
    Let $F$ be the free module $\bigoplus_{i = 1}^n Ae_i$ and let $g: F\to M$ be given by $e_i\mapsto m_i$ and $f: M\to F$ given by 
    \begin{equation*}
        f(m) = \sum_{i = 1}^n f_i(m)e_i.
    \end{equation*}
    By our construction, $g\circ f = \id_M$, and hence $M$ is a direct summand of $F$, i.e. $M$ is projective.
\end{proof}
\end{document}