\begin{definition}
    An $A$-module $M$ is said to be \define{injective} if the (contravariant) functor $\Hom_A(-,M):\mathfrak{Mod}_A^{op}\to\mathfrak{Mod}_A$ is exact.
\end{definition}

\subsection{Injective Hulls}

\begin{definition}
    Let $M\le E$ be $A$-modules. Then $E$ is said to be an \define{essential extension} of $M$ if every non-zero submodule of $E$ intersects $M$ non-trivially. We denote this by $M\le_e E$.
\end{definition}
\begin{remark}
    The above is equivalent to requiring that for every $x\in E\setminus\{0\}$, there is an $a\in A\setminus\{0\}$ such that $ax\in M\setminus\{0\}$.
\end{remark}

We note some trivial properties of essential extensions before proceeding.

\begin{proposition}
    Let $L\le M\le N$ be $A$-modules. Then 
    \begin{equation*}
        L\le_e M\text{ and } M\le_e N\iff L\le_e N.
    \end{equation*}
\end{proposition}
\begin{proof}
    Straightforward.
\end{proof}

\begin{proposition}
    Let $M\le E$ be $A$-modules. Consider the set 
    \begin{equation*}
        \mathcal E = \{N\le E\colon M\le_e N\}.
    \end{equation*}
    Then $\mathcal E$ has a maximal element.
\end{proposition}
\begin{proof}
    Standard application of Zorn's lemma.
\end{proof}

\begin{proposition}
    If $N_1\le_e M_1$ and $N_2\le_e M_2$, then $N_1\oplus N_2\le_e M_1\oplus M_2$.
\end{proposition}
\begin{proof}
    Trivial.
\end{proof}