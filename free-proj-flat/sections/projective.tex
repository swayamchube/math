\begin{definition}
    An $A$-module $M$ is said to be \define{projective} if the functor $\Hom_A(M,-):\mathfrak{Mod}_A\to\mathfrak{Mod}_A$ is exact.
\end{definition}

\subsection{Kaplansky's Theorem}

\begin{theorem}\thlabel{kaplansky-projective}
    Let $(A,\frakm,k)$ be a local ring. If $M$ is a projective $A$-module, then $M$ is free.
\end{theorem}

We begin by proving two lemmas. 

\begin{lemma}\thlabel{countably-generated-direct-summand}
    Let $R$ be any (commutative) ring, and $F$ an $A$-module which is a direct sum of countably generated submodules. If $M$ is a direct summand of $F$, then $M$ is also a direct sum of countably generated submodules.
\end{lemma}
\begin{proof}
    Let $F = M\oplus N$ and $\displaystyle F = \bigoplus_{\lambda\in\Lambda} E_\lambda$ where each $E_\lambda$ is a countably generated $R$-submodule of $F$. Our first order of business will be to construct, using transfinite induction, a sequence of submodules $(F_\alpha)_{\alpha\in\mathbf{Ord}}$ of $F$ such that 
    \begin{enumerate}[label=(\roman*)]
        \item if $\alpha < \beta$, then $F_\alpha\subseteq F_\beta$. 
        \item $\displaystyle F = \bigcup_{\alpha} F_\alpha$. 
        \item if $\alpha$ is a limit ordinal, then $\displaystyle F_\alpha = \bigcup_{\beta < \alpha} F_\beta$. 
        \item $F_{\alpha + 1}/F_\alpha$ is countably generated. 
        \item $F_\alpha = M_\alpha\oplus N_\alpha$, where $M_\alpha = F_\alpha\cap M$ and $N_\alpha = F_\alpha\cap N$.
        \item each $F_\alpha$ is a direct sum of a suitable subset of $\{E_\lambda\colon\lambda\in\Lambda\}$.
    \end{enumerate}
    Begin by setting $F_0 = 0$. Suppose for an ordinal $\alpha > 0$, $F_\beta$ has been defined for all ordinals $\beta < \alpha$. If $\alpha$ is a limit ordinal then set 
    \begin{equation*}
        F_\alpha = \bigcup_{\beta < \alpha} F_\beta.
    \end{equation*}
    We must show that $F_\alpha$ satisfies the aforementioned conditions. Clearly (i) and (iii) are satisfied; and further since each $F_\beta$ is a direct sum of a subset of $\{E_\lambda\colon \lambda\in\Lambda\}$, it would follow that so is $F_\alpha$, thereby verifying (vi). To verify (v), it suffices to show that $F_\alpha = M_\alpha + N_\alpha$, but this is clear since any element of $F_\alpha$ is also an element of $F_\beta$ for some $\beta < \alpha$.

    Next, suppose $\alpha$ is not a limit ordinal so that $\alpha = \beta + 1$ for some ordinal $\beta$. This construction is a bit involved. First, if $F_\beta = F$, then the construction stops at $\beta$. Suppose now that $F_\beta\subsetneq F$. Let $Q_1$ be any one of the $E_\lambda$ \emph{not} contained in $F_\beta$. Take a countable set of generators $x_{11}, x_{12},\dots$ of $Q_1$. Since $F = M\oplus N$, we can write 
    \begin{equation*}
        x_{11} = m_{11} + n_{11} \quad\text{ for }m_{11}\in M\text{ and } n_{11}\in N.
    \end{equation*}
    Further, using the decomposition $\displaystyle F = \bigoplus_{\lambda\in\Lambda} E_\lambda$, we can write 
    \begin{equation*}
        m_{11} = \sum_{\substack{\lambda\in\Lambda\\\text{finite}}} m_{11}^\lambda\quad\text{ and }\quad n_{11} = \sum_{\substack{\lambda\in\Lambda\\\text{finite}}} n_{11}^\lambda.
    \end{equation*}
    Now let $Q_2$ be the sum of those $E_\lambda$'s for which $\lambda$ occurs in the two expressions above. Since $Q_2$ is a finite direct sum of some $E_\lambda$'s, it is countably generated. Let $x_{21},x_{22},\dots$ be a countable generating set of $Q_2$. Just as before, we can (uniquely) decompose $x_{12} = m_{12} + n_{12}$ with $m_{12}\in M$ and $n_{12}\in N$; and further decompose 
    \begin{equation*}
        m_{12} = \sum_{\substack{\lambda\in\Lambda\\\text{finite}}} m_{12}^\lambda\quad\text{ and }\quad n_{12} = \sum_{\substack{\lambda\in\Lambda\\\text{finite}}} n_{12}^\lambda.
    \end{equation*}
    Again, set $Q_3$ to be the direct sum of those $E_\lambda$'s for which $\lambda$ occurs in the two expressions above, so that $Q_3$ is countably generated too. Pick a countable generating set $x_{31},x_{32},\dots$ of $Q_3$. Next decompose $x_{21}$ and repeat the procedure above to obtain $Q_4$ and its countable generating set $x_{41}, x_{42},\dots$. Decompose $x_{13}$ next and repeat ad infinitum.
    \begin{center}
        \begin{tabular}{ccccc}
            $x_{11}$ & $x_{12}$ & $x_{13}$ & $x_{14}$ & $\dots$\\
            $x_{21}$ & $x_{22}$ & $x_{23}$ & $x_{24}$ & $\dots$\\
            $x_{31}$ & $x_{32}$ & $x_{33}$ & $x_{34}$ & $\dots$\\
            $x_{41}$ & $x_{42}$ & $x_{43}$ & $x_{44}$ & $\dots$\\
            $\vdots$ & $\vdots$ & $\vdots$ & $\vdots$ & $\ddots$
        \end{tabular}
    \end{center}
    To be explicit, the order in which we decompose the $x_{ij}$'s is 
    \begin{equation*}
        x_{11},~ x_{12},~ x_{21},~ x_{13},~ x_{22},~ x_{31},~ x_{14},~\dots.
    \end{equation*}
    Finally, set $F_{\alpha}$ to be the submodule of $F$ generated by $F_\beta$ and $\{x_{ij}\colon i,j\ge 1\}$. Clearly $F_\alpha/F_\beta$ is countably generated and $F_\beta\subseteq F_\alpha$, which verifies (i) and (iv). Since $\{x_{ni}\colon i\ge 1\}$ generates $Q_n$, we in fact have
    \begin{equation*}
        F_\alpha = F_\beta + \sum_{n\ge 1} Q_n,
    \end{equation*}
    whence $F_\alpha$ is a direct sum of a subset of $\{E_\lambda\colon\lambda\in\Lambda\}$. It remains to verify (v), and to this end, it suffices to show that $F_\alpha = M_\alpha + N_\alpha$. An element of $F_\alpha$ can be written as 
    \begin{equation*}
        f_\beta + \sum_{\substack{i, j\\\text{finite}}} a_{ij}x_{ij},
    \end{equation*}
    for some $f_\beta\in F_\beta$ and $a_{ij}\in R$. Recall that we can write 
    \begin{equation*}
        x_{ij} = m_{ij} + n_{ij},\quad m_{ij} = \sum_{\substack{\lambda\in\Lambda\\\text{finite}}}m_{ij}^\lambda,\quad\text{ and }\quad n_{ij} = \sum_{\substack{\lambda\in\Lambda\\\text{finite}}} n_{ij}^\lambda.
    \end{equation*}
    Note that each $m_{ij}^\lambda$ is contained in one of the $Q_n$'s, and hence, in $F_\alpha$. Therefore $m_{ij}$ and $n_{ij}$ are elements of $F_{\alpha}$, and hence, are elements of $M_\alpha$ and $N_\alpha$ respectively. Further, by the inductive hypothesis, $f_\beta = m_\beta + n_\beta$ for some $m_\beta\in M_\beta\subseteq M_\alpha$ and $n_\beta\in N_\beta\subseteq N_\alpha$, whence it follows that $F_\alpha = M_\alpha + N_\alpha$, thereby verifying (v).

    Next, note that the composition
    \begin{equation*}
        F_{\alpha + 1}\onto M_{\alpha + 1}\onto M_{\alpha + 1}/M_\alpha
    \end{equation*}
    has kernel containing $F_\alpha$ and therefore, $M_{\alpha + 1}/M_\alpha$ is a quotient of $F_{\alpha + 1}/F_\alpha$, which is countably generated, and hence so is $M_{\alpha + 1}/M_\alpha$. Next, since $M_\alpha$ is a direct summand of $F_\alpha$, it is also a direct summand of $F$. Hence, $M_\alpha$ is a direct summand of $M_{\alpha + 1}$. Thus, we can write 
    \begin{equation*}
        M_{\alpha + 1} = M_\alpha\oplus M_{\alpha + 1}',
    \end{equation*}
    where $M_{\alpha + 1}'$ is countably generated. When $\alpha$ is a limit ordinal, set $M_{\alpha}' = 0$. It is now easy to see that 
    \begin{equation*}
        M_\alpha = \bigoplus_{\beta\le\alpha} M_\beta'.
    \end{equation*}
    And since $\displaystyle M = \bigcup_{\alpha} M_\alpha$, it follow that 
    \begin{equation*}
        M = \bigoplus_{\alpha} M_\alpha',
    \end{equation*}
    thereby completing the proof.
\end{proof}

\begin{lemma}\thlabel{free-direct-summand-exists}
    Let $M$ be a projective module over a local ring $(A,\frakm)$ and $x\in M$. Then there exists a direct summand of $M$ containing $x$ which is a free module.
\end{lemma}
\begin{proof}
    We can write $F$ as a direct summand of a free $A$-module $F = M\oplus N$. Choose a basis $B = \{u_i\}_{i\in I}$ such that $x$ has the minimum possible non-zero coefficients when expressed as an $A$-linear combination of the $u_i$'s. Write 
    \begin{equation*}
        x = a_1 u_1 + \dots + a_nu_n
    \end{equation*}
    for some $0\ne a_i\in A$. Note that we must have $a_i\notin\sum\limits_{j\ne i} Aa_j$ for $1\le i\le n$. Indeed, if we could write 
    \begin{equation*}
        a_n = b_1a_1 + \dots + b_{n - 1}a_n, 
    \end{equation*}
    then 
    \begin{equation*}
        x = \sum_{i = 1}^{n - 1}a_i(u_i + b_iu_n),
    \end{equation*}
    and $\{u_1 + b_1u_n,\dots,u_{n - 1} + b_{n - 1}u_n, u_n\}\cup\{u_j\colon j\ne 1,\dots,n\}$ is also a basis of $F$, which would contradict the minimality in the choice of $B$.

    Set $u_i = y_i + z_i$ where $y_i\in M$ and $z_i\in N$. Since $x\in M$, we must have 
    \begin{equation*}
        x = a_1y_1 + \dots + a_ny_n.
    \end{equation*}
    We can write each $y_i$ in coordinates as 
    \begin{equation*}
        y_i = \sum_{j = 1}^n c_{ij}u_j + t_i,
    \end{equation*}
    for some $c_{ij}\in A$ and $t_i\in F$ which is a linear combination of $u_k$'s for $k\ne 1,\dots, n$. Thus 
    \begin{equation*}
        x = \sum_{i = 1}^n a_iy_i = \sum_{i = 1}^n\sum_{j = 1}^n a_i c_{ij}u_j + \sum_{i = 1}^n a_it_i.
    \end{equation*}
    By the uniqueness of coordinate representation with respect to a basis, we get 
    \begin{equation*}
        a_i = \sum_{j = 1}^n a_j c_{ji}\implies\sum_{j = 1}^n a_j\left(c_{ji} - \delta_{ji}\right) = 0
    \end{equation*}
    for $1\le i\le n$. Since elements in $A\setminus\frakm$ are invertible, we must have that $c_{ii}\in 1 + \frakm$ for all $1\le i\le n$ and $c_{ij}\in\frakm$ for $1\le i\ne j\le n$. In particular, this means the matrix $\mathbf C = \left(c_{ij}\right)$ is invertible since its determinant is in $1 + \frakm$.

    We claim that $\wt B = \{y_1,\dots,y_n\}\cup\{u_i\colon i\ne 1,\dots,n\}$ is a basis for $F$. The invertibility of $\mathbf C$ shows that each $u_i$ can be written as an $A$-linear combination of elements in $\wt B$, and hence, the $A$-linear span of $\wt B$ is all of $F$. To see that $\wt B$ is $A$-linearly independent, suppose 
    \begin{equation*}
        0 = \sum_{i = 1}^n f_iy_i + \sum_{\lambda\ne 1,\dots,n} f_\lambda u_\lambda.
    \end{equation*}
    Substituting the representation of $y_i$ in the basis $B$, we have 
    \begin{equation*}
        0 = \sum_{i = 1}^n f_i\left(\sum_{j = 1}^n c_{ij}u_j + t_i\right) + \sum_{\lambda\ne 1,\dots, n}f_\lambda u_\lambda.
    \end{equation*}
    Therefore, in particular, 
    \begin{equation*}
        \begin{pmatrix}
            f_1 & \cdots & f_n
        \end{pmatrix}\mathbf C = 0,
    \end{equation*}
    and the invertibility of $\mathbf C$ would mean $f_i = 0$ for $1\le i\le n$; consequently, 
    \begin{equation*}
        \sum_{\lambda\ne 1,\dots, n} f_\lambda u_\lambda = 0,
    \end{equation*}
    so that $f_\lambda = 0$ for all $\lambda$. Hence $\wt B$ is a basis of $F$. Let $F_1$ denote the $A$-submodule generated by $\{y_1,\dots,y_n\}$. This is a free direct summand of $F$ contained in $M$, and hence, is a free direct summand of $M$ containing $x$.
\end{proof}

\begin{proof}[Proof of \thref{kaplansky-projective}]
    $M$ is a direct summand of a free module, and every free module is a direct sum of countably generated submodules. Hence $M$ itself is a direct sum of countably generated projective modules. Therfore, it is sufficient to prove the theorem assuming $M$ is countably generated. 

    Let $\{\omega_1,\omega_2,\dots\}$ be a countable generating set for $M$. By \thref{free-direct-summand-exists}, there exists a free direct summand $F_1$ of $M$ containing $\omega_1$. Write $M = F_1\oplus M_1$ and let $\omega_2'$ denote the $M_1$ component of $\omega_2$. Since $M_1$ is projective, using \thref{free-direct-summand-exists}, there exists a free direct summand $F_2$ of $M_1$ containing $\omega_2$. Then $M_1 = F_2\oplus M_2$ so that $M = F_1\oplus F_2\oplus M_2$. Let $\omega_3'$ denote the $M_2$-component of $\omega_3$ and repeat the above process ad infinitum. That would yield $M = F_1\oplus F_2\oplus\cdots$, whence $M$ is free.
\end{proof}