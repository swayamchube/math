\begin{definition}
    An $A$-module $M$ is said to be \define{projective} if the functor $\Hom_A(M,-):\mathfrak{Mod}_A\to\mathfrak{Mod}_A$ is exact.
\end{definition}

\subsection{Kaplansky's Theorem}

\begin{theorem}\thlabel{kaplansky-projective}
    Let $(A,\frakm,k)$ be a local ring. If $M$ is a projective $A$-module, then $M$ is free.
\end{theorem}

We begin by proving two lemmas. 

\begin{lemma}\thlabel{countably-generated-direct-summand}
    Let $R$ be any (commutative) ring, and $F$ an $A$-module which is a direct sum of countably generated submodules. If $M$ is a direct summand of $F$, then $M$ is also a direct sum of countably generated submodules.
\end{lemma}
\begin{proof}
    Let $F = M\oplus N$ and $\displaystyle F = \bigoplus_{\lambda\in\Lambda} E_\lambda$ where each $E_\lambda$ is a countably generated $R$-submodule of $F$. Our first order of business will be to construct, using transfinite induction, a sequence of submodules $(F_\alpha)_{\alpha\in\mathbf{Ord}}$ of $F$ such that 
    \begin{enumerate}[label=(\roman*)]
        \item if $\alpha < \beta$, then $F_\alpha\subseteq F_\beta$. 
        \item $\displaystyle F = \bigcup_{\alpha} F_\alpha$. 
        \item if $\alpha$ is a limit ordinal, then $\displaystyle F_\alpha = \bigcup_{\beta < \alpha} F_\beta$. 
        \item $F_{\alpha + 1}/F_\alpha$ is countably generated. 
        \item $F_\alpha = M_\alpha\oplus N_\alpha$, where $M_\alpha = F_\alpha\cap M$ and $N_\alpha = F_\alpha\cap N$.
        \item each $F_\alpha$ is a direct sum of a suitable subset of $\{E_\lambda\colon\lambda\in\Lambda\}$.
    \end{enumerate}
    Begin by setting $F_0 = 0$. Suppose for an ordinal $\alpha > 0$, $F_\beta$ has been defined for all ordinals $\beta < \alpha$. If $\alpha$ is a limit ordinal then set 
    \begin{equation*}
        F_\alpha = \bigcup_{\beta < \alpha} F_\beta.
    \end{equation*}
    We must show that $F_\alpha$ satisfies the aforementioned conditions. Clearly (i) and (iii) are satisfied; and further since each $F_\beta$ is a direct sum of a subset of $\{E_\lambda\colon \lambda\in\Lambda\}$, it would follow that so is $F_\alpha$, thereby verifying (vi). To verify (v), it suffices to show that $F_\alpha = M_\alpha + N_\alpha$, but this is clear since any element of $F_\alpha$ is also an element of $F_\beta$ for some $\beta < \alpha$.

    Next, suppose $\alpha$ is not a limit ordinal so that $\alpha = \beta + 1$ for some ordinal $\beta$. This construction is a bit involved. First, if $F_\beta = F$, then the construction stops at $\beta$. Suppose now that $F_\beta\subsetneq F$. Let $Q_1$ be any one of the $E_\lambda$ \emph{not} contained in $F_\beta$. Take a countable set of generators $x_{11}, x_{12},\dots$ of $Q_1$. Since $F = M\oplus N$, we can write 
    \begin{equation*}
        x_{11} = m_{11} + n_{11} \quad\text{ for }m_{11}\in M\text{ and } n_{11}\in N.
    \end{equation*}
    Further, using the decomposition $\displaystyle F = \bigoplus_{\lambda\in\Lambda} E_\lambda$, we can write 
    \begin{equation*}
        m_{11} = \sum_{\substack{\lambda\in\Lambda\\\text{finite}}} m_{11}^\lambda\quad\text{ and }\quad n_{11} = \sum_{\substack{\lambda\in\Lambda\\\text{finite}}} n_{11}^\lambda.
    \end{equation*}
    Now let $Q_2$ be the sum of those $E_\lambda$'s for which $\lambda$ occurs in the two expressions above. Since $Q_2$ is a finite direct sum of some $E_\lambda$'s, it is countably generated. Let $x_{21},x_{22},\dots$ be a countable generating set of $Q_2$. Just as before, we can (uniquely) decompose $x_{12} = m_{12} + n_{12}$ with $m_{12}\in M$ and $n_{12}\in N$; and further decompose 
    \begin{equation*}
        m_{12} = \sum_{\substack{\lambda\in\Lambda\\\text{finite}}} m_{12}^\lambda\quad\text{ and }\quad n_{12} = \sum_{\substack{\lambda\in\Lambda\\\text{finite}}} n_{12}^\lambda.
    \end{equation*}
    Again, set $Q_3$ to be the direct sum of those $E_\lambda$'s for which $\lambda$ occurs in the two expressions above, so that $Q_3$ is countably generated too. Pick a countable generating set $x_{31},x_{32},\dots$ of $Q_3$. Next decompose $x_{21}$ and repeat the procedure above to obtain $Q_4$ and its countable generating set $x_{41}, x_{42},\dots$. Decompose $x_{13}$ next and repeat ad infinitum.
    \begin{center}
        \begin{tabular}{ccccc}
            $x_{11}$ & $x_{12}$ & $x_{13}$ & $x_{14}$ & $\dots$\\
            $x_{21}$ & $x_{22}$ & $x_{23}$ & $x_{24}$ & $\dots$\\
            $x_{31}$ & $x_{32}$ & $x_{33}$ & $x_{34}$ & $\dots$\\
            $x_{41}$ & $x_{42}$ & $x_{43}$ & $x_{44}$ & $\dots$\\
            $\vdots$ & $\vdots$ & $\vdots$ & $\vdots$ & $\ddots$
        \end{tabular}
    \end{center}
    To be explicit, the order in which we decompose the $x_{ij}$'s is 
    \begin{equation*}
        x_{11},~ x_{12},~ x_{21},~ x_{13},~ x_{22},~ x_{31},~ x_{14},~\dots.
    \end{equation*}
    Finally, set $F_{\alpha}$ to be the submodule of $F$ generated by $F_\beta$ and $\{x_{ij}\colon i,j\ge 1\}$. Clearly $F_\alpha/F_\beta$ is countably generated and $F_\beta\subseteq F_\alpha$, which verifies (i) and (iv). Since $\{x_{ni}\colon i\ge 1\}$ generates $Q_n$, we in fact have
    \begin{equation*}
        F_\alpha = F_\beta + \sum_{n\ge 1} Q_n,
    \end{equation*}
    whence $F_\alpha$ is a direct sum of a subset of $\{E_\lambda\colon\lambda\in\Lambda\}$. It remains to verify (v), and to this end, it suffices to show that $F_\alpha = M_\alpha + N_\alpha$. An element of $F_\alpha$ can be written as 
    \begin{equation*}
        f_\beta + \sum_{\substack{i, j\\\text{finite}}} a_{ij}x_{ij},
    \end{equation*}
    for some $f_\beta\in F_\beta$ and $a_{ij}\in R$. Recall that we can write 
    \begin{equation*}
        x_{ij} = m_{ij} + n_{ij},\quad m_{ij} = \sum_{\substack{\lambda\in\Lambda\\\text{finite}}}m_{ij}^\lambda,\quad\text{ and }\quad n_{ij} = \sum_{\substack{\lambda\in\Lambda\\\text{finite}}} n_{ij}^\lambda.
    \end{equation*}
    Note that each $m_{ij}^\lambda$ is contained in one of the $Q_n$'s, and hence, in $F_\alpha$. Therefore $m_{ij}$ and $n_{ij}$ are elements of $F_{\alpha}$, and hence, are elements of $M_\alpha$ and $N_\alpha$ respectively. Further, by the inductive hypothesis, $f_\beta = m_\beta + n_\beta$ for some $m_\beta\in M_\beta\subseteq M_\alpha$ and $n_\beta\in N_\beta\subseteq N_\alpha$, whence it follows that $F_\alpha = M_\alpha + N_\alpha$, thereby verifying (v).

    Next, note that the composition
    \begin{equation*}
        F_{\alpha + 1}\onto M_{\alpha + 1}\onto M_{\alpha + 1}/M_\alpha
    \end{equation*}
    has kernel containing $F_\alpha$ and therefore, $M_{\alpha + 1}/M_\alpha$ is a quotient of $F_{\alpha + 1}/F_\alpha$, which is countably generated, and hence so is $M_{\alpha + 1}/M_\alpha$. Next, since $M_\alpha$ is a direct summand of $F_\alpha$, it is also a direct summand of $F$. Hence, $M_\alpha$ is a direct summand of $M_{\alpha + 1}$. Thus, we can write 
    \begin{equation*}
        M_{\alpha + 1} = M_\alpha\oplus M_{\alpha + 1}',
    \end{equation*}
    where $M_{\alpha + 1}'$ is countably generated. When $\alpha$ is a limit ordinal, set $M_{\alpha}' = 0$. It is now easy to see that 
    \begin{equation*}
        M_\alpha = \bigoplus_{\beta\le\alpha} M_\beta'.
    \end{equation*}
    And since $\displaystyle M = \bigcup_{\alpha} M_\alpha$, it follow that 
    \begin{equation*}
        M = \bigoplus_{\alpha} M_\alpha',
    \end{equation*}
    thereby completing the proof.
\end{proof}

\begin{lemma}\thlabel{free-direct-summand-exists}
    Let $M$ be a projective module over a local ring $(A,\frakm)$ and $x\in M$. Then there exists a direct summand of $M$ containing $x$ which is a free module.
\end{lemma}
\begin{proof}
    We can write $F$ as a direct summand of a free $A$-module $F = M\oplus N$. Choose a basis $B = \{u_i\}_{i\in I}$ such that $x$ has the minimum possible non-zero coefficients when expressed as an $A$-linear combination of the $u_i$'s. Write 
    \begin{equation*}
        x = a_1 u_1 + \dots + a_nu_n
    \end{equation*}
    for some $0\ne a_i\in A$. Note that we must have $a_i\notin\sum\limits_{j\ne i} Aa_j$ for $1\le i\le n$. Indeed, if we could write 
    \begin{equation*}
        a_n = b_1a_1 + \dots + b_{n - 1}a_n, 
    \end{equation*}
    then 
    \begin{equation*}
        x = \sum_{i = 1}^{n - 1}a_i(u_i + b_iu_n),
    \end{equation*}
    and $\{u_1 + b_1u_n,\dots,u_{n - 1} + b_{n - 1}u_n, u_n\}\cup\{u_j\colon j\ne 1,\dots,n\}$ is also a basis of $F$, which would contradict the minimality in the choice of $B$.

    Set $u_i = y_i + z_i$ where $y_i\in M$ and $z_i\in N$. Since $x\in M$, we must have 
    \begin{equation*}
        x = a_1y_1 + \dots + a_ny_n.
    \end{equation*}
    We can write each $y_i$ in coordinates as 
    \begin{equation*}
        y_i = \sum_{j = 1}^n c_{ij}u_j + t_i,
    \end{equation*}
    for some $c_{ij}\in A$ and $t_i\in F$ which is a linear combination of $u_k$'s for $k\ne 1,\dots, n$. Thus 
    \begin{equation*}
        x = \sum_{i = 1}^n a_iy_i = \sum_{i = 1}^n\sum_{j = 1}^n a_i c_{ij}u_j + \sum_{i = 1}^n a_it_i.
    \end{equation*}
    By the uniqueness of coordinate representation with respect to a basis, we get 
    \begin{equation*}
        a_i = \sum_{j = 1}^n a_j c_{ji}\implies\sum_{j = 1}^n a_j\left(c_{ji} - \delta_{ji}\right) = 0
    \end{equation*}
    for $1\le i\le n$. Since elements in $A\setminus\frakm$ are invertible, we must have that $c_{ii}\in 1 + \frakm$ for all $1\le i\le n$ and $c_{ij}\in\frakm$ for $1\le i\ne j\le n$. In particular, this means the matrix $\mathbf C = \left(c_{ij}\right)$ is invertible since its determinant is in $1 + \frakm$.

    We claim that $\wt B = \{y_1,\dots,y_n\}\cup\{u_i\colon i\ne 1,\dots,n\}$ is a basis for $F$. The invertibility of $\mathbf C$ shows that each $u_i$ can be written as an $A$-linear combination of elements in $\wt B$, and hence, the $A$-linear span of $\wt B$ is all of $F$. To see that $\wt B$ is $A$-linearly independent, suppose 
    \begin{equation*}
        0 = \sum_{i = 1}^n f_iy_i + \sum_{\lambda\ne 1,\dots,n} f_\lambda u_\lambda.
    \end{equation*}
    Substituting the representation of $y_i$ in the basis $B$, we have 
    \begin{equation*}
        0 = \sum_{i = 1}^n f_i\left(\sum_{j = 1}^n c_{ij}u_j + t_i\right) + \sum_{\lambda\ne 1,\dots, n}f_\lambda u_\lambda.
    \end{equation*}
    Therefore, in particular, 
    \begin{equation*}
        \begin{pmatrix}
            f_1 & \cdots & f_n
        \end{pmatrix}\mathbf C = 0,
    \end{equation*}
    and the invertibility of $\mathbf C$ would mean $f_i = 0$ for $1\le i\le n$; consequently, 
    \begin{equation*}
        \sum_{\lambda\ne 1,\dots, n} f_\lambda u_\lambda = 0,
    \end{equation*}
    so that $f_\lambda = 0$ for all $\lambda$. Hence $\wt B$ is a basis of $F$. Let $F_1$ denote the $A$-submodule generated by $\{y_1,\dots,y_n\}$. This is a free direct summand of $F$ contained in $M$, and hence, is a free direct summand of $M$ containing $x$.
\end{proof}

\begin{proof}[Proof of \thref{kaplansky-projective}]
    $M$ is a direct summand of a free module, and every free module is a direct sum of countably generated submodules. Hence $M$ itself is a direct sum of countably generated projective modules. Therfore, it is sufficient to prove the theorem assuming $M$ is countably generated. 

    Let $\{\omega_1,\omega_2,\dots\}$ be a countable generating set for $M$. By \thref{free-direct-summand-exists}, there exists a free direct summand $F_1$ of $M$ containing $\omega_1$. Write $M = F_1\oplus M_1$ and let $\omega_2'$ denote the $M_1$ component of $\omega_2$. Since $M_1$ is projective, using \thref{free-direct-summand-exists}, there exists a free direct summand $F_2$ of $M_1$ containing $\omega_2$. Then $M_1 = F_2\oplus M_2$ so that $M = F_1\oplus F_2\oplus M_2$. Let $\omega_3'$ denote the $M_2$-component of $\omega_3$ and repeat the above process ad infinitum. That would yield $M = F_1\oplus F_2\oplus\cdots$, whence $M$ is free.
\end{proof}

\subsection{Projective Covers}

\begin{definition}
    Let $R$ be a ring and $M$ an $R$-module. A submodule $K$ of $M$ is said to be \define{small} if for any $R$-submodule $N$ of $M$ 
    \begin{equation*}
        K + N = M \implies N = M.
    \end{equation*}
    We denote this by $K\ll M$.
\end{definition}

We give two standard examples of small submodules.

\begin{enumerate}[label=(\roman*)]
    \item Let $(R,\frakm, k)$ be a local ring, $M$ a finite $R$-module, and $K = \frakm M$. Due to Nakayama's lemma, for any $R$-submodule $N$ of $M$, if $N + \frakm M = M$, then $N = M$. Thus $K\ll M$.
    \item Similarly, let $(R,\frakm, k)$ be an Artinian local ring, $M$ any $R$-module, and $K = \frakm M$. If $N$ is an $R$-submodule of $M$ such that $N + \frakm M = M$, then $\frakm\left(M/N\right) = M/N$. But since $\frakm$ is nilpotent, we must have that $M/N = 0$, so that $M = N$. Thus $K\ll M$.\label{artinian-example}
\end{enumerate}

\begin{definition}
    Let $R$ be a ring and $M$ an $R$-module. A \define{projective cover} of $M$ is a pair $(P, f)$, where $P$ is a projective $R$-module and $f\colon P\to M$ a surjective $R$-linear map such that $\ker f\ll M$.
\end{definition}

\begin{remark}
    Unlike the situation for injective hulls, projective covers need not always exist. For example, consider the $\Z$-module $\Z/p\Z$, where $p > 0$ is a rational prime. Suppose $f\colon P\to\Z/p\Z$ is a projective cover. Since $P$ is a projective $\Z$-module, it must be free. Set $K = \ker f\ll P$. Since $P/K$ is a simple $\Z$-module, $K$ is a maximal submodule of $P$. On the other hand, since $K\ll P$, for any proper submodule $N$ of $K$, if $N$ were not contained in $K$, then $K + N = P$, whence $N = P$, a contradiction. Thus, $K$ must contain every proper submodule of $P$. This is absurd, since $P$ admits quotients of the form $\Z/q\Z$ for primes $q\ne p$.
\end{remark}

\begin{theorem}\thlabel{projective-cover-exists-over-artinian}
    Every module over an Artinian ring admits a projective cover.
\end{theorem}
\begin{proof}
    Let $A$ be an Artinian ring, so that we can write $A = A_1\times\dots\times A_n$ as a product of Artinian local rings $(A_i,\frakm_i, k_i)$ for $1\le i\le n$. Any $A$-module $M$ is of the form $M_1\times\dots\times M_n$ where each $M_i$ is an $A_i$-module. From this reduction, it is easy to see that it suffices to prove the theorem in the Artinian local case.

    Therefore, let $(A,\frakm, k)$ be an Artinian local ring and $M$ an $A$-module. Choose a $k$-basis $\{\overline x_\lambda\colon\lambda\in\Lambda\}$ of $M/\frakm M$, where $x_\lambda\in M$ for all $\lambda\in\Lambda$. Let $F$ denote the free $A$-module on $\Lambda$, i.e., 
    \begin{equation*}
        F = \bigoplus_{\lambda\in\Lambda} Ae_\lambda,
    \end{equation*}
    and let $f\colon F\to M$ be the unique $A$-linear map sending $e_\lambda\mapsto\overline x_\lambda$ for all $\lambda\in\Lambda$. The $k$-linear independence of $\Lambda$ forces $\ker f = \frakm F\ll F$ due to \ref{artinian-example}. Using the projectivity of $F$ as an $R$-module, we can lift $f$ to a map $\wt f\colon F\to M$ making 
    \begin{equation*}
        \xymatrix {
            & M\ar[d] \\
            F\ar[ru]^{\wt f}\ar[r]_-f & M/\frakm M
        }
    \end{equation*}
    commute. Since $f$ is surjective, we have $\frakm M + \im\wt f = M$, that is, $\frakm\left(M/\im\wt f\right) = M/\im\wt f$. Again, since $\frakm$ is nilpotent, we have $M/\im\wt f = 0$, that is, $\wt f$ is surjective. Finally, since $\ker \wt f\subseteq\ker f$, it is also a small submodule of $F$, and hence $(F, \wt f)$ is a projective cover of $M$.
\end{proof}

\begin{proposition}
    Let $M$ be an $R$-module. Then 
    \begin{equation*}
        \bigcap\bigg\{\text{maximal proper submodules of }M\bigg\} = \sum\bigg\{\text{small submodules of }M\bigg\}.
    \end{equation*}
    This submodule is called the \define{radical} of $M$ and is denoted by $\rad(M)$.
\end{proposition}
\begin{proof}
    If $K\ll M$ and $L$ is any maximal proper submodule of $M$, then $K$ must be contained in $L$, else $K + L = M$ which would impoly $L = M$, a contradiction. Thus every small submodule of $M$ is contained in every maximal proper submodule of $M$. Hence, the sum of all small submodules of $M$ is contained in the intersection of all maximal proper submodules of $M$.

    Conversely, suppose $x\in M$ is contained in the intersection of all maximal proper submodules of $M$. We claim that $Rx\ll M$. Indeed, if $N$ is a proper submodule of $M$ such that $Rx + N = M$, then $M/N$ is a cyclic $R$-module, so that it admits a non-zero simple quotient. In particular, $N$ is contained in a maximal proper submodule $K$ of $M$. But $x\in K$ by assumption; and hence $Rx + N\subseteq K\subsetneq M$, a contradiction. This shows that $Rx\ll M$, thereby completing the proof.
\end{proof}

\begin{proposition}\thlabel{rad-commutes-with-direct-sum}
    If $\{M_\lambda\}_{\lambda\in\Lambda}$ is a collection of $R$-module, then 
    \begin{equation*}
        \rad\left(\bigoplus_{\lambda\in\Lambda} M_\lambda\right) = \bigoplus_{\lambda\in\Lambda} \rad(M_\lambda).
    \end{equation*}
\end{proposition}
\begin{proof}
    Straightforward.
\end{proof}

\begin{proposition}\thlabel{jacobson-contained-in-rad}
    Let $\frakR$ denote the Jacobson radical of a ring $R$. Then for any $R$-module $M$, $\frakR M\subseteq\rad(M)$.
\end{proposition}
\begin{proof}
    If $N\subsetneq M$ is a maximal proper submodule of $M$, then $M/N$ is isomorphic to $R/\frakm$ for some maximal ideal $\frakm$ of $R$. Since $\frakR\subseteq\frakm$, $\frakR M\subseteq N$. Thus $\frakR M\subseteq\rad (M)$.
\end{proof}

\begin{proposition}\thlabel{rad-of-projective}
    Let $R$ be a ring and $P$ a projective $R$-module. Then $\rad(P) = \frakR P$, where $\frakR$ denotes the Jacobson radical of $R$.
\end{proposition}
\begin{proof}
    There is an $R$-module $Q$ such that $F = P\oplus Q$ is a free module. In view of \thref{rad-commutes-with-direct-sum} and \thref{jacobson-contained-in-rad}
    \begin{equation*}
        \frakR P\oplus\frakR Q\subseteq\rad(P)\oplus\rad(Q) = \rad(F) = \frakR F = \frakR P\oplus\frakR Q.
    \end{equation*}
    Hence $\rad P = \frakR P$.
\end{proof}

\begin{theorem}\thlabel{flat-over-artin-is-projective}
    A flat module over an Artinian ring is projective.
\end{theorem}
\begin{proof}
    Let $R$ be an Artinian ring and $E$ a flat $R$-module. In view of \thref{projective-cover-exists-over-artinian}, there exists a projective cover $f\colon P\to E$. Set $K = \ker f\ll P$, and hence $\ker f\subseteq\frakR P$. Now since $P$ and $E$ are flat $R$-modules, the ``multiplication maps''
    \begin{equation*}
        \mu_1\colon\frakR\otimes_R P\to\frakR P\quad\text{ and }\quad\mu_2\colon\frakR\otimes_R E\to \frakR E
    \end{equation*}
    are isomorphisms, which can be seen either using the equational criterion of flatness of just invoking the $\Tor$ long exact sequence. Further note that the diagram 
    \begin{equation*}
        \xymatrix {
            0\ar[r] & \frakR\otimes_R K\ar[r]^-{\mathbbm 1\otimes\iota} & \frakR\otimes_R P\ar[r]^-{\mathbbm 1\otimes f}\ar[d]_{\mu_1} & \frakR\otimes_R E\ar[d]^{\mu_2}\ar[r] & 0\\
            & & \frakR P\ar[r]_-{g} & \frakR E
        }
    \end{equation*}
    commutes where $g$ is the restriction of $f$ to $\frakR P$. Since $E$ is flat, the $\Tor$ long exact sequence gives that the top row is short exact. Using the fact that $\mu_1$ and $\mu_2$ are isomorphisms, we can write 
    \begin{equation*}
        K = \ker f = \ker g = \mu_1\left(\ker(\mathbbm{1}\otimes f)\right) = \mu_1\left(\im(\mathbbm{1}\otimes\iota)\right) = \frakR K.
    \end{equation*}
    But $\frakR$ is nilpotent in $R$, and hence $K = 0$, that is, $E\cong P$ is projective.
\end{proof}

\begin{corollary}
    An arbitrary product of projective modules over an Artinian ring is projective.
\end{corollary}
\begin{proof}
    This follows immediately from \thref{product-of-flat-noetherian} and \thref{flat-over-artin-is-projective}.
\end{proof}