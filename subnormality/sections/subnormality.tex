\begin{definition}
    Let $G$ be a group. A subgroup $S\subseteq G$ is said to be \define{subnormal} in $G$ if there exist subgroups $H_i$ of $G$ such that 
    \begin{equation*}
        S = H_0\noreq H_1\noreq\cdots\noreq H_r = G.
    \end{equation*}
    In this situation, we write $S\nor\nor G$. The smallest integer $r$ for which the above holds is called the \define{subnormal depth} of $S$ in $G$.
\end{definition}

\begin{remark}
    Note that the definition of a subnormal subgroup behaves well with respect to ``contraction''. That is, if $S\subnor G$ and $H$ is any subgroup of $G$, then $S\cap H\subnor H$. As a result, if $S, T\subnor G$, then $S\cap T\subnor G$.

    Now, suppose $\varphi: G\to\overline G$ is a surjective group homomorphism and $S\subnor G$. Then, $\varphi(S)\subnor\overline G$, since the image of a subnormal series under $\varphi$ is still subnormal.
\end{remark}

\begin{lemma}
    Let $G$ be a finite group. Then $G$ is nilpotent if and only if every subgroup of $G$ is subnormal.
\end{lemma}
\begin{proof}
    Suppose $G$ is nilpotent and $H$ is a proper subgroup of $G$. Define $H_0 = H$ and $H_{i + 1} = N_G(H_i)$. Then, either $H_{i + 1} = G$ or $H_i\subsetneq H_{i + 1}$. This gives us a subnormal series for $H$.

    Conversely, suppose every subgroup of $G$ is subnormal and let $H$ be a proper subgroup. There is a sequence 
    \begin{equation*}
        H = H_0\nor H_1\nor\cdots\nor H_n = G.
    \end{equation*}
    In particular, we may assume that $H_i\subsetneq H_{i + 1}$ for $0\le i\le n - 1$. Hence, $H\subsetneq H_1\subseteq N_G(H)$. Due to \thref{prop:normalizer-strict-implies-nilpotent}, we see that $G$ must be nilpotent.
\end{proof}

\begin{proposition}
    Let $G$ be a finite group and $H\le G$. Then $H\subseteq\bfF(G)$ if and only if $H$ is nilpotent and subnormal in $G$.
\end{proposition}
\begin{proof}
    Since $\bfF(G)$ is nilpotent, if $H$ were contained in $\bfF(G)$, then it would be niloptent too. Further, due to the preceding lemma, $H\subnor G$ and $\bfF(G)\nor G$, whence $H\subnor G$.

    We prove the converse by induction on $|G|$. If $H = G$, then there is nothing to prove, since $G$ would be nilpotent and $\bfF(G) = G$. Suppose now that $H\subsetneq G$. There is a subnormal series
    \begin{equation*}
        H = H_0\nor H_1\nor\cdots\nor H_n = G.
    \end{equation*}
    where every successive containment is proper. Set $M = H_{n - 1}\nor G$. The inductive hypothesis applies since $H$ is nilpotent and subnormal in $M$, consequently, $H\subseteq\bfF(M)\subseteq\bfF(G)$, due to \thref{prop:fitting-of-normal}, thereby completing the proof.
\end{proof}

\begin{definition}
    A \define{minimal normal subgroup} of a group $G$ is a non-identity normal subgroup of $G$ that does not admit any non-trivial normal subgroups. The \define{socle} of a \caution{finite} group $G$ is defined to be the subgroup generated by all minimal normal subgroups of $G$, which is precisely their product.
\end{definition}

If $M$ and $N$ are two minimal normal subgroups of $G$, then $M\cap N = \{1\}$ and hence, every element of $M$ commutes with every element of $N$. Thus, $\Soc(G)$ is precisely the product of all minimal normal subgroups of $G$ and is a normal subgroup of $G$. Further, if $G$ is a finite group that is not trivial, then it admits a non-trivial minimal finite group, and hence, $\Soc(G)$ is non-trivial.

\begin{proposition}
    Let $G$ be a finite group. Then $\Soc(G)$ is characteristic in $G$.
\end{proposition}
\begin{proof}
    Let $\varphi\in\Aut(G)$. For a minimal normal subgroup $M$ of $G$, $\varphi(M)$ is also a minimal normal subgroup of $G$. Consequently, $\varphi$ permutes the minimal normal subgroups of $G$ and thus stabilizes the socle.
\end{proof}

\begin{theorem}\thlabel{thm:min-normal-normalizer}
    Let $G$ be a finite group, $S\subnor G$, and $M$ a minimal normal subgroup of $G$. Then $M\subseteq N_G(S)$.
\end{theorem}
\begin{proof}
    Induction on $|G|$. If $S = G$, then there is nothing to prove, so we can suppose that $S\subsetneq G$. Since $S\subnor G$, arguing as in the preceding proof, we can choose a normal subgroup $N\subsetneq G$ such that $S\subnor N\nor G$.

    If $M\cap N = 1$, then every element of $M$ commutes with every element of $N$, and hence, $M\subseteq C_G(N)\subseteq C_G(S)\subseteq N_G(S)$. Suppose now that $M\cap N$ is non-trivial. But since $M$ is a minimal normal subgroup, $M = M\cap N$, i.e. $M\subseteq N$.

    The inductive hypothesis applies to $N$, whence every minimal normal subgroup of $N$ normalizes $S$, consequently, $\Soc(N)$ normalizes $S$. Therefore, it suffices to show that $M\subseteq\Soc(N)$. 

    Since $N$ is a finite group and $M$ is a non-trivial normal subgroup of $N$, it contains a minimal normal subgroup. That is, $M\cap\Soc(N)\ne 1$. Since $\Soc(N)$ is characteristic in $N$, it must be normal in $G$. Owing to the minimality of $M$ in $G$, $M\cap\Soc(N) = M$, that is, $M\subseteq\Soc(N)$ as desired.
\end{proof}

\begin{theorem}[Wielandt]\thlabel{thm:wielandt-join}
    Let $G$ be a finite group and $S,T\subnor G$. Then $\langle S, T\rangle\subnor G$.
\end{theorem}
\begin{proof}
    Induction on $|G|$. Suppose $G$ is non-trivial, choose a minimal normal subgroup $M$ of $G$ and set $\overline G = G/M$. By abuse of notation, we use the ``overbar'' to denote the homomorphism $G\to\overline G$. Note that 
    \begin{equation*}
        \langle\overline S,\overline T\rangle = \overline{\langle S, T\rangle} = \overline{\langle S, T\rangle M},
    \end{equation*}
    since $M$ is the kernel of $G\to\overline G$. The inductive hypothesis applies to $\overline G$ and hence, $\langle\overline S,\overline T\rangle\subnor\overline G$. There is a natural bijection between the subgroups of $G$ containing $M$ and the subgroups of $\overline G$, which preserves normality and hence, subnormality. Therefore, $\langle S, T\rangle M\subnor G$.

    Finally, note that $M\subseteq N_G(S), N_G(T)$ and hence, $M\subseteq N_G(\langle S, T\rangle)$, whence $\langle S, T\rangle\nor\langle S, T\rangle M\subnor G$, whence the conclusion follows.
\end{proof}

\begin{lemma}\thlabel{lem:product-conjugate}
    Let $G$ be a group and $H\le G$. If $HH^x = G$ for some $x\in G$, then $H = G$.
\end{lemma}
\begin{proof}
    Write $x = uv$, where $u\in H$ and $v\in H^x$. Then $xv^{-1} = u$ and we have 
    \begin{equation*}
        H^x = \left(H^x\right)^{v^{-1}} = H^{uv^{-1}} = H^u = H.
    \end{equation*}
    Then $G = HH^x = HH = H$, as desired.
\end{proof}

% \begin{proposition}
%     Let $G$ be a finite group and $S\le G$. If $SS^x = S^xS$ for all $x\in G$, then $S\subnor G$.
% \end{proposition}
% \begin{proof}
    
% \end{proof}

\begin{theorem}[Wielandt Zipper Lemma]
    Let $G$ be a finite group and $S\le G$ such that $S\subnor H$ for every proper subgroup $H$ of $G$ containing $S$. If $S$ is not subnormal in $G$, then there is a unique maximal subgroup of $G$ containing $S$.
\end{theorem}
\begin{proof}
    We induct on $|G : S|$. Since $S$ is not normal, $N_G(S)\subsetneq G$, and thus $N_G(S)\subseteq M$ for some maximal subgroup $M$ of $G$. We must show that this $M$ is unique. Suppose that $S\subseteq K$ is another maximal subgroup of $G$. We shall show that $K = M$.

    By our hypothesis, $S\subnor K$. Suppose first that $S\noreq K$. Then $K\subseteq N_G(S)\subseteq M$ and hence due to maximality, $K = M$, as desired. We can suppose, therefore, that $S$ is not normal in $K$. Choose the shortest subnormal series 
    \begin{equation*}
        S = H_0\nor H_1\nor\cdots\nor H_r = K,
    \end{equation*}
    where $r\ge 2$, since $S$ is not normal in $K$. Also, $S$ is not normal in $H_2$ since otherwise we could delete $H_1$ to obtain a shorter subnormal series. Let $x\in H_2$ be such that $S^x\ne S$, and write $T = \langle S, S^x\rangle\supsetneq S$. Note that $T\subseteq K$. Also, $S^x\subseteq H_1^x = H_1\subseteq N_G(S)$, and thus, $T\subseteq N_G(S)\subseteq M$. Furthermore, we have that $S\nor T\subsetneq G$.

    Note that $S^x$ also satisfies the hypothesis of the theorem because conjugation by $x$ is an automorphism of $G$. We claim that the subgroup $T = \langle S, S^x\rangle$ also satisfies the same hypothesis. In particular, we need to show that if $T\subseteq H\subsetneq G$, then $T\subnor H$ and $T$ is not subnormal in $G$.

    First, if $T\subseteq H\subsetneq G$, then $S\subseteq H$, and thus $S\subnor H$, and similarly, $S^x\subnor H$, consequently, due to \thref{thm:wielandt-join}, $T\subnor  H$. Also, $S\nor T$ and so if $T\subnor G$, then it would follow that $S\subnor G$, a contradiction. Thus $T$ is not subnormal in $G$.

    Our inductivev hypothesis applies to $T$ since it properly contains $S$, and hence $T$ is contained in a unique maximal subgroup of $G$. But since $T\subseteq M$ and $T\subseteq K$, we have that $M = K$, as desired.
\end{proof}

\begin{definition}
    For a subgroup $H$ of a group $G$, let $H^G$ denote the smallest normal subgroup of $G$ containing $H$. This is known as the \define{normal closure} of $H$ in $G$.
\end{definition}

\begin{theorem}[Baer]\thlabel{thm:baer}
    Let $G$ be a finite group and $H\le G$. Then $H\subseteq\bfF(G)$ if and only if $\langle H, H^x\rangle$ is nilpotent for all $x\in G$.
\end{theorem}
\begin{proof}
    If $H\subseteq\bfF(G)$, then $H^x\subseteq\bfF(G)$ for every $x\in G$, since $\bfF(G)\noreq G$. Hence, $\langle H, H^x\rangle\subseteq\bfF(G)$. But since $\bfF(G)$ is nilpotent, so is $\langle H, H^x\rangle$.

    Conversely, suppose $\langle H, H^x\rangle$ is nilpotent for every $x\in G$. We induct on $|G|$. Taking $x = 1$, we see that $H$ is nilpotent, whence it suffices to prove that $H\subnor G$. 

    Suppose $H$ is not subnormal in $G$. For any proper subgroup $K$ of $G$ containing $H$, the induction hypothesis applies to $K$ and hence, $H\subseteq\bfF(K)$, that is, $H\subnor K$. Due to Wielandt's Zipper Lemma, there is a unique maximal subgroup $M$ of $G$ containing $H$.

    If $\langle H, H^x\rangle = G$, then $G$ is nilpotent and $\bfF(G) = G$, and $H\subnor G$, a contradiction. Thus, $\langle H, H^x\rangle\subsetneq G$ for all $x\in G$. This subgroup must be contained in a maximal subgroup of $G$; but since it contains $H$, and there is a unique maximal subgroup $M$ containing $H$, we conclude that $H^x\subseteq M$ for all $x\in G$. Therefore, $H^G\subseteq M\subsetneq G$. 

    Since $H^G$ is normal and properly contained in $G$, the induction hypothesis applies and $H\subnor H^G\nor G$, that is, $H\subnor G$, a contradiction. This completes the proof.
\end{proof}

\begin{theorem}[Zenkov]\thlabel{thm:zenkov}
    Let $G$ be a finite group and $A, B\le G$ be abelian subgroups. If $M$ is a minimal element in the set 
    \begin{equation*}
        \left\{A\cap B^g\colon g\in G\right\},
    \end{equation*}
    then $M\subseteq\bfF(G)$.
\end{theorem}
\begin{proof}
    The set $\{A\cap B^g\colon g\in G\}$ remains unchanged upon replacing $B$ with $B^g$. Therefore, we may assume that $M = A\cap B$. We prove the statement by induction on $|G|$. First, suppose that $G = \langle A, B^g\rangle$ for some $g\in G$. Since $A$ and $B^g$ are abelian, we have $A\cap B^g\subseteq Z(G)$, and hence, 
    \begin{equation*}
        A\cap B^g = \left(A\cap B^g\right)^{g^{-1}} = A^{g^{-1}}\cap B\subseteq B.
    \end{equation*}
    It follows that $A\cap B^g\subseteq A\cap B\subseteq M$, and by the minimality of $M$, we have $M = A\cap B^g\subseteq Z(G)\subseteq F(G)$, as desired.

    Next, assume that $\langle A, B^g\rangle\subsetneq G$ for all $g\in G$. To show that $M$ is contained in $\bfF(G)$, it suffices to show that every Sylow $p$-subgroup $P$ of $M$ is contained in $\bfF(G)$ (because every group is generated by its Sylow subgroups). Due to \thref{thm:baer}, it suffices to show that $\langle P, P^g\rangle$ is nilpotent for every $g\in G$.

    Fix $g\in G$, and let $H = \langle A, B^g\rangle\subsetneq G$, and $C = B\cap H$. For $h\in H$, we have 
    \begin{equation*}
        A\cap C^h = A\cap\left(B\cap H\right)^h = A\cap B^h\cap H = A\cap B^h.
    \end{equation*}
    In particular, $M = A\cap B = A\cap B\cap H = A\cap C$ is minimal in the set $\{A\cap C^h\colon h\in H\}$ since its minimal in the larger set $\{A\cap B^g\colon g\in G\}$. By the inductive hypothesis, $P\subseteq M\subseteq\bfF(H)$, and hence, $P\subseteq\bfO_p(H)$, since $\bfO_p(H)$ is the unique Sylow $p$-subgroup of $\bfF(H)$. Also, $P^g\subseteq B^g\subseteq H$, and since $\bfO_p(H)$ is a normal subgroup, we have that $\bfO_p(H)P^g$ is a $p$-group containing $\langle P, P^g\rangle$. In particular, $\langle P, P^g\rangle$ is a $p$-group, whence is nilpotent, as desired.
\end{proof}

\begin{corollary}\thlabel{cor:for-luccini}
    Let $A$ be an abelian subgroup of a non-trivial finite group $G$, and suppose that $|A|\ge |G : A|$. Then $A\cap\bfF(G)$ is non-trivial.
\end{corollary}
\begin{proof}
    If $A = G$, then there is nothing to prove. Suppose now that $A\subsetneq G$. If $g\in G$, then $|A||A^g| = |A|^2\ge |A||G : A| = |G|$. Further, due to \thref{lem:product-conjugate}, $AA^g\subsetneq G$. Hence, 
    \begin{equation*}
        |G| > |AA^g| = \frac{|A||A|^g}{|A\cap A^g|}\ge\frac{|G|}{|A\cap A^g|},
    \end{equation*}
    and thus $A\cap A^g$ is non-trivial. Since this holds for all $g\in G$, we can apply \thref{thm:zenkov} to deduce that there is a $g\in G$ such that $A\cap A^g\subseteq\bfF(G)$, whence $A\cap\bfF(G)$ is non-trivial.
\end{proof}

\subsection{Theorems of Luccini and Horosevskii}

\begin{theorem}[Luccini]\thlabel{thm:luccini}
    Let $A$ be a proper cyclic subgroup of a finite group $G$, and let $K = \core_G(A)$. Then $|A : K| < |G : A|$, and in particular, if $|A|\ge |G : A|$, then $K$ is non-trivial.
\end{theorem}
\begin{proof}
    Induction on $|G|$. Note that $A/K$ is a proper cyclic subgroup of $G/K$ and the core of $A/K$ in $G/K$ is trivial. If $K$ is non-trivial, then the inductive hypothesis applies and we deduce that 
    \begin{equation*}
        |A/K| = \left|A/K : \core_{G/K}(A/K)\right| < |G/K : A/K| = |G : A|.
    \end{equation*}

    We may now assume  that $K = 1$, and we shall show that $|A| < |G : A|$. Suppose not, that is, $|A|\ge |G : A|$. Due to \thref{cor:for-luccini}, $A\cap\bfF(G)$ is non-trivial. In particular, $\bfF(G)$ is non-trivial, so we can choose a minimal normal subgroup $E$ of $G$ with $E\subseteq\bfF(G)$ (since $\bfF(G)$ is normal in $G$). Due to \thref{lem:normal-in-nilpotent}, $E\cap Z(\bfF(G))$ is non-trivial; but since $Z(\bfF(G))$ is characteristic in $\bfF(G)$, it is normal in $G$. Due to the minimality of $E$, we must have $E\subseteq Z(\bfF(G))$, in particular, $E$ is abelian. Being abelian, every Sylow subgroup of $E$ is characteristic in $G$, whence due to minimality, $E$ itself must be a $p$-group. We contend that $E$ is an elementary abelian $p$-group. Indeed, consider $\wt E = \{x^p\colon x\in E\}$, which is proper and characteristic in $E$, and hence, is normal in $G$. Due to minimality of $E$, $\wt E = 1$, as desired.

    Since $E\subseteq Z(\bfF(G))$, we see that $E$ normalizes the non-trivial group $A\cap\bfF(G)$, and of course $A$ normalizes this too. Then $A\cap\bfF(G)\noreq AE$. Since $\core_G(A) = 1$, we cannot have $AE = G$, else $A\cap\bfF(G)$ would be contained in the core. It follows that $AE\subseteq G$.

    Set $\overline G = G/E$, $\overline A = AE/E\subsetneq\overline G$, $\overline{M} = \core_{\overline{G}}(\overline A)$, with $E\subseteq M$ and $M\noreq G$. Note that $M\subseteq AE$, and hence, $AE\subseteq AM\subseteq AE$, whence $AM = AE$. Due to the inductive hypothesis, we must have $|\overline A : \overline M| < |\overline G : \overline A|$, that is, $|AE : M| < |G : AE|$.

    \begin{equation*}
        \xymatrix {
            G\ar@{-}[dd]\ar@{-}[rd] & & & \\
            & AE\ar@{-}[rd]\ar@{-}[ld] & & \\
            A\ar@{-}[rd] & & M\ar@{-}[rd]\ar@{-}[ld] & \\
            & B\ar@{-}[rd] & & E\ar@{-}[ld] \\
            & & B\cap E & 
        }
    \end{equation*}

    Let $B = A\cap M$ so that $B$ is cyclic. We have 
    \begin{equation*}
        |AE : A| = |AM : A| = |M : A\cap M| = |M : B|,
    \end{equation*}
    and hence, $|AE : M| = |A : B|$. Therefore, 
    \begin{equation*}
        |M : B| = |AE : A| = \frac{|G : A|}{|G : AE|} < \frac{|G : A|}{|AE : M|} = \frac{|G : A|}{|A : B|}\le\frac{|A|}{|A : B|} = |B|.
    \end{equation*}
    Before we proceed, note that $E\subseteq M\subseteq AE = EA$, and hence, because of what's colloquially known as Dedekind's rule, $M = E(A\cap M) = EB = BE$ (since $E\noreq G$).

    Suppose $M$ is abelian, and let $\varphi: M\to M$ be the endomorphism $\varphi(m) = m^p$. Then $E\subseteq\ker\varphi$ since it is an elementary abelian $p$-group. It follows that 
    \begin{equation*}
        \varphi(M) = \varphi(EB) = \varphi(B)\subseteq B\subseteq A.
    \end{equation*}
    Now, $M\noreq G$, and hence, $\varphi(M)\noreq G$, and we conclude that $\varphi(M) = 1$, since $\core_G(A) = 1$. Then $\varphi(B) = 1$, and since $B$ is cyclic, it follows that $|B|\le p$. Then $|M : B| < |B|\le p$, and since $M/B\cong E/B\cap E$\footnote{These quotients make sense because $M$ is abelian.}, it is a $p$-group, it follows that $M/B = 1$, that is, $M = B\subseteq A$. But $M\noreq G$, and since $M\subseteq A$, we have $M = 1$, whence $E = 1$, a contradiction.

    It follows that $M$ is non-abelian, and since $M/E\cong B/B\cap E$ is cyclic, we conclude that $E$ is not central in $M$\footnote{Recall that if $G/Z(G)$ is cyclic, then $G$ is abelian.}, and so $E\cap Z(M)\subsetneq E$. Again recall that $Z(M)$ is characteristic in $M$ and hence normal in $G$. Due to the minimality of $E$, we must have $E\cap Z(M) = 1$, and thus $Z(M)$ is cyclic because the restriction of the surjection $M\onto M/E$ is injective on $Z(M)$.

    Since $B$ is an abelian subgroup of $M$ and $|M : B| < |B|$, due to \thref{cor:for-luccini}, we have that $B\cap\bfF(M)$ is non-trivial. Due to \thref{prop:fitting-of-normal}, $\bfF(M)\subseteq\bfF(G)$, and so $E$ centralizes $\bfF(M)$ because $E\subseteq Z(\bfF(G))$. Since every element of $B\cap \bfF(M)$ commutes with every element of $B$ (since $B$ is abelian) and every element of $E$, we see that $B\cap \bfF(M)$ is a non-trivial central subgroup of $EB = M$. Since $Z(M)$ is cyclic, we see that $B\cap\bfF(M)\subseteq Z(M)$ is characteristic in $Z(M)\noreq G$\footnote{Every subgroup of a cyclic group is characteristic.}, and hence, $B\cap\bfF(M)$ is a non-trivial normal subgroup of $G$ contained in $A$, a contradiction. This completes the proof.
\end{proof}

\begin{theorem}[Horosevskii]
    Let $\sigma\in\Aut(G)$, where $G$ is a non-trivial finite group. Then the order $o(\sigma)$ of $\sigma$ as an element of $\Aut(G)$ is strictly smaller than $|G|$.
\end{theorem}
\begin{proof}
    Let $A = \langle\sigma\rangle\subseteq\Aut(G)$, so that $A$ is a cyclic group of order equal to the order of $\sigma$ as an element of $\Aut(G)$. Set $\Gamma = G\rtimes_\theta A$, where $\theta: A\to\Aut(G)$ is the obvious inclusion map. We identify $G$ and $A$ with subgroups $G\times\{1\}$ and $\{1\}\times A$ of $\Gamma$. Note that the conjugation action of $A$ on $G$ as elements of $\Gamma$ is given by $g^\tau = \tau(g)\in G$ for $\tau\in A$. By definition of an automorphism, every non-identity element of $A$ acts non-trivially on $G$, and hence, $A\cap C_\Gamma(G) = 1$.

    Since $G$ is non-trivial and $A$ is cyclic, due to \thref{thm:luccini}, $|A : K| < |\Gamma : A|$, where $K = \core_\Gamma(A)$. But then $K\cap G\subseteq A\cap G = 1$, and both $K$ and $G$ are normal in $\Gamma$, consequently, their elements commute, that is, $K\subseteq C_\Gamma(G)$. Since $K\subseteq A$, we see that $K\subseteq A\cap C_\Gamma(G) = 1$, that is, $K$ is trivial. Thus,
    \begin{equation*}
        o(\sigma) = |A| = |A : K| < |\Gamma : A| = G,
    \end{equation*}
    as desired.
\end{proof}

\subsection{Quasisimple Groups}

Recall that for a group $G$, we denote the commutator subgroup $[G, G]$ by $G'$. A group is said to be \define{perfect} if $G = G'$. We denote the further commutators of $G$ by $G'' = [G', G']$ and $G''' = [G'', G'']$. A group is said to be \define{simple} if it admits precisely two normal subgroups. In particular, the trivial group is \emph{not} simple.

\begin{lemma}
    Let $G$ be a group and suppose that $G/Z(G)$ is simple. Then $G/Z(G)$ is non-abelian, and $G'$ is perfect. Also $G'/Z(G')$ is isomorphic to the simple group $G/Z(G)$.
\end{lemma}
\begin{proof}
    Let $Z = Z(G)$. If $G/Z$ abelian simple, then it must be cyclic, and hence, $G$ is abelian, whence $G = Z$, a contradiction. Thus, $G/Z$ is a non-abelian group, in particular, $G$ is not solvable, thus $G'''\ne 1$, so $G''$ is not abelian, and hence, $G''\not\subseteq Z$.

    Since $G/Z$ is simple, $Z$ is a maximal normal subgroup of $G$ and $G''\not\subseteq G$, and thus, $G''Z\supsetneq Z$ is a normal subgroup of $G$, and we conclude that $G''Z = G$. Then 
    \begin{equation*}
        \frac{G}{G''} = \frac{G''Z}{G''}\cong\frac{Z}{Z\cap G''},
    \end{equation*}
    which is abelian. Thus, $G'\subseteq G''\subseteq G'$, whence $G'$ is perfect.

    Finally, since $G = G''Z = G'Z$, we have 
    \begin{equation*}
        \frac{G'}{Z\cap G'}\cong\frac{G'Z}{Z} = \frac{G}{Z}
    \end{equation*}
    is simple. It follows that $Z\cap G'$ is a maximal normal subgroup of $G'$, and since $G'$ is non-abelian, we see that $Z\cap G'\subseteq Z(G')\subsetneq G'$, and hence, $Z\cap G' = Z(G')$. Thus, 
    \begin{equation*}
        \frac{G'}{Z(G')} = \frac{G'}{Z\cap G'}\cong\frac{G'Z}{Z} = \frac{G}{Z},
    \end{equation*}
    as desired.
\end{proof}

\begin{definition}
    A group $G$ is said to be \define{quasisimple} if $G/Z(G)$ is simple and $G$ is perfect.
\end{definition}

\begin{lemma}\thlabel{lem:quasi-simple}
    Let $G$ be quasisimple. If $N$ is a proper normal subgroup of $G$, then $N\subseteq Z(G)$. Also, every nonidentity homomorphic image of $G$ is quasisimple.
\end{lemma}
\begin{proof}
    Again, let $Z = Z(G)$, so that $Z$ is a maximal normal subgroup of $G$, and let $N\nor G$ with $N\subsetneq G$. If $N\not\subseteq Z$, then $NZ\supsetneq Z$ is a normal subgroup of $G$, and hence, $NZ = G$. Then, we have that 
    \begin{equation*}
        \frac{G}{N} = \frac{NZ}{N} = \frac{Z}{N\cap Z}
    \end{equation*}
    is abelian, and so $G = G'\subseteq N\subsetneq G$, a contradiction. Hence, $N\subseteq G$.

    Next, we must show that $\overline G = G/N$ is quasisimple. We know that $(\overline G)' = \overline{G'} = \overline G$, and thus $\overline G$ is perfect. Further, since $N\subseteq Z$, we have $\overline G/\overline Z\cong G/Z$ is simple and non-abelian. Thus, $Z(\overline G) = \overline Z$, thereby completing the proof.
\end{proof}

\begin{definition}
    A subnormal quasisimple subgroup of an arbitrary finite group $G$ is called a \define{component} of $G$.
\end{definition}

Before proceeding, we present a technical lemma due to P. Hall. 

\begin{lemma}[P. Hall]\thlabel{lem:hall-commutator}
    Let $G$ be a group (possibly infinite). Let $x,y,z\in G$, then 
    \begin{equation*}
        [x, y^{-1}, z]^y [y, z^{-1}, x]^{z} [z, x^{-1}, y]^{x} = 1.
    \end{equation*}
\end{lemma}
\begin{proof}
    Just write it out :-)
\end{proof}

\begin{lemma}[Three subgroups]\thlabel{lem:three-subgroups}
    Let $X, Y, Z\le G$ and suppose 
    \begin{equation*}
        [X, Y, Z] = 1\qquad\text{and}\qquad[Y, Z, X] = 1.
    \end{equation*}
    Then $[Z, X, Y] = 1$.
\end{lemma}
\begin{proof}
    Let $x\in X, y\in Y, z\in Z$. Then $[x, y^{-1}, z] = 1$ and $[y, z^{-1}, x] = 0$, consequently due to \thref{lem:hall-commutator}, $[z, x^{-1}, y]^{x} = 1$ and hence $[z, x^{-1}, y] = 1$. That is, $[z, x^{-1}]\in C_G(y)$ for all $x\in X$, $y\in Y$, and $z\in Z$. It follows that $[Z,X]\subseteq C_G(Y)$, and hence $[Z, X, Y] = 1$.
\end{proof}

\begin{lemma}\thlabel{lem:minimal-normal-commutes-with-component}
    Let $N$ be a minimal normal subgroup of a finite group $G$, and suppose that $H$ is a component of $G$ with $H\not\subseteq N$. Then $[N, H] = 1$.
\end{lemma}
\begin{proof}
    Note that $H\cap N\subsetneq H$ and $H\cap N\nor H$, whence by \thref{lem:quasi-simple}, $H\cap N\subseteq Z(H)$. Now, $H\subnor G$, and $N$ is minial normal in $G$, whence due to \thref{thm:min-normal-normalizer}, $N\subseteq N_G(H)$, and hence, $[N, H]\subseteq H$. Since $N$ is normal, we have $[N, H]\subseteq N$, consequently, $[N, H]\subseteq N\cap H\subseteq Z(H)$. Then $[N, H, H] = 1$ and $[H, N, H] = 1$. Due to \thref{lem:three-subgroups}, we must have $[H, H, N] = 1$. Since $H' = H$, we have $[H, N] = 1$ as desired.
\end{proof}

\begin{theorem}
    Let $H$ and $K$ be distinct components of a finite group $G$. Then $[H, K] = 1$.
\end{theorem}
\begin{proof}
    Induction on $|G|$. If both $H$ and $K$ are contained in a proper subgroup $X$ of $G$, then $H$ and $K$ are subnormal in $X$ and hence, are distinct components of $X$. The inductive hypothesis applies and $[H, K] = 1$. So we can assume henceforth that no proper subgroup of $G$ contains both $H$ and $K$.

    If $G$ is simple, then being subnormal, both $H$ and $K$ must be one of $\{1, G\}$. If one of $H$ or $K$ is $1$, then there is nothing to prove. On the other hand, since $H\ne K$, we cannot have $H = G = K$. Thus, we may assume that $G$ is a non-trivial non-simple group. Let $N\nor G$ be a minimal normal subgroup (hence $N\subsetneq G$). If one of the components, say $K$ were contained in $N$, then $H\not\subseteq N$ (since they cannot be contained in a proper subgroup of $G$), and due to \thref{lem:minimal-normal-commutes-with-component} $[H, K]\subseteq [H, N] = 1$, as desired. We can therefore assume that for every minimal normal subgroup $N$ of $G$, we have $H\not\subseteq N$, and $K\not\subseteq N$.

    Let $\overline G = G/N$, where $N$ is a minimal normal subgroup of $G$, and observe that $\overline H$ and $\overline K$ are non-identity subnormal subgroups of $\overline G$. Due to \thref{lem:quasi-simple}, both $\overline H$ and $\overline K$ are quasisimple., and so they are components of $\overline G$. If $\overline H\ne\overline K$, then by the inductive hypothesis, $\overline{[H, K]} = [\overline H,\overline K] = 1$, and hence, $[H, K]\subseteq N$. Due to \thref{lem:minimal-normal-commutes-with-component}, $[N, H] = [N, K] = 1$, and thus, 
    \begin{equation*}
        [H, K, H] = 1\qquad\text{and}\qquad[K, H, H] = 1.
    \end{equation*}
    Due to \thref{lem:three-subgroups}, $1 = [H, H, K] = [H, K]$, since $H' = H$ owing to it being quasisimple.

    It remains to analyze the case $\overline H = \overline K$, that is, $HN = KN$, and we can assume that this equality holds for every minimal normal subgroup $N$ of $G$. Since $HN$ contains both $H$ and $K$, it follows that $HN = G$ (since both $H$ and $K$ cannot be contained in a proper subgroup of $G$). By \thref{thm:min-normal-normalizer}, $N\subseteq N_G(H)$, and thus $H\nor HN = G$, and similarly, $K\nor G$, and hence, $[H\cap K]\subseteq H\cap K$. If $1\ne [H, K]\nor G$, we could choose a minimal normal subgroup $N$ such that $N\subseteq [H, K]\subseteq H\cap K$. Thus $H = HN = KN = K$, a contradiction.
\end{proof}

% \begin{lemma}
%     Suppose that a finite simple group $G$ is the product of the members of some collection $\cal X$ of non-abelian simple normal subgroups. Then the product is direct, and $\cal X$ is the set of all minimal normal subgroups of $G$.
% \end{lemma}
% \begin{proof}
    
% \end{proof}