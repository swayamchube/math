\documentclass[12pt]{article}

% \usepackage{./arxiv}

\title{Subnormality in Group Theory}
\author{Swayam Chube}
\date{\today}

\usepackage[utf8]{inputenc} % allow utf-8 input
\usepackage[T1]{fontenc}    % use 8-bit T1 fonts
\usepackage{hyperref}       % hyperlinks
\usepackage{url}            % simple URL typesetting
\usepackage{booktabs}       % professional-quality tables
\usepackage{amsfonts}       % blackboard math symbols
\usepackage{nicefrac}       % compact symbols for 1/2, etc.
\usepackage{microtype}      % microtypography
\usepackage{graphicx}
\usepackage{natbib}
\usepackage{doi}
\usepackage{amssymb}
\usepackage{bbm}
\usepackage{amsthm}
\usepackage{amsmath}
\usepackage{xcolor}
\usepackage{theoremref}
\usepackage{enumitem}
\usepackage{mathpazo}
% \usepackage{euler}
\usepackage{mathrsfs}
\setlength{\marginparwidth}{2cm}
\usepackage{todonotes}
\usepackage{stmaryrd}
\usepackage[all,cmtip]{xy} % For diagrams, praise the Freyd–Mitchell theorem 
\usepackage{marvosym}
\usepackage{geometry}
\usepackage{titlesec}
\usepackage{tikz}
\usetikzlibrary{cd}

\renewcommand{\qedsymbol}{$\blacksquare$}

% Uncomment to override  the `A preprint' in the header
% \renewcommand{\headeright}{}
% \renewcommand{\undertitle}{}
% \renewcommand{\shorttitle}{}

\hypersetup{
    pdfauthor={Lots of People},
    colorlinks=true,
}

\newtheoremstyle{thmstyle}%               % Name
  {}%                                     % Space above
  {}%                                     % Space below
  {}%                             % Body font
  {}%                                     % Indent amount
  {\bfseries\scshape}%                            % Theorem head font
  {.}%                                    % Punctuation after theorem head
  { }%                                    % Space after theorem head, ' ', or \newline
  {\thmname{#1}\thmnumber{ #2}\thmnote{ (#3)}}%                                     % Theorem head spec (can be left empty, meaning `normal')

\newtheoremstyle{defstyle}%               % Name
  {}%                                     % Space above
  {}%                                     % Space below
  {}%                                     % Body font
  {}%                                     % Indent amount
  {\bfseries\scshape}%                            % Theorem head font
  {.}%                                    % Punctuation after theorem head
  { }%                                    % Space after theorem head, ' ', or \newline
  {\thmname{#1}\thmnumber{ #2}\thmnote{ (#3)}}%                                     % Theorem head spec (can be left empty, meaning `normal')

\theoremstyle{thmstyle}
\newtheorem{theorem}{Theorem}[section]
\newtheorem{lemma}[theorem]{Lemma}
\newtheorem{proposition}[theorem]{Proposition}

\theoremstyle{defstyle}
\newtheorem{definition}[theorem]{Definition}
\newtheorem{corollary}[theorem]{Corollary}
\newtheorem{remark}[theorem]{Remark}
\newtheorem{example}[theorem]{Example}
\newtheorem*{notation}{Notation}

% Common Algebraic Structures
\newcommand{\R}{\mathbb{R}}
\newcommand{\Q}{\mathbb{Q}}
\newcommand{\Z}{\mathbb{Z}}
\newcommand{\N}{\mathbb{N}}
\newcommand{\bbC}{\mathbb{C}} 
\newcommand{\K}{\mathbb{K}} % Base field which is either \R or \bbC
\newcommand{\calA}{\mathcal{A}} % Banach Algebras
\newcommand{\calB}{\mathcal{B}} % Banach Algebras
\newcommand{\calI}{\mathcal{I}} % ideal in a Banach algebra
\newcommand{\calJ}{\mathcal{J}} % ideal in a Banach algebra
\newcommand{\frakM}{\mathfrak{M}} % sigma-algebra
\newcommand{\calO}{\mathcal{O}} % Ring of integers
\newcommand{\bbA}{\mathbb{A}} % Adele (or ring thereof)
\newcommand{\bbI}{\mathbb{I}} % Idele (or group thereof)

% Categories
\newcommand{\catTopp}{\mathbf{Top}_*}
\newcommand{\catGrp}{\mathbf{Grp}}
\newcommand{\catTopGrp}{\mathbf{TopGrp}}
\newcommand{\catSet}{\mathbf{Set}}
\newcommand{\catTop}{\mathbf{Top}}
\newcommand{\catRing}{\mathbf{Ring}}
\newcommand{\catCRing}{\mathbf{CRing}} % comm. rings
\newcommand{\catMod}{\mathbf{Mod}}
\newcommand{\catMon}{\mathbf{Mon}}
\newcommand{\catMan}{\mathbf{Man}} % manifolds
\newcommand{\catDiff}{\mathbf{Diff}} % smooth manifolds
\newcommand{\catAlg}{\mathbf{Alg}}
\newcommand{\catRep}{\mathbf{Rep}} % representations 
\newcommand{\catVec}{\mathbf{Vec}}

% Group and Representation Theory
\newcommand{\chr}{\operatorname{char}}
\newcommand{\Aut}{\operatorname{Aut}}
\newcommand{\GL}{\operatorname{GL}}
\newcommand{\im}{\operatorname{im}}
\newcommand{\tr}{\operatorname{tr}}
\newcommand{\id}{\mathbf{id}}
\newcommand{\cl}{\mathbf{cl}}
\newcommand{\Gal}{\operatorname{Gal}}
\newcommand{\Tr}{\operatorname{Tr}}
\newcommand{\sgn}{\operatorname{sgn}}
\newcommand{\Sym}{\operatorname{Sym}}
\newcommand{\Alt}{\operatorname{Alt}}

% Commutative and Homological Algebra
\newcommand{\spec}{\operatorname{spec}}
\newcommand{\mspec}{\operatorname{m-spec}}
\newcommand{\Tor}{\operatorname{Tor}}
\newcommand{\tor}{\operatorname{tor}}
\newcommand{\Ann}{\operatorname{Ann}}
\newcommand{\Supp}{\operatorname{Supp}}
\newcommand{\Hom}{\operatorname{Hom}}
\newcommand{\End}{\operatorname{End}}
\newcommand{\coker}{\operatorname{coker}}
\newcommand{\limit}{\varprojlim}
\newcommand{\colimit}{%
  \mathop{\mathpalette\colimit@{\rightarrowfill@\textstyle}}\nmlimits@
}
\makeatother


\newcommand{\fraka}{\mathfrak{a}} % ideal
\newcommand{\frakb}{\mathfrak{b}} % ideal
\newcommand{\frakc}{\mathfrak{c}} % ideal
\newcommand{\frakf}{\mathfrak{f}} % face map
\newcommand{\frakg}{\mathfrak{g}}
\newcommand{\frakh}{\mathfrak{h}}
\newcommand{\frakm}{\mathfrak{m}} % maximal ideal
\newcommand{\frakn}{\mathfrak{n}} % naximal ideal
\newcommand{\frakp}{\mathfrak{p}} % prime ideal
\newcommand{\frakq}{\mathfrak{q}} % qrime ideal
\newcommand{\fraks}{\mathfrak{s}}
\newcommand{\frakt}{\mathfrak{t}}
\newcommand{\frakz}{\mathfrak{z}}
\newcommand{\frakA}{\mathfrak{A}}
\newcommand{\frakI}{\mathfrak{I}}
\newcommand{\frakJ}{\mathfrak{J}}
\newcommand{\frakK}{\mathfrak{K}}
\newcommand{\frakL}{\mathfrak{L}}
\newcommand{\frakN}{\mathfrak{N}} % nilradical 
\newcommand{\frakO}{\mathfrak{O}} % dedekind domain
\newcommand{\frakP}{\mathfrak{P}} % Prime ideal above
\newcommand{\frakQ}{\mathfrak{Q}} % Qrime ideal above 
\newcommand{\frakR}{\mathfrak{R}} % jacobson radical
\newcommand{\frakU}{\mathfrak{U}}
\newcommand{\frakX}{\mathfrak{X}}

% General/Differential/Algebraic Topology 
\newcommand{\scrA}{\mathscr A}
\newcommand{\scrB}{\mathscr B}
\newcommand{\scrF}{\mathscr F}
\newcommand{\scrN}{\mathscr N}
\newcommand{\scrP}{\mathscr P}
\newcommand{\scrR}{\mathscr R}
\newcommand{\scrS}{\mathscr S}
\newcommand{\bbH}{\mathbb H}
\newcommand{\Int}{\operatorname{Int}}
\newcommand{\psimeq}{\simeq_p}
\newcommand{\wt}[1]{\widetilde{#1}}
\newcommand{\RP}{\mathbb{R}\text{P}}
\newcommand{\CP}{\mathbb{C}\text{P}}

% Miscellaneous
\newcommand{\wh}[1]{\widehat{#1}}
\newcommand{\calM}{\mathcal{M}}
\newcommand{\calP}{\mathcal{P}}
\newcommand{\onto}{\twoheadrightarrow}
\newcommand{\into}{\hookrightarrow}
\newcommand{\Gr}{\operatorname{Gr}}
\newcommand{\Span}{\operatorname{Span}}
\newcommand{\ev}{\operatorname{ev}}
\newcommand{\weakto}{\stackrel{w}{\longrightarrow}}

\newcommand{\define}[1]{\textcolor{blue}{\textit{#1}}}
\newcommand{\caution}[1]{\textcolor{red}{\textit{#1}}}
\renewcommand{\mod}{~\mathrm{mod}~}
\renewcommand{\le}{\leqslant}
\renewcommand{\leq}{\leqslant}
\renewcommand{\ge}{\geqslant}
\renewcommand{\geq}{\geqslant}
\newcommand{\Res}{\operatorname{Res}}
\newcommand{\floor}[1]{\left\lfloor #1\right\rfloor}
\newcommand{\ceil}[1]{\left\lceil #1\right\rceil}
\newcommand{\gl}{\mathfrak{gl}}
\newcommand{\ad}{\operatorname{ad}}
\newcommand{\Stab}{\operatorname{Stab}}
\newcommand{\bfX}{\mathbf{X}}
\newcommand{\Ind}{\operatorname{Ind}}
\newcommand{\bfG}{\mathbf{G}}
\newcommand{\rank}{\operatorname{rank}}
\newcommand{\calo}{\mathcal{o}}
\newcommand{\frako}{\mathfrak{o}}
\newcommand{\Cl}{\operatorname{Cl}}

\newcommand{\idim}{\operatorname{idim}}
\newcommand{\pdim}{\operatorname{pdim}}
\newcommand{\Ext}{\operatorname{Ext}}
\newcommand{\co}{\operatorname{co}}
\newcommand{\bfO}{\mathbf{O}}
\newcommand{\bfF}{\mathbf{F}} % Fitting Subgroup
\newcommand{\Syl}{\operatorname{Syl}}
\newcommand{\nor}{\vartriangleleft}
\newcommand{\noreq}{\trianglelefteqslant}
\newcommand{\subnor}{\nor\!\nor}
\newcommand{\Soc}{\operatorname{Soc}}
\newcommand{\core}{\operatorname{core}}

\geometry {
    margin = 1in
}

\titleformat
{\section}
[block]
{\Large\bfseries\scshape}
{\S\thesection}
{0.5em}
{\centering}
[]


\titleformat
{\subsection}
[block]
{\normalfont\bfseries\sffamily}
{\S\S}
{0.5em}
{\centering}
[]


\begin{document}
\maketitle

\tableofcontents

\section{Sylow Theory}

\subsection{The Three Theorems}

In this section, we shall state and prove the three Sylow theorems.

\begin{theorem}[Sylow's First Theorem]\thlabel{sylow-first}
    Let $G$ be a finite group and $p$ be a prime dividing the order of $G$ with $k\in\N$ such that $p^k\||G|$. Then, there is a subgroup $P\le G$ with $|P| = p^k$.
\end{theorem}

We denote the set of all $p$-Sylow subgroups by $\Syl_p(G)$.

\begin{theorem}[Sylow's Second Theorem]\thlabel{sylow-second}
    Let $G$ be a finite group and $p$ be a prime dividing the order of $G$. Then, all subgroups in $\Syl_p(G)$ are conjugate.
\end{theorem}

In order to prove the above theorem, we require the following lemmas: 
\begin{lemma}\thlabel{p-group-normalizer-lemma}
    Let $G$ be a finite group, $p$ a prime dividing $|G|$ and $P\in\Syl_p(G)$. If $H$ is a $p$-group contained in $N_G(P)$, then $H$ is contained in $P$.
\end{lemma}

\begin{lemma}\thlabel{p-group-conjugate-sylow-lemma}
    Let $G$ be a finite group, $p$ a prime dividing $|G|$, $H$ a $p$-subgroup and $P\in\Syl_p(G)$. Then, there is $x\in G$ such that $xHx^{-1}\subseteq P$.
\end{lemma}

\begin{theorem}[Sylow's Third Theorem]
    Let $G$ be a finite grouop and $p$ a prime dividing $|G|$. Let $n_p$ be the cardinality of $\Syl_p(G)$. Then, 
    \begin{enumerate}
        \item $n_p = |G|/|N_G(P)|$ for any $P\in\Syl_p(G)$
        \item $n_p\mid |G|$
        \item $n_p\equiv1\pmod p$
    \end{enumerate}
\end{theorem}

\subsection{Some Related Results}

Henceforth, unless specified otherwise, $G$ is a finite group and $p$ is a prime dividing the order of $G$.

\begin{lemma}\thlabel{p-group-containment-lemma}
    Let $G$ be a finite group and $P$ be a $p$-subgroup of $G$. Then, there is a $p$-Sylow subgroup of $G$ containing $P$.
\end{lemma}
\begin{proof}
    Choose any $Q\in\Syl_p(G)$. Using \thref{p-group-conjugate-sylow-lemma}, there is $x\in G$ such that $xPx^{-1}\subseteq Q$, and equivalently, $P\subseteq x^{-1}Qx$, which is also a $p$-Sylow subgroup. This completes the proof.
\end{proof}

\begin{corollary}
    Let $G$ be a finite group and $H$ a subgroup. If $P\in\Syl_p(H)$, then there is $Q\in\Syl_p(G)$ such that $P = H\cap Q$.
\end{corollary}
\begin{proof}
    Since $P$ is a $p$-subgroup of $G$, due to \thref{p-group-containment-lemma}, there is a $p$-Sylow subgroup $Q$ containing it. We shall show that $P = H\cap Q$. Obviously, $P\subseteq H\cap Q$, therefore, $v_p(|H\cap Q|)\ge v_p(|P|) = v_p(H)$. But since $H\cap Q$ is a subgroup of $H$, we must have $v_p(|H|)\ge v_p(|H\cap Q|)$, as a result, $v_p(|H|) = v_p(|H\cap Q|)$ and $P = H\cap Q$, since $H\cap Q$ is a $p$-group owing the fact that it is a subgroup of $Q$.
\end{proof}


\begin{theorem}\thlabel{generalized-normalizer-stability}
    Let $P\in\Syl_p(G)$ and $H$ be a subgroup of $G$ such that $N_G(P)\subseteq H$. Then, $N_G(H) = H$ and $[G:H]\equiv 1\pmod p$.
\end{theorem}
\begin{proof}
    Let $x\in N_G(H)$. Then, $P^x\subseteq H$ and is also an element of $\Syl_p(H)$. Using \thref{sylow-second}, there is $h\in H$ such that $P^x = P^h$, equivalently, $x^{-1}h\in N_G(P)\subseteq H$, implying that $x\in H$.

    Now, we have 
    \begin{equation*}
        [G:H] = \frac{[G:N_G(P)]}{[H:N_G(P)]} = \frac{n_p(G)}{n_p(H)}\equiv1\pmod p
    \end{equation*}
\end{proof}

In particular, we have the following attractive result:
\begin{corollary}\thlabel{normalizer-stability-lemma}
    Let $P\in\Syl_p(G)$. Then, $N_G(N_G(P)) = N_G(P)$.
\end{corollary}

\begin{theorem}[Frattini Argument]\thlabel{thm:frattini-argument}
    Let $N$ be a normal subgroup of $G$ and $P\in\Syl_p(N)$, then $G = N_G(P)N$.
\end{theorem}
\begin{proof}
    Let $g\in G$. Since $N\noreq G$, $P^g\subseteq N^g\subseteq N$, $P^g\in\Syl_p(N)$, as a result, there is $n\in N$ such that $(P^g) = P^n$, equivalently, $P^{n^{-1}g} = P$. This immediately implies $n^{-1}g\in N_G(P)$, therefore, $g\in NN_G(P) = N_G(P)N$, completing the proof.
\end{proof}

\section{Nilpotent Groups}

\begin{definition}[Nilpotent Groups]\thlabel{def-nilpotent}
    A group $G$ is said to be \textit{nilpotent} if there is a finite collection of normal subgroups $H_0,\ldots,H_n$ with 
    \begin{equation*}
        1 = H_0\subseteq H_1\subseteq\cdots\subseteq H_n = G
    \end{equation*}
    and such that 
    \begin{equation*}
        H_{i + 1}/H_i\subseteq Z(G/H_{i})
    \end{equation*}
    for $0\le i < n$.
\end{definition}

The Upper Central Series and the Lower Central Series are often useful in the analysis of nilpotent groups.

\begin{definition}[Upper Central Series]\thlabel{def-upper-central}
    For any group $G$, define the \textit{Upper Central Series} as a sequence of groups,
    \begin{equation*}
        1 = Z_0\noreq Z_1\noreq\cdots
    \end{equation*}
    such that 
    \begin{enumerate}
        \item Each $Z_i$ is characteristic in $G$ 
        \item $Z_{i + 1}/Z_i = Z(G/Z_i)$
    \end{enumerate}
\end{definition}

\begin{definition}[Lower Central Series]\thlabel{def-lower-central}
    For any group $G$, define the \textit{Lower Central Series} as a sequence of groups,
    \begin{equation*}
        G = G_0\unrhd G_1\unrhd\cdots
    \end{equation*}
    such that $G_{i + 1} = [G,G_{i}]$
\end{definition}

\subsection{Analyzing The Upper And Lower Central Series}
\begin{lemma}\thlabel{lem-zi-construct}
    For all $i\ge 0$, let $\pi_i:G\twoheadrightarrow G/Z_i$ denote the projection. Then, $Z_{i + 1} = \pi_i^{-1}\left(Z(G/Z_i)\right)$.
\end{lemma}
\begin{proof}
    Obvious.
\end{proof}

\begin{lemma}\thlabel{lem-zi-characteristic}
    For all $i\ge0$, $Z_i$ is characteristic in $G$
\end{lemma}
\begin{proof}
    We shall show this by induction on $i$. The statement is obviously true for $Z_0 = \{1\}$. Suppose we have shown that the statement holds up to $i\ge0$. Let $\varphi: G\to G$ be an automorphism of groups. We now have the following commutative diagram: 

    \begin{center}
    \begin{tikzcd}
        G\arrow[r,"\varphi"]\arrow[d, two heads, "\pi_i" left]\arrow[rd,dotted,"f" description] & G\arrow[d, two heads, "\pi_i" right]\\
        G/Z_i\arrow[r,dotted, "\exists!~\psi" below] & G/Z_i
    \end{tikzcd}
    \end{center}

    Since $\ker\pi_i\circ\varphi = \varphi^{-1}(\ker\pi_i) = Z_i$, due to the \textbf{universal property} of the quotient, there is a unique homomorphism $\varphi: G/Z_i\to G/Z_i$ such that the above diagram commutes. Define $f = \pi_i\circ\varphi$. Then, $Z_i = \ker f = \pi_i^{-1}(\ker\psi)$, and thus, $\ker\psi = 1$. This implies that $\psi$ is injective. Further, since $\pi_i$ is surjective, so is $f = \pi_i\circ\varphi$, implying that $\psi$ must be surjective. As a result, $\psi$ is an automorphism of groups.

    Let $g\in Z_{i + 1}$, then $\pi_i(\varphi(g)) = \psi(\pi_i(g))$. We know, due to \thref{lem-zi-construct}, that $\pi(g)\in Z(G/Z_i)$ and therefore, $\psi(\pi_i(g))\in Z(G/Z_i)$, consequently $\pi_i(\varphi(g))\in Z(G/Z_i)$ and thus, $\varphi(g)\in Z_{i + 1}$.

    Since we have shown for all automorphisms $\varphi: G\to G$, that $\varphi(Z_{i + 1})\subseteq Z_{i + 1}$, then $\varphi^{-1}(Z_{i + 1})\subseteq Z_{i + 1}$. This immediately gives us that $\varphi(Z_{i + 1}) = Z_{i + 1}$ for all automorphisms $\varphi: G\to G$ and $Z_{i + 1}$ is characteristic.
\end{proof}

\begin{lemma}\thlabel{lem-commutator-upper-containment}
    For all $i\ge 0$, we have $[G,Z_{i + 1}]\subseteq Z_i$.
\end{lemma}
\begin{proof}
    Let $g\in G$ and $x\in Z_{i + 1}$. Let $\pi_i: G\to G/Z_i$ be the natural projection. Then, 
    \begin{equation*}
        \pi_i([g,x]) = [\pi_i(g), \pi_i(x)] = 1
    \end{equation*}
    where the last equality follows from the fact that $\pi_i(x)\in\pi_i(Z_{i + 1}) = Z(G/Z_i)$. This immediately implies that $[g,x]\in Z_i$ and the desired conclusion.
\end{proof}

\begin{lemma}\thlabel{lem-gi-characteristic}
    For all $i\ge 0$, $G_i$ is characteristic in $G$.
\end{lemma}
\begin{proof}
    We shall show this by induction on $i$. The base case with $G_0 = G$ is trivial. Let $\varphi:G\to G$ be an automorphism of groups. Then, for all $g\in G$ and $x\in G_i$, it is not hard to see that $\varphi([g,x]) = [\varphi(g),\varphi(x)]\in [G,G_i] = G_{i + 1}$. Therefore, for all automorphisms $\varphi: G\to G$, $\varphi(G_{i + 1})\subseteq G_{i + 1}$. This implies that $\varphi(G_{i + 1}) = G_{i + 1}$, and completes the induction.
\end{proof}

\begin{lemma}\thlabel{lem-gi-containment-center}
    For all $i\ge 0$, $G_i/G_{i + 1} \subseteq Z(G/G_{i + 1})$.
\end{lemma}
\begin{proof}
    Let $\pi_{i + 1}: G\to G/G_{i + 1}$ denote the natural projection. Let $x\in G_i$ and $g\in G$, then 
    \begin{equation*}
        1 = \pi_{i + 1}([x, g]) = [\pi_{i + 1}(x), \pi_{i + 1}(g)]
    \end{equation*}
    since $\pi_{i + 1}$ is surjective, $\pi_{i + 1}(x)\in Z(G/G_{i + 1})$. This completes the proof.
\end{proof}

\begin{theorem}\thlabel{thm-equivalence-nilpotent}
    For a group $G$, the following are equivalent,
    \begin{enumerate}
        \item For some $n\ge 0$, $Z_n = G$
        \item For some $m\ge 0$, $G_m = 1$
        \item $G$ is nilpotent
    \end{enumerate}
\end{theorem}
\begin{proof}
We shall show that $(1)\Longrightarrow (2)\Longrightarrow (3)\Longrightarrow (1)$, which would imply the desired conclusion.
\begin{itemize}
    \item $\underline{(1)\Longrightarrow(2):}$ We have a finite series
    \begin{equation*}
        1 = Z_0\subseteq Z_1\subseteq\cdots\subseteq Z_n = G
    \end{equation*}
    We shall show, through induction on $i$, that $G_i\subseteq Z_{n - i}$. The base case with $i = 0$ is obviously true. Using \thref{lem-commutator-upper-containment}, we have, for all $i\le n - 1$,
    \begin{equation*}
        G_{i + 1} = [G, G_i]\subseteq [G, Z_{n - i}]\subseteq [G, Z_{n - i - 1}]\subseteq Z_{i + 1}
    \end{equation*}
    which completes the induction. Finally, we have $G_n\subseteq Z_0 = 1$, implying the desired conclusion.

    \item $\underline{(2)\Longrightarrow(3):}$ Simply define $H_i = G_{n - i}$ for all $0\le i\le n$. Due to \thref{lem-gi-containment-center}, we have that $H_{i + 1}/H_i\subseteq Z(G/H_i)$.

    \item $\underline{(3)\Longrightarrow(1):}$ We shall show that for all $i\ge 0$, $H_i\subseteq Z_i$. The base case with $i = 0$ is trivial. Consider the following commutative diagram:
    \begin{center}
    \begin{tikzcd}
        G\arrow[r, two heads, "\pi_i"]\arrow[d, two heads, "\pi_i'" left] & G/Z_i\\
        G/H_i\arrow[ru, dotted, two heads, "\exists!~\phi" below]
    \end{tikzcd}
    \end{center}
    Since $H_i\subseteq Z_i$, using the universal property of the quotient, there is an epimorphism $\phi: G/H_i\to G/Z_i$ such that the above diagram commutes. Let $x\in H_{i + 1}$. Then, $\pi_i'(x)\in Z(G/H_i)$, therefore, for all $g\in G$
    \begin{equation*}
        1 = \phi\left(\pi_i'([g, x])\right) = \pi_i([g,x]) = [\pi_i(g), \pi_i(x)]
    \end{equation*}
    Now, since $\pi_i$ is surjective, $\pi_i(x)\in Z(G/Z_i)$, and thus, $x\in Z_{i + 1}$. This implies the desired conclusion.
\end{itemize}
\end{proof}

\subsection{Related Results for Nilpotent Groups}

\begin{lemma}\thlabel{lem-p-group-nilpotent}
    Every finite $p$-group is nilpotent.
\end{lemma}
\begin{proof}
    Let $G$ be a finite $p$-group. We shall show that the upper central series is finite by showing the proper containment $Z_i\subsetneq Z_{i + 1}$ whenever $Z_i\subsetneq G$ which would imply the desired conclusion. Let $\pi_i: G\to G/Z_i$ denote then natural projection. We know, due to \thref{lem-zi-construct}, that $Z_{i + 1} = \pi_i^{-1}(Z(G/Z_i))$ and since $G/Z_i$ is a non-trivial $p$-group, it must have a non-trivial center, therefore, $Z_i\subsetneq Z_{i + 1}$. This completes the proof.
\end{proof}

\begin{lemma}\thlabel{lem-normalizer-proper-containment}
    Let $G$ be a nilpotent group and $H$, a proper subgroup of $G$. Then, $H\subsetneq N_G(H)$.\\
    \textit{Note that finiteness of $G$ is \underline{NOT} required}.
\end{lemma}
\begin{proof}
    Since $G$ is nilpotent, the upper central series $1 = Z_0\subseteq Z_1\subseteq\cdots\subseteq Z_n = G$ is strictly increasing (with respect to containment). Let $k$ be the maximal index such that $Z_k\subseteq H$, that is to say, $Z_{k + 1}\subsetneq H$. Now, using \thref{lem-commutator-upper-containment},
    \begin{equation*}
        [Z_{k + 1}, H]\subseteq [Z_{k + 1}, G]\subseteq Z_k\subseteq H
    \end{equation*}
    as a result, $Z_{k + 1}\subseteq N_G(H)$ which completes the proof.
\end{proof}

\begin{lemma}\thlabel{lem-nilpotent-sylow}
    Let $G$ be a finite nilpotent group. For every prime $p$ dividing the order of $G$, the $p$-Sylow subgroup $P$ is normal and therefore unique.
\end{lemma}
\begin{proof}
    Recall from the study of Sylow subgroups that $N_G(N_G(P)) = N_G(P)$. This combined with \thref{lem-normalizer-proper-containment} implies that $N_G(P) = G$, and $P$ is normal in $G$ which immediately implies uniqueness.
\end{proof}

\begin{lemma}\thlabel{lem-product-nilpotent}
    Let $G_1,\ldots,G_n$ be nilpotent groups. Then, their direct product $G_1\times\cdots\times G_n$ is also nilpotent.
\end{lemma}
\begin{proof}
    The central series of the product is the pointwise product of the individual central series.
\end{proof}

\begin{theorem}\thlabel{thm-nilpotent-decomposition}
    A finite group is nilpotent if and only if it is a direct product of $p$-groups.
\end{theorem}
\begin{proof}
    Suppose $G$ is a finite nilpotent group, then due to \thref{lem-nilpotent-sylow}, the Sylow subgroups of $G$ are normal and it is well known that in this case, $G$ is the direct product of the Sylow subgroups.

    Conversely, if $G$ is the direct product of $p$-groups, then using \thref{lem-product-nilpotent} and \thref{lem-p-group-nilpotent}, we have that $G$ is nilpotent.
\end{proof}

\begin{proposition}\thlabel{prop:normalizer-strict-implies-nilpotent}
    Let $G$ be a finite group. If $H\subsetneq N_G(H)$ for every proper subgroup $H$ of $G$, then $G$ is nilpotent.
\end{proposition}
\begin{proof}
    Let $P$ be a Sylow subgroup of $G$. Since $N_G(P) = N_G(N_G(P))$, we must have that $N_G(P) = G$, consequently, $P$ is normal in $G$. It follows that $G$ is a (internal) direct product of its Sylow subgroups, i.e., a direct product of $p$-groups, each of which is nilpotent. Hence, $G$ is nilpotent.
\end{proof}

\begin{theorem}\thlabel{thm-nilpotent-subgroup-quotient}
    Every subgroup and quotient of a nilpotent group is nilpotent.
\end{theorem}
\begin{proof}
    Let $G$ be a nilpotent group and $H$ a subgroup of $G$. Let $H_0\supseteq H_1\supseteq\cdots$ be the lower central series of $H$. We shall show by induction on $i$, that $H_i\subseteq G_i$. The base case with $i = 0$ is trivial. We now have 
    \begin{equation*}
        H_{i + 1} = [H, H_i]\subseteq [G, H_i]\subseteq [G, G_i] = G_{i + 1}
    \end{equation*}
    this completes the induction. Finally, since the lower central series of $G$ is finite, the lower central series of $H$ must be finite too, implying that $H$ is nilpotent.

    On the other hand, let $N$ be a normal subgroup of $G$ and $G' = G/N$. Let $\pi: G\to G'$ denote the natural projection. We shall show by induction on $i$ that $G_i' = \pi(G_i)$. The base case with $i = 0$ is trivial. We have 
    \begin{equation*}
        G_{i + 1}' = [G', G_i'] = \pi([G, G_i]) = \pi(G_{i + 1})
    \end{equation*}
    This completes the induction and implies that the lower central series of $G'$ is finite.
\end{proof}

\begin{lemma}\thlabel{lem-nilpotent-iff-quotient}
    A group $G$ is nilpotent if and only if $G/Z(G)$ is nilpotent.
\end{lemma}
\begin{proof}
    One direction of the statement is trivial due to \thref{thm-nilpotent-subgroup-quotient}. Now suppose $\widetilde{G} = G/Z(G)$ is nilpotent and let $\pi: G\to G/Z(G)$ denote the natural projection. Let $\widetilde{G} = \widetilde{G}_0\supseteq\widetilde{G}_1\supseteq\cdots\supseteq\widetilde{G}_n = 1$ denote the lower central series of $\widetilde{G}$. We shall show by induction on $i$ that $G_i\subseteq\pi^{-1}(\widetilde{G}_i)$. We have 
    \begin{equation*}
        \pi(G_{i + 1}) = \pi([G, G_{i}]) = [\pi(G), \pi(G_i)]\subseteq[\widetilde{G},\widetilde{G}_i] = \widetilde{G}_{i + 1}
    \end{equation*}
    This completes the induction and implies the desired conclusion.
\end{proof}

\begin{lemma}\thlabel{lem:normal-in-nilpotent}
    Let $G$ be a nilpotent group and $N$ a non-trivial normal subgroup of $G$. Then, $Z(G)\cap N$ is non-trivial.
\end{lemma}
\begin{proof}
    Let $1 = Z_0\subseteq Z_1\subseteq\cdots\subseteq Z_n = G$ denote the upper central series of $G$. Let $k$ be the unique index such that $Z_k\cap N = 1$ while $Z_{k + 1}\cap N\ne 1$. We shall show that $G\cap Z_{k + 1}\subseteq Z(G)$. Indeed, we have 
    \begin{equation*}
        [G, N\cap Z_{k + 1}]\subseteq [G, N]\cap[G, Z_{k + 1}]\subseteq N\cap Z_{k} = 1
    \end{equation*}
    where we used that for all normal subgroups $N$, $[G,N]\subseteq N$ and \thref{lem-commutator-upper-containment}. 

    Since $[G, N\cap Z_{k + 1}] = 1$, we must have that $1\ne N\cap Z_{k + 1}\subseteq Z(G)$, which completes the proof.
\end{proof}


\subsection{The Fitting Subgroup}

\begin{definition}
    Let $G$ be a finite group. For every prime $p$, let $\Syl_p(G)$ denote the collection of all Sylow $p$-subgroups of $G$. Define 
    \begin{equation*}
        \bfO(G) = \bigcap_{H\in\Syl_p(G)} H.
    \end{equation*}
\end{definition}

Since all Sylow $p$-subgroups of $G$ are conjugate, $\bfO(G)$ is a normal $p$-subgroup of $G$. For distinct primes $p\ne q$, $\bfO_p(G)\cap\bfO_q(G) = \{1\}$ and hence, $\bfO_p(G)$ commutes with $\bfO_q(G)$.

\begin{proposition}
    $\bfO_p(G)$ contains every normal $p$-subgroup of $G$.
\end{proposition}
\begin{proof}
    Let $P\noreq G$ be a normal $p$-subgroup. It is well-known that there is a Sylow $p$-subgroup of $G$ containing $P$. But since all the Sylow $p$-subgroups of $G$ are conjugate, $P$ must be contained in all of them, and hence, in $\bfO_p(G)$.
\end{proof}

Consider the product map 
\begin{equation*}
    \mu: \prod_{p\mid G}\bfO_p(G)\longrightarrow G,
\end{equation*}
given by $\mu\left((x_p)\right) = \prod x_p$. We contend that this map is injective. Let $H$ be the image of $\mu$. Since each $\bfO_p(G)$ is contained in $H$, their orders must divide the order of $H$. Further, since they are coprime, we have that the order of $H$ is equal to the order of the product $\prod_p\bfO_p(G)$ and hence, the map must be injective.

\begin{definition}
    The image of $\mu$ is denoted by $\bfF(G)$ and is called the \define{Fitting subgroup}.
\end{definition}

\begin{proposition}
    $\bfF(G)$ is a normal nilpotent subgroup of $G$. Further, it contains every nilpotent normal subgroup of $G$.
\end{proposition}
\begin{proof}
    Being a product of normal subgroups, $\bfF(G)$ is normal. It is nilpotent as it is isomorphic to a direct product of $p$-groups, each of which is nilpotent.

    Let $N\noreq G$ be a normal nilpotent subgroup of $G$ and suppose $P\in\Syl_P(N)$. Then, $P$ is normal in $G$. For any $g\in G$, $P^g$ is also contained in $N$ (owing to $N$ being normal in $G$) and has the same cardinality as $P$, i.e. is a Sylow $p$-subgroup of $N$. Consequently, $P = P^g$ and $P$ is normal in $G$, whence $P$ is contained in $\bfO_p(G)\subseteq\bfF(G)$. This shows that all Sylow subgroups of $N$ are contained in $\bfF(G)$. Since $N$ is the product of its Sylow subgroups, we have shown that $N$ is contained in $\bfF(G)$.
\end{proof}

\begin{proposition}
    $\bfF(G)$ is characteristic in $G$.
\end{proposition}
\begin{proof}
    Let $\varphi\in\Aut(G)$. Note that $\varphi(\bfF(G))$ is also nilpotent and normal in $G$. Consequently, it must be contained in $\bfF(G)$, whence the conclusion follows.
\end{proof}

\begin{proposition}\thlabel{prop:fitting-of-normal}
    If $N\noreq G$, then $\bfF(N)\subseteq\bfF(G)$.
\end{proposition}
\begin{proof}
    We know that $\bfF(N)$ is nilpotent and hence, it suffices to show that it is normal in $G$. For any $g\in G$, the map $x\mapsto g^{-1}xg = x^g$ is an automorphism of $N$. Since $\bfF(N)$ is characteristic in $N$, we have that $\bfF(N)^g\subseteq\bfF(N)$, whence the conclusion follows.
\end{proof}

\section{Solvable Groups}
\begin{definition}[Derived Series]\thlabel{def-derived-series}
    Let $G$ be a group. The \textit{derived series} of a group is given by the sequence of subgroups 
    \begin{equation*}
        G = G^{(0)}\supseteq G^{(1)}\supseteq\cdots
    \end{equation*}
    such that $G^{(i + 1)} = [G^{(i)}, G^{(i)}]$.
\end{definition}

\begin{definition}[Solvable Groups]\thlabel{def-solvable-group}
    A group $G$ is said to be solvable if there is $n\ge 0$ and a series $G = H^{(0)}\supseteq H^{(1)}\supseteq\cdots H^{(n)} = 1$ such that for all $0\le i\le n - 1$, each $H^{(i + 1)}$ is normal in $H^{(i)}$ and $H^{(i)}/H^{(i + 1)}$ is Abelian.
\end{definition}

\subsection{Analyzing the Derived Series}
\begin{lemma}\thlabel{lem-gi-characteristic}
    For all $i\ge 0$, $G^{(i)}$ is characteristic in $G$.
\end{lemma}
\begin{proof}
    We shall show this statement by induction on $i$. The base case with $i = 0$ is trivial. Let $\varphi: G\to G$ be an automorphism of groups. Then, 
    \begin{equation*}
        \varphi(G^{(i + 1)}) = \varphi([G^{(i)}, G^{(i)}]) = [\varphi(G^{(i)}), \varphi(G^{(i)})] = G^{(i + 1)}
    \end{equation*}
\end{proof}

\begin{theorem}\thlabel{thm-solvable-equivalence}
    For any group $G$, the following are equivalent
    \begin{enumerate}
        \item There is $n\ge 0$ such that $G^{(n)} = 1$
        \item $G$ is solvable
    \end{enumerate}
\end{theorem}
\begin{proof}
\hfill 
\begin{itemize}
    \item $\underline{(1)\Longrightarrow(2):}$ Simply choose $H^{(i)} = G^{(i)}$.
    \item $\underline{(2)\Longrightarrow(1):}$ We shall show by induction on $i$ that $G^{(i)}\subseteq H^{(i)}$. The base case with $i = 0$ is trivial. Now, for all $0\le i\le n - 1$,
    \begin{equation*}
        G^{(i + 1)} = [G^{(i)}, G^{(i)}]\subseteq [H^{(i)}, H^{(i)}]\subseteq H^{(i + 1)}
    \end{equation*}
    where the last containment follows from the fact that $H^{(i)}/H^{(i + 1)}$ is Abelian. This completes the proof.
\end{itemize}
\end{proof}

\begin{lemma}\thlabel{lem-nilpotent-solvable}
    All nilpotent groups are solvable.
\end{lemma}
\begin{proof}
    Let $G = G_0\supseteq G_1\supseteq\cdots\supseteq G_n = 1$ be the lower central series. We shall show by induction on $i$ that for all $0\le i\le n$, $G^{(i)}\subseteq G_i$. The base case with $i = 0$ is trivial. For $i\ge 0$, we have 
    \begin{equation*}
        G^{(i + 1)} = [G^{(i)}, G^{(i)}] \subseteq [G_i, G_i]\subseteq [G,G_i] = G_{i + 1}
    \end{equation*}
    This completes the induction step and implies the desired conclusion.
\end{proof}

\begin{corollary}\thlabel{cor-p-group-solvable}
    All $p$-groups are solvable.
\end{corollary}

\begin{theorem}\thlabel{lem-solvable-extension}
    Let $1\to N\stackrel{\alpha}{\longrightarrow} G\stackrel{\pi}{\longrightarrow} H\to 1$ be a short exact sequence. Then, $G$ is solvable if and only if both $N$ and $H$ are solvable.
\end{theorem}
\begin{proof}
    Without loss of generality, we may assume $N$ to be a normal subgroup in $G$ and $H$ its corresponding quotient.

    Suppose $G$ is solvable. Then, we can inductively show that $N^{(i)}\subseteq G^{(i)}$, implying the solvability of $N^{(i)}$. On the other hand, $\pi(G^{(i)}) = H^{(i)}$, again implying the solvability of $H$.

    Conversely, suppose both $N$ and $H$ are solvable. Then, $\pi(G^{(n)}) = 1$ for some $n\ge 0$, therefore, $G^{(n)}\subseteq N$. From here, it isn't hard to show that $G^{(n + i)}\subseteq N^{(i)}$, implying the solvability of $G$. This completes the proof.
\end{proof}

\begin{corollary}\thlabel{cor-solvable-subgroup}
    Let $G$ be a solvable group. If $H$ is a subgroup of $G$, then $H$ is solvable.
\end{corollary}

\begin{proposition}\thlabel{prop:minimal-normal-solvable}
    A minimal normal subgroup of a solvable group is an elementary abelian $p$-group.
\end{proposition}

\subsection{Two theorems of P. Hall}

\begin{theorem}[Hall]
    Let $G$ be a solvable group of order $|G| = ab$, where $\gcd(a, b) = 1$.
    \begin{description}
        \item[Existence:] $G$ admits a subgroup of order $a$.
        \item[Conjugacy:] Any two subgroups of order $a$ are conjugate in $G$.
    \end{description}
\end{theorem}
\begin{proof}
    Induct on $|G|$. The base cases where $|G|$ is a prime number are trivially established.

    \noindent\boxed{\textbf{\itshape Case 1.}} $G$ contains a non-trivial normal subgroup $H$ of order $a'b'$, where $a'\mid a$, $b'\mid b$, and $b' < b$.

    \noindent\emph{Existence.} In this case, $G/H$ is a solvable group of order group of order $(a/a')(b/b') < ab$. Due to the induction hypothesis, $G/H$ admits a subgroup $A/H$ of order $a/a'$, where $A$ is a subgroup of $G$ of order $ab' < ab$. Since $A$ is solvable, the induction hypothesis applies to $A$, which then admits a subgroup of order $a$.

    \noindent\emph{Conjugacy.} Let $A$ and $A'$ be subgroups of $G$ of order $a$. Note that $AH$ is a subgroup of $G$ of order 
    \begin{equation*}
        |AH| = \frac{|A| |H|}{|A\cap H|}\le|A|\frac{|H|}{|A\cap H|}.
    \end{equation*}
    Note that $|A\cap H|$ divides $|H| = a'b'$ and since $\gcd(a', b') = 1$ and $|A\cap H|$ divides $|A| = a$, we see that $|H|/|A\cap H|\le b'$. It follows that $|AH|\le ab'$. But, on the other hand, $AH$ contains $A$ and $H$ as subgroups, whence $a\mid |AH|$ and $a'b'\mid |AH|$, whence $ab'\mid |AH|$, that is, $|AH| = ab'$. Similarly, one can argue that $|A'H| = ab'$.

    Now, $|G/H| = a/a'\cdot b/b'$ and $|AH/H| = |A'H/H| = a/a'$. The induction hypothesis applies and these groups are conjugate in $G/H$, whence $AH$ and $A'H$ are conjugate in $G$. That is, there is an $x\in G$ such that $x AH x^{-1} = A'H$. Therefore, $xAx^{-1}$ and $A'$ are subgroups of $A'H$ of order $a$, and since $|A'H| < |G|$, the induction hypothesis applies once again, and $A$ nad $A'$ are conjugate in $G$.\\

    It follows from the first case that if there is a non-trivial proper normal subgroup whose order is not divisible by $b$, then the theorem has been proved. We may therefore assume that $b\mid |H|$ for every non-trivial normal subgroup $H$ of $G$. If $H$ is a minimal normal subgroup of $G$, then due to \thref{prop:minimal-normal-solvable}, $H$ is an elementary abelian $p$-group. It follows that $b = p^m = |H|$ for some $m \ge 1$. Thus, $H$ is a normal (hence unique) Sylow $p$-subgroup of $G$. So we have shown that every minimal normal subgroup of $G$ is the Sylow $p$-subgroup, and hence, $G$ admits a unique minimal normal subgroup. The problem is no reduced to the following:\\

    \noindent\boxed{\textbf{\itshape Case 2.}} $|G| = ap^m$, where $p\nmid a$, and $G$ has a normal abelian Sylow $p$-subgroup $H$, and $H$ is the unique minimal normal subgroup in $G$.

    \noindent\emph{Existence.} The group $G/H$ is solvable of order $a$. If $K/H$ is a minimal normal subgroup of $G/H$, then $|K/H| = q^n$ for some prime $q\ne p$ due to \thref{prop:minimal-normal-solvable}; and so $|K| = p^mq^n$, also note that $K\noreq G$. If $Q$ is a Sylow $q$-subgroup of $K$, then $K = HQ$. Let $N^\ast = N_G(Q)$ and let $N = N^\ast\cap K = N_K(Q)$. Then \thref{thm:frattini-argument} gives $G = KN^\ast$. Since 
    \begin{equation*}
        G/K\cong KN^\ast/K\cong N^\ast/N^\ast\cap K = N^\ast/N,
    \end{equation*}
    we have $|N^\ast| = |G||N|/|K|$. But $K = HQ$, and $Q\subseteq N\subseteq K$ gives $K = HN$, whence $|K| = |HN| = |H||N|/|H\cap N|$, so that 
    \begin{equation*}
        |N^\ast| = \frac{|G||N|}{|K|} = \frac{|G||N||H\cap N|}{|H||N|} = \frac{|G|}{|H|}|H\cap N| = a|H\cap N|.
    \end{equation*}
    We claim that $H\cap N = 1$. We show this in two stages: 
    \begin{itemize}
        \item First, we show that $H\cap N\subseteq Z(K)$. Let $x\in H\cap N$. Every $k\in K$ has the form $k = hs$ for some $h\in H$ and $s\in Q$. Since $H$ is abelian, it suffices to show that $x$ commutes commutes with $s$. Note that the commutator $[x, s]\in Q$, since $x$ normalizes $Q$. On the other hand, $[x, s] = x(sx^{-1}s{^-1})\in H$, because $H$ is normal in $G$. Therefore, $[x, s]\in Q\cap H = 1$. Thus, $H\cap N\subseteq Z(K)$.
        \item Next, we show that $Z(K) = 1$. Since $Z(K)$ is characteristic in $K$ and $K$ is normal in $G$, we have that $Z(K)\noreq G$. If $Z(K)$ were non-trivial, then it would contain a minimal normal subgroup of $G$, i.e., $H$ due to uniqueness. But since $K = HQ$, and $H$ is central in $K$, we see that $Q$ must be normal in $K$. A normal Sylow subgroup is characteristic (owing to its uniqueness), and hence, $Q\noreq G$. Again, this means $H\subseteq Q$, because $Q$ must also contain a minimal normal subgroup of $G$. This is absurd, since $H$ is a $p$-group. Thus, $Z(K) = 1$.
    \end{itemize}
    We have shown that $|N^\ast| = a$, thereby proving existence. 

    \noindent\emph{Conjugacy.} Let $A$ be another subgroup of $G$ of order $a$. Since $|AK|$ is divisible by $a$ and by $|K| = p^mq^n$, it follows that $|AK| = ap^m = |G|$, that is, $AK = G$. Hence, 
    \begin{equation*}
        \frac{G}{K}\cong\frac{AK}{K}\cong\frac{A}{A\cap K},
    \end{equation*}
    so $|A\cap K| = q^n$. From Sylow's theorem, $A\cap K$ is conjugate to $Q$. It follows that $N^\ast = N_G(Q)$ is conjugate to $N_G(A\cap K)$, whence $a = |N_G(A\cap K)|$. Since $A\subseteq N_G(A\cap K)$, we must have $A = N_G(A\cap K)$ and that $A$ is conjugate to $N^\ast$ as desired.
\end{proof}

\section{Subnormality}

\begin{definition}
    Let $G$ be a grouop. A subgroup $S\subseteq G$ is said to be \define{subnormal} in $G$ if there exist subgroups $H_i$ of $G$ such that 
    \begin{equation*}
        S = H_0\noreq H_1\noreq\cdots\noreq H_r = G.
    \end{equation*}
    In this situation, we write $S\nor\nor G$. The smallest integer $r$ for which the above holds is called the \define{subnormal depth} of $S$ in $G$.
\end{definition}

\begin{remark}
    Note that the definition of a subnormal subgroup behaves well with respect to ``contraction''. That is, if $S\subnor G$ and $H$ is any subgroup of $G$, then $S\cap H\subnor H$. As a result, if $S, T\subnor G$, then $S\cap T\subnor G$.

    Now, suppose $\varphi: G\to\overline G$ is a surjective group homomorphism and $S\subnor G$. Then, $\varphi(S)\subnor\overline G$, since the image of a subnormal series under $\varphi$ is still subnormal.
\end{remark}

\begin{lemma}
    Let $G$ be a finite group. Then $G$ is nilpotent if and only if every subgroup of $G$ is subnormal.
\end{lemma}
\begin{proof}
    Suppose $G$ is nilpotent and $H$ is a proper subgroup of $G$. Define $H_0 = H$ and $H_{i + 1} = N_G(H_i)$. Then, either $H_{i + 1} = G$ or $H_i\subsetneq H_{i + 1}$. This gives us a subnormal series for $H$.

    Conversely, suppose every subgroup of $G$ is subnormal and let $H$ be a proper subgroup. There is a sequence 
    \begin{equation*}
        H = H_0\nor H_1\nor\cdots\nor H_n = G.
    \end{equation*}
    In particular, we may assume that $H_i\subsetneq H_{i + 1}$ for $0\le i\le n - 1$. Hence, $H\subsetneq H_1\subseteq N_G(H)$. Due to \thref{prop:normalizer-strict-implies-nilpotent}, we see that $G$ must be nilpotent.
\end{proof}

\begin{proposition}
    Let $G$ be a finite group and $H\le G$. Then $H\subseteq\bfF(G)$ if and only if $H$ is nilpotent and subnormal in $G$.
\end{proposition}
\begin{proof}
    Since $\bfF(G)$ is nilpotent, if $H$ were contained in $\bfF(G)$, then it would be niloptent too. Further, due to the preceding lemma, $H\subnor G$ and $\bfF(G)\nor G$, whence $H\subnor G$.

    We prove the converse by induction on $|G|$. If $H = G$, then there is nothing to prove, since $G$ would be nilpotent and $\bfF(G) = G$. Suppose now that $H\subsetneq G$. There is a subnormal series
    \begin{equation*}
        H = H_0\nor H_1\nor\cdots\nor H_n = G.
    \end{equation*}
    where every successive containment is proper. Set $M = H_{n - 1}\nor G$. The inductive hypothesis applies since $H$ is nilpotent and subnormal in $M$, consequently, $H\subseteq\bfF(M)\subseteq\bfF(G)$, due to \thref{prop:fitting-of-normal}, thereby completing the proof.
\end{proof}

\begin{definition}
    A \define{minimal normal subgroup} of a group $G$ is a non-identity normal subgroup of $G$ that does not admit any non-trivial normal subgroups. The \define{socle} of a \caution{finite} group $G$ is defined to be the subgroup generated by all minimal normal subgroups of $G$, which is precisely their product.
\end{definition}

If $M$ and $N$ are two minimal normal subgroups of $G$, then $M\cap N = \{1\}$ and hence, every element of $M$ commutes with every element of $N$. Thus, $\Soc(G)$ is precisely the product of all minimal normal subgroups of $G$ and is a normal subgroup of $G$. Further, if $G$ is a finite group that is not trivial, then it admits a non-trivial minimal finite group, and hence, $\Soc(G)$ is non-trivial.

\begin{proposition}
    Let $G$ be a finite group. Then $\Soc(G)$ is characteristic in $G$.
\end{proposition}
\begin{proof}
    Let $\varphi\in\Aut(G)$. For a minimal normal subgroup $M$ of $G$, $\varphi(M)$ is also a minimal normal subgroup of $G$. Consequently, $\varphi$ permutes the minimal normal subgroups of $G$ and thus stabilizes the socle.
\end{proof}

\begin{theorem}
    Let $G$ be a finite group, $S\subnor G$, and $M$ a minimal normal subgroup of $G$. Then $M\subseteq N_G(S)$.
\end{theorem}
\begin{proof}
    Induction on $|G|$. If $S = G$, then there is nothing to prove, so we can suppose that $S\subsetneq G$. Since $S\subnor G$, arguing as in the preceding proof, we can choose a normal subgroup $N\subsetneq G$ such that $S\subnor N\nor G$.

    If $M\cap N = 1$, then every element of $M$ commutes with every element of $N$, and hence, $M\subseteq C_G(N)\subseteq C_G(S)\subseteq N_G(S)$. Suppose now that $M\cap N$ is non-trivial. But since $M$ is a minimal normal subgroup, $M = M\cap N$, i.e. $M\subseteq N$.

    The inductive hypothesis applies to $N$, whence every minimal normal subgroup of $N$ normalizes $S$, consequently, $\Soc(N)$ normalizes $S$. Therefore, it suffices to show that $M\subseteq\Soc(N)$. 

    Since $N$ is a finite group and $M$ is a non-trivial normal subgroup of $N$, it contains a minimal normal subgroup. That is, $M\cap\Soc(N)\ne 1$. Since $\Soc(N)$ is characteristic in $N$, it must be normal in $G$. Owing to the minimality of $M$ in $G$, $M\cap\Soc(N) = M$, that is, $M\subseteq\Soc(N)$ as desired.
\end{proof}

\begin{theorem}[Wielandt]\thlabel{thm:wielandt-join}
    Let $G$ be a finite group and $S,T\subnor G$. Then $\langle S, T\rangle\subnor G$.
\end{theorem}
\begin{proof}
    Induction on $|G|$. Suppose $G$ is non-trivial, choose a minimal normal subgroup $M$ of $G$ and set $\overline G = G/M$. By abuse of notation, we use the ``overbar'' to denote the homomorphism $G\to\overline G$. Note that 
    \begin{equation*}
        \langle\overline S,\overline T\rangle = \overline{\langle S, T\rangle} = \overline{\langle S, T\rangle M},
    \end{equation*}
    since $M$ is the kernel of $G\to\overline G$. The inductive hypothesis applies to $\overline G$ and hence, $\langle\overline S,\overline T\rangle\subnor\overline G$. There is a natural bijection between the subgroups of $G$ containing $M$ and the subgroups of $\overline G$, which preserves normality and hence, subnormality. Therefore, $\langle S, T\rangle M\subnor G$.

    Finally, note that $M\subseteq N_G(S), N_G(T)$ and hence, $M\subseteq N_G(\langle S, T\rangle)$, whence $\langle S, T\rangle\nor\langle S, T\rangle M\subnor G$, whence the conclusion follows.
\end{proof}

\begin{lemma}\thlabel{lem:product-conjugate}
    Let $G$ be a group and $H\le G$. If $HH^x = G$ for some $x\in G$, then $H = G$.
\end{lemma}
\begin{proof}
    Write $x = uv$, where $u\in H$ and $v\in H^x$. Then $xv^{-1} = u$ and we have 
    \begin{equation*}
        H^x = \left(H^x\right)^{v^{-1}} = H^{uv^{-1}} = H^u = H.
    \end{equation*}
    Then $G = HH^x = HH = H$, as desired.
\end{proof}

% \begin{proposition}
%     Let $G$ be a finite group and $S\le G$. If $SS^x = S^xS$ for all $x\in G$, then $S\subnor G$.
% \end{proposition}
% \begin{proof}
    
% \end{proof}

\begin{theorem}[Wielandt Zipper Lemma]
    Let $G$ be a finite group and $S\le G$ such that $S\subnor H$ for every proper subgroup $H$ of $G$ containing $S$. If $S$ is not subnormal in $G$, then there is a unique maximal subgroup of $G$ containing $S$.
\end{theorem}
\begin{proof}
    We induct on $|G : S|$. Since $S$ is not normal, $N_G(S)\subsetneq G$, and thus $N_G(S)\subseteq M$ for some maximal subgroup $M$ of $G$. We must show that this $M$ is unique. Suppose that $S\subseteq K$ is another maximal subgroup of $G$. We shall show that $K = M$.

    By our hypothesis, $S\subnor K$. Suppose first that $S\noreq K$. Then $K\subseteq N_G(S)\subseteq M$ and hence due to maximality, $K = M$, as desired. We can suppose, therefore, that $S$ is not normal in $K$. Choose the shortest subnormal series 
    \begin{equation*}
        S = H_0\nor H_1\nor\cdots\nor H_r = K,
    \end{equation*}
    where $r\ge 2$, since $S$ is not normal in $K$. Also, $S$ is not normal in $H_2$ since otherwise we could delete $H_1$ to obtain a shorter subnormal series. Let $x\in H_2$ be such that $S^x\ne S$, and write $T = \langle S, S^x\rangle\supsetneq S$. Note that $T\subseteq K$. Also, $S^x\subseteq H_1^x = H_1\subseteq N_G(S)$, and thus, $T\subseteq N_G(S)\subseteq M$. Furthermore, we have that $S\nor T\subsetneq G$.

    Note that $S^x$ also satisfies the hypothesis of the theorem because conjugation by $x$ is an automorphism of $G$. We claim that the subgroup $T = \langle S, S^x\rangle$ also satisfies the same hypothesis. In particular, we need to show that if $T\subseteq H\subsetneq G$, then $T\subnor H$ and $T$ is not subnormal in $G$.

    First, if $T\subseteq H\subsetneq G$, then $S\subseteq H$, and thus $S\subnor H$, and similarly, $S^x\subnor H$, consequently, due to \thref{thm:wielandt-join}, $T\subnor  H$. Also, $S\nor T$ and so if $T\subnor G$, then it would follow that $S\subnor G$, a contradiction. Thus $T$ is not subnormal in $G$.

    Our inductivev hypothesis applies to $T$ since it properly contains $S$, and hence $T$ is contained in a unique maximal subgroup of $G$. But since $T\subseteq M$ and $T\subseteq K$, we have that $M = K$, as desired.
\end{proof}

\begin{definition}
    For a subgroup $H$ of a group $G$, let $H^G$ denote the smallest normal subgroup of $G$ containing $H$. This is known as the \define{normal closure} of $H$ in $G$.
\end{definition}

\begin{theorem}[Baer]\thlabel{thm:baer}
    Let $G$ be a finite group and $H\le G$. Then $H\subseteq\bfF(G)$ if and only if $\langle H, H^x\rangle$ is nilpotent for all $x\in G$.
\end{theorem}
\begin{proof}
    If $H\subseteq\bfF(G)$, then $H^x\subseteq\bfF(G)$ for every $x\in G$, since $\bfF(G)\noreq G$. Hence, $\langle H, H^x\rangle\subseteq\bfF(G)$. But since $\bfF(G)$ is nilpotent, so is $\langle H, H^x\rangle$.

    Conversely, suppose $\langle H, H^x\rangle$ is nilpotent for every $x\in G$. We induct on $|G|$. Taking $x = 1$, we see that $H$ is nilpotent, whence it suffices to prove that $H\subnor G$. 

    Suppose $H$ is not subnormal in $G$. For any proper subgroup $K$ of $G$ containing $H$, the induction hypothesis applies to $K$ and hence, $H\subseteq\bfF(K)$, that is, $H\subnor K$. Due to Wielandt's Zipper Lemma, there is a unique maximal subgroup $M$ of $G$ containing $H$.

    If $\langle H, H^x\rangle = G$, then $G$ is nilpotent and $\bfF(G) = G$, and $H\subnor G$, a contradiction. Thus, $\langle H, H^x\rangle\subsetneq G$ for all $x\in G$. This subgroup must be contained in a maximal subgroup of $G$; but since it contains $H$, and there is a unique maximal subgroup $M$ containing $H$, we conclude that $H^x\subseteq M$ for all $x\in G$. Therefore, $H^G\subseteq M\subsetneq G$. 

    Since $H^G$ is normal and properly contained in $G$, the induction hypothesis applies and $H\subnor H^G\nor G$, that is, $H\subnor G$, a contradiction. This completes the proof.
\end{proof}

\begin{theorem}[Zenkov]\thlabel{thm:zenkov}
    Let $G$ be a finite group and $A, B\le G$ be abelian subgroups. If $M$ is a minimal element in the set 
    \begin{equation*}
        \left\{A\cap B^g\colon g\in G\right\},
    \end{equation*}
    then $M\subseteq\bfF(G)$.
\end{theorem}
\begin{proof}
    The set $\{A\cap B^g\colon g\in G\}$ remains unchanged upon replacing $B$ with $B^g$. Therefore, we may assume that $M = A\cap B$. We prove the statement by induction on $|G|$. First, suppose that $G = \langle A, B^g\rangle$ for some $g\in G$. Since $A$ and $B^g$ are abelian, we have $A\cap B^g\subseteq Z(G)$, and hence, 
    \begin{equation*}
        A\cap B^g = \left(A\cap B^g\right)^{g^{-1}} = A^{g^{-1}}\cap B\subseteq B.
    \end{equation*}
    It follows that $A\cap B^g\subseteq A\cap B\subseteq M$, and by the minimality of $M$, we have $M = A\cap B^g\subseteq Z(G)\subseteq F(G)$, as desired.

    Next, assume that $\langle A, B^g\rangle\subsetneq G$ for all $g\in G$. To show that $M$ is contained in $\bfF(G)$, it suffices to show that every Sylow $p$-subgroup $P$ of $M$ is contained in $\bfF(G)$ (because every group is generated by its Sylow subgroups). Due to \thref{thm:baer}, it suffices to show that $\langle P, P^g\rangle$ is nilpotent for every $g\in G$.

    Fix $g\in G$, and let $H = \langle A, B^g\rangle\subsetneq G$, and $C = B\cap H$. For $h\in H$, we have 
    \begin{equation*}
        A\cap C^h = A\cap\left(B\cap H\right)^h = A\cap B^h\cap H = A\cap B^h.
    \end{equation*}
    In particular, $M = A\cap B = A\cap B\cap H = A\cap C$ is minimal in the set $\{A\cap C^h\colon h\in H\}$ since its minimal in the larger set $\{A\cap B^g\colon g\in G\}$. By the inductive hypothesis, $P\subseteq M\subseteq\bfF(H)$, and hence, $P\subseteq\bfO_p(H)$, since $\bfO_p(H)$ is the unique Sylow $p$-subgroup of $\bfF(H)$. Also, $P^g\subseteq B^g\subseteq H$, and since $\bfO_p(H)$ is a normal subgroup, we have that $\bfO_p(H)P^g$ is a $p$-group containing $\langle P, P^g\rangle$. In particular, $\langle P, P^g\rangle$ is a $p$-group, whence is nilpotent, as desired.
\end{proof}

\begin{corollary}\thlabel{cor:for-luccini}
    Let $A$ be an abelian subgroup of a non-trivial finite group $G$, and suppose that $|A|\ge |G : A|$. Then $A\cap\bfF(G)$ is non-trivial.
\end{corollary}
\begin{proof}
    If $A = G$, then there is nothing to prove. Suppose now that $A\subsetneq G$. If $g\in G$, then $|A||A^g| = |A|^2\ge |A||G : A| = |G|$. Further, due to \thref{lem:product-conjugate}, $AA^g\subsetneq G$. Hence, 
    \begin{equation*}
        |G| > |AA^g| = \frac{|A||A|^g}{|A\cap A^g|}\ge\frac{|G|}{|A\cap A^g|},
    \end{equation*}
    and thus $A\cap A^g$ is non-trivial. Since this holds for all $g\in G$, we can apply \thref{thm:zenkov} to deduce that there is a $g\in G$ such that $A\cap A^g\subseteq\bfF(G)$, whence $A\cap\bfF(G)$ is non-trivial.
\end{proof}

\begin{theorem}[Luccini]\thlabel{thm:luccini}
    Let $A$ be a proper cyclic subgroup of a finite group $G$, and let $K = \core_G(A)$. Then $|A : K| < |G : A|$, and in particular, if $|A|\ge |G : A|$, then $K$ is non-trivial.
\end{theorem}
\begin{proof}
    Induction on $|G|$. Note that $A/K$ is a proper cyclic subgroup of $G/K$ and the core of $A/K$ in $G/K$ is trivial. If $K$ is non-trivial, then the inductive hypothesis applies and we deduce that 
    \begin{equation*}
        |A/K| = \left|A/K : \core_{G/K}(A/K)\right| < |G/K : A/K| = |G : A|.
    \end{equation*}

    We may now assume  that $K = 1$, and we shall show that $|A| < |G : A|$. Suppose not, that is, $|A|\ge |G : A|$. Due to \thref{cor:for-luccini}, $A\cap\bfF(G)$ is non-trivial. In particular, $\bfF(G)$ is non-trivial, so we can choose a minimal normal subgroup $E$ of $G$ with $E\subseteq\bfF(G)$ (since $\bfF(G)$ is normal in $G$). Due to \thref{lem:normal-in-nilpotent}, $E\cap Z(\bfF(G))$ is non-trivial; but since $Z(\bfF(G))$ is characteristic in $\bfF(G)$, it is normal in $G$. Due to the minimality of $E$, we must have $E\subseteq Z(\bfF(G))$, in particular, $E$ is abelian. Being abelian, every Sylow subgroup of $E$ is characteristic in $G$, whence due to minimality, $E$ itself must be a $p$-group. We contend that $E$ is an elementary abelian $p$-group. Indeed, consider $\wt E = \{x^p\colon x\in E\}$, which is proper and characteristic in $E$, and hence, is normal in $G$. Due to minimality of $E$, $\wt E = 1$, as desired.

    Since $E\subseteq Z(\bfF(G))$, we see that $E$ normalizes the non-trivial group $A\cap\bfF(G)$, and of course $A$ normalizes this too. Then $A\cap\bfF(G)\noreq AE$. Since $\core_G(A) = 1$, we cannot have $AE = G$, else $A\cap\bfF(G)$ would be contained in the core. It follows that $AE\subseteq G$.

    Set $\overline G = G/E$, $\overline A = AE/E\subsetneq\overline G$, $\overline{M} = \core_{\overline{G}}(\overline A)$, with $E\subseteq M$ and $M\noreq G$. Note that $M\subseteq AE$, and hence, $AE\subseteq AM\subseteq AE$, whence $AM = AE$. Due to the inductive hypothesis, we must have $|\overline A : \overline M| < |\overline G : \overline A|$, that is, $|AE : M| < |G : AE|$.

    \begin{equation*}
        \xymatrix {
            G\ar@{-}[dd]\ar@{-}[rd] & & & \\
            & AE\ar@{-}[rd]\ar@{-}[ld] & & \\
            A\ar@{-}[rd] & & M\ar@{-}[rd]\ar@{-}[ld] & \\
            & B\ar@{-}[rd] & & E\ar@{-}[ld] \\
            & & B\cap E & 
        }
    \end{equation*}

    Let $B = A\cap M$ so that $B$ is cyclic. We have 
    \begin{equation*}
        |AE : A| = |AM : A| = |M : A\cap M| = |M : B|,
    \end{equation*}
    and hence, $|AE : M| = |A : B|$. Therefore, 
    \begin{equation*}
        |M : B| = |AE : A| = \frac{|G : A|}{|G : AE|} < \frac{|G : A|}{|AE : M|} = \frac{|G : A|}{|A : B|}\le\frac{|A|}{|A : B|} = |B|.
    \end{equation*}
    Before we proceed, note that $E\subseteq M\subseteq AE = EA$, and hence, because of what's colloquially known as Dedekind's rule, $M = E(A\cap M) = EB = BE$ (since $E\noreq G$).

    Suppose $M$ is abelian, and let $\varphi: M\to M$ be the endomorphism $\varphi(m) = m^p$. Then $E\subseteq\ker\varphi$ since it is an elementary abelian $p$-group. It follows that 
    \begin{equation*}
        \varphi(M) = \varphi(EB) = \varphi(B)\subseteq B\subseteq A.
    \end{equation*}
    Now, $M\noreq G$, and hence, $\varphi(M)\noreq G$, and we conclude that $\varphi(M) = 1$, since $\core_G(A) = 1$. Then $\varphi(B) = 1$, and since $B$ is cyclic, it follows that $|B|\le p$. Then $|M : B| < |B|\le p$, and since $M/B\cong E/B\cap E$\footnote{These quotients make sense because $M$ is abelian.}, it is a $p$-group, it follows that $M/B = 1$, that is, $M = B\subseteq A$. But $M\noreq G$, and since $M\subseteq A$, we have $M = 1$, whence $E = 1$, a contradiction.

    It follows that $M$ is non-abelian, and since $M/E\cong B/B\cap E$ is cyclic, we conclude that $E$ is not central in $M$\footnote{Recall that if $G/Z(G)$ is cyclic, then $G$ is abelian.}, and so $E\cap Z(M)\subsetneq E$. Again recall that $Z(M)$ is characteristic in $M$ and hence normal in $G$. Due to the minimality of $E$, we must have $E\cap Z(M) = 1$, and thus $Z(M)$ is cyclic because the restriction of the surjection $M\onto M/E$ is injective on $Z(M)$.

    Since $B$ is an abelian subgroup of $M$ and $|M : B| < |B|$, due to \thref{cor:for-luccini}, we have that $B\cap\bfF(M)$ is non-trivial. Due to \thref{prop:fitting-of-normal}, $\bfF(M)\subseteq\bfF(G)$, and so $E$ centralizes $\bfF(M)$ because $E\subseteq Z(\bfF(G))$. Since every element of $B\cap \bfF(M)$ commutes with every element of $B$ (since $B$ is abelian) and every element of $E$, we see that $B\cap \bfF(M)$ is a non-trivial central subgroup of $EB = M$. Since $Z(M)$ is cyclic, we see that $B\cap\bfF(M)\subseteq Z(M)$ is characteristic in $Z(M)\noreq G$\footnote{Every subgroup of a cyclic group is characteristic.}, and hence, $B\cap\bfF(M)$ is a non-trivial normal subgroup of $G$ contained in $A$, a contradiction. This completes the proof.
\end{proof}

\begin{theorem}[Horosevskii]
    Let $\sigma\in\Aut(G)$, where $G$ is a non-trivial finite group. Then the order $o(\sigma)$ of $\sigma$ as an element of $\Aut(G)$ is strictly smaller than $|G|$.
\end{theorem}
\begin{proof}
    Let $A = \langle\sigma\rangle\subseteq\Aut(G)$, so that $A$ is a cyclic group of order equal to the order of $\sigma$ as an element of $\Aut(G)$. Set $\Gamma = G\rtimes_\theta A$, where $\theta: A\to\Aut(G)$ is the obvious inclusion map. We identify $G$ and $A$ with subgroups $G\times\{1\}$ and $\{1\}\times A$ of $\Gamma$. Note that the conjugation action of $A$ on $G$ as elements of $\Gamma$ is given by $g^\tau = \tau(g)\in G$ for $\tau\in A$. By definition of an automorphism, every non-identity element of $A$ acts non-trivially on $G$, and hence, $A\cap C_\Gamma(G) = 1$.

    Since $G$ is non-trivial and $A$ is cyclic, due to \thref{thm:luccini}, $|A : K| < |\Gamma : A|$, where $K = \core_\Gamma(A)$. But then $K\cap G\subseteq A\cap G = 1$, and both $K$ and $G$ are normal in $\Gamma$, consequently, their elements commute, that is, $K\subseteq C_\Gamma(G)$. Since $K\subseteq A$, we see that $K\subseteq A\cap C_\Gamma(G) = 1$, that is, $K$ is trivial. Thus,
    \begin{equation*}
        o(\sigma) = |A| = |A : K| < |\Gamma : A| = G,
    \end{equation*}
    as desired.
\end{proof}

\end{document}