\documentclass[12pt]{article}

% \usepackage{./arxiv}

\title{Theorems of Levi and Ado-Iwasawa}
\author{Swayam Chube}
\date{\today}

\usepackage[utf8]{inputenc} % allow utf-8 input
\usepackage[T1]{fontenc}    % use 8-bit T1 fonts
\usepackage{hyperref}       % hyperlinks
\usepackage{url}            % simple URL typesetting
\usepackage{booktabs}       % professional-quality tables
\usepackage{amsfonts}       % blackboard math symbols
\usepackage{nicefrac}       % compact symbols for 1/2, etc.
\usepackage{microtype}      % microtypography
\usepackage{graphicx}
\usepackage{natbib}
\usepackage{doi}
\usepackage{amssymb}
\usepackage{bbm}
\usepackage{amsthm}
\usepackage{amsmath}
\usepackage{xcolor}
\usepackage{theoremref}
\usepackage{enumitem}
\usepackage{mathpazo}
% \usepackage{euler}
\usepackage{mathrsfs}
\usepackage{todonotes}
\usepackage{stmaryrd}
\usepackage[all,cmtip]{xy} % For diagrams, praise the Freyd–Mitchell theorem 
\usepackage{marvosym}
\usepackage{geometry}
\usepackage{titlesec}

\renewcommand{\qedsymbol}{$\blacksquare$}

% Uncomment to override  the `A preprint' in the header
% \renewcommand{\headeright}{}
% \renewcommand{\undertitle}{}
% \renewcommand{\shorttitle}{}

\hypersetup{
    pdfauthor={Lots of People},
    colorlinks=true,
}

\newtheoremstyle{thmstyle}%               % Name
  {}%                                     % Space above
  {}%                                     % Space below
  {}%                             % Body font
  {}%                                     % Indent amount
  {\bfseries\scshape}%                            % Theorem head font
  {.}%                                    % Punctuation after theorem head
  { }%                                    % Space after theorem head, ' ', or \newline
  {\thmname{#1}\thmnumber{ #2}\thmnote{ (#3)}}%                                     % Theorem head spec (can be left empty, meaning `normal')

\newtheoremstyle{defstyle}%               % Name
  {}%                                     % Space above
  {}%                                     % Space below
  {}%                                     % Body font
  {}%                                     % Indent amount
  {\bfseries\scshape}%                            % Theorem head font
  {.}%                                    % Punctuation after theorem head
  { }%                                    % Space after theorem head, ' ', or \newline
  {\thmname{#1}\thmnumber{ #2}\thmnote{ (#3)}}%                                     % Theorem head spec (can be left empty, meaning `normal')

\theoremstyle{thmstyle}
\newtheorem{theorem}{Theorem}[section]
\newtheorem{lemma}[theorem]{Lemma}
\newtheorem{proposition}[theorem]{Proposition}

\theoremstyle{defstyle}
\newtheorem{definition}[theorem]{Definition}
\newtheorem*{corollary}{Corollary}
\newtheorem{remark}[theorem]{Remark}
\newtheorem{example}[theorem]{Example}
\newtheorem*{notation}{Notation}

% Common Algebraic Structures
\newcommand{\R}{\mathbb{R}}
\newcommand{\Q}{\mathbb{Q}}
\newcommand{\Z}{\mathbb{Z}}
\newcommand{\N}{\mathbb{N}}
\newcommand{\bbC}{\mathbb{C}} 
\newcommand{\K}{\mathbb{K}} % Base field which is either \R or \bbC
\newcommand{\calA}{\mathcal{A}} % Banach Algebras
\newcommand{\calB}{\mathcal{B}} % Banach Algebras
\newcommand{\calI}{\mathcal{I}} % ideal in a Banach algebra
\newcommand{\calJ}{\mathcal{J}} % ideal in a Banach algebra
\newcommand{\frakM}{\mathfrak{M}} % sigma-algebra
\newcommand{\calO}{\mathcal{O}} % Ring of integers
\newcommand{\bbA}{\mathbb{A}} % Adele (or ring thereof)
\newcommand{\bbI}{\mathbb{I}} % Idele (or group thereof)

% Categories
\newcommand{\catTopp}{\mathbf{Top}_*}
\newcommand{\catGrp}{\mathbf{Grp}}
\newcommand{\catTopGrp}{\mathbf{TopGrp}}
\newcommand{\catSet}{\mathbf{Set}}
\newcommand{\catTop}{\mathbf{Top}}
\newcommand{\catRing}{\mathbf{Ring}}
\newcommand{\catCRing}{\mathbf{CRing}} % comm. rings
\newcommand{\catMod}{\mathbf{Mod}}
\newcommand{\catMon}{\mathbf{Mon}}
\newcommand{\catMan}{\mathbf{Man}} % manifolds
\newcommand{\catDiff}{\mathbf{Diff}} % smooth manifolds
\newcommand{\catAlg}{\mathbf{Alg}}
\newcommand{\catRep}{\mathbf{Rep}} % representations 
\newcommand{\catVec}{\mathbf{Vec}}

% Group and Representation Theory
\newcommand{\chr}{\operatorname{char}}
\newcommand{\Aut}{\operatorname{Aut}}
\newcommand{\GL}{\operatorname{GL}}
\newcommand{\im}{\operatorname{im}}
\newcommand{\tr}{\operatorname{tr}}
\newcommand{\id}{\mathbf{id}}
\newcommand{\cl}{\mathbf{cl}}
\newcommand{\Gal}{\operatorname{Gal}}
\newcommand{\Tr}{\operatorname{Tr}}
\newcommand{\sgn}{\operatorname{sgn}}
\newcommand{\Sym}{\operatorname{Sym}}
\newcommand{\Alt}{\operatorname{Alt}}

% Commutative and Homological Algebra
\newcommand{\spec}{\operatorname{spec}}
\newcommand{\mspec}{\operatorname{m-spec}}
\newcommand{\Tor}{\operatorname{Tor}}
\newcommand{\tor}{\operatorname{tor}}
\newcommand{\Ann}{\operatorname{Ann}}
\newcommand{\Supp}{\operatorname{Supp}}
\newcommand{\Hom}{\operatorname{Hom}}
\newcommand{\End}{\operatorname{End}}
\newcommand{\coker}{\operatorname{coker}}
\newcommand{\limit}{\varprojlim}
\newcommand{\colimit}{%
  \mathop{\mathpalette\colimit@{\rightarrowfill@\textstyle}}\nmlimits@
}
\makeatother


\newcommand{\fraka}{\mathfrak{a}} % ideal
\newcommand{\frakb}{\mathfrak{b}} % ideal
\newcommand{\frakc}{\mathfrak{c}} % ideal
\newcommand{\frakf}{\mathfrak{f}} % face map
\newcommand{\frakg}{\mathfrak{g}}
\newcommand{\frakh}{\mathfrak{h}}
\newcommand{\frakm}{\mathfrak{m}} % maximal ideal
\newcommand{\frakn}{\mathfrak{n}} % naximal ideal
\newcommand{\frakp}{\mathfrak{p}} % prime ideal
\newcommand{\frakq}{\mathfrak{q}} % qrime ideal
\newcommand{\fraks}{\mathfrak{s}}
\newcommand{\frakt}{\mathfrak{t}}
\newcommand{\frakz}{\mathfrak{z}}
\newcommand{\frakA}{\mathfrak{A}}
\newcommand{\frakI}{\mathfrak{I}}
\newcommand{\frakJ}{\mathfrak{J}}
\newcommand{\frakK}{\mathfrak{K}}
\newcommand{\frakL}{\mathfrak{L}}
\newcommand{\frakN}{\mathfrak{N}} % nilradical 
\newcommand{\frakO}{\mathfrak{O}} % dedekind domain
\newcommand{\frakP}{\mathfrak{P}} % Prime ideal above
\newcommand{\frakQ}{\mathfrak{Q}} % Qrime ideal above 
\newcommand{\frakR}{\mathfrak{R}} % jacobson radical
\newcommand{\frakU}{\mathfrak{U}}
\newcommand{\frakX}{\mathfrak{X}}

% General/Differential/Algebraic Topology 
\newcommand{\scrA}{\mathscr A}
\newcommand{\scrB}{\mathscr B}
\newcommand{\scrF}{\mathscr F}
\newcommand{\scrN}{\mathscr N}
\newcommand{\scrP}{\mathscr P}
\newcommand{\scrR}{\mathscr R}
\newcommand{\scrS}{\mathscr S}
\newcommand{\bbH}{\mathbb H}
\newcommand{\Int}{\operatorname{Int}}
\newcommand{\psimeq}{\simeq_p}
\newcommand{\wt}[1]{\widetilde{#1}}
\newcommand{\RP}{\mathbb{R}\text{P}}
\newcommand{\CP}{\mathbb{C}\text{P}}

% Miscellaneous
\newcommand{\wh}[1]{\widehat{#1}}
\newcommand{\calM}{\mathcal{M}}
\newcommand{\calP}{\mathcal{P}}
\newcommand{\onto}{\twoheadrightarrow}
\newcommand{\into}{\hookrightarrow}
\newcommand{\Gr}{\operatorname{Gr}}
\newcommand{\Span}{\operatorname{Span}}
\newcommand{\ev}{\operatorname{ev}}
\newcommand{\weakto}{\stackrel{w}{\longrightarrow}}

\newcommand{\define}[1]{\textcolor{blue}{\textit{#1}}}
\newcommand{\caution}[1]{\textcolor{red}{\textit{#1}}}
\renewcommand{\mod}{~\mathrm{mod}~}
\renewcommand{\le}{\leqslant}
\renewcommand{\leq}{\leqslant}
\renewcommand{\ge}{\geqslant}
\renewcommand{\geq}{\geqslant}
\newcommand{\Res}{\operatorname{Res}}
\newcommand{\floor}[1]{\left\lfloor #1\right\rfloor}
\newcommand{\ceil}[1]{\left\lceil #1\right\rceil}
\newcommand{\gl}{\mathfrak{gl}}
\newcommand{\ad}{\operatorname{ad}}
\newcommand{\Stab}{\operatorname{Stab}}
\newcommand{\bfX}{\mathbf{X}}
\newcommand{\Ind}{\operatorname{Ind}}
\newcommand{\bfG}{\mathbf{G}}
\newcommand{\rank}{\operatorname{rank}}
\newcommand{\calo}{\mathcal{o}}
\newcommand{\frako}{\mathfrak{o}}
\newcommand{\Cl}{\operatorname{Cl}}

\newcommand{\idim}{\operatorname{idim}}
\newcommand{\pdim}{\operatorname{pdim}}
\newcommand{\Ext}{\operatorname{Ext}}
\newcommand{\co}{\operatorname{co}}

\geometry {
    margin = 1in
}

\titleformat
{\section}
[block]
{\Large\bfseries\scshape}
{\S\thesection}
{0.5em}
{\centering}
[]


\titleformat
{\subsection}
[block]
{\normalfont\bfseries\sffamily}
{\S\S}
{0.5em}
{\centering}
[]

\begin{document}
\maketitle
\begin{abstract}
    It is known that every Euclidean Domain (ED) is a Principal Ideal Domain (PID). We present two exammples of PIDs that are not EDs, namely, $\R[X, Y]/(X^2 + Y^2 + 1)$ and $\Z\left[\frac{1 + \sqrt{-19}}{2}\right]$.
\end{abstract}

\section{\texorpdfstring{$\R[X, Y]/(X^2 + Y^2 + 1)$}{}}

We first begin with two important lemmas.

\begin{lemma}\thlabel{lem:prime-principal-pir}
    Let $A$ be a commutative ring in which every prime ideal is principal. Then, $A$ is a principal ring.
\end{lemma}
\begin{proof}
    Suppose not and let $\Sigma$ denote the poset of all proper ideals that are not principal. Let $\mathscr C$ denote a chain in $\Sigma$ and let $\fraka = \bigcup\mathscr C$. If $\fraka = (a)$ is principal, then there is an ideal $\frakb\in\mathscr C$ that contains $a$, consequently, $\frakb = (a)$, a contradiction. Thus, $\fraka\in\Sigma$ and is an upper bound for $\mathscr C$. Due to Zorn's Lemma, $\Sigma$ contains a maximal element, say $\frakp$.

    We contend that $\frakp$ is prime. Suppose not. Then, there are $a,b\notin\frakp$ with $ab\in\frakp$. Note that $(\frakp : b)$ is an ideal properly containing $\frakp$ (since it also contains $a$) and hence, must be principal, say $(c)$. Next, $\frakp + (b)$ properly contains $\frakp$ and hence, must be principal, say $(d)$. Clearly, $\frakp\supseteq(\frakp : b)(\frakp + (b)) = (cd)$. On the other hand, if $x\in\frakp$, then $x = \alpha d$ for some $\alpha\in A$. Since $\alpha d\in\frakp$, we have $\alpha\in (\frakp : b) = (c)$. Thus, $\frakp\subseteq(\frakp : b)(\frakp + (b))$ and $\frakp = (cd)$ is principal, a contradiction. Hence, $\frakp$ is prime, and must be principal, again, a contradiction. This completes the proof.
\end{proof}

\begin{lemma}\thlabel{lem:group-surjection}
    Let $A$ be a Euclidean Domain with Euclidean function $\delta: A\backslash\{0\}\to\N_0$. Then, there is a non-zero prime $p\in A$ such that $\pi: A\twoheadrightarrow A/p$ restricts to a surjective group homomorphism $\pi: A^\times\to (A/p)^\times$.
\end{lemma}
\begin{proof}
    Let $p\in A$ be a non-zero element in $A\backslash A^\times$ that minimizes $\delta$. Then, $p$ must be irreducible, for if $p = ab$ with $a$ non-unit, then 
    \begin{equation*}
        \delta(p) = \delta(ab)\ge\delta(a)\ge\delta(p),
    \end{equation*}
    consequently, $\delta(a) = \delta(ab)$ whence $b$ must be a unit. This shows that $p$ is prime. 

    Now, let $\overline a\in A/p$ be invertible. Then, there is a non-zero $a\in A$ with $\pi(a) = \overline a$. Thus, there are $q$ and $r$ with $a = pq + r$. Since $r\ne 0$, we must have $\delta(r) < \delta(p)$, whereby, $r\in A^\times$. Note that $\pi(r) = \pi(a) = \overline a$ and hence, the restriction of $\pi$ to $A^\times\to (A/p)^\times$ is surjective.
\end{proof}

We are now ready to prove the main of this section. Let $A = R[X, Y]/(X^2 + Y^2 + 1)$ and let $x$ and $y$ denote the images of $X$ and $Y$ in $A$.

\begin{proposition}\thlabel{prop:primes-in-A}
    Every non-zero prime ideal in $A$ is of the form $(ax + by + c)$ where $(a,b)\ne 0$.
\end{proposition}
\begin{proof}
    Let $\frakp$ be a non-zero prime ideal of $A$. Note first that 
    \begin{equation*}
        \dim A = \dim R[X, Y] - \operatorname{ht}((X^2 + Y^2 + 1)) = 1,
    \end{equation*}
    whence $\frakp$ is maximal. Further, $A/\frakp$ is a finitely generated $\R$-algebra and also a field, and due to Zariski's Lemma, must be a finite extension of $\R$. Thus, $[A/\frakp : \R]\le 2$. Let $\overline x, \overline y$ denote the images of $x$ and $y$ in $A/\frakp$. Since $1,\overline x,\overline y$ cannot be linearly independent over $\R$, we must have a non-trivial linear combination $a\overline x + b\overline y + c = 0$ in $A/\frakp$. Hence, $ax + by + c\in\frakp$. If $(a,b) = 0$, then $\frakp$ would contain a unit which is impossible.

    Note that $(aX + bY + c)$ was a maximal ideal in $R[X, Y]$. Hence, $(ax + by + c)$ is a maximal ideal in $A$. Further, the quotient $A/(ax + by + c)$ strictly contains $\R$ and due to Zariski's Lemma, must be a finite extension of it, whence is isomorphic to $\bbC$. This shows that $\frakp = (ax + by + c)$ and $A/\frakp\cong\bbC$ thereby completing the proof.
\end{proof}

\begin{proposition}
    $A$ is a PID but not an ED.
\end{proposition}
\begin{proof}
    Due to \thref{prop:primes-in-A} and \thref{lem:prime-principal-pir}, $A$ is a PID. Suppose $A$ were an ED. According to \thref{lem:group-surjection}, there is a non-zero prime $p\in A$ and a group surjection $\pi: A^\times\to (A/p)^\times$. Note that $A^\times\cong\R^\times$ and $(A/p)^\times\cong\bbC^\times$. But there is no surjective group homomorphism $\R^\times\twoheadrightarrow\bbC^\times$, a contradiction.
\end{proof}

\section{\texorpdfstring{$\Z\left[\frac{1 + \sqrt{-19}}{2}\right]$}{}}
Let $K = \Q[\sqrt{-19}]$ be a number field and let $\mathcal O_K$ denote the ring of integers in $K$. It is well known that $\mathcal O_K = \Z\left[\frac{1 + \sqrt{-19}}{2}\right]$ and that it has class number $1$. Hence, every fractional ideal over $\mathcal O_K$ is principal. In particular, every integral ideal of $\mathcal O_K$ is principal and $\mathcal O_K$ is a PID.

We shall now argue that $\calO_K$ is not an ED. Suppose $\delta:\calO_K\backslash\{0\}\to\N_0$ is a Euclidean function and let $p\in\mathcal O_K$ be a non-zero, non-invertible element that minimizes $\delta$. Consider the canonical projection $\pi: \calO_K\twoheadrightarrow\calO_K/(p)$. 

If $0\ne\overline a\in\calO_K/(p)$, then there is an $a\in\calO_K$ that maps to it under $\pi$. We may write $a = pq + r$ where $q\in\calO_K$, $0\ne r$ and $\delta(r) < \delta(p)$. Due to the minimality of $\delta(p)$, we must have that $r$ is a unit. Note that the only units in $\calO_K$ are $\pm 1$. Indeed, if $x\in\calO_K$ is a unit, then there are integers $m$ and $n$ such that 
\begin{equation*}
    x = m + n\left(\frac{1 + \sqrt{-19}}{2}\right) = \frac{(2m + n) + n\sqrt{-19}}{2}.
\end{equation*}
Since $x$ is a unit, we have $N_{K/\Q}(x) = \pm 1$, that is, 
\begin{equation*}
    (2m + n)^2 + 19 n^2 = 4.
\end{equation*}
It is not hard to see, from the above equation, that the only solutions are $x = \pm 1$.

Hence, $r\in\{\pm 1\}$, in particular, $\calO_K/(p)$ can have atmost $3$ elements and at least $2$ elements. Thus, the \emph{ideal norm} of $(p)$ is either $2$ or $3$. Hence, $N_{K/\Q}(p)\in\{2, 3\}$.

We may suppose $p = m + n\frac{1 + \sqrt{-19}}{2}$. The equation involving norm gives us 
\begin{equation*}
    (2m + n)^2 + 19n^2\in\{8, 12\}.
\end{equation*}
Due to size reasons, $n = 0$. And we are left with $m^2\in\{2,3\}$, which is impossible. Thus, $\calO_K$ cannot be an ED. This completes the proof.
\end{document}