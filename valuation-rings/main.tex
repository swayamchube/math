\documentclass[11pt]{article}

\usepackage[utf8]{inputenc} % allow utf-8 input
\usepackage[T1]{fontenc}    % use 8-bit T1 fonts
\usepackage{hyperref}       % hyperlinks
\usepackage{url}            % simple URL typesetting
\usepackage{booktabs}       % professional-quality tables
\usepackage{amsfonts}       % blackboard math symbols
\usepackage{nicefrac}       % compact symbols for 1/2, etc.
\usepackage{microtype}      % microtypography
\usepackage{graphicx}
\usepackage{natbib}
\usepackage{doi}
\usepackage{amssymb}
\usepackage{bbm}
\usepackage{amsthm}
\usepackage{amsmath}
\usepackage{xcolor}
\usepackage{theoremref}
\usepackage{enumitem}
\usepackage{fouriernc}
\usepackage{mdframed}
\usepackage{mathrsfs}
\setlength{\marginparwidth}{2cm}
\usepackage{todonotes}
\usepackage{stmaryrd}
\usepackage[all,cmtip]{xy} % For diagrams, praise the Freyd-Mitchell theorem 
\usepackage{marvosym}
\usepackage{geometry}
\usepackage{titlesec}
\usepackage{mathtools}
\usepackage{tikz}
\usetikzlibrary{cd}
\usepackage{epigraph}
\setlength\epigraphwidth{0.4\textwidth}

\renewcommand{\qedsymbol}{$\blacksquare$}
% \renewcommand{\familydefault}{\sfdefault} % Do you want this font? 

% Uncomment to override  the `A preprint' in the header
% \renewcommand{\headeright}{}
% \renewcommand{\undertitle}{}
% \renewcommand{\shorttitle}{}

\hypersetup{
    pdfauthor={Swayam Chube},
    colorlinks=true,
	citecolor=blue,
}

\newtheoremstyle{thmstyle}%               % Name
  {}%                                     % Space above
  {}%                                     % Space below
  {}%                             % Body font
  {}%                                     % Indent amount
  {\bfseries\scshape}%                            % Theorem head font
  {.}%                                    % Punctuation after theorem head
  { }%                                    % Space after theorem head, ' ', or \newline
  {\thmname{#1}\thmnumber{ #2}\thmnote{ (#3)}}%                                     % Theorem head spec (can be left empty, meaning `normal')

\newtheoremstyle{defstyle}%               % Name
  {}%                                     % Space above
  {}%                                     % Space below
  {}%                                     % Body font
  {}%                                     % Indent amount
  {\bfseries\scshape}%                            % Theorem head font
  {.}%                                    % Punctuation after theorem head
  { }%                                    % Space after theorem head, ' ', or \newline
  {\thmname{#1}\thmnumber{ #2}\thmnote{ (#3)}}%                                     % Theorem head spec (can be left empty, meaning `normal')

\theoremstyle{thmstyle}
\newtheorem{theorem}{Theorem}[section]
\newtheorem{lemma}[theorem]{Lemma}
\newtheorem{proposition}[theorem]{Proposition}

\theoremstyle{defstyle}
\newtheorem{definition}[theorem]{Definition}
\newtheorem{corollary}[theorem]{Corollary}
\newtheorem{porism}[theorem]{Porism}
\newtheorem{remark}[theorem]{Remark}
\newtheorem{interlude}[theorem]{Interlude}
\newtheorem{example}[theorem]{Example}
\newtheorem*{notation}{Notation}
\newtheorem*{claim}{Claim}

% Common Algebraic Structures
\newcommand{\R}{\mathbb{R}}
\newcommand{\Q}{\mathbb{Q}}
\newcommand{\Z}{\mathbb{Z}}
\newcommand{\N}{\mathbb{N}}
\newcommand{\bbC}{\mathbb{C}} 
\newcommand{\K}{\mathbb{K}} % Base field which is either \R or \bbC
\newcommand{\F}{\mathbb{F}} % Base field which is either \R or \bbC
\newcommand{\calA}{\mathcal{A}} % Banach Algebras
\newcommand{\calB}{\mathcal{B}} % Banach Algebras
\newcommand{\calI}{\mathcal{I}} % ideal in a Banach algebra
\newcommand{\calJ}{\mathcal{J}} % ideal in a Banach algebra
\newcommand{\frakM}{\mathfrak{M}} % sigma-algebra
\newcommand{\calO}{\mathcal{O}} % Ring of integers
\newcommand{\bbA}{\mathbb{A}} % Adele (or ring thereof)
\newcommand{\bbI}{\mathbb{I}} % Idele (or group thereof)

% Categories
\newcommand{\catTopp}{\mathbf{Top}_*}
\newcommand{\catGrp}{\mathbf{Grp}}
\newcommand{\catTopGrp}{\mathbf{TopGrp}}
\newcommand{\catSet}{\mathbf{Set}}
\newcommand{\catTop}{\mathbf{Top}}
\newcommand{\catRing}{\mathbf{Ring}}
\newcommand{\catCRing}{\mathbf{CRing}} % comm. rings
\newcommand{\catMod}{\mathbf{Mod}}
\newcommand{\catMon}{\mathbf{Mon}}
\newcommand{\catMan}{\mathbf{Man}} % manifolds
\newcommand{\catDiff}{\mathbf{Diff}} % smooth manifolds
\newcommand{\catAlg}{\mathbf{Alg}}
\newcommand{\catRep}{\mathbf{Rep}} % representations 
\newcommand{\catVec}{\mathbf{Vec}}

% Group and Representation Theory
\newcommand{\chr}{\operatorname{char}}
\newcommand{\Aut}{\operatorname{Aut}}
\newcommand{\GL}{\operatorname{GL}}
\newcommand{\im}{\operatorname{im}}
\newcommand{\tr}{\operatorname{tr}}
\newcommand{\id}{\mathbf{id}}
\newcommand{\cl}{\mathbf{cl}}
\newcommand{\Gal}{\operatorname{Gal}}
\newcommand{\Tr}{\operatorname{Tr}}
\newcommand{\sgn}{\operatorname{sgn}}
\newcommand{\Sym}{\operatorname{Sym}}
\newcommand{\Alt}{\operatorname{Alt}}

% Commutative and Homological Algebra
\newcommand{\spec}{\operatorname{spec}}
\newcommand{\mspec}{\operatorname{m-spec}}
\newcommand{\Spec}{\operatorname{Spec}}
\newcommand{\MaxSpec}{\operatorname{MaxSpec}}
\newcommand{\Tor}{\operatorname{Tor}}
\newcommand{\tor}{\operatorname{tor}}
\newcommand{\Ann}{\operatorname{Ann}}
\newcommand{\Supp}{\operatorname{Supp}}
\newcommand{\Hom}{\operatorname{Hom}}
\newcommand{\End}{\operatorname{End}}
\newcommand{\coker}{\operatorname{coker}}
\newcommand{\limit}{\varprojlim}
\newcommand{\colimit}{%
  \mathop{\mathpalette\colimit@{\rightarrowfill@\textstyle}}\nmlimits@
}
\makeatother


\newcommand{\fraka}{\mathfrak{a}} % ideal
\newcommand{\frakb}{\mathfrak{b}} % ideal
\newcommand{\frakc}{\mathfrak{c}} % ideal
\newcommand{\frakf}{\mathfrak{f}} % face map
\newcommand{\frakg}{\mathfrak{g}}
\newcommand{\frakh}{\mathfrak{h}}
\newcommand{\frakm}{\mathfrak{m}} % maximal ideal
\newcommand{\frakn}{\mathfrak{n}} % naximal ideal
\newcommand{\frakp}{\mathfrak{p}} % prime ideal
\newcommand{\frakq}{\mathfrak{q}} % qrime ideal
\newcommand{\fraks}{\mathfrak{s}}
\newcommand{\frakt}{\mathfrak{t}}
\newcommand{\frakz}{\mathfrak{z}}
\newcommand{\frakA}{\mathfrak{A}}
\newcommand{\frakB}{\mathfrak{B}}
\newcommand{\frakI}{\mathfrak{I}}
\newcommand{\frakJ}{\mathfrak{J}}
\newcommand{\frakK}{\mathfrak{K}}
\newcommand{\frakL}{\mathfrak{L}}
\newcommand{\frakN}{\mathfrak{N}} % nilradical 
\newcommand{\frakO}{\mathfrak{O}} % dedekind domain
\newcommand{\frakP}{\mathfrak{P}} % Prime ideal above
\newcommand{\frakQ}{\mathfrak{Q}} % Qrime ideal above 
\newcommand{\frakR}{\mathfrak{R}} % jacobson radical
\newcommand{\frakU}{\mathfrak{U}}
\newcommand{\frakV}{\mathfrak{V}}
\newcommand{\frakW}{\mathfrak{W}}
\newcommand{\frakX}{\mathfrak{X}}

% General/Differential/Algebraic Topology 
\newcommand{\scrA}{\mathscr{A}}
\newcommand{\scrB}{\mathscr{B}}
\newcommand{\scrF}{\mathscr{F}}
\newcommand{\scrM}{\mathscr{M}}
\newcommand{\scrN}{\mathscr{N}}
\newcommand{\scrP}{\mathscr{P}}
\newcommand{\scrO}{\mathscr{O}} % sheaf
\newcommand{\scrR}{\mathscr{R}}
\newcommand{\scrS}{\mathscr{S}}
\newcommand{\bbH}{\mathbb H}
\newcommand{\Int}{\operatorname{Int}}
\newcommand{\psimeq}{\simeq_p}
\newcommand{\wt}[1]{\widetilde{#1}}
\newcommand{\RP}{\mathbb{R}\text{P}}
\newcommand{\CP}{\mathbb{C}\text{P}}

% Miscellaneous
\newcommand{\wh}[1]{\widehat{#1}}
\newcommand{\calM}{\mathcal{M}}
\newcommand{\calP}{\mathcal{P}}
\newcommand{\onto}{\twoheadrightarrow}
\newcommand{\into}{\hookrightarrow}
\newcommand{\Gr}{\operatorname{Gr}}
\newcommand{\Span}{\operatorname{Span}}
\newcommand{\ev}{\operatorname{ev}}
\newcommand{\weakto}{\stackrel{w}{\longrightarrow}}

\newcommand{\define}[1]{\textcolor{blue}{\textit{#1}}}
% \newcommand{\caution}[1]{\textcolor{red}{\textit{#1}}}
\newcommand{\important}[1]{\textcolor{red}{\textit{#1}}}
\renewcommand{\mod}{~\mathrm{mod}~}
\renewcommand{\le}{\leqslant}
\renewcommand{\leq}{\leqslant}
\renewcommand{\ge}{\geqslant}
\renewcommand{\geq}{\geqslant}
\newcommand{\Res}{\operatorname{Res}}
\newcommand{\floor}[1]{\left\lfloor #1\right\rfloor}
\newcommand{\ceil}[1]{\left\lceil #1\right\rceil}
\newcommand{\gl}{\mathfrak{gl}}
\newcommand{\ad}{\operatorname{ad}}
\newcommand{\Stab}{\operatorname{Stab}}
\newcommand{\bfX}{\mathbf{X}}
\newcommand{\Ind}{\operatorname{Ind}}
\newcommand{\bfG}{\mathbf{G}}
\newcommand{\rank}{\operatorname{rank}}
\newcommand{\calo}{\mathcal{o}}
\newcommand{\frako}{\mathfrak{o}}
\newcommand{\Cl}{\operatorname{Cl}}

\newcommand{\idim}{\operatorname{idim}}
\newcommand{\pdim}{\operatorname{pdim}}
\newcommand{\Ext}{\operatorname{Ext}}
\newcommand{\co}{\operatorname{co}}
\newcommand{\bfO}{\mathbf{O}}
\newcommand{\bfF}{\mathbf{F}} % Fitting Subgroup
\newcommand{\Syl}{\operatorname{Syl}}
\newcommand{\nor}{\vartriangleleft}
\newcommand{\noreq}{\trianglelefteqslant}
\newcommand{\subnor}{\nor\!\nor}
\newcommand{\Soc}{\operatorname{Soc}}
\newcommand{\core}{\operatorname{core}}
\newcommand{\Sd}{\operatorname{Sd}}
\newcommand{\mesh}{\operatorname{mesh}}
\newcommand{\sminus}{\setminus}
\newcommand{\diam}{\operatorname{diam}}
\newcommand{\Ass}{\operatorname{Ass}}
\newcommand{\projdim}{\operatorname{proj~dim}}
\newcommand{\injdim}{\operatorname{inj~dim}}
\newcommand{\gldim}{\operatorname{gl~dim}}
\newcommand{\embdim}{\operatorname{emb~dim}}
\newcommand{\hght}{\operatorname{ht}}
\newcommand{\depth}{\operatorname{depth}}
\newcommand{\ul}[1]{\underline{#1}}
\newcommand{\type}{\operatorname{type}}
\newcommand{\CM}{\operatorname{CM}}
\newcommand{\cech}[1]{\mathbin{\check{#1}}}
\newcommand{\cdim}{\operatorname{cdim}}
\newcommand{\Der}{\operatorname{Der}}
\newcommand{\trdeg}{\operatorname{trdeg}}

\geometry {
    margin = 1in
}

\titleformat
{\section}
[block]
{\Large\bfseries\sffamily}
{\S\thesection}
{0.5em}
{\centering}
[]


\titleformat
{\subsection}
[block]
{\normalfont\bfseries\sffamily}
{\S\S}
{0.5em}
{\centering}
[]


\begin{document}

\title{Valuation Rings and Dedekind Domains}
\author{Swayam Chube}
\date{Last Updated: \today}

\maketitle

\section{General Valuation Rings}

\begin{definition}
    An integral domain $R$ with fraction field $K$ is said to be a \define{valuation ring} if for each $x\in K^\times$, either $x\in R$ or $x^{-1}\in R$.
\end{definition}

We remark that if $R$ is a valuation ring with fraction field $K$, then any subring of $K$ containing $R$ is also a valuation ring.

\begin{proposition}\thlabel{ideals-totally-ordered}
    The ideals in a valuation ring $R$ are totally ordered by inclusion.
\end{proposition}
\begin{proof}
    Let $I$ and $J$ be ideals in a valuation ring. If either $I$ or $J$ is zero or $I = J$, then there is nothing to prove. Assume hence that both are non-zero and $I\ne J$. Without loss of generality, suppose $x\in I\setminus J$. Then for any $0\ne y\in J$, $\frac{x}{y}\notin R$ lest $x\in J$. But since $R$ is a valuation ring, $\frac{y}{x}\in R$, consequently, $y\in I$, so that $J\subseteq I$.
\end{proof}

\begin{corollary}
    A valuation ring is a local domain.
\end{corollary}
\begin{proof}
    Since the ideals are totally ordered, there must be a unique maximal ideal.
\end{proof}

\begin{corollary}
    A finitely generated ideal in a valuation ring is principal, i.e., a valuation ring is a B\'ezout domain.
\end{corollary}
\begin{proof}
    Suppose $I = (a_1,\dots,a_n)\noreq R$, a valuation ring. In view of \thref{ideals-totally-ordered} there exists an index $i$ such that $(a_j)\subseteq(a_i)$ for all $1\le j\le n$, and hence $I = (a_i)$ is principal.
\end{proof}

\begin{definition}
    A ring is said to be B\'ezout if every finitely generated ideal is principal.
\end{definition}

\begin{proposition}
    Let $R$ be a ring. Then $R$ is a valuation ring if and only if it is a local B\'ezout domain.
\end{proposition}
\begin{proof}
    We have shown above that every valuation ring is a local B\'ezout domain. Conversely, suppose $(R,\frakm)$ is a local B\'ezout domain and let $0\ne x\in K$, the fraction field of $R$. Then there exist $f, g\in R\setminus\{0\}$ such that $x = \frac{f}{g}$. Since $R$ is a B\'ezout domain, there exists $h\in R$ such that $(f, g) = (h)$. Let $a, b\in R$ be such that $f = ah$ and $g = bh$. Then $(a, b) = (1)$, and hence, at least one of $a$ or $b$ must be a unit. In any case, either $\frac{f}{g}$ or $\frac{g}{h}$ is an element of $R$, that is, $R$ is a valuation ring.
\end{proof}

\begin{proposition}\thlabel{valuation-ring-integrally-closed}
    A valuation ring is integrally closed in its field of fractions.
\end{proposition}
\begin{proof}
    Let $(R,\frakm)$ be a valuation ring with field of fractions $K$. Suppose $R$ is not integrally closed in $K$, then there exists $0\ne x\in R$ such that $x^{-1}\in K\setminus R$ is integral over $R$, and hence, satisfies an equation of the form 
    \begin{equation*}
        x^{-n} + a_{1}x^{-n + 1} + \cdots + a_n = 0,
    \end{equation*}
    where $a_i\in R$ for $1\le i\le n$. Further, since $K$ is a field, we may assume that $a_n\ne 0$. Multiplying by $x^n$, we obtain 
    \begin{equation*}
        a_nx^n + \dots + a_1x + 1 = 0.
    \end{equation*}
    Since $x$ is not a unit in $R$, $x\in\frakm$, but the above equation would then imply that $1\in\frakm$, a contradiction. Thus $R$ is integrally closed in $K$.
\end{proof}

There is a very simple characterization of flat modules over valuation rings which we include here, although it will never be used throughout this article.

\begin{theorem}
    A module over a B\'ezout domain is flat if and only if it is torsion-free. In particular, this is true for valuation rings.
\end{theorem}
\begin{proof}
    Let $R$ be a B\'ezout domain. It is well-known that a flat module over an integral domain is torsion-free; this follows by considering, for each $0\ne a\in R$, the injective map $0\to R\xrightarrow{\cdot a} R$ and tensoring it with $M$.

    Conversely, let $M$ be a torsion-free $R$-module. It suffices to show that $\Tor^R_1(R/\fraka, M) = 0$ for every finitely generated ideal $\fraka$ of $R$. Disregarding the trivial case, we may assume that $\fraka\ne 0$. Since $R$ is a B\'ezout domain, $\fraka = (a)$ for some $0\ne a\in R$. Tensoring the short exact sequence 
    \begin{equation*}
        0\to R\xrightarrow{\cdot a} R\to R/aR\to 0
    \end{equation*}
    with $M$ and taking the induced long exact sequence, we get 
    \begin{equation*}
        \cdots\to 0 = \Tor^1_R(R, M)\to \Tor^1_R(R/aR, M)\to R\otimes_R M\xrightarrow{\cdot a} R\otimes_R M\to R/aR\otimes_R M\to 0.
    \end{equation*}
    But since $R\otimes_R M$ is canonically isomorphic to $M$, and $M$ is torsion-free, we have that $\Tor^1_R(R/aR, M) = 0$, whence $M$ is a flat $R$-module.
\end{proof}

\begin{theorem}
    Let $R$ be a valuation ring with fraction field $K$, and let $R'$ be another subring of $K$ properly containing $R$. Let $\frakm$ denote the maximal ideal of $R$ and $\frakp$ the maximal ideal of $R'$. Then 
    \begin{enumerate}[label=(\arabic*)]
        \item $\frakp\subsetneq\frakm\subseteq R\subseteq R'$.
        \item $\frakp$ is a prime ideal in $R$, and $R' = R_\frakp$. 
        \item $R/\frakp$ is a valuation ring of the field $R'/\frakp$.
        \item Given any valuation ring $\overline S$ of the field $R/\frakm$, let $S$ be its inverse image in $R$. Then $S$ is a valuation ring having the same fraction field $K$ as $R$.
    \end{enumerate}
\end{theorem}
\begin{proof}
\begin{enumerate}[label=(\arabic*)]
    \item Let $x\in\frakp$ so that $x$ is not a unit in $R'$, i.e., $x^{-1}\notin R'$. Thus $x^{-1}\notin R$, equivalently, $x\in\frakm$. Next, to see that the inclusion $\frakp\subseteq\frakm$ is strict, choose some $y\in R'\setminus R$. Then $y^{-1}\in R$ and is not a unit in $R$, whence $y^{-1}\in\frakm$, but $y^{-1}\notin\frakp$, else $y\notin R'$. Thus the inclusion $\frakp\subseteq\frakm$ is strict. 
    
    \item Since $\frakp = \frakp\cap R$, it is a prime ideal in $R$. Clearly every element in $R\setminus\frakp$ is invertible in $R'$, so that $R\subseteq R_\frakp\subseteq R'$. But by construction, the maximal ideal $\frakp R_\frakp$ of $R_\frakp$ is contained in the maximal ideal $\frakp$ of $R'$; in view of (1), this means $R_\frakp = R'$. 
    
    \item Let $\pi\colon R'\to R'/\frakp$ denote the natural surjection. Let $0\ne \overline x\in R'/\frakp$ and choose some $x\in R'\setminus p$ such that $\overline x = \pi(x)$. If $x\in R$, then $\overline x\in R/\frakp$, else $x^{-1}\in R$ and $\overline x^{-1} = \pi(x^{-1})\in R/\frakp$, as desired. 
    
    \item Note that $S$ is a subring of $R$ containing $\frakm$ and $S/\frakm = \overline S\subseteq R/\frakm$. Let $0\ne x\in K$. Since $R$ is a valuation ring, either $x\in R$ or $x^{-1}\in R$. We may suppose without loss of generality that $x\in R$. Let $\overline x \in R/\frakm$ denote its image. If $\overline x = 0$, then $x\in\frakm\subseteq S$. Otherwise, either $\overline x\in\overline S$ or $\overline x^{-1}\in\overline S$. Hence, either $x\in S$ or $x^{-1}\in S$, i.e., $S$ is a valuation ring with fraction field $K$. \qedhere
\end{enumerate}
\end{proof}

\begin{theorem}\thlabel{existence-of-dominating-valuation-ring}
    Let $K$ be a field, $A\subseteq K$ a subring, and $\frakp$ a prime ideal of $A$. Then there exists a valuation ring $(R, \frakm)$ of $K$ satisfying 
    \begin{equation*}
        A\subseteq R\quad\text{ and }\quad \frakm\cap A = \frakp.
    \end{equation*}
\end{theorem}
\begin{proof}
    Replacing $A$ by $A_\frakp$, we may assume that $A$ is a local ring with $\frakp = \frakm_A$ the maximal ideal of $A$. This can be done because $A\cap\frakp A_\frakp = \frakp$. Next, let $\scrF$ denote the subrings of $K$ containing $A$ such that $1\notin\frakp B$; and order $\scrF$ by inclusion of rings. Clearly every chain in $\scrF$ has an upper bound given by union of rings constituting the chain. In view of Zorn's lemma, $\scrF$ admits a maximal element, say $R$. Since $\frakp R\subsetneq R$, there exists a maximal ideal $\frakm$ of $R$ containing $\frakp R$. Then $R_\frakm\in\scrF$, since the extension of $\frakm$ to $R_\frakm$ is $\frakm R_\frakm$, which is a proper ideal. Thus $R_\frakm\in\scrF$ and by maximality of $R$, we have $R = R_\frakm$, that is, $(R,\frakm)$ is a local ring. 

    Since $\frakm\cap A\supseteq\frakp$ and $\frakp$ is a maximal ideal, we have $\frakm\cap A = \frakp$. It remains to show that $R$ is a valuation ring. Let $0\ne x\in K$ and suppose that $x, x^{-1}\notin R$. Since $x\notin R$, $R\subsetneq R[x]$, so that $R[x]\notin\scrF$ and hence $\frakp$ generates the unit ideal in $R[x]$. Thus there exists a relation 
    \begin{equation*}
        1 = a_0 + a_1x + \dots + a_nx^n 
    \end{equation*}
    for some positive integer $n$ and $a_i\in\frakp R$ for $0\le i\le n$. Note that $n\ge 1$ since $1\notin\frakp R$. Since $1 - a_0$ is a unit in $R$, we can multiply by its inverse to get a relation of the form 
    \begin{equation}
        1 = b_1x + \dots + b_nx^n\quad\text{ where }b_i\in\frakm\text{ for }1\le i\le n.\label{first}\tag{$\star$}
    \end{equation}
    Choose such a relation that minimizes $n\ge 1$. Similarly, since $x^{-1}\notin R$, we can find another relation 
    \begin{equation}
        1 = c_1x^{-1} + \dots + c_mx^{-m}\quad\text{ where }c_i\in\frakm\text{ for }1\le i\le m.\label{second}\tag{$\star\star$}
    \end{equation}
    Again, choose such a relation that minimizes $n\ge 1$. If $n\ge m$, multiply \eqref{second} by $b_nx^n$ and substitute in \eqref{first} to obtain a relation of smaller $x$-degree, a contradiction. On the other hand, if $n < m$, then we obtain a similar contradiction by interchanging the roles of $x$ and $x^{-1}$. This completes the proof.
\end{proof}

\begin{theorem}
    Let $K$ be a field, $A\subseteq K$ a subring, and $B$ the integral closure of $A$ in $K$. Then $B$ is the intersection of all the valuation rings of $K$ containing $A$.
\end{theorem}
\begin{proof}
    Let $B'$ denote the intersection of all valuation rings of $K$ containing $A$. Due to \thref{valuation-ring-integrally-closed}, every such valuation ring contains $B$, that is, $B\subseteq B'$. Let $x\in K\setminus B$, that is, $x$ is not integral over $A$. It suffices to find a valuation ring of $K$ containing $A$ but not $x$. Set $y = x^{-1}$, and consider the ideal $yA[y]$ of the ring $A[y]$. Note that this ideal is proper, else, there would exist a relation 
    \begin{equation*}
        1 = a_1 y + \dots + a_n y^n
    \end{equation*}
    for some $a_1,\dots,a_n\in A$; which upon multiplying by $x^n$ forces $x$ to be integral over $A$, a contradiction. Let $\frakp$ be a maximal ideal of $A[y]$ containing $yA[y]$. In view of \thref{existence-of-dominating-valuation-ring}, there exists a valuation ring $(V,\frakm)$ of $K$ containing $A[y]$ such that $\frakm\cap A[y] = \frakp$, in particular, $y\in\frakm$, and hence $x = y^{-1}\notin V$, as desired.
\end{proof}

\begin{definition}
    An abelian group $(H, +)$ together with a total ordering $(H, \leqq)$ is said to be an \define{ordered group} if 
    \begin{equation*}
        \forall~x,y,z,w\in H\quad x\geqq y\text{ and }z\geqq w\implies x + z\geqq y + w.
    \end{equation*}
\end{definition}
Note that if $x > 0$ and $y\geqq 0$ in $H$, then 
\begin{equation*}
    x + y\geqq x > 0,
\end{equation*}
and if $x\geqq y$ in $H$, then adding $-(x + y)$ to both sides of the inequality, we obtain: $-y\geqq -x$.

Given an ordered abliean group $(H,\leqq)$, we can extend the ordering to the set $H\cup\{\infty\}$ by setting $\infty\geqq x$ for all $x\in H$, $\infty + x = \infty$ for all $x\in H$, and $\infty + \infty = \infty$.

\begin{definition}
    A(n) (additive) \define{valuation} of a field $K$ is a map $v\colon K\to H\cup\{\infty\}$ where $(H, \leqq)$ is an ordered abelian group such that for all $x,y\in K$, 
    \begin{enumerate}[label=(\roman*)]
        \item $v(xy) = v(x) + v(y)$,
        \item $v(x + y)\geqq\min\left\{v(x), v(y)\right\}$, and 
        \item $v(x) = \infty$ if and only if $x = 0$.
    \end{enumerate}
\end{definition}
Clearly, the restriction $v\colon K^\times\to H$ defines a group homomorphism. Set 
\begin{equation*}
    R_v = \left\{x\in K\colon v(x)\ge 0\right\}\quad\text{ and }\quad \frakm_v = \left\{x\in K\colon v(x) > 0\right\}.
\end{equation*}
It is easy to see that $(R_v, \frakm_v)$ is a valuation ring of the field $K$. The image of the group homomorphism $v\colon K^\times\to H$ is called the \define{value group} of the valuation $v$. Note that we may restrict the codomain of $v$ to its value group without changing the valuation ring. 

Now, let $(R,\frakm)$ be a valuation ring with fraction field $K$. Let $G$ denote the set of non-zero principal $R$-submodules of $K$, that is, 
\begin{equation*}
    G = \left\{xR\colon x\in K^\times\right\}.
\end{equation*}
Note that $G$ is clearly an abelian group under the ``multiplication'' defined by 
\begin{equation*}
    xR\cdot yR = xy R.
\end{equation*}
The identity element is $R$ and the inverse of $xR$ is given by $x^{-1}R$. The canonical map $v\colon K^\times\to G$ given by $x\mapsto xR$ is a surjective group homomorphism with kernel $R^\times$. Thus $G\cong K^\times/R^\times$ as abelian groups. Note that the submodules in $G$ are totally ordered, indeed, if $x, y\in K^\times$, either $\frac{x}{y}$ or $\frac{y}{x}\in R$, and thus, one of $xR$ and $yR$ must be contained in the other. Define the relation 
\begin{equation*}
    xR\leqq yR\iff xR\supseteq yR
\end{equation*}
on $G$. Clearly $G$ forms an ordered abelian group under this relation. Extend the map $v\colon K^\times\to G\subseteq G\cup\{\infty\}$ by setting $v(0) = \infty$. We contend that $v$ is a valuation. To this end, it suffices to verify axiom (ii); for this, we may assume $x, y\in K^\times$. Now, 
\begin{equation*}
    v(x + y) = xR + yR = \min\{v(x), v(y)\},
\end{equation*}
since the submodules in $G$ are totally ordered. Thus $v$ is a valuation, and the corresponding valuation ring is $(R, \frakm)$ by construction. Call this the \define{canonical valuation} of the valuation ring $(R,\frakm)$. In essence, we have shown that there's no substantial difference between ``abstract'' valuation rings and those valuation rings that come from (additive) valuations of a field.

\begin{proposition}
    Let $v$ and $v'$ be two (additive) valuations of the field $K$ with value groups $H$ and $H'$ respectively. If both $v$ and $v'$ give rise to the same valuation ring, then there is an order isomorphism $\varphi\colon H\to H'$ such that $v' = \varphi\circ v$.

    Thus, in some sense, the value group of a valuation ring is determined up to order-isomorphism, and in particular, is isomorphic to $K^\times/R^\times$.
\end{proposition}
\begin{proof}
    Note that it sufficess to assume $v'$ is the canonical valuation with value group $G = \{xR\colon x\in K^\times\}$. Since $\ker v = R^\times$ and $\ker v' = R^\times$, there is an injective map $\varphi\colon G\to H$ such that $\varphi\circ v' = v$. Since $v$ is surjective, so is $\varphi$. That is, $v$ is an isomorphism. Finally, suppose $v'(x)\leqq v'(y)$, that is, $xR\supseteq yR$, equivalently, $\frac{y}{x}\in R$, whence $v\left(\frac{x}{y}\right)\geqq 0$, equivalently $v(x)\geqq v(y)$, as desired.
\end{proof}

\section{Discrete Valuation Rings and Dedekind Domains}

\begin{definition}
    A valuation ring whose value group is isomorphic to $\Z$ is called a \define{discrete valuation ring (DVR)}.
\end{definition}

\begin{theorem}
    Let $R$ be a valuation ring. The following are equivalent: 
    \begin{enumerate}[label=(\arabic*)]
        \item $R$ is a DVR. 
        \item $R$ is a PID. 
        \item $R$ is Noetherian.
    \end{enumerate}
\end{theorem}
\begin{proof}
    
\end{proof}

\begin{theorem}
    Let $R$ be a ring. The following are equivalent: 
    \begin{enumerate}[label=(\arabic*)]
        \item $R$ is a DVR.
        \item $R$ is a local PID which is not a field. 
        \item $R$ is a Noetherian local ring of positive Krull dimension with principal maximal ideal. 
        \item $R$ is a one-dimensional normal Noetherian local domain.
    \end{enumerate}
\end{theorem}
\begin{proof}
    
\end{proof}

\subsection{Fractional Ideals and Dedekind Domains}

\begin{definition}
    Let $R$ be an integral domain with fraction field $K$. A \define{fractional ideal} of $R$ is an $R$-submodule $I$ of $K$ such that there exists $0\ne\alpha\in R$ such that $\alpha I\subseteq R$.
\end{definition}

Just like ordinary ideals of $R$, we can take the sum and product of $R$-submodules of $K$: 
\begin{equation*}
    I + J = \left\{x + y\colon x\in I,~y\in J\right\}\quad\text{ and } I\cdot J = \left\{xy\colon x\in I,~y\in J\right\}R.
\end{equation*}
Note that the sum and product of fractional ideals is again a fractional ideal. Indeed, suppose $\alpha, \beta\in R\setminus\{0\}$ such that $\alpha I, \beta J\subseteq R$. Then it is clear that $\alpha\beta(I + J)\subseteq R$ and $\alpha\beta(I\cdot J)\subseteq R$. 

Next, we take a look at localization. Let $S\subseteq R$ be a multiplicative subset. Then 
\begin{equation*}
    S^{-1}I = \left\{\frac{x}{s}\colon x\in I,~s\in S\right\}
\end{equation*}
is an $S^{-1}R$ submodule of $K$ such that $\alpha(S^{-1}I)\subseteq S^{-1}R$, so that $S^{-1}I$ is a fractional ideal of $S^{-1}R$. The usual properties of localization for ideals carries over to the case of fractional ideals. Indeed, if $I$ and $J$ are $R$-submodules of $K$, then:
\begin{enumerate}[label=(\roman*)]
    \item $S^{-1}I\cdot S^{-1}J = S^{-1}(I\cdot J)$
    \item $S^{-1}I\colon_{S^{-1}R} S^{-1}J = S^{-1}\left(I\colon_R J\right)$.
\end{enumerate}
The first one is clear. For the second one, the inclusion $S^{-1}(I\colon_R J)\subseteq S^{-1}I\colon_{S^{-1}R} S^{-1}J$ is also clear. Conversely, if $\frac{\alpha}{s}\in S^{-1}I\colon_{S^{-1}R} S^{-1}J$, then for any $\frac{y}{t}\in S^{-1}J$, we have $\frac{\alpha y}{st}\in S^{-1}I$, that is, $\alpha y\in I$, whence $\alpha\in I\colon_R J$. This establishes the equality.

\begin{definition}
    An $R$-submodule $I$ of $K$ is said to be \define{invertible} if there exists an $R$-submodule $J$ of $K$ such that $I\cdot J = R$.
\end{definition}
Clearly, if $I$ is an invertible $R$-submodule of $K$, then it must be a fractional ideal. Further, if $I\cdot J = R$, then 
\begin{equation*}
    J = \left\{\alpha\in K\colon\alpha I\subseteq R\right\} = R\colon_R I.
\end{equation*}
Indeed, we have the inclusion $J\subseteq R\colon_R I = (R\colon_R I)\cdot I\cdot J\subseteq J$, and hence, equality holds everywhere.

\begin{theorem}
    Let $R$ be an integral domain and $I$ a fractional ideal of $R$. The following are equivalent: 
    \begin{enumerate}[label=(\arabic*)]
        \item $I$ is invertible. 
        \item $I$ is a projective $R$-module. 
        \item $I$ is finitely generated, and for every maximal ideal $\frakm$ of $R$, the fractional ideal $I_\frakm$ of $R_\frakm$ is principal.
    \end{enumerate}
\end{theorem}
\begin{proof}
    
\end{proof}

\end{document}