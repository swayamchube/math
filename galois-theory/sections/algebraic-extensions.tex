\subsection{Algebraic Elements}

\begin{definition}
    A \define{field extension} is a containment of fields $F\subseteq E$. The dimension of $E$ when viewed as a vector space over $F$ is called the \define{degree} of the extension and is denoted $[E : F]$.
    
    An element $\alpha\in E$ is said to be \define{algebraic} over $F$ if it satisfies an equation of the form 
    \begin{equation*}
        a_n\alpha^n + \dots + a_0 = 0,
    \end{equation*}
    where not all the $a_i$'s are zero. An element of $E$ which is not algebraic is said to be transcendental over $F$. The extension $F\subseteq E$ is said to be \define{algebraic} if every element of $E$ is algebraic over $F$.
\end{definition}

Let $\alpha\in E$ be algebraic over $F$ as in the above definition. Consider the ring homomorphism $\varphi\colon F[X]\to E$ which is identity on $F$ and sends $X\mapsto\alpha$. Clearly the kernel of this homomorphism is non-zero since it contains the non-zero polynomial $a_nX^n + \dots + a_0$. Further, the image, being a subring of $E$ is an integral domain, whence the kernel is a prime ideal. Since $F[X]$ is a PID, every prime ideal is maximal and is generated by an irreducible polynomial. Further, since $\left(F[X]\right)^\times = F^\times$, there is a unique monic (irreducible) polynomial $p(X)$ such that $\ker\varphi = \left(p(X)\right)$. This is called \emph{the} \define{minimal polynomial} of $\alpha$ over $F$, denoted by $\Irr(\alpha, F, X)$. The image of $\varphi$ is clearly $F[\alpha]$ and since we have argued that it is a field, we have shown that $F[\alpha] = F(\alpha)$. To summarize: 
\begin{proposition}\thlabel{prop:single-algebraic-element}
    Let $F\subseteq E$ be a field extension and $\alpha\in E$ be algebraic over $F$ with minimal polynomial $p(X)\in F[X]$. Then $F[\alpha] = F(\alpha)$, and $[F(\alpha) : F] = \deg p(X)$. In particular, $F\subseteq F(\alpha)$ is a finite extension.
\end{proposition}

\begin{proposition}\thlabel{prop:finite-implies-algebraic}
    If $[E : F] < \infty$, then $F\subseteq E$ is algebraic.
\end{proposition}
\begin{proof}
    Let $n = [E : F]$ and $\alpha\in E$. Thus the set $\{1, \alpha, \dots, \alpha^n\}$ is linearly dependent whence the conclusion follows.
\end{proof}

\begin{proposition}\thlabel{prop:multiplicativity-of-degree}
    Let $k\subseteq F\subseteq E$ be a tower of finite extensions. Then 
    \begin{equation*}
        [E : k] = [E : F][F : k].
    \end{equation*}
\end{proposition}
\begin{proof}
    Let $\{x_i\}$ be a $k$-basis for $F$ and $\{y_j\}$ an $F$-basis for $E$. It is straightforward to check that $\{x_iy_j\}$ is a $k$-basis for $E$.
\end{proof}

\begin{corollary}\thlabel{cor:finitely-generated-algebraic-implies-finite}
    A finitely generated algebraic extension is finite.
\end{corollary}
\begin{proof}
    Let $F\subseteq E$ be a finitely generated algebraic extension, that is, there exist $\alpha_1,\dots,\alpha_n\in E$ such that $E = F(\alpha_1,\dots,\alpha_n)$. Considering the tower of field extensions: 
    \begin{equation*}
        F\subseteq F(\alpha_1)\subseteq F(\alpha_1, \alpha_2)\subseteq\dots\subseteq F(\alpha_1,\dots,\alpha_n) = E,
    \end{equation*}
    and using \thref{prop:single-algebraic-element} and \thref{prop:multiplicativity-of-degree}, the conclusion follows.
\end{proof}

\begin{definition}
    A class $\scrC$ of field extensions is said to form a \define{distinguished class} if the following two conditions are satisfied: 
    \begin{enumerate}[label=(DC \arabic*)]
        \item if $k\subseteq F\subseteq E$, then $E/k$ is an element of $\scrC$ if and only if both $E/F$ and $F/k$ lie in $\scrC$. \label{DC1}
        \item if $E/k$ is an element of $\scrC$ and $F/k$ is any extension with both $E$ and $F$ contained in a larger ambient field, then $EF/F$ lies in $\scrC$. \label{DC2}
    \end{enumerate}
\end{definition}

\begin{remark}
    It follows formally from \ref{DC1} and \ref{DC2} that if $E/k$ and $F/k$ lie in $\scrC$ with both $E$ and $F$ contained in a larger ambient field, then $EF/k$ lies in $\scrC$.
\end{remark}

\begin{theorem}\thlabel{thm:algebraic-distinguished-class}
    Algebraic extensions form a distinguished class. 
\end{theorem}
\begin{proof}
    Let $k\subseteq F\subseteq E$ be a tower of extensions. Clearly if $E/k$ is algebraic, then both $E/F$ and $F/k$ are algebraic from the definition. Suppose now that both $E/F$ and $F/k$ are algebraic and let $\alpha\in E$. Then $\alpha$ satisfies an algebraic equation over $F$: 
    \begin{equation*}
        a_n\alpha^n + \dots + a_0 = 0,
    \end{equation*}
    where not all the $a_i$'s are zero. Consider the tower of fields: 
    \begin{equation*}
        k\subseteq k(a_0,\dots,a_n) = K\subseteq K(\alpha).
    \end{equation*}
    Note that $\alpha$ is algebraic over $K$, and hence both $K/k$ and $K(\alpha)/K$ are finite due to \thref{cor:finitely-generated-algebraic-implies-finite} and \thref{prop:single-algebraic-element} respectively. It follows from \thref{prop:multiplicativity-of-degree} that $K(\alpha)/k$ is finite, and hence algebraic due to \thref{prop:finite-implies-algebraic}. This verifies \ref{DC1}.

    Next, let $k\subseteq E$ be algebraic and $k\subseteq F$ be arbitrary with both $E$ and $F$ contained in a larger ambient field. Note that every $\alpha\in EF = F(E)$ is contained in a finitely generated subextension $F(\alpha_1,\dots,\alpha_n)$ for some $\alpha_1,\dots,\alpha_n\in E$. Since each $\alpha_i$ is algebraic over $k$ and $k\subseteq F$, the extension $F\subseteq F(\alpha_1,\dots,\alpha_n)$ is algebraic due to \thref{cor:finitely-generated-algebraic-implies-finite}, and hence $\alpha$ is algebraic over $F$. This verifies \ref{DC2}, thereby completing the proof.
\end{proof}

\subsection{Algebraic Closure}

\begin{definition}
    A field $\Omega$ is said to be \define{algebraically closed} if every non-constant polynomial in $\Omega[X]$ has a root in $\Omega$.
\end{definition}

\begin{lemma}
    Let $k$ be a field and $p(X)\in k[X]$ a non-constant polynomial. Then there is an extension $k\subseteq E$ in which $p(X)$ has a root.
\end{lemma}
\begin{proof}
    Without loss of generality, we may assume that $p(X)$ is irreducible. Consider the embedding of fields $k\into k[X]/p(X)$. We may identify $k$ with its image under the above embedding. Clearly $\overline X\in k[X]/p(X)$ is a root of $p(X)$, thereby completing the proof.
\end{proof}

\begin{corollary}
    Inductively, it is clear that given any finite collection of polynomials $f_1(X),\dots, f_n(X)\in k[X]$, there exists an extension $k\subseteq E$ in which each of them have a root.
\end{corollary}

\begin{theorem}[Artin]
    Every field is contained in an algebraically closed field. 
\end{theorem}
\begin{proof}
    Let $k$ be a field. Set $K_0 = k$. For each non-constant polynomial $f\in k[X]$, introduce a new variable $X_f$, and let $R = k[\{X_f\}]$ be a polynomial ring over those infinite variables. Let $I\noreq R$ denote the ideal generated as 
    \begin{equation*}
        I\coloneq\left(f(X_f)\colon f\in k[X]\text{ is a non-constant polynomial}\right).
    \end{equation*}
    We contend that $I$ is a proper ideal. Suppose not, then there exist $g_1,\dots,g_n\in R$ and non-constant polynomials $f_1,\dots,f_n\in k[X]$ 
    \begin{equation*}
        g_1f_1(X_{f_1}) + \dots + g_nf_n(X_{f_n}) = 1
    \end{equation*}
    in $R$. Let $F$ be an extension of $k$ in which each $f_i$ has a root for $1\le i\le n$. Pick a root $\alpha_i\in F$ for each $f_i$. Substituting $X_{f_i}\mapsto\alpha_i$, we obtain an immediate contradiction. Thus $I$ is a proper ideal, whence is contained in a maximal ideal $\frakm$ of $R$. Set $K_1 = R/\frakm$. Clearly $K_0\subseteq K_1$ and every non-constant polynomial in $K_0[X]$ has a root in $K_1$. Similarly construct $K_1\subseteq K_2\subseteq \dots$ and set $K\coloneq\displaystyle\bigcup_{i = 0}^\infty K_i$. 

    We contend that $K$ is algebraically closed. Indeed, let $f(X) = a_nX^n + \dots + a_0$ be a non-constant polynomial. Then, there is a sufficiently large $N\gg 0$ with $a_i\in K_N$ for all $0\le i\le n$. Then, by construction, $f(X)$ has a root in $K_{N + 1}\subseteq K$, thereby completing the proof.
\end{proof}

\begin{remark}
    In \cite[Chapter VI, Exercise 28]{lang-algebra}, one shows that $K_1$ itself is algebraically closed.
\end{remark}

\begin{corollary}
    Let $k$ be a field. Then there exists an algebraic extension $k\subseteq k^a$ where $k^a$ is algebraically closed.
\end{corollary}
\begin{proof}
    Let $\Omega$ be an algebraically closed field containing $k$ and set 
    \begin{equation*}
        k^a = \left\{\alpha\in \Omega\colon \alpha\text{ is algebraic over } k\right\}.
    \end{equation*}
    To see that $k^a$ forms a field, note that if $\alpha,\beta\in k^a$, with $\alpha,\beta\ne 0$, then $k\subseteq k(\alpha,\beta)$ is an algebraic extension due to \thref{cor:finitely-generated-algebraic-implies-finite}. In particular, $\alpha\pm\beta, \alpha\beta, \alpha\beta^{-1}\in k(\alpha,\beta)$ are algebraic over $k$ and hence lie in $k^a$. Thus $k^a$ is a field. 

    Let $f(X)\in k^a[X]$ be a non-constant polynomial. Since $\Omega$ is algebraically closed, $f(X)$ has a root $\alpha\in\Omega$ and $\alpha$ is algebraic over $k^a$, therefore is algebraic over $k$ due to \thref{thm:algebraic-distinguished-class}. In particular, $\alpha\in k^a$ and hence $k^a$ is algebraically closed.
\end{proof}

\begin{remark}
    Using an analogous proof, one can show that given any extension of fields $k\subseteq E$, the subset 
    \begin{equation*}
        E^a \coloneq\left\{\alpha\in E\colon\alpha\text{ is algebraic over }k\right\}
    \end{equation*}
    is a field containing $k$ and is algebraic over $k$.
\end{remark}

\begin{definition}
    Let $k\subseteq F$ and $k\subseteq E$ be field extensions. A \define{$k$-embedding} $\sigma\colon F\to E$ is a field homomorphism which restricts to the identity map on $k$.
\end{definition}

Next we examine extensions of field embeddings. Let $k$ be a field and $\sigma\colon k\to\Omega$ be an embedding into an algebraically closed field $\Omega$. Let $k(\alpha)\supseteq k$ be an algebraic extension. Let $p(X)\in k[X]$ be the minimal polynomial of $\alpha$ over $k$. For any root $\beta\in\Omega$ of $p^\sigma$, it is clear from the isomorphism $k(\alpha)\cong k[X]/p(X)$ that the embedding $\sigma$ can be extended to an embedding $\wt\sigma\colon k(\alpha)\to\Omega$ sending $\alpha\mapsto\beta$. Conversely, it is also clear that any such embedding must send $\alpha$ to a root of $p^\sigma$ in $\Omega$. In particular, we have: 
\begin{lemma}\thlabel{lem:extending-one-element}
    The number of extensions of $\sigma$ to $k(\alpha)$ is equal to the number of distinct roots of $p^\sigma(X)$ in $\Omega$.
\end{lemma}

\begin{lemma}[Extension Lemma]
    Let $E/k$ be an algebraic extension and $\Omega$ an algebraically closed field. Any embedding $\sigma\colon k\to\Omega$ can be extended to an embedding $E\to \Omega$.
\end{lemma}
\begin{proof}
    This is a standard application of Zorn's lemma. We only sketch the proof. Let 
    \begin{equation*}
        \Sigma = \left\{(F,\sigma_F)\colon k\subseteq F\subseteq E,~\sigma_F\text{ is an embedding of $F$ into $\Omega$ extending $\sigma$}\right\}.
    \end{equation*}
    Clearly $\Sigma$ is a poset containing a maximal element, say $(M,\sigma_M)$. If $M\ne E$, then choose some $\alpha\in E\setminus M$ and using \thref{lem:extending-one-element}, derive a contradiction.
\end{proof}

\begin{corollary}\thlabel{cor:uniqueness-algebraic-closure}
    Let $k$ be a field with algebraic extensions $k\subseteq E$ and $k\subseteq E'$ where both $E$ and $E'$ are algebraically closed. Then there is a $k$-isomorphism $\tau\colon E\to E'$.
\end{corollary}
\begin{proof}
    The inclusion $k\into E$ extends to a $k$-embedding $\tau\colon E\to E'$. Clearly $\tau(E)$ is an algebraically closed subfield of $E'$ and hence must be equal to $E'$.
\end{proof}

\begin{definition}
    Let $k$ be a field. An algebraically closed field $k^a\supseteq k$ which is algebraic over $k$ is said to be an \define{algebraic closure} of $k$. Due to \thref{cor:uniqueness-algebraic-closure}, the algebraic closure is unique up to isomorphism.
\end{definition}

\begin{proposition}
    Let $E/k$ be an algebraic extension and $\sigma\colon E\to E$ be a $k$-embedding (i.e., $\sigma|_k = \id_k$). Then $\sigma$ is an automorphism of $E$.
\end{proposition}
\begin{proof}
    It suffices to show that $\sigma$ is surjective. Indeed, let $\alpha\in E$ and $p(X)\in k[X]$ denote the minimal polynoial of $\alpha$ over $k$. Since the coefficients of $p$ are invariant under the action of the embedding, $\sigma$ must send a root of $p$ to another root of $p$. Further, since $\sigma$ is injective and there are only finitely many roots of $p$ in $E$, $\sigma$ permutes the roots of $p$. In particular, there is some root $\beta$ of $p$ in $E$ such that $\sigma\beta = \alpha$, whence surjectivity follows.
\end{proof}


\subsection{Splitting Fields and Normal Extensions}

\subsection{Separable Extensions}

\subsection{Finite Fields}

\subsection{Purely Inseparable Extensions}