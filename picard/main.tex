\documentclass[12pt]{article}

% \usepackage{./arxiv}

\title{Picard's Theorems}
\author{Swayam Chube}
\date{\today}

\usepackage[utf8]{inputenc} % allow utf-8 input
\usepackage[T1]{fontenc}    % use 8-bit T1 fonts
\usepackage{hyperref}       % hyperlinks
\usepackage{url}            % simple URL typesetting
\usepackage{booktabs}       % professional-quality tables
\usepackage{amsfonts}       % blackboard math symbols
\usepackage{nicefrac}       % compact symbols for 1/2, etc.
\usepackage{microtype}      % microtypography
\usepackage{graphicx}
\usepackage{natbib}
\usepackage{doi}
\usepackage{amssymb}
\usepackage{bbm}
\usepackage{amsthm}
\usepackage{amsmath}
\usepackage{xcolor}
\usepackage{theoremref}
\usepackage{enumitem}
\usepackage{mathpazo}
% \usepackage{euler}
\usepackage{mathrsfs}
\usepackage{todonotes}
\usepackage{stmaryrd}
\usepackage[all,cmtip]{xy} % For diagrams, praise the Freyd–Mitchell theorem 
\usepackage{marvosym}
\usepackage{geometry}
\usepackage{titlesec}

\renewcommand{\qedsymbol}{$\blacksquare$}

% Uncomment to override  the `A preprint' in the header
% \renewcommand{\headeright}{}
% \renewcommand{\undertitle}{}
% \renewcommand{\shorttitle}{}

\hypersetup{
    pdfauthor={Lots of People},
    colorlinks=true,
}

\newtheoremstyle{thmstyle}%               % Name
  {}%                                     % Space above
  {}%                                     % Space below
  {}%                             % Body font
  {}%                                     % Indent amount
  {\bfseries\scshape}%                            % Theorem head font
  {.}%                                    % Punctuation after theorem head
  { }%                                    % Space after theorem head, ' ', or \newline
  {\thmname{#1}\thmnumber{ #2}\thmnote{ (#3)}}%                                     % Theorem head spec (can be left empty, meaning `normal')

\newtheoremstyle{defstyle}%               % Name
  {}%                                     % Space above
  {}%                                     % Space below
  {}%                                     % Body font
  {}%                                     % Indent amount
  {\bfseries\scshape}%                            % Theorem head font
  {.}%                                    % Punctuation after theorem head
  { }%                                    % Space after theorem head, ' ', or \newline
  {\thmname{#1}\thmnumber{ #2}\thmnote{ (#3)}}%                                     % Theorem head spec (can be left empty, meaning `normal')

\theoremstyle{thmstyle}
\newtheorem{theorem}{Theorem}[section]
\newtheorem{lemma}[theorem]{Lemma}
\newtheorem{proposition}[theorem]{Proposition}

\theoremstyle{defstyle}
\newtheorem{definition}[theorem]{Definition}
\newtheorem*{corollary}{Corollary}
\newtheorem{remark}[theorem]{Remark}
\newtheorem{example}[theorem]{Example}
\newtheorem*{notation}{Notation}

% Common Algebraic Structures
\newcommand{\R}{\mathbb{R}}
\newcommand{\Q}{\mathbb{Q}}
\newcommand{\Z}{\mathbb{Z}}
\newcommand{\N}{\mathbb{N}}
\newcommand{\bbC}{\mathbb{C}} 
\newcommand{\K}{\mathbb{K}} % Base field which is either \R or \bbC
\newcommand{\calA}{\mathcal{A}} % Banach Algebras
\newcommand{\calB}{\mathcal{B}} % Banach Algebras
\newcommand{\calI}{\mathcal{I}} % ideal in a Banach algebra
\newcommand{\calJ}{\mathcal{J}} % ideal in a Banach algebra
\newcommand{\frakM}{\mathfrak{M}} % sigma-algebra
\newcommand{\calO}{\mathcal{O}} % Ring of integers
\newcommand{\bbA}{\mathbb{A}} % Adele (or ring thereof)
\newcommand{\bbI}{\mathbb{I}} % Idele (or group thereof)

% Categories
\newcommand{\catTopp}{\mathbf{Top}_*}
\newcommand{\catGrp}{\mathbf{Grp}}
\newcommand{\catTopGrp}{\mathbf{TopGrp}}
\newcommand{\catSet}{\mathbf{Set}}
\newcommand{\catTop}{\mathbf{Top}}
\newcommand{\catRing}{\mathbf{Ring}}
\newcommand{\catCRing}{\mathbf{CRing}} % comm. rings
\newcommand{\catMod}{\mathbf{Mod}}
\newcommand{\catMon}{\mathbf{Mon}}
\newcommand{\catMan}{\mathbf{Man}} % manifolds
\newcommand{\catDiff}{\mathbf{Diff}} % smooth manifolds
\newcommand{\catAlg}{\mathbf{Alg}}
\newcommand{\catRep}{\mathbf{Rep}} % representations 
\newcommand{\catVec}{\mathbf{Vec}}

% Group and Representation Theory
\newcommand{\chr}{\operatorname{char}}
\newcommand{\Aut}{\operatorname{Aut}}
\newcommand{\GL}{\operatorname{GL}}
\newcommand{\SL}{\operatorname{SL}}
\newcommand{\im}{\operatorname{im}}
\newcommand{\tr}{\operatorname{tr}}
\newcommand{\id}{\mathbf{id}}
\newcommand{\cl}{\mathbf{cl}}
\newcommand{\Gal}{\operatorname{Gal}}
\newcommand{\Tr}{\operatorname{Tr}}
\newcommand{\sgn}{\operatorname{sgn}}
\newcommand{\Sym}{\operatorname{Sym}}
\newcommand{\Alt}{\operatorname{Alt}}

% Commutative and Homological Algebra
\newcommand{\spec}{\operatorname{spec}}
\newcommand{\mspec}{\operatorname{m-spec}}
\newcommand{\Tor}{\operatorname{Tor}}
\newcommand{\tor}{\operatorname{tor}}
\newcommand{\Ann}{\operatorname{Ann}}
\newcommand{\Supp}{\operatorname{Supp}}
\newcommand{\Hom}{\operatorname{Hom}}
\newcommand{\End}{\operatorname{End}}
\newcommand{\coker}{\operatorname{coker}}
\newcommand{\limit}{\varprojlim}
\newcommand{\colimit}{%
  \mathop{\mathpalette\colimit@{\rightarrowfill@\textstyle}}\nmlimits@
}
\makeatother


\newcommand{\fraka}{\mathfrak{a}} % ideal
\newcommand{\frakb}{\mathfrak{b}} % ideal
\newcommand{\frakc}{\mathfrak{c}} % ideal
\newcommand{\frakf}{\mathfrak{f}} % face map
\newcommand{\frakg}{\mathfrak{g}}
\newcommand{\frakh}{\mathfrak{h}}
\newcommand{\frakm}{\mathfrak{m}} % maximal ideal
\newcommand{\frakn}{\mathfrak{n}} % naximal ideal
\newcommand{\frakp}{\mathfrak{p}} % prime ideal
\newcommand{\frakq}{\mathfrak{q}} % qrime ideal
\newcommand{\fraks}{\mathfrak{s}}
\newcommand{\frakt}{\mathfrak{t}}
\newcommand{\frakz}{\mathfrak{z}}
\newcommand{\frakA}{\mathfrak{A}}
\newcommand{\frakI}{\mathfrak{I}}
\newcommand{\frakJ}{\mathfrak{J}}
\newcommand{\frakK}{\mathfrak{K}}
\newcommand{\frakL}{\mathfrak{L}}
\newcommand{\frakN}{\mathfrak{N}} % nilradical 
\newcommand{\frakO}{\mathfrak{O}} % dedekind domain
\newcommand{\frakP}{\mathfrak{P}} % Prime ideal above
\newcommand{\frakQ}{\mathfrak{Q}} % Qrime ideal above 
\newcommand{\frakR}{\mathfrak{R}} % jacobson radical
\newcommand{\frakU}{\mathfrak{U}}
\newcommand{\frakX}{\mathfrak{X}}

% General/Differential/Algebraic Topology 
\newcommand{\scrA}{\mathscr A}
\newcommand{\scrB}{\mathscr B}
\newcommand{\scrF}{\mathscr F}
\newcommand{\scrN}{\mathscr N}
\newcommand{\scrP}{\mathscr P}
\newcommand{\scrR}{\mathscr R}
\newcommand{\scrS}{\mathscr S}
\newcommand{\bbH}{\mathbb H}
\newcommand{\Int}{\operatorname{Int}}
\newcommand{\psimeq}{\simeq_p}
\newcommand{\wt}[1]{\widetilde{#1}}
\newcommand{\RP}{\mathbb{R}\text{P}}
\newcommand{\CP}{\mathbb{C}\text{P}}

% Miscellaneous
\newcommand{\wh}[1]{\widehat{#1}}
\newcommand{\calM}{\mathcal{M}}
\newcommand{\calP}{\mathcal{P}}
\newcommand{\onto}{\twoheadrightarrow}
\newcommand{\into}{\hookrightarrow}
\newcommand{\Gr}{\operatorname{Gr}}
\newcommand{\Span}{\operatorname{Span}}
\newcommand{\ev}{\operatorname{ev}}
\newcommand{\weakto}{\stackrel{w}{\longrightarrow}}

\newcommand{\define}[1]{\textcolor{blue}{\textit{#1}}}
\newcommand{\caution}[1]{\textcolor{red}{\textit{#1}}}
\renewcommand{\mod}{~\mathrm{mod}~}
\renewcommand{\le}{\leqslant}
\renewcommand{\leq}{\leqslant}
\renewcommand{\ge}{\geqslant}
\renewcommand{\geq}{\geqslant}
\newcommand{\Res}{\operatorname{Res}}
\newcommand{\floor}[1]{\left\lfloor #1\right\rfloor}
\newcommand{\ceil}[1]{\left\lceil #1\right\rceil}
\newcommand{\gl}{\mathfrak{gl}}
\newcommand{\ad}{\operatorname{ad}}
\newcommand{\Stab}{\operatorname{Stab}}
\newcommand{\bfX}{\mathbf{X}}
\newcommand{\Ind}{\operatorname{Ind}}
\newcommand{\bfG}{\mathbf{G}}
\newcommand{\rank}{\operatorname{rank}}
\newcommand{\calo}{\mathcal{o}}
\newcommand{\frako}{\mathfrak{o}}
\newcommand{\Cl}{\operatorname{Cl}}

\newcommand{\idim}{\operatorname{idim}}
\newcommand{\pdim}{\operatorname{pdim}}
\newcommand{\Ext}{\operatorname{Ext}}
\newcommand{\co}{\operatorname{co}}
\newcommand{\bbD}{\mathbb{D}}

\geometry {
    margin = 1in
}

\titleformat
{\section}
[block]
{\Large\bfseries\scshape}
{\S\thesection}
{0.5em}
{\centering}
[]


\titleformat
{\subsection}
[block]
{\normalfont\bfseries\sffamily}
{\S\S}
{0.5em}
{\centering}
[]


\begin{document}
\maketitle
\begin{abstract}
    Taming the big bad wolves of $\bbC$omplex Analysis by nuking it.
\end{abstract}

\section{Analytic Covering Maps}

Recall first the definiton of a covering map in a general topological space. 
\begin{definition}[Abstract Covering Map]
    A map $\pi: E\to B$ is said to be a \emph{covering map} if there is an open cover $\{U_\alpha\}$ of $B$ such that $\pi^{-1}(U_\alpha)$ is homeomorphic to $U_\alpha\times D_\alpha$ where $D_\alpha$ is a topological space with the discrete topology.
\end{definition}

\begin{definition}[Analytic Covering Map]
    Let $\Omega,G\subseteq\bbC$ be open sets. An abstract covering map $\pi: \Omega\to G$ is said to be \emph{analytic} if $\pi$ is a holomorphic map.
\end{definition}

\begin{proposition}
    Let $\pi: \Omega\to G$ be an analytic covering map and $f: H\to G$ a holomorphic map. If there is a continuous map $\wt f: H\to\Omega$ such that $\pi\circ\wt f = f$, then $\wt f$ is holomorphic.
\end{proposition}
\begin{proof}
    Let $z_0\in H$. Then, there is a neighborhood $U$ of $f(z_0)$ in $G$ and a neighborhood $V$ of $\wt f(z_0)$ such that $\pi$ is a biholomorphism from $V$ to $U$. Let $W$ be a neighborhood of $z_0$ that maps into $V$ under $\wt f$. Then, on $W$, we have $\wt f = \pi^{-1}\circ f$, which is holomorphic.
\end{proof}

\section{Modular Function}

\begin{definition}[Modular Transformation]
    A \emph{modular transformation} is a M\"obius transformation 
    \begin{equation*}
        M(z) = \frac{az + b}{cz + d}
    \end{equation*}
    such that $\begin{pmatrix}a & b\\ c & d\end{pmatrix}\in\operatorname{SL}_2(\Z)$. The set of all modular transformations form a group, known as the \emph{modular group}. We often identify this group with $\operatorname{SL}_2(\Z)$.
\end{definition}

\begin{definition}
    Let $\Gamma$ denote the subgroup of $\SL_2(\Z)$ generated by 
    \begin{equation*}
        \tau = 
        \begin{pmatrix}
            1 & 2\\
            0 & 1
        \end{pmatrix}
        \text{ and }
        \sigma = 
        \begin{pmatrix}
            1 & 0\\
            2 & 1
        \end{pmatrix}.
    \end{equation*}
    This group will be of particular interest during the construction of a modular function. It is customary to denote this group by $\Gamma(2)$ but we drop the ``(2)'' for brevity.
\end{definition}

\begin{definition}
    Let $G$ denote the region 
    \begin{equation*}
        \{z = x + iy \in\bbH\colon -1\le x < 1,~|2z - 1| > 1\text{ and }|2z + 1|\ge 1\}.
    \end{equation*}
\end{definition}

\begin{theorem}
    Let $G$ and $\Gamma$ be as defined above. Then, 
    \begin{enumerate}
        \item $\varphi_1(G)\cap\varphi_2(G) = \emptyset$ whenever $\varphi_1\ne\varphi_2$ in $\Gamma$.
        \item $\bbH = \bigsqcup_{\varphi\in\Gamma}\varphi(G)$.
        \item   
        \begin{equation*}
            \Gamma = \left\{
            \begin{pmatrix}
                a & b\\
                c & d
            \end{pmatrix}\in\SL_2(\Z)
            \Bigg\vert~a,d\equiv1\pmod 2\text{ and } b,c\equiv0\pmod 2
            \right\}.
        \end{equation*}
    \end{enumerate}
\end{theorem}
\begin{proof}
    
\end{proof}

\begin{theorem}
    Let $G$ and $\Gamma$ be as defined above. Then, there is a holomorphic function $\lambda: \bbH\to\bbC$ having the following properties: 
    \begin{enumerate}
        \item $\lambda\circ\varphi = \lambda$ for all $\varphi\in\Gamma$.
        \item $\lambda$ is injective on $G$.
        \item $\lambda(\bbH) = \bbC\backslash\{0,1\}$.
        \item $\lambda:\bbH\to\bbC\backslash\{0,1\}$ is a covering map.
    \end{enumerate}
\end{theorem}
\begin{proof}
    Let 
    \begin{equation*}
        G_0 = \{z = x + iy\in\bbH\colon 0 < x < 1\text{ and }|2z - 1| > 1\}.
    \end{equation*}
    Note that $G_0$ is simply connected and thus, there is a conformal equivalence $f_0: G_0\to\bbH$. Then, there is an extension of $f$ to a homeomorphism $f:\overline G_0\to\overline\bbH$ that maps $\partial G_0\to\partial\bbH$. Upon composing with a suitable M\"obius transformation, we may suppose that $f(0) = 0$, $f(1) = 1$ and $f(\infty) = \infty$. 

    Consider the following three pieces of $\partial G_0$, 
    \begin{align*}
        L_1 &= \{z\in\overline\bbH\colon \Re(z) = 0\}\\
        L_2 &= \{z\in\overline\bbH\colon |2z - 1| = 1\}\\
        L_3 &= \{z\in\overline\bbH\colon \Re z = 1\}.
    \end{align*}
    First, note that $f$ is a bijection $L_1\cup L_2\cup L_3\to\partial\bbH = \R$. Further, $L_1$ and $L_3$ must map to half lines with $0$ and $1$ mapping to themselves. Therefore, $L_1$ must map to $(-\infty, 1]$, $L_2$ to $[0,1]$ and $L_3$ to $[1,\infty)$. Next, note that $f(L_1)\subseteq\R$ and hence, due to the Schwarz Reflection Principle, there is an extension of $f$ to all of $G$, defined by $f(-x + iy) = \overline{f(x + iy)}$. Note that this gives $f(G) = \bbC\backslash\{0,1\}$ and $f(\operatorname{int}G) = \bbC\backslash[0,\infty)$. Finally, define $\lambda:\bbH\to\bbC$ by 
    \begin{equation*}
        \lambda(z) = \lambda(\varphi^{-1}(z))\text{ when }z\in\varphi(G).
    \end{equation*}
    We contend that the $\lambda$ defined above is holomorphic. Consider the set $\Delta = G\cup\sigma^{-1}(G)\cup\tau^{-1}(G)$, whose interior contains $G$. It is not hard to argue, from the definition of $\lambda$ that it is continuous on $\Delta$ and holomorphic on the interiors of the aforementioned three sets that form it. Therefore, $\lambda$ is holomorphic on the interior of $\Delta$, in particular, on $G$.

    Lastly, we show that $\lambda$ is a covering map. To do this, we shall show that every point in $\bbC\backslash\{0,1\}$ has an evenly covered neighborhood. First, suppose $\zeta\in\bbC\backslash[0,\infty)$ and choose $\delta > 0$ small enough so that $B_0 = B(\zeta,\delta)\subseteq\bbC\backslash[0,\infty)$ and $U = f^{-1}(B_0)\subseteq G$. Obviously, $\lambda^{-1}(B_0) = \bigsqcup_{\varphi\in\Gamma}\varphi(U)$. Thus $B_0$ is an evenly covered neighborhood of $\zeta$.

    Next, suppose $t\in(0,1)$ and choose $\delta > 0$ small enough so that $B_0 = B(t,\delta)\subseteq\bbC\backslash\{0,1\}$. From the explicit definition of $f$, note that $f^{-1}(t)$ contains two points, $\{z_+, z_-\}$ and $f^{-1}(B_0)$ contains two components $U_+$ and $U_-$ containing $z_+$ and $z_-$ respectively, such that $f^{-1}(B\cap\pm\overline\bbH) = U_{\pm}$. The transformation $\sigma$ defined previously maps $|2z + 1| = 1$ to $|2z - 1| = 1$, $z_-$ to $z_+$ and hence, maps $U_-$ to $U_+$. Consequently, $U_0 = U_+\cup\sigma(U_-)$ is a neighborhood of $z_+$ such that $\lambda(U_0) = \lambda(U_+)\cup\lambda(\sigma(U_-)) = B_0$. Consequently, the components of $\lambda^{-1}(B_0)$ that are biholomorphically mapped to $B_0$ are $\varphi(U_0)$ where $\varphi\in\Gamma$.

    Finally, suppose $t\in(1,\infty)$. Recall that $L_3$ is mapped to $[1,\infty)$ under $f$, which was initially defined on $\overline G_0$. Arguing as in the previous paragraph, we see that there are two points $z_{\pm}$ with neighborhoods $U_{\pm}$ that are mapped to one another under $\tau$. Thus, it follow again, that $t$ has an evenly covered neighborhood. This completes the proof.
\end{proof}

\begin{corollary}
    There is a covering map $\mu:\bbD\to\bbC\backslash\{0,1\}$
\end{corollary}

\section{Normal Families}

\begin{theorem}[Montel-Carath\'eodory]\thlabel{thm:montel-caratheodory}
    Let $\Omega\subseteq\bbC$ be a region and 
    \begin{equation*}
        \mathscr F = \{f:\Omega\to\bbC\mid f\text{ is holomorphic and } f(\Omega)\subseteq\bbC\backslash\{0,1\}\}.
    \end{equation*}
    Then, $\mathscr F$ is a normal family in $C(\Omega,\wh\bbC)$.
\end{theorem}
\begin{proof}
    To prove that $\mathscr F$ is normal, it suffices to show that for every disk $D$ in $\Omega$, the restriction of $\mathscr F$ to $D$ is normal. Hence, we may suppose without loss of generality that $\Omega = \bbD$. To show normality, we shall show that every sequence of functions in $\mathscr F$ has a subsequence that is uniformly bounded on compact subsets of $\bbD$ or has a subsequence that converges uniformly to $\infty$ on compact subsets of $\bbD$.

    Let $\{f_n\}$ be a sequence of functions in $\mathscr F$. Then, there is a point $\alpha\in\wh C$ such that a subsequence $\{f_{n_k}\}$ of $\{f_n(0)\}$ converges to $\alpha$. Replace $\{f_n\}$ by $\{f_{n_k}\}$.

    \noindent\textbf{Case 1:} $\alpha\in\bbC\backslash\{0,1\}$.

    Consider the analytic covering map $\mu:\bbD\to\bbC\backslash\{0,1\}$ and let $U$ be an evenly covered neighborhood of $\alpha$. Pick a component $V$ of $\mu^{-1}(U)$. Since $\bbD$ is simply connected, there are holomorphic lifts $\wt f_n: \bbD\to\bbD$ such that $\mu\circ\wt f_n = f_n$ and $\wt f_n(0)\in V$ for sufficiently large $n$. This is a sequence of functions that is uniformly bounded on compact subsets of $\bbD$ and hence, has a subsequence $\{\wt f_{n_k}\}$ that converges to a holomorphic function $f:\bbD\to\bbC$. Note that $|f(z)|\le 1$ for all $z\in\bbD$ and if $|f(z)| = 1$ for some $z\in\bbD$, then $f$ must be a constant function $\beta$ due to the Maximum Modulus Principle. In particular, this means that $\wt f_{n_k}(0)\to\beta$. Note that $\mu|_V$ is a biholomorphism and hence, admits a holomorphic (in particular, continuous) inverse. Then, we have 
    \begin{equation*}
        (\mu|_V)^{-1}(\alpha) = \lim_{k\to\infty}(\mu|_V)^{-1}(f_{n_k}(0)) = \beta,
    \end{equation*}
    which is absurd, since $\beta\notin\bbD$. Hence, $|f(z)| < 1$ for all $z\in\bbD$.

    We shall now show that $\{f_{n_k}\}$ is uniformly bounded on compact subsets of $\bbD$, whence we would be done by Montel's Theorem. Let $K\subseteq\bbD$ be a compact set. Then, there is an $M < 1$ such that $|f(z)|\le M$ on $K$. Choose $M < r < 1$. Then, for sufficiently large $k$, we have $|f(z) - \wt f_{n_k}(z)| < r - M$. Thus, for all such $k$, we have $|\wt f_{n_k}(z)| < r$. Note that $\mu$ is bounded on $\overline B(0, r)$ and hence, $f_{n_k} = \mu\circ\wt f_{n_k}$ is uniformly bounded on $K$.

    \textbf{Case 2:} $\alpha = 1$.

    Since $\bbD$ is simply connected and $f_n$ never vanishes on $\bbD$ for all $n$, there is a ``square root'' $g_n: \bbD\to\bbC$. Replacing $g_n$ by $-g_n$ if necessary, we may suppose that $g_n(0)\to -1$ as $n\to\infty$. Further, note that the $g_n$'s have image contained in $\bbC\backslash\{0,1\}$. From our analysis in \textbf{Case 1}, there is a subsequence $\{g_{n_k}\}$ that converges uniformly on compact subsets of $\bbD$. Since $f_{n_k} = g_{n_k}^2$, we are done by once again invoking Montel's Theorem.

    \textbf{Case 3:} $\alpha = 0$.
    Simply replace $f_n$ by $1 - f_n$. This brings us to \textbf{Case 2}.

    \textbf{Case 4:} $\alpha = \infty$.

    Let $g_n = 1/f_n$, which are holomorphic on $\bbD$ since $f_n$'s never vanish on $\bbD$ for all $n$. Since the images of the $g_n$'s are contained in $\bbC\backslash\{0,1\}$, invoking the analysis of the previous cases, there must be a subsequence $\{g_{n_k}\}$ that converges uniformly on compact subsets of $\bbD$ to a holomorphic function $g:\bbD\to\bbC$. Note that $g(0) = 0$ but the $g_n$'s have no zeros and hence, due to Hurwitz's Theorem, $g$ must identically be $0$. It follows that $f_{n_k}(z)\to\infty$ uniformly on compact subsets of $\bbD$.
\end{proof}

\section{Picard's Theorems}

\begin{theorem}[Little Picard]
    Let $f$ be an entire function. If there are two distinct complex numbers that are not in the image of $f$, then $f$ must be constant.
\end{theorem}
\begin{proof}
    Without loss of generality, suppose $f$ misses $0$ and $1$. Recall the analytic covering map $\mu:\bbD\to\bbC\backslash\{0,1\}$. There is a holomorphic lift $\wt f: \bbC\to\bbD$ of $f$. Due to Liouville, $\wt f$ must be constant and hence, so must $f$.
\end{proof}

\begin{theorem}[Great Picard]
    Let $f:\Omega\to\bbC$ have an essential singularity at $0\in\Omega$. Then, there is an $\alpha\in\bbC$ such that for all $\zeta\ne\alpha$, the equation $f(z) = \zeta$ has infinitely many solutions in any punctured neighborhood of $0$ that is contained in $\Omega$.
\end{theorem}
\begin{proof}
    Suppose there is an $R > 0$ such that $B(0,R)\subseteq\Omega$ and $f(B(0, R))$ misses atleast two points in $\bbC$. We may suppose without loss of generality that $0$ and $1$ are missed. Note that we may also choose $R < 1/2$.

    Since $0$ is not a pole, the limit $|f(z)|$ as $z\to 0$ does not tend to $\infty$. Consequently, there is a positive constant $P > 0$ such that for all $R > \delta > 0$, there is a $z\in B(0,\delta)$ with $|f(z)|\le P$. Begin with $\delta = R$ and choose such a $z_1$. Next, set $\delta = z_1$ and pick a corresponding $z_2$ and continue in this fashion. The sequence $\{z_i\}$ is bounded and hence, has a convergent subsequence, say $\{z_{n_k}\}$. Call this sequence $\{\alpha_k\}$.
    
    Define $f_n:\Omega\to\bbC$ by $f_n(z) = f(2\alpha_n z/R)$. Then, due to \thref{thm:montel-caratheodory} $\{f_n\}$ is a normal family and hence, admits a subsequence $\{f_{n_k}\}$ that either converges uniformly on compact subsets of $\Omega$ to either a holomorphic function $g:\Omega\to\bbC$ or to the identically $\infty$ function on $\Omega$.

    Suppose the former case and let $M = \max\{|g(z)|\colon |z| = R/2\}$. Due to uniform convergence on compact subsets of $\Omega$, there is a $k_0$ such that for all $k\ge k_0$, we have $|f_{n_k}(z) - g(z)|\le M$ whenever $|z| = R/2$ and hence, $|f(\alpha_{n_k} z)| = |f_{n_k}(z)|\le 2M$ whenever $|z| = R/2$. Due to the Maximum Modulus Principle, $f(z)$ is bounded by $2M$ on the annulus $|\alpha_{n_k}| < |z| < R/2$. Since $|\alpha_{n_k}|$ grows arbitrarily small, we see that $f(z)$ Is bounded by $2M$ on the annulus $0 < |z| < R/2$. This would mean that $z = 0$ is a removable singularity, a contradiction.

    Consider the latter case, $g\equiv\infty$. But this is obviously not possible since for sufficiently large $n$, $f_n(R/2)$ converges to a finite limit. This completes the proof.
\end{proof}

\end{document}