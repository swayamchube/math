\subsection{Discrete Valuation Rings}

\begin{definition}
    A valuation ring with value group order-isomorphic to $\Z$ is called a \define{discrete valuation ring (DVR)}.
\end{definition}

\begin{theorem}
    Let $R$ be a valuation ring. Then the following are equivalent: 
    \begin{enumerate}[label=(\arabic*)]
        \item $R$ is a DVR. 
        \item $R$ is a PID. 
        \item $R$ is Noetherian.
    \end{enumerate}
\end{theorem}
\begin{proof}
    Let $K$ be the field of fractions of $R$ and $\frakm$ its maximal ideal.

    \noindent $(1)\implies(2)$ Let $v\colon K^\times\to\Z$ be a surjective valuation corresponding to $R$. Let $t\in\frakm$ be such that $v(t) = 1$. For $0\ne x\in\frakm$, $v(x) = n > 0$ for some positive integer $n$. Then $v(x/t^n) = 0$, i.e., $x = ut^n$ for some unit $u\in R^\times$. In particular, this shows that $\frakm = tR$. Now let $0\ne I$ be a proper ideal in $R$, and let 
    \begin{equation*}
        n = \min\left\{v(a)\colon 0\ne a\in I\right\}.
    \end{equation*}
    Clearly $n$ is a positive integer since $I$ is proper. Let $x\in I$ with $v(x) = n$. Then $x R\subseteq I$, and for every $y\in I$, $v(y/x)\ge 0$, so that $y\in xR$. Thus $I = xR$, and $R$ is a PID.


    \noindent$(2)\implies(3)$ is clear.

    \noindent$(3)\implies(2)$ A Noetherian B\'ezout domain is a PID. 

    \noindent$(2)\implies(1)$ Let $t\in\frakm$ be such that $\frakm = tR$. Recall that a PID is a UFD, and let $v$ denote the $t$-adic valuation on $K$. It is not hard to see that $R$ is the valuation ring corresponding to $v$.
\end{proof}

\begin{definition}
    If $R$ is a DVR with maximal ideal $\frakm$, then any element $t\in\frakm$ such that $\frakm = t R$ is said to be a \define{uniformizer} or a \define{uniformizing element} of $R$.
\end{definition}

\begin{remark}
    We note here that a valuation ring whose maximal ideal is principal need not necessarily be a DVR. Indeed, let $K$ be a field, $(R,\frakm_R)$ a DVR of $K$, set $k = R/\frakm_R$, and suppose $\mathfrak R$ is a DVR of $k$. Let $S$ denote the composite of $R$ and $\mathfrak R$. We contend that $S$ is our desired counterexample.

    Let $f\in\frakm_R$ be a uniformizer of $R$, $g\in S$ such that $\overline g = g + \frakm_R$ is a uniformizer of $\mathfrak R$. Then $\frakm_S = \frakm_R + gS$. Further, since $g\notin\frakm_R$, it is a unit in $R$, so that $g^{-1}\in R$. Hence, for any element $h\in\frakm_R$, we can write $h = g\cdot\left(g^{-1}h\right)$, so that $\frakm_R\subseteq gS$. Hence $\frakm_S = gS$ is principal.

    Next, we show that $S$ is not Noetherian. Indeed, consider the ascending chain of ideals in $S$: 
    \begin{equation*}
        (f)\subseteq (f, fg^{-1})\subseteq (f, fg^{-1}, fg^{-2})\subseteq\cdots.
    \end{equation*}
    We claim that all the above inclusions are proper. Indeed, if $fg^{-(n + 1)}\in (f, fg^{-1},\dots, fg^{-n})$ for some $n\ge 0$, then there exist $a_0,\dots a_n\in S$ such that 
    \begin{equation*}
        fg^{-(n + 1)} = a_0f + a_1 fg^{-1} + \dots + a_n fg^{-n}.
    \end{equation*}
    Multiplying out by $f^{-1}g^{n + 1}$, we obtain 
    \begin{equation*}
        a_0 g^{n + 1} + \dots + a_n g = 1,
    \end{equation*}
    which is absurd, since $1\notin\frakm_S = gS$, thereby completing the proof.
\end{remark}

\begin{theorem}\thlabel{equivalent-dvr-non-valuation}
    Let $R$ be a ring. The following conditions are equivalent: 
    \begin{enumerate}[label=(\arabic*)]
        \item $R$ is a DVR. 
        \item $R$ is a local PID but not a field. 
        \item $R$ is a Noetherian local ring, $\dim R > 0$, and the maximal ideal of $R$ is principal. 
        \item $R$ is a one-dimensional normal Noetherian local domain.
    \end{enumerate}
\end{theorem}
\begin{proof}
    $(1)\implies(2)\implies(3)$ is clear. 

    $(3)\implies(1)$ Let $\frakm = tR$ denote the maximal ideal of $R$. Due to Krull's intersection theorem, 
    \begin{equation*}
        \bigcap_{n = 1}^\infty\frakm^n = (0).
    \end{equation*}
    Hence, for every $0\ne x\in R$, there is a non-negative integer $n$ such that $x\in\frakm^n\setminus\frakm^{n + 1}$. Set $v(x) = n$. Now if $x, y\in R\setminus\{0\}$ are such that $v(x) = n$ and $v(y) = m$, then we can find units $u, v\in R^\times$ such that $x = t^n u$ and $y = t^m v$. This shows that $xy = t^{m + n}uv\ne 0$, so that $R$ is an integral domain and $v(xy) = v(x) + v(y)$. Let $K$ denote the fraction field of $R$, and set 
    \begin{equation*}
        v\left(\frac{a}{b}\right) = v(a) - v(b)
    \end{equation*}
    for all $a, b\in R\setminus\{0\}$. This is clearly well-defined, and defines a valuation on $K$ whose value group is $\Z$ and corresponding valuation ring is $R$. Hence $R$ is a DVR.

    $(1)\implies(4)$ Recall that in a DVR the only ideals are $(0)$ and powers of the maximal ideal. Thus the only prime ideals are $(0)$ and $\frakm$, so that $\dim R = 1$. That $R$ is Noetherian follows from it being a PID, and finally recall that every valuation ring is normal.

    $(4)\implies(3)$ Let $\frakm$ denote the maximal ideal of $R$. Due to Nakayama's lemma and the fact that $\dim R = 1$, $\frakm\ne\frakm^2$. Choose $x\in\frakm\setminus\frakm^2$. We shall show that $\frakm = xR$. Note that $\frakm\in\Ass_R\left(R/xR\right)$, so that there exists $y\in R\setminus xR$ such that $(xR : y) = \frakm$. Set $a = yx^{-1}\in K\coloneq\operatorname{Frac}(R)$. Note that $a\notin R$ lest $y\in xR$. Set 
    \begin{equation*}
        \frakm^{-1}\coloneq\left\{\alpha\in K\colon\alpha\frakm\subseteq R\right\}.
    \end{equation*}
    Note that $\frakm^{-1}$ is an $R$-submodule of $K$, and further, by construction, 
    \begin{equation*}
        \frakm\frakm^{-1}\coloneq\left\{\text{$R$-submodule of $K$ generated by } x_iy_i\colon x_i\in \frakm,~y_i\in\frakm^{-1}\right\}\subseteq R\quad\text{ and }\quad R\subseteq\frakm^{-1}.
    \end{equation*}
    In particular, we have the inclusions $\frakm\subseteq\frakm\frakm^{-1}\subseteq R$. Hence, $\frakm\frakm^{-1}\in\left\{\frakm, R\right\}$. If $\frakm^{-1}\frakm = \frakm$, then $a\frakm\subseteq\frakm$. Since $\frakm$ is a finite $R$-module, due to Nakayama's lemma, $a$ must be integral over $R$, but since $R$ is normal, $a\in R$, a contradiction. Thus $\frakm^{-1}\frakm = R$. Next, note that $x\frakm^{-1}\subseteq R$. If $x\frakm^{-1}\subseteq\frakm$, then 
    \begin{equation*}
        xR = x\frakm^{-1}\frakm \subseteq\frakm^2,
    \end{equation*}
    a contradiction. Hence, $x\frakm^{-1} = R$, so that 
    \begin{equation*}
        xR = x\frakm^{-1}\frakm = \frakm,
    \end{equation*}
    thereby completing the proof.
\end{proof}

\subsection{Fractional Ideals and Dedekind domains}

\begin{definition}
    Let $R$ be an integral domain with fraction field $K$. An $R$-submodule $I$ of $K$ is said to be a \define{fractional ideal} if there is a non-zero $\alpha\in R$ such that $\alpha I\subseteq R$. 
\end{definition}

The standard operations on ideals readily generalize to operations on fractional ideals. Indeed, if $I$ and $J$ are fractional ideals, define 
\begin{align*}
    I + J &= \left\{x + y \colon x\in I,~y\in J\right\}\\
    IJ &= \left\{\text{$R$-submodule generated by }x_iy_i\colon x_i\in I,~y_i\in J\right\}.
\end{align*}
Clearly, both $I + J$ and $IJ$ are fractional ideals of $R$. Further, if $S\subseteq R$ is a multiplicative set, then 
\begin{equation*}
    S^{-1}I = \left\{\frac{x}{s}\colon x\in I,~s\in S\right\}
\end{equation*}
is a fractional ideal of $S^{-1}R$.

\begin{lemma}\thlabel{localisation-commutes-with-colon}
    Let $R$ be an integral domain, $M$ and $N$ $R$-submodules of $K = \operatorname{Frac}(R)$. If $N$ is finitely generated, then 
    \begin{equation*}
        S^{-1}\left(M : N\right) = \left(S^{-1}M : S^{-1}N\right).
    \end{equation*}
\end{lemma}
\begin{proof}
    Clearly $S^{-1}\left(M : N\right)\subseteq\left(S^{-1}M : S^{-1} N\right)$. Since $N$ is finitely generated, we can write $N = a_1 R + \dots + a_n R$ for some $a_1, \dots, a_n\in K$. If $x\in \left(S^{-1}M : S^{-1}N\right)$, then for all $1\le i\le n$, $xa_i\in S^{-1}M$. Therefore, there exist $c_i\in S$ such that $c_ixa_i\in M$, whence $c_ix\in\left(M : N\right)$, so that $x\in S^{-1}\left(M : N\right)$.
\end{proof}

\begin{definition}
    An $R$-submodule $I$ of $K$ is said to be \define{invertible} if there exists an $R$-submodule $J$ of $K$ such that $IJ = R$.
\end{definition}

\begin{proposition}\thlabel{invertible-is-finite}
    An invertible $R$-submodule of $K$ must be a finite $R$-module.
\end{proposition}
\begin{proof}
    Let $I$ be an invertible fractional ideal of $R$ and $J$ an $R$-submodule of $K$ such that $IJ = R$. Then we can find $a_1,\dots,a_n\in I$ and $b_1,\dots, b_n\in J$ such that 
    \begin{equation*}
        1 = a_1b_1 + \dots + a_n b_n.
    \end{equation*}
    For each $x\in I$, we can write 
    \begin{equation*}
        x = (xb_1)a_1 + \dots + (xb_n)a_n,
    \end{equation*}
    and note that $xb_i\in R$ for $1\le i\le n$. It follows that $I = a_1 R + \dots + a_n R$.
\end{proof}
\begin{corollary}
    Every invertible $R$-submodule of $K$ is a fractional ideal.
\end{corollary}

\begin{remark}
    Suppose $I$ is an invertible fractional ideal of $R$, and set 
    \begin{equation*}
        I^{-1}\coloneq\left\{\alpha\in K\colon \alpha I\subseteq R\right\}.
    \end{equation*}
    This is clearly an $R$-submodule of $K$. Further, note that $II^{-1}\subseteq R$, so that 
    \begin{equation*}
        I^{-1} = JII^{-1}\subseteq J.
    \end{equation*}
    But since $JI\subseteq R$, we clearly have $J\subseteq I^{-1}$. This shows that $J = I^{-1}$. 
\end{remark}

\begin{theorem}\thlabel{equivalent-invertible-ideal}
    Let $R$ be an integral domain and $I$ an $R$-submodule of $K$. The following are equivalent: 
    \begin{enumerate}[label=(\arabic*)]
        \item $I$ is an invertible fractional ideal.
        \item $I$ is a projective $R$-module. 
        \item $I$ is a finite $R$-module, and for each maximal ideal $\frakm$ of $R$, the fractional ideal $I_\frakm \coloneq I R_\frakm$ of $R_\frakm$ is principal.
    \end{enumerate}
\end{theorem}
\begin{proof}
    $(1)\implies(2)$ Let $a_1,\dots,a_n\in I$ and $b_1,\dots,b_n\in I^{-1}$ be such that $1 = a_1 b_1 + \dots + a_nb_n$. As we have seen in the proof of \thref{invertible-is-finite}, $I = a_1 R + \dots + a_n R$. Let $\pi\colon F\coloneq\displaystyle\bigoplus_{i = 1}^n Re_i\to I$ be the map sending $e_i\mapsto a_i$. Define $\sigma\colon I\to F$ by
    \begin{equation*}
        \sigma(x) = (xb_1)e_1 + \dots + (xb_n)e_n.
    \end{equation*}
    It is then clear that $\pi\circ\sigma = \id_I$, so that $I$ is projective.

    $(2)\implies(1)$ There is a free module $\displaystyle F = \bigoplus_i Re_i$ and an epimorphism $\pi\colon F\to I$ which splits through an $R$-linear map $\sigma\colon I\to F$. Let $a_i = \pi(e_i)$ and $\lambda_i = \pi_i\circ\sigma\colon I\to R$. Note that every $R$-linear map $I\to R$ is multiplication by some element of $K$. Say $\lambda_i(x) = b_i x$ for all $x\in I$. Note that $b_i\in I^{-1}$ for all $i$. Then 
    \begin{equation*}
        \pi(\sigma(x)) = \sum_{i}a_ib_i x\implies\sum_{i}a_ib_i = 1.
    \end{equation*}
    Set $J$ denote the $R$-submodule of $K$ generated by the $b_i$'s. Then $R\subseteq IJ\subseteq R$, and hence $I$ is an invertible fractional ideal.

    $(1)\implies(3)$ That $I$ is a finite $R$-module is the content of \thref{invertible-is-finite}. Further, 
    \begin{equation*}
        R_\frakm = \left(II^{-1}\right)_\frakm = I_\frakm\left(I^{-1}\right)_\frakm,
    \end{equation*}
    so that $I_\frakm$ is an invertible fractional ideal of $R_\frakm$. Due to (2), $I_\frakm$ is a projective $R_\frakm$-module. But since any two elements of $K$ are $R_\frakm$-linearly dependent, $I_\frakm$ must be principal. 

    $(3)\implies(1)$ Note that for every maximal ideal $\frakm$ of $R$,
    \begin{equation*}
        \left(I^{-1}\right)_\frakm = \left(R : I\right)_\frakm = \left(R_\frakm : I_\frakm\right) = \left(I_\frakm\right)^{-1}.
    \end{equation*}
    If $II^{-1}\subsetneq R$, then there is a maximal ideal $\frakm$ containing $II^{-1}$, so that 
    \begin{equation*}
        \frakm R_\frakm\supseteq\left(II^{-1}\right)_\frakm = I_\frakm\left(I^{-1}\right)_\frakm = I_\frakm I_\frakm^{-1} = R_\frakm,
    \end{equation*}
    a contradiction. This completes the proof.
\end{proof}

\begin{theorem}
    Let $R$ be a Noetherian domain, and $\frakp$ a non-zero prime ideal of $R$. If $\frakp$ is invertible, then $\hght\frakp = 1$ and $R_\frakp$ is a DVR.
\end{theorem}
\begin{proof}
    Since $\frakp$ is invertible, due to \thref{equivalent-invertible-ideal}, $\frakp R_\frakp$ is a principal ideal. Since $\dim R_\frakp = \hght\frakp > 0$, it follows from \thref{equivalent-dvr-non-valuation} that $R_\frakp$ is a DVR, and hence $\hght\frakp = \dim R_\frakp = 1$.
\end{proof}

\begin{theorem}\thlabel{intersection-of-localisations}
    Let $R$ be a normal Noetherian domain. Then 
    \begin{enumerate}[label=(\arabic*)]
        \item all prime divisors of a non-zero principal ideal have height $1$. 
        \item $\displaystyle R = \bigcap_{\hght\frakp = 1} R_\frakp$.
    \end{enumerate}
\end{theorem}
\begin{proof}
\begin{enumerate}[label=(\arabic*)]
    \item This follows from Krull's Hauptidealsatz.
    \item Suppose $\displaystyle\frac{a}{b}\in\bigcap_{\hght\frakp = 1}R_\frakp$ with $b\ne 0$. We shall show that $a\in bR$. If $b$ is a unit in $R$, then there is nothing to prove. If $b$ is not a unit in $R$, consider the primary decomposition 
    \begin{equation*}
        bR = \frakq_1 \cap\dots\cap\frakq_r
    \end{equation*}
    with corresponding associated primes $\frakp_1,\dots,\frakp_r$. Due to (1), we know that $\hght\frakp_i = 1$ for all $i$, so that there are no embedded associated primes. Localising at $\frakp_i$ and contracting back to $R$, we have 
    \begin{equation*}
        \frakq_i = bR_\frakp\cap R\ni a.
    \end{equation*}
    Therefore, 
    \begin{equation*}
        a\in\bigcap_{i = 1}^r \frakq_i = bR,
    \end{equation*}
    as desired. \qedhere
\end{enumerate}
\end{proof}

\begin{porism}\thlabel{porism-intersection-localisations}
    Let $R$ be a Noetherian domain. If all associated primes of a non-zero principal ideal have height $1$, then 
    \begin{equation*}
        R = \bigcap_{\hght\frakp = 1} R_\frakp.
    \end{equation*}
\end{porism}

\begin{corollary}
    Let $R$ be a Noetherian domain. Then $R$ is normal if and only if the following two conditions are satisfied: 
    \begin{enumerate}[label=(\roman*)]
        \item for every height $1$ prime $\frakp$, $R_\frakp$ is a DVR; and 
        \item all associated primes of a non-zero principal ideal of $R$ have height $1$.
    \end{enumerate}
\end{corollary}
\begin{proof}
    Necessity is the content of \thref{intersection-of-localisations}. For sufficiency, note that due to \thref{porism-intersection-localisations}, $R = \displaystyle\bigcap_{\hght\frakp = 1} R_\frakp$. But since each $R_\frakp$ is a DVR, it is normal, so that $R$ is normal.
\end{proof}

\begin{definition}
    An integral domain for which every ideal is invertible is called a \define{Dedekind domain}.
\end{definition}

\begin{theorem}
    For an integral domain $R$, the following conditions are equivalent: 
    \begin{enumerate}[label=(\arabic*)]
        \item $R$ is a Dedekind domain. 
        \item $R$ is either a field or a one-dimensional Noetherian normal domain. 
        \item every non-zero ideal of $R$ can be written as a product of a finite number of prime ideals.
    \end{enumerate}
\end{theorem}
\begin{proof}
    $(1)\implies(2)$ Suppose $R$ is a field. Since every ideal is invertible, it is finitely generated, so that $R$ is Noetherian. If $\frakp$ is a non-zero prime ideal of $R$, then due to \thref{equivalent-invertible-ideal}, $\frakp R_\frakp$ is a principal ideal. Due to \thref{equivalent-dvr-non-valuation}, $R_\frakp$ is a DVR and $\hght\frakp = 1$. Since $R$ is not a field, every prime ideal is a maximal ideal, and hence, we can write 
    \begin{equation*}
        R = \bigcap_{0\ne \frakp\in\Spec R} R_\frakp.
    \end{equation*}
    Being the intersection of normal domains, $R$ is also a normal domain. 

    $(2)\implies(1)$ Let $I$ be an ideal of $R$. We shall show that $I$ is invertible. For a maximal ideal $\frakm$ of $R$, note that $R_\frakm$ is a one-dimensional Noetherian normal local ring, which, due to \thref{equivalent-dvr-non-valuation}, is a DVR. In particular, $I R_\frakm$ is a principal ideal. Thus, due to \thref{equivalent-invertible-ideal}, $I$ is invertible.

    $(1)\implies(3)$ That $R$ is Noetherian follows from (2). We prove (3) by Noetherian induction. Let $\scrF$ denote the set of all non-zero proper ideals in $R$ that are not products of prime ideals. If $\scrF$ is non-empty, using the Noetherian-ness of $R$, choose a maximal elmeent $I\in\scrF$. Note that every proper ideal of $R$ properly containing $I$ can be expressed as a product of prime ideals. Let $\frakm$ be a maximal ideal containing $I$. Clearly $I\ne\frakm$ else it has a trivial expression as a product of prime ideals. Since $R\subseteq\frakm^{-1}$, we have $I\subseteq I\frakm^{-1}\subseteq\frakm\frakm^{-1} = R$. If $I\frakm^{-1} = I$, then due to Nakayama's lemma, every element of $\frakm^{-1}$ would be integral over $R$, and therefore must be an element of $R$ due to (2), a contradiction. Thus $I\frakm^{-1}\supsetneq I$, so that it can be expressed as a product of prime ideals. Multiplying this expression by $\frakp$, we see that $I$ can be expressed as a product of prime ideals too, a contradiction. Thus $\scrF$ is empty, as desired.

    $(3)\implies(1)$ We shall show that every prime ideal in $R$ is invertible. The factorization property would then imply the same for all non-zero ideals of $R$. 

    \noindent\underline{Step 1.} Suppose $I$ and $J$ are fractional ideals of $R$ such that $B = IJ$ is an invertible fractional ideal. We shall show that $I$ and $J$ are invertible. First, note that 
    \begin{equation*}
        I^{-1}J^{-1}B = I^{-1}J^{-1}JI\subseteq R\implies I^{-1}J^{-1} = I^{-1}J^{-1}BB^{-1}\subseteq B^{-1}.
    \end{equation*}
    Also, since $B^{-1}IJ = R$, we have the two obvious inclusions $B^{-1}I\subseteq J^{-1}$ and $B^{-1}J\subseteq I^{-1}$. Multiplying these two inclusions, we obtain 
    \begin{equation*}
        B^{-1}\subseteq I^{-1}J^{-1}\implies I^{-1}J^{-1} = B^{-1}.
    \end{equation*}
    Finally, note that 
    \begin{equation*}
        R = BB^{-1} = (II^{-1})(JJ^{-1}),
    \end{equation*}
    so that $II^{-1} = JJ^{-1} = R$, i.e., $I$ and $J$ are invertible. 

    \noindent\underline{Step 2.} Let $\frakp$ be a non-zero prime ideal in $R$. We shall show that for any ideal $I$ of $R$ properly containing $\frakp$, $\frakp = I\frakp$. To this end, it suffices to show that $\frakp\subseteq I\frakp$. Let $a\in I\setminus\frakp$ and set $J = aR + \frakp$. It suffices to show that $\frakp\subseteq J\frakp$ so that we may replace $I$ by $J$ and continue our analysis. Consider prime decompositions of the two ideals 
    \begin{equation*}
        I^2 = \frakp_1\cdots\frakp_r\quad\text{ and }\quad a^2 R + \frakp = \frakq_1 \cdots \frakq_s.
    \end{equation*}
    Clearly each of the $\frakq_i$'s contain $\frakp$. Since $I^2\subseteq\frakp_i$, we have $I\subseteq\frakp_i$, so that $\frakp\subseteq\frakp_i$. Let $\overline R \coloneq R/\frakp$, and for each $x\in R$, let $\overline x$ denote its image in $\overline R$. Note that $\overline R$ is an integral domain with the factorization property (3), and in $\overline R$, we have the decompositions 
    \begin{equation*}
        \overline\frakp_1\cdots\overline\frakp_r = \overline a^2\overline R = \overline\frakq_1 \cdots \overline\frakq_s.
    \end{equation*}
    Since $\overline a^2\overline R$ is an invertible ideal of $\overline R$, due to Step 1, all the $\overline\frakp_i$'s and the $\overline\frakq_j$'s are invertible ideals of $\overline R$. 
    Now let $\overline\frakp_1$ be a minimal element of the set $\{\overline\frakp_1,\dots,\overline\frakp_r\}$. Since $\overline\frakp_1\supseteq\overline\frakq_1\cdots\overline\frakq_s$, using Prime Avoidance, $\overline\frakp_1$ must contain one of the $\overline\frakq_i$'s, say without loss of generality, $\overline\frakq_1$. An analogous argument would give that $\overline\frakq_1$ must contain some $\overline\frakp_j$, which, due to the minimality of $\overline\frakp_1$ must be equal to $\overline\frakp_1$ -- that is, $\overline\frakp_1 = \overline\frakq_1$. Since these ideals are invertible, multiplying with their inverses, we are left with 
    \begin{equation*}
        \overline\frakp_2\cdots\overline\frakp_r = \overline\frakq_2\cdots\overline\frakq_s.
    \end{equation*}
    Continuing in this way, we obtain $r = s$ and $\overline\frakp_i = \overline\frakq_i$ for $1\le i\le r$. In particular, $\frakp_i = \frakq_i$ for $1\le i\le r$, so that 
    \begin{equation*}
        a^2 R + \frakp = \left(aR + \frakp\right)^2 = a^2 R + a\frakp + \frakp^2.
    \end{equation*}
    Hence, every $x\in\frakp$ can be written as 
    \begin{equation*}
        x = a^2 y + az + w\qquad\text{where}\quad y\in R,~z\in\frakp,~\text{ and }~w\in\frakp^2.
    \end{equation*}
    Thus $a^2y\in\frakp$ -- but since $a\notin\frakp$, $y\in\frakp$. This gives
    \begin{equation*}
        \frakp\subseteq a\frakp + \frakp^2 = I\frakp,
    \end{equation*}
    as desired. 

    \noindent\underline{Step 3.} Let $0\ne a\in R$. Then in the factorization $bR = \frakp_1\cdots\frakp_r$, all the $\frakp_i$'s are maximal. Indeed, if $I$ is any ideal properly containing $\frakp_i$, then due to Step 2, $\frakp_i = I\frakp_i$. As we have already argued, every $\frakp_i$ is invertible, so that $I = R$. 

    \noindent\underline{Step 4.} Let $\frakp$ be a non-zero prime ideal in $R$, and let $0\ne a\in R$. If $aR = \frakp_1\cdots\frakp_r$, then due to Step 3, each $\frakp_i$ is maximal. Since $\frakp\supseteq aR$, there is an index $i$ such that $\frakp_i\subseteq\frakp$. The maximality of $\frakp_i$ forces $\frakp_i = \frakp$. In particular, $\frakp$ is invertible, thereby completing the proof.
\end{proof}