\subsection{Hilbert Polynomials}

\begin{theorem}\thlabel{noetherian-graded-ring}
    An $\N$-graded ring $\displaystyle R = \bigoplus_{n\ge 0} R_n$ is Noetherian if and only if $R_0$ is Noetherian and $R$ is a finitely generated $R_0$-algebra.
\end{theorem}
\begin{proof}
    The converse direction follows from Hilbert's basis theorem. We prove the forward direction. Suppose $R$ is a Noetherian ring. Since $R_0 = R/R_+$ where $\displaystyle R_+ = \bigoplus_{n > 0} R_n$, it follows that $R_0$ is Noetherian. Next, since $R_+$ is a finitely generated ideal, we may pick homogeneous elements $x_1,\dots,x_r\in R_+$ generating it as an $R$-module. Using induction on $n\ge 0$, we shall show that $R_n\subseteq R_0[x_1,\dots,x_r]$ for all $n\ge 0$. Let $d_i > 0$ denote the degree of $x_i$. Then
    \begin{equation*}
        R_n = \sum_{i = 1}^r x_i R_{n - d_i},
    \end{equation*}
    with the convention that $R_j = 0$ for $j < 0$. Using the inductive hypothesis, every element of $R_{n - d_i}$ can be written as a polynomial in the $x_i$'s with coefficients in $R_0$. From the above relation, it follows that every element of $R_n$ can be written as a polynomial in the $x_i$'s with coefficients in $R_0$.
\end{proof}

\begin{proposition}
    Let $R = \displaystyle\bigoplus_{n\ge 0} R_n$ be a Noetherian graded ring, and $M = \displaystyle\bigoplus_{n\ge 0} M_n$ a finite graded $R$-module. Then each $M_n$ is a finite $R_0$-module.
\end{proposition}
\begin{proof}
    The case $M = R$ has been dealt with in the proof of \thref{noetherian-graded-ring}. We may choose the generators $\omega_1,\dots,\omega_r$ of $M$ as an $R$-module to be homogeneous. Let $d_i$ denote the degree of each $\omega_i$. Note that $M_+\coloneq\displaystyle\bigoplus_{n\ge 0} M_n$ is a submodule of $M$, it follows that $M_0 = M/M_+$ is a finite $R$-module which is annihilated by $R_+$. Therefore, $M_0$ is a finite $R$-module. Next, for $n > 0$, we can write 
    \begin{equation*}
        M_n = \sum_{i = 1}^r R_{n - d_i}\omega_i.
    \end{equation*}
    Since each $R_j$ is a finite $R_0$-module, it follows that each $M_n$ is a finite $R_0$-module.
\end{proof}

\begin{definition}
    Let $R = \displaystyle\bigoplus_{n\ge 0} R_n$ be a Noetherian graded ring with $R_0$ Artinian. For a finite graded $R$-module $M = \displaystyle\bigoplus_{n\ge 0} M_n$, we define its \define{Hilbert series} to be 
    \begin{equation*}
        P(M, t)\coloneq \sum_{n = 0}^\infty\lambda_{R_0}(M_n) t^n\in\Z\llbracket t\rrbracket.
    \end{equation*}
    Note that each $M_n$ is a finite $R_0$-module, and hence, is Artinian, so that it has finite length as an $R_0$-module.
\end{definition}

\begin{theorem}[Hilbert-Serre]\thlabel{hilbert-serre}
    Let $R = \displaystyle\bigoplus_{n\ge 0} R_n$ be a Noetherian graded ring with $R_0$ Artinian, and let $M =\displaystyle\bigoplus_{n\ge 0} M_n$ be a finite graded $R$-module. Suppose $x_1,\dots,x_r\in R$ are homogeneous elements such that $R$ is generated by $x_1,\dots,x_r$ as an $R_0$-algebra. If each $x_i$ has degree $d_i > 0$, then the Hilbert series $P(M, t)$ is a rational function, that is, there is a polynomial $f(t)\in\Z[t]$ such that 
    \begin{equation*}
        P(M, t) = \frac{f(t)}{\prod_{i = 1}^r \left(1 - t^{d_i}\right)}\quad\text{ in } \Z\llbracket t\rrbracket.
    \end{equation*}
\end{theorem}
\begin{proof}
    We shall prove this statement by induction on $r\ge 0$. If $r = 0$, then $R = R_0$, so that $M$ is an Artinian $R$-module. Since there is a descending chain of $R$-submodules 
    \begin{equation*}
        M\supseteq\bigoplus_{n\ge 1} M_n\supseteq \bigoplus_{n\ge 2} M_n\supseteq\cdots,
    \end{equation*}
    we must have that $M_n = 0$ for $n\gg 0$, that is, $P(M, t)$ is a polynomial in $t$, thereby establishing the theorem in this case. 

    Suppose now that $r > 0$. For $n\ge 0$, let $K_n$ and $L_{n + d_r}$ denote the kernel and cokernel of the map $M_n\xrightarrow{\cdot x_r} M_{n + d_r}$. The short exact sequence 
    \begin{equation*}
        0\to K_n\to M_n\xrightarrow{\cdot x_r} M_{n + d_r}\to L_{n + d_r}\to 0
    \end{equation*}
    gives us 
    \begin{equation}
        \lambda(K_n) - \lambda(M_n) + \lambda(M_{n + d_r}) - \lambda(L_{n + d_r}) = 0.
    \end{equation}
    The above convention is such that $L_n = M_n$ for $n < d_r$. Set $K = \displaystyle\bigoplus_{n\ge 0} K_n$ and $L = \displaystyle\bigoplus_{n\ge 0} L_n$. Note that $K_n$ is annihilated by $x_r$, and since $L = M/x_r M$ as a graded $R$-module, it is also annihilated by $x_r$. In particular, $K$ and $L$ are graded $\overline R\coloneq R/x_r R$-modules. Multiplying the above equation by $t^{n + d_r}$ and summing over $n\in\Z$, we obtain:
    \begin{equation*}
        t^{d_r}P(K, t) - t^{d_r} P(M, t) + P(M, t) - P(L, t) = 0.
    \end{equation*}
    Since $\overline R$ is an $R_0$-algebra generated by $x_1,\dots,x_{r - 1}$, the induction hypothesis applies so that there are polynomials $g(t), h(t)\in\Z[t]$ such that 
    \begin{equation*}
        P(K, t) = \frac{g(t)}{\prod_{i = 1}^{r - 1}(1 - t^{d_i})}\quad\text{ and }\quad P(L, t) = \frac{h(t)}{\prod_{i = 1}^{r - 1}(1 - t^{d_i})},
    \end{equation*}
    whence the conclusion follows.
\end{proof}

\begin{definition}
    In \thref{hilbert-serre} if $d_1 = \dots = d_r = 1$, then there is an integer polynomial $f(t)\in\Z[t]$ with $f(1)\ne 0$ and a $d\ge 0$ such that 
    \begin{equation*}
        P(M, t) = \frac{f(t)}{(1 - t)^d}\in\Z\llbracket t\rrbracket.
    \end{equation*}
    This integer $d$ is denoted by $d(M)$.
\end{definition}

\begin{corollary}\thlabel{corollary-hilbert-serre}
    If $d_1 = \dots = d_r = 1$ in \thref{hilbert-serre} and $d = d(M)$, then there is a polynomial $\varphi_M(X)\in\Q[X]$ such that for $n\gg 0$, $\lambda(M_n) = \varphi_M(n)$.
\end{corollary}

\begin{definition}
    The polynomial $\varphi_M(X)\in\Q[X]$ is called the \define{Hilbert polynomial} of the graded $R$-module $M$. 
\end{definition}

\subsection{Samuel Functions}

Let $(A,\frakm)$ be a Noetherian local ring, $M$ a finite $A$-module, and $\frakq$ an $\frakm$-primary ideal. The maximal ideal defines filtrations on $A$ and $M$: 
\begin{equation*}
    A\supseteq\frakq\supseteq\frakq^2\supseteq\cdots\quad\text{ and }\quad M\supseteq\frakq M\supseteq\frakq^2 M\supseteq\cdots.
\end{equation*}
Let 
\begin{equation*}
    \gr_\frakq(A) \coloneq \bigoplus_{n\ge 0}\frakq^n/\frakq^{n + 1}\quad\text{ and }\quad\gr_\frakq(M)\coloneq\bigoplus_{n\ge 0}\frac{\frakq^nM}{\frakq^{n + 1}M}
\end{equation*}
denote the corresponding associated graded objects. Note that $\gr_\frakq(M)$ is a finite graded $\gr_\frakq(A)$-module. If $\frakq$ is generated as an $A$-module by $x_1,\dots,x_r$, and $\overline x_1,\dots,\overline x_r\in\frakq/\frakq^2$ denote the corresponding images, then $\gr_\frakq(A)$ is generated as an $A/\frakq$-algebra by $\overline x_1,\dots,\overline x_r$, which are homogeneous elements of degree $1$. Further, since $A/\frakq$ is Artinian, we may talk about the Hilbert series of $\gr_\frakq(M)$ with respect to $\gr_\frakq(A)$. Now define 
\begin{equation*}
    \chi_M^\frakq(n) = \sum_{i = 0}^n \lambda_{A/\frakq}\left(\frac{\frakq^i M}{\frakq^{i + 1} M}\right) = \lambda_{A/\frakq}\left(\frac{M}{\frakq^{n + 1}M}\right).
\end{equation*}
Recall from \thref{corollary-hilbert-serre} that $\lambda_{A/q}\left(\frac{\frakq^n M}{\frakq^{n + 1}M}\right)$ is a polynomial of degree $d\left(\gr_\frakq(M)\right)$ for $n\gg 0$. It follows that $\chi_M^\frakq(n)$ is a polynomial of degree $d\left(\gr_{\frakq}(M)\right) + 1$ for $n\gg 0$.

\begin{proposition}
    If $\frakp$ and $\frakq$ are two $\frakm$-primary ideals, then $\deg\chi_M^\frakp = \deg\chi_M^\frakq$.
\end{proposition}
\begin{proof}
    There are integers $a > 0$ and $b > 0$ such that $\frakp^a\subseteq\frakq$ and $\frakq^b\subseteq\frakp$. Thus, $\frakp^{a(n + 1)}\subseteq\frakq^{n + 1}$ and $\frakq^{b(n + 1)}\subseteq\frakp^{n + 1}$. In particular, this gives the inequalities 
    \begin{equation*}
        \chi^\frakp_M(a(n + 1) - 1)\ge\chi^\frakq_M(n)\quad\text{ and }\quad\chi^\frakq_M(b(n + 1) - 1)\ge\chi^\frakp_M(n)
    \end{equation*}
    for every positive integer $n$. Thus $\deg\chi^\frakp_M = \deg\chi^\frakq_M$, as desired. 
\end{proof}

\begin{definition}
    We write $d(M)$ to denote $\deg\chi^\frakq_M$ for some $\frakm$-primary ideal $\frakq$.
\end{definition}

\begin{theorem}\thlabel{d-in-ses}
    Let $(A,\frakm)$ be a Noetherian local ring and $0\to M'\to M\to M''\to 0$ a short exact sequence of finite $A$-modules. Then 
    \begin{equation*}
        d(M) = \max\left\{d(M'), d(M'')\right\}.
    \end{equation*}
    Further, if $\frakq$ is an $\frakm$-primary ideal of $A$, then $\chi^\frakq_{M} - \chi^\frakq_{M''}$ and $\chi^\frakq_{M'}$ have the same leading coefficient.
\end{theorem}
\begin{proof}
    Let $\frakq$ be an $\frakm$-primary ideal and let $\lambda\coloneq\lambda_{A/\frakq}$ for brevity. Further, we may assume that $M'\subseteq M$ and $M'' = M/M'$. We have 
    \begin{equation*}
        \chi^\frakq_{M''}(n) = \lambda\left(\frac{M''}{\frakq^{n + 1}M''}\right) = \lambda\left(\frac{M}{M' + \frakq^{n + 1}M}\right) = \lambda\left(\frac{M}{\frakq^{n + 1}M}\right) - \underbrace{\lambda\left(\frac{M' + \frakq^{n + 1}M}{\frakq^{n + 1}M}\right)}_{\varphi(n)}.
    \end{equation*}
    Note that $\varphi(n)$ is a polynomial for $n\gg0$. Note that 
    \begin{equation*}
        \frac{M' + \frakq^{n + 1}M}{\frakq^{n + 1}M} = \frac{M'}{M'\cap\frakq^{n + 1}M}, 
    \end{equation*}
    and from the Artin-Rees lemma, we know that the filtration $\left(M'\cap\frakq^{n + 1}\right)_{n\ge 0}$ is $\frakq$-stable. In particular, there is an integer $N > 0$ such that whenever $n > N$, we have 
    \begin{equation*}
        \frakq^{n + 1}M'\subseteq M'\cap\frakq^{n + 1}M = \frakq^{n - N}\left(M'\cap\frakq^{N + 1}M\right)\subseteq\frakq^{n - N}M'.
    \end{equation*}
    This gives $\chi^\frakq_{M'}(n)\ge\varphi(n)\ge\chi^\frakq_{M'}(n - N)$ for all $n > N$. This shows that $\deg\varphi = d(M')$ and $\varphi$ and $\chi^\frakq_{M'}$ have the same leading coefficient. The conclusion now follows, since all polynomials involved have positive leading coefficients.
\end{proof}

\begin{corollary}
    If $M\onto N$ is an epimorphism of $A$-modules, then $d(N)\le d(M)$.
\end{corollary}
\begin{proof}
    Apply the theorem to the short exact sequence $0\to\ker\to M\to N\to 0$.
\end{proof}

\begin{definition}
    Let $(A,\frakm)$ be a Noetherian local ring and $M$ a finite $A$-module. We define the \define{Chevalley dimension} of $M$ to be 
    \begin{equation*}
        \delta(M)\coloneq\inf\left\{n\colon\exists~x_1,\dots,x_n\in\frakm\text{ such that }\lambda_A\left(\frac{M}{(x_1,\dots,x_n)M}\right) < \infty\right\}.
    \end{equation*}
\end{definition}

\begin{theorem}[Dimension Theorem]\thlabel{dimension-theorem}
    Let $(A,\frakm)$ be a Noetherian local ring and $M$ a finite $A$-module. Then 
    \begin{equation*}
        \dim\Supp_R(M) \eqcolon\dim M = d(M) = \delta(M).
    \end{equation*}
\end{theorem}
\begin{proof}
    We prove this theorem in three steps by proving the sequence of inequalities $\dim M\le d(M)\le\delta(M)\le\dim M$.

    \noindent\underline{Step 1.} We show that $d(M)\ge\dim M$. To this end, we first establish this inequality for $M = A$ by induction on $d(A)$. If $d(A) = 0$, then $\lambda(A/\frakm^n)$ is eventually constant, so that $\frakm^n = \frakm^{n + 1}$ for $n\gg 0$, equivalently, due to Nakayama's lemma, $\frakm^n = 0$, that is, $\dim A = 0$. Next, suppose $d(A) > 0$ and consider a strictly ascending chain of prime ideals $\frakp_0\subseteq\frakp_1\subseteq\dots\subseteq\frakp_e$. If $e = 0$, then there is nothing to prove. Suppose now that $e > 0$ and choose $x\in\frakp_1\setminus\frakp_0$. Let $B = A/(\frakp_0 + xA)$, which is a ring and an $A$-module fitting into a short exact sequence of $A$-modules 
    \begin{equation*}
        0\to A/\frakp_0\xrightarrow{\cdot x} A/\frakp_0\to B\to 0.
    \end{equation*}
    Due to \thref{d-in-ses}, $d_A(B) < d_A(A/\frakp_0)\le d_A(A)$. Further, using the induction hypothesis, $\dim B\le d_B(B) = d_A(B) < d_A(A)$. Since the image of the strictly ascending chain of prime ideals $\frakp_1\subseteq\dots\subseteq\frakp_e$ in $B$ is also a strictly ascending chain of prime ideals, $e - 1 < d_A(A)$, so that $e\le d_A(A)$. Taking a supremum over all such $e$'s, $\dim A\le d(A)$.

    Now, let $M$ be a finite $A$-module. One can find a filtration 
    \begin{equation*}
        0 = M_0\subseteq M_1\subseteq\dots\subseteq M_r = M
    \end{equation*}
    such that $M_i/M_{i - 1}\cong A/\frakp_i$ as an $A$-module for some prime ideal $\frakp_i$. Repeatedly invoking \thref{d-in-ses}, 
    \begin{equation*}
        d(M) = \sup_{1\le i\le r}d_A(A/\frakp_i) = \sup_{1\le i\le r} d_{A/\frakp_i}(A/\frakp_i)\ge\sup_{1\le i\le r}\dim A/\frakp_i = \dim M,
    \end{equation*}
    since $\Supp_R(M) = V(\frakp_1)\cup\dots\cup V(\frakp_r)$, as desired.

    \hfill

    \noindent\underline{Step 2.} We show that $\delta(M)\ge d(M)$. Let $s = \delta(M)$. If $s = 0$, then $M$ is Artinian, so that $\lambda_R(M) < \infty$, i.e., $\chi_M^\frakm$ is bounded, and hence $d(M) = 0$.
    
    Let $x_1,\dots,x_s\in\frakm$ be such that $\lambda_A\left(\frac{M}{(x_1, \dots, x_s)M}\right) < \infty$, and set $M_i = M/(x_1,\dots,x_i)M$. Clearly $\delta(M_i) = s - i$. Further, we have 
    \begin{align*}
        \lambda_R(M_1/\frakm^n M_1) &= \lambda_R\left(\frac{M}{x_1 M + \frakm^n M}\right)\\
        &= \lambda_R(M/\frakm^n M) - \lambda_R\left(\frac{x_1 M + \frakm^n M}{\frakm^n M}\right).
    \end{align*}
    The map $M\xrightarrow{\cdot x_1} M/\frakm^n M$ has kernel $(\frakm^n M\colon x_1)$ and image equal to $\frac{x_1 M + \frakm^n M}{\frakm^n M}$, and hence 
    \begin{equation*}
        \lambda_R(M_1/\frakm^n M_1) = \lambda_R(M/\frakm^n M) - \lambda_R\left(\frac{M}{(\frakm^n\colon x_1) M}\right)\ge\lambda_R(M/\frakm^n M) - \lambda_R(M/\frakm^{n - 1}M) = \lambda_R\left(\frakm^{n - 1}M /\frakm^n M\right).
    \end{equation*}
    The right hand side is a polynomial of degree $d(M) - 1$ for $n\gg 0$. Thus $d(M_1)\ge d(M) - 1$. Inductively, $d(M_i)\ge d(M) - i$, in particular, $0 = d(M_s)\ge d(M) - s$, whence $\delta(M)\ge d(M)$.

    \hfill

    \noindent\underline{Step 3.} Finally, we show that $\dim M\ge \delta(M)$ by induction on $\dim M$. If $\dim M = 0$, then $\Supp_R(M) = \{\frakm\}$, so that $\lambda_R(M) < \infty$, i.e., $\delta(M) = 0$. Now suppose $\dim M > 0$ and let $\frakp_1,\dots,\frakp_r$ denote the associated primes of $M$. Since $\dim M > 0$, $\frakm$ is not an associated prime, so that by prime avoidance, there is an element $\displaystyle x_1\in\frakm\setminus\bigcup_{i = 1}^r\frakp_r$. Let $M_1 = M/x_1 M$, so that $\dim M_1\le\dim M - 1$, and clearly $\delta(M)\le\delta(M_1) + 1$, so that $\delta(M)\le\dim M$, as desired.
\end{proof}

\begin{theorem}[Krull's Hauptidealsatz]\thlabel{hauptidealsatz}
    Let $R$ be a Noetherian ring and $I = (a_1,\dots,a_r)$ a proper ideal in $R$. If $\frakp$ is a minimal prime ideal containing $I$, then $\hght\frakp\le r$.
\end{theorem}
\begin{proof}
    Note that $I R_\frakp$ is an $\frakp R_\frakp$-primary ideal, so that $\delta(R_\frakp)\le r$. Hence, $\hght\frakp = \dim R_\frakp\le r$.
\end{proof}

\begin{theorem}
    Let $\frakp$ be a prime ideal of height $r\ge 0$ in a Noetherian ring $R$. 
    \begin{enumerate}[label=(\arabic*)]
        \item $\frakp$ is a minimal prime over some ideal $(a_1,\dots,a_r)$ generated by $r$ elements.
        \item if $b_1,\dots,b_s\in\frakp$, then $\hght\frakp/(b_1,\dots,b_s)\ge r - s$.
        \item if $a_1,\dots,a_r$ are as in (1), we have 
        \begin{equation*}
            \hght\frakp/(a_1,\dots,a_i) = r - i\quad\text{ for }\quad 1\le i\le r.
        \end{equation*}
    \end{enumerate}
\end{theorem}
\begin{proof}
\begin{enumerate}[label=(\arabic*)]
    \item We have $\dim R_\frakp = \hght\frakp = r$, so that there exist $\frac{a_1}{s_1},\dots,\frac{a_r}{s_r}\in R_\frakp$ generating an $\frakp R_\frakp$-primary ideal. Let $I = (a_1,\dots,a_r)$. Then $IR_\frakp$ is $\frakp R_\frakp$-primary, so that $\frakp$ is a minimal prime over $I$, as desired. 
    \item Set $\overline R = A/(b_1,\dots,b_s)$ and $\overline\frakp = \frakp/(b_1,\dots,b_s)$. Let $t = \hght\overline\frakp$. Due to (1), there exist $\overline c_1,\dots,\overline c_t\in\overline R$ such that $\overline\frakp$ is a minimal prime containing $(\overline c_1,\dots,\overline c_t)\overline R$. Pick lifts $c_i$ of $\overline c_i$. Then $\frakp$ is a minimal prime containing $(b_1,\dots,b_s,c_1,\dots,c_t)$. By \thref{hauptidealsatz}, $s + t\ge\hght\frakp = r$, as desired. 
    \item Note that $\overline\frakp = \frakp/(a_1,\dots,a_i)$ is a minimal prime containing $(\overline a_{i + 1},\dots,\overline a_r)$. Thus, due to \thref{hauptidealsatz}, $\hght\overline\frakp\le r - i$. But due to (2), $\hght\overline\frakp\ge r - i$ whence the equality follows. \qedhere
\end{enumerate}
\end{proof}

