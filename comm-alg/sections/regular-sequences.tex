\subsection{The Koszul Complex}

\begin{definition}
    Let $A$ be a ring and $M$ an $A$-module. An element $a\in A$ is said to be \define{$M$-regular} if $a$ is a non-zero-divisor on $M$. A sequence $\ul a = a_1,\dots, a_n$ is said to be an \define{$M$-sequence} if the following two conditions hold: 
    \begin{enumerate}[label=(\arabic*)]
        \item $a_i$ is $M/(a_1,\dots,a_{i - 1})M$-regular for $1\le i\le n$. 
        \item $M\ne (a_1,\dots, a_n)M$.
    \end{enumerate}
\end{definition}

\begin{remark}
    Note that the permutation of an $M$-sequence need not be an $M$-sequence.
\end{remark}

\begin{lemma}\thlabel{non-trivial-relation-regular-sequence}
    If $a_1,\dots,a_n$ is an $M$-sequence, and if $a_1\xi_1 + \dots + a_n\xi_n = 0$ with $\xi_i\in M$ for $1\le i\le n$, then $\xi_i\in (a_1,\dots,a_n)M$ for all $1\le i\le n$.
\end{lemma}
\begin{proof}
    Note that $a_n\xi_n = -a_1\xi_1 - \dots - a_{n - 1}\xi_{n - 1}\in (a_1,\dots,a_{n - 1})M$, so that $\xi_n\in (a_1,\dots,a_{n - 1})M$, and we can write 
    \begin{equation*}
        \xi_n = a_1\eta_1 + \dots + a_{n - 1}\eta_{n - 1}
    \end{equation*}
    for some $\eta_1,\dots,\eta_{n - 1}\in M$. Substituting this back, we have 
    \begin{equation*}
        a_1(\xi_1 + a_n\eta_1) + \dots + a_{n - 1}(\xi_{n - 1} + a_n\eta_{n - 1}) = 0.
    \end{equation*}
    Repeating our above argument, we have 
    \begin{equation*}
        \xi_{n - 1} + a_n\eta_{n - 1}\in (a_1,\dots,a_{n - 2}) M,
    \end{equation*}
    so that $\xi_{n - 1}\in (a_1,\dots,a_n) M$. Continuing this way, we have our desired conclusion.
\end{proof}

\begin{theorem}\thlabel{exponentiating-regular-sequences}
    If $a_1,\dots,a_n$ is an $M$-sequence, then so is $a_1^{m_1},\dots,a_n^{m_n}$ for any positive integers $m_1,\dots,m_n$.
\end{theorem}
\begin{proof}
    Note that it suffices to show that for any positive integer $m$, $a_1^m, a_2,\dots, a_n$ is an $M$-sequence. We shall prove this statement by induction on $m\ge 1$. The base case with $m = 1$ is trivial. Since $a_1$ is $M$-regular, so is $a_1^m$. Suppose for $i > 1$, we have the relation 
    \begin{equation*}
        a_i\omega = a_1^m\xi_1 + \dots + a_{i - 1}\xi_{i - 1}\quad\text{ with }\quad \xi_j\in M.
    \end{equation*}
    Due to the induction hypothesis, we know that $a_1^{m - 1}, a_2,\dots,a_i$ is an $M$-sequence, so that by \thref{non-trivial-relation-regular-sequence}, there exist $\eta_1,\dots,\eta_{i - 1}\in M$ such that 
    \begin{equation*}
        \omega = a_1^{m - 1}\eta_1 + a_2\eta_2 + \dots + a_{i - 1}\eta_{i - 1}.
    \end{equation*}
    Hence, we obtain 
    \begin{equation*}
        0 = a_1^{m - 1}(a_1\xi_1 - a_i\eta_1) + a_2(\xi_2 - a_i\eta_2) + \dots + a_{i - 1}(\xi_{i - 1} - a_i\eta_{i - 1}).
    \end{equation*}
    Invoking \thref{non-trivial-relation-regular-sequence} again, 
    \begin{equation*}
        a_1\xi_1 - a_i\eta_1\in(a_1^{m - 1}, a_2,\dots,a_{i - 1})M,
    \end{equation*}
    and hence, 
    \begin{equation*}
        a_i\eta_1\in(a_1, \dots, a_{i - 1})M.
    \end{equation*}
    But since $a_1,\dots,a_i$ is an $M$-sequence, 
    \begin{equation*}
        \eta_1\in (a_1,\dots,a_{i - 1})M.
    \end{equation*}
    This shows that $\omega\in (a_1^m, a_2,\dots,a_{i - 1})M$, as desired.
\end{proof}

\begin{definition}
    Let $A$ be a ring and $x_1,\dots,x_n\in A$. We define a complex $K_\bullet$ as follows: set $K_0 = A$, and $K_p = 0$ if $p < 0$ or $p > n$. For $1\le p\le n$, set 
    \begin{equation*}
        K_p = \bigoplus_{1\le i_1 < \dots < i_p\le n} Ae_{i_1,\dots,i_p}.
    \end{equation*}
    The differential $\rmd\colon K_p\to K_{p - 1}$ is defined by 
    \begin{equation*}
        \rmd\left(e_{i_1,\dots,i_p}\right) = \sum_{k = 1}^p (-1)^{k - 1} x_{i_k} e_{i_1,\dots,\wh{i_k},\dots,i_p}\in K_{p - 1}
    \end{equation*}
    for $1\le p\le n$, and $\rmd = 0$ for $p\le 0$ or $p > n$. This is called the \define{Koszul complex} corresponding to the sequence $\ul x = x_1,\dots,x_n$ and is often denoted by $K(\ul x)_\bullet$. If $M$ is an $A$-module, we set $K(\ul x, M)_\bullet = K(\ul x)_\bullet\otimes_A M$.
\end{definition}

\begin{interlude}[Tensor Product of Complexes]
    Let $\left(A_\bullet, \rmd\right)$ and $\left(B_\bullet, \rmd'\right)$ be two complexes of $R$-modules. We define the complex $\left(A\otimes_R B\right)_\bullet$ as 
    \begin{equation*}
        \left(A\otimes_R B\right)_p = \bigoplus_{i + j = p} A_i\otimes_R B_j,
    \end{equation*}
    with the differential given by 
    \begin{equation*}
        \rmd\left(a_i\otimes b_j\right) = \rmd a_i\otimes b_j + (-1)^i a_i\otimes\rmd b_j
    \end{equation*}
    where $a_i\in A_i$ and $b_j\in B_j$ for some integers $i$ and $j$.
\end{interlude}

\begin{remark}
    It is straightforward to check that $K(\ul x)_\bullet$ is a complex, i.e., $\rmd^2 = 0$. Further, for each $x\in A$, consider the complex $C(x)_\bullet$ given by 
    \begin{equation*}
        \cdots\to 0\to A\xrightarrow{\cdot x} A\to 0\to\cdots
    \end{equation*}
    with $A$ in degrees $0$ and $1$. One can then check that 
    \begin{equation*}
        K(\ul x) \cong C(x_1)\otimes_A \cdots\otimes_A C(x_n).
    \end{equation*}
    Since the tensor product of complexes is commutative (up to isomorphism), we note that permuting the sequence $\ul x = x_1,\dots,x_n$ results in an isomorphic complex.
\end{remark}

\begin{definition}
    Let $M$ be an $A$-module and $\ul x = x_1,\dots,x_n$ be a sequence in $M$. We define the \define{Koszul homology modules} to be the homology modules of the complex $K(\ul x, M)_\bullet$ and are denoted $H_i\left(\ul x, M\right)$ for all $i\in\Z$.
\end{definition}

It is straightforward to see that 
\begin{equation*}
    H_0(\ul x, M)\cong M/(x_1,\dots,x_n)M\quad\text{ and }\quad H_n(\ul x, M) = \left(0 :_M \ul x\right) = \left\{\xi\in M\colon x_1\xi = \dots = x_n\xi = 0\right\}.
\end{equation*}

For a complex $C_\bullet$ and $x\in A$, let $C(x)_\bullet$ denote the complex $C_\bullet\otimes K(x)_\bullet$.

\begin{theorem}\thlabel{koszul-long-exact-sequence}
    Let $C_\bullet$ be a complex of $A$-modules and $x\in A$. Then there is an exact sequence of complexes 
    \begin{equation*}
        0\to C_\bullet\to C(x)_\bullet\to \wt C_\bullet\to 0,
    \end{equation*}
    where $\wt C_\bullet$ is given by $\wt C_p = C_{p + 1}$ with the same differential. This short exact sequence gives rise to a corresponding homology long exact sequence: 
    \begin{equation*}
        \cdots\to H_p\left(C_\bullet\right)\to H_p\left(C(x)_\bullet\right)\to H_{p - 1}\left(C_\bullet\right) = H_p\left(\wt C_\bullet\right)\xrightarrow{(-1)^{p - 1}x} H_{p - 1}\left(C_\bullet\right)\to\cdots.
    \end{equation*}
    Further, we have $x\cdot H_p\left(C(x)_\bullet\right) = 0$ for all $p\in\Z$.
\end{theorem}
\begin{proof}
    Let $\rmd$ denote the differential of the complex $C_\bullet$. The complex $K(x)_\bullet$ is given by 
    \begin{equation*}
        \cdots\to 0 \to A e_1\xrightarrow{e_1\mapsto x} A\to 0\to\cdots,
    \end{equation*}
    so that we may identify
    \begin{equation*}
        C(x)_p = C_p\oplus C_{p - 1},
    \end{equation*}
    with the differential 
    \begin{equation*}
        \wt\rmd\left(\xi, \eta\right) = \left(\rmd\xi + (-1)^{p - 1} x\eta, \rmd\eta\right).
    \end{equation*}
    Consider the short exact sequence of complexes: 
    \begin{equation*}
        \xymatrix{
            & 0\ar[d] & 0\ar[d] & \\
            \cdots\ar[r] & C_p\ar[r]\ar[d] & C_{p - 1}\ar[d]\ar[r] & \cdots\\
            \cdots\ar[r] & C_p\oplus C_{p - 1}\ar[r]\ar[d] & C_{p - 1}\oplus C_{p - 2}\ar[r]\ar[d] & \cdots\\
            \cdots\ar[r] & C_{p - 1}\ar[r]\ar[d] & C_{p - 2}\ar[r]\ar[d] & \cdots\\
            & 0 & 0 & 
        }
    \end{equation*}
    where the vertical maps are the canonical inclusions and the projections. That these maps are indeed chain maps is easy to check. 
    
    It remains to compute the boundary map $H_p(\wt C_\bullet)\to H_{p - 1}(C_\bullet)$. Let $\eta\in\wt C_p = C_{p -1}$ be such that $\rmd \eta = 0$. Consider $(0, \eta)\in C_p\oplus C_{p - 1}$ whose image under $\wt\rmd$ is $\left((-1)^{p - 1}x\eta, 0\right)$, which lifts to $(-1)^{p - 1}x\eta\in C_{p - 1}$. Thus, the boundary map $H_{p - 1}(C_\bullet)\to H_{p - 1}(C_\bullet)$ is multiplication by $(-1)^{p - 1}x$.

    Finally, we show that $x$ annihilates $H_p\left(C(x)_\bullet\right)$. Indeed, let $(\xi, \eta)\in C_p\oplus C_{p - 1}$ be such that $\wt\rmd(\xi, \eta) = 0$, equivalently, $\rmd\xi = (-1)^p x\eta$ and $\rmd\eta = 0$. Thus 
    \begin{equation*}
        x\cdot(\xi, \eta) = (x\xi, x\eta) = \rmd\left(0, (-1)^p\xi\right),
    \end{equation*}
    consequently, $x$ annihilates $H_p(C(x)_\bullet)$ for all $p\in\Z$. 
\end{proof}

\begin{corollary}
    Let $\ul x = x_1,\dots,x_n$ be a sequence in $A$. Then $(x_1,\dots,x_n)$ annihilates $H_p(\ul x, M)$ for all $p\in\Z$.
\end{corollary}
\begin{proof}
    This follows from the fact that the Koszul complex is precisely $C(x_1)\otimes_A\dots\otimes_A C(x_n)\otimes_A M$.
\end{proof}

\begin{theorem}
    Let $A$ be a ring, $M$ an $A$-module, and $\ul x = x_1,\dots,x_n\in A$. 
    \begin{enumerate}[label=(\arabic*)]
        \item If $\ul x$ is an $M$-sequence, then $H_p(\ul x, M) = 0$ for all $p\ne 0$ and $H_0(\ul x, M) = M/(\ul x)M$.
        \item Suppose that $(A,\frakm)$ is a local ring, $x_1,\dots,x_n\in\frakm$, and $M$ is a finitely presented $A$-module. If $H_1(\ul x, M) = 0$ and $M\ne 0$, then $\ul x$ is an $M$-sequence.
    \end{enumerate}
\end{theorem}
\begin{proof}
We prove both statements by induction on $n\ge 1$.
\begin{enumerate}[label=(\arabic*)]
    \item The base case with $n = 1$ is trivial, since $H_1(x_1, M) = (0 :_M x_1) = 0$ since $x_1$ is $M$-regular. Suppose now that $n > 1$. If $p > 1$, using \thref{koszul-long-exact-sequence} we have a long exact sequence: 
    \begin{equation*}
        0 = H_p(x_1, \dots, x_{n - 1}, M)\to H_p(x_1,\dots,x_n, M)\to H_{p - 1}(x_1,\dots,x_{n - 1}, M) = 0
    \end{equation*},
    where the two terms are zero from the inductive hypothesis. It follows that $H_p(x_1,\dots,x_n, M) = 0$ too. Next, if $p = 1$, then due to \thref{koszul-long-exact-sequence}, we have an exact sequence
    \begin{equation*}
        0 = H_1(x_1,\dots, x_{n - 1}, M)\to H_1(x_1,\dots,x_n, M)\to H_0(x_1,\dots,x_{n - 1}, M) = M/(x_1,\dots,x_{n - 1})M\xrightarrow{\cdot x_n} M/(x_1,\dots,x_{n - 1})M.
    \end{equation*}
    Since $x_n$ is $M/(x_1,\dots,x_{n - 1})M$-regular, the last map in the above sequence is injective, so that $H_1(x_1,\dots,x_n, M) = 0$, as desired. 

    \item Again, the base case with $n = 1$ is trival since $H_1(x_1, M) = (0 :_M x_1)$, which is zero if and only if $x_1$ is $M$-regular.
    
    Since $M$ is a finitely presented, all Koszul homology groups $K(\ul x, M)$ of $M$ must be finitely generated. The long exact sequence furnished by \thref{koszul-long-exact-sequence} is of the form 
    \begin{equation*}
        H_1(x_1,\dots,x_{n - 1}, M)\xrightarrow{\cdot(-x_n)}H_1(x_1,\dots,x_{n - 1}, M)\to H_1(x_1,\dots,x_n, M) = 0.
    \end{equation*}
    Thus the multiplication map by $x_n$ on $H_1(x_1,\dots,x_{n - 1}, M)$ is surjective. Since $x_n\in\frakm$ and the Koszul homology is finitely generated, we must have that $H_1(x_1,\dots,x_{n - 1}, M) = 0$ by Nakayama's lemma. Due to the inductive hypothesis, we must have that $x_1,\dots, x_{n - 1}$ is an $M$-sequence. Finally, due to \thref{koszul-long-exact-sequence}, there is an exact sequence 
    \begin{equation*}
        0 = H_1(x_1,\dots,x_n, M)\to H_0(x_1,\dots,x_{n - 1}, M) = M/(x_1,\dots,x_{n - 1})M\xrightarrow{\cdot x_n} M/(x_1,\dots,x_{n - 1})M,
    \end{equation*}
    so that the map $M/(x_1,\dots,x_{n - 1})M\xrightarrow{\cdot x_n} M/(x_1,\dots,x_{n - 1})M$ is injective, that is, $x_n$ is $M/(x_1,\dots,x_{n - 1})M$-regular. \qedhere
\end{enumerate}
\end{proof}

\begin{theorem}
    Let $A$ be a Noetherian ring, $M$ a finite $A$-module, and $I$ an ideal in $A$ with $IM\ne M$. For a positive integer $n > 0$, the following are equivalent: 
    \begin{enumerate}[label=(\arabic*)]
        \item $\Ext^i_A(N, M) = 0$ for all $i < n$ and for any finite $A$-module $N$ with $\Supp_A(N)\subseteq V(I)$. 
        \item $\Ext^i_A(A/I, M) = 0$ for all $i < n$. 
        \item $\Ext^i_A(N, M) = 0$ for all $i < n$ and for some finite $A$-module $N$ with $\Supp_A(N) = V(I)$. 
        \item there exists an $M$-sequence of length $n$ contained in $I$.
    \end{enumerate}
\end{theorem}
\begin{proof}
    The implications $(1)\implies(2)\implies(3)$ are clear. 

    \noindent$(3)\implies(4)$ We shall prove this statement by induction on $n$. For the base case, suppose that $n = 1$, that is, $\Hom_A(N, M) = 0$ where $N$ is a finite $A$-module with $\Supp_A(N) = V(I)$. If $I$ does not contain a non-zero-divisor on $M$, then 
    \begin{equation*}
        I\subseteq\bigcup_{\frakp\in\Ass_A(M)}\frakp,
    \end{equation*}
    so that by Prime Avoidance, there exists an associated prime $\frakp\in\Ass_A(M)$ such that $I\subseteq\frakp$ -- in particular, $\frakp\in V(I) = \Supp_A(N)$, that is, $N_\frakp\ne 0$. Due to Nakayama's lemma, this is equivalent to $N_\frakp\otimes_{A_\frakp}\kappa(\frakp)\ne 0$. Thus, there is a non-zero $\kappa(\frakp)$-linear, and hence $A_\frakp$-linear map $N_\frakp\otimes_{A_\frakp}\kappa(\frakp)\to\kappa(\frakp)$. Further, since $\frakp\in\Ass_A(M)$, there is an injective $A$-linear map $A/\frakp\into M$, which localizes into an injective $A_\frakp$-linear map $\kappa(\frakp)\into M_\frakp$. As a result, the composite: 
    \begin{equation*}
        N_\frakp\onto N_\frakp\otimes_{A_\frakp}\kappa(\frakp)\to\kappa(\frakp)\into M_\frakp
    \end{equation*}
    is non-zero. That is, 
    \begin{equation*}
        0\ne\Hom_{A_\frakp}\left(N_\frakp, M_\frakp\right) = \left(\Hom_A(N, M)\right)_\frakp,
    \end{equation*}
    whence $\Ext^0_A(N, M) = \Hom_A(N, M)\ne 0$, a contradiction. Thus $I$ must contain an $M$-regular element. This commpletes our proof of the base case. 

    Suppose now that $n > 1$. We know again that $\Ext^i_A(N, M) = 0$ for $i < 1$, so by the above argument, $I$ contains an $M$-regular element, say $x_1$. Set $M_1 = M/x_1 M$. There is a short exact sequence 
    \begin{equation*}
        0\to M\xrightarrow{x_1} M\to M_1\to 0.
    \end{equation*}
    This gives rise to a long exact sequence: 
    \begin{equation*}
        \cdots\Ext^i_A(N, M)\xrightarrow{x_1}\Ext^i_A(N, M)\to\Ext^i_A(N, M_1)\to\Ext^{i + 1}(N, M_1)\to \cdots.
    \end{equation*}
    If $i < n - 1$, then $\Ext^i_A(N, M) = \Ext^{i + 1}_A(N, M) = 0$, so that $\Ext^i_A(N, M) = 0$. Due to the inductive hypothesis, there exists an $M_1$-sequence $x_2,\dots,x_n$ contained in $I$. Hence, $x_1,\dots,x_n$ is an $M$-sequence contained in $I$.

    \noindent$(4)\implies(1)$ We shall proceed by induction on $n\ge 1$. Consider the base case $n = 1$, that is, $I$ contains an $M$-regular element $x$, and $\Supp_A(N)\subseteq V(I)$. Since $N$ is a finite $A$-module, $\Supp_A(N) = V(\Ann_A(N))$, so that $I\subseteq\sqrt{\Ann_A(N)}$. Now $x^r$ annihilates $N$ for some $r > 0$ but $x^r$ is still $M$-regular (\thref{exponentiating-regular-sequences}). This immediately implies that $\Ext^0_A(N, M) = \Hom_A(N, M) = 0$, thereby establishing the base case. 

    Suppose now that $n > 1$, and let $x_1,\dots,x_n\in I$ be an $M$-sequence. Set $M_1 = M/x_1 M$ and consider the short exact sequence 
    \begin{equation*}
        0\to M\xrightarrow{x_1} M\to M_1\to 0,
    \end{equation*}
    which gives rise to the long exact sequence
    \begin{equation*}
        \cdots\to\Ext^{n - 2}_A(N, M_1)\to\Ext^{n - 1}_A(N, M)\xrightarrow{x_1}\Ext^{n - 1}_A(N, M)\to\cdots.
    \end{equation*}
    Since $I$ contains an $M_1$-sequence of length $n - 1$, due to the inductive hypothesis, $\Ext^{n - 2}_A(N, M) = 0$, that is, multiplication by $x_1$ induces an injective map on $\Ext^{n - 1}_A(N, M)$. Now, as we have observed above, $I\subseteq\sqrt{\Ann_A(N)}$, so that there is a positive integer $r > 0$ such that $x_1^r$ annihilates $N$, and hence must annihilate $\Ext^{n - 1}_A(N, M)$. This is possible if and only if $\Ext^{n - 1}_A(N, M) = 0$, as desired.
\end{proof}