\subsection{The topology on \texorpdfstring{$\scrD_K$}{DK}}

Let $\Omega\subseteq\R^n$ be an open set. We begin by topologizing $C^\infty(\Omega)$. Fix an exhaustion $\{K_i\}$ of $\Omega$ by compact sets. That is,
\begin{itemize}
    \item $\displaystyle\Omega = \bigcup_{i = 1}^\infty K_i$, and 
    \item $K_i\subseteq K_{i + 1}^\circ$ for all $i\ge 1$.
\end{itemize}

Define the seminorms $p_N: C^\infty(\Omega)\to\R$ given by 
\begin{equation*}
    p_N(\phi) = \sup\left\{|\partial^\alpha\phi(x)|\colon x\in K_N,~|\alpha|\le N\right\}.
\end{equation*}
That this is a separating family of seminorms is obvious, and since this is a countable family, the induced locally convex vector topology on $C^\infty(\Omega)$ is metrizable.

It is easy to see that the ``evaluation functionals'' on $C^\infty(\Omega)$ equipped with this topology are continuous, therefore, 
\begin{equation*}
    \scrD_K := \bigcap_{x\in\Omega\setminus K} \ker\ev_x
\end{equation*}
is closed in $C^\infty(\Omega)$. It is easy to see that a (countable) local base at $0$ is given by the sets 
\begin{equation*}
    V_N = \left\{f\in C^\infty(\Omega)\colon p_N(f) < \frac{1}{N}\right\}
\end{equation*}
for $N\ge 1$. Further, in this topology $C^\infty(\Omega)$ is a Fr\'echet space\footnote{I might add in the details to this some day.} and since $\scrD_K$ is closed, it to is a Fr\'echet space. It can also be showed that $C^\infty(\Omega)$ has the Heine-Borel property and hence, the same conclusion holds for $\scrD_K$.


\subsection{Distributions}
Let $\Omega\subseteq\R^n$ be an open set. Define
\begin{equation*}
    \scrD(\Omega) = \bigcup_{K\Subset\Omega}\scrD_K.
\end{equation*}
Introduce the seminorms $\|\cdot\|_N: \scrD(\Omega)\to\R$ given by 
\begin{equation*}
    \|\phi\|_N = \max\left\{|\partial^\alpha\phi(x)|\colon x\in\Omega,~|\alpha|\le N\right\},
\end{equation*}
for $\phi\in\scrD(\Omega)$ and $N\ge 0$. The restrictions of these seminorms to $\scrD_K$ are still seminorms. We claim that they induce the same topology as the canoical topology of $\scrD_K$ discussed in the preceding (sub)section. First, there is a positive integer $N_0$ such that $K\subseteq K_N$ for all $N\ge N_0$. For these $N$, $\|\phi\|_N = p_N(\phi)$ if $\phi\in\scrD_K$. Further, since $\|\phi\|_N\le\|\phi\|_{N + 1}$, the topologies induced by either sequence of seminorms are unchanged if we let $N$ start at $N_0$ instead of $1$. Thus, the two topologies coincide and a local base is given by sets of the form 
\begin{equation*}
    V_N = \left\{\phi\in\scrD_K\colon\|\phi\|_N < \frac{1}{N}\right\}.
\end{equation*}
In particular, $\scrD_K$ is still a Fr\'echet space having the Heine-Borel property.

\begin{definition}
    Let $\Omega\subseteq\R^n$ be a nonempty open set.
    \begin{enumerate}[label=(\alph*)]
        \item For every compact $K\Subset\Omega$, let $\tau_K$ denote the standard Fr\'echet space topology of $\scrD_K$. 
        \item Let $\beta$ denote the collection of all convex balanced sets $W\subseteq\scrD(\Omega)$ such that $\scrD_K\cap W\in\tau_K$ for every $K\Subset\Omega$.
        \item $\tau$ is the collection of all unions of sets of the form $\phi + W$, with $\phi\in\scrD(\Omega)$ and $W\in\beta$.
    \end{enumerate}
\end{definition}

\begin{theorem}
    \begin{enumerate}[label=(\alph*)]
        \item $\tau$ is a topology on $\scrD(\Omega)$, and $\beta$ is a local base for $\tau$.
        \item $\tau$ makes $\scrD(\Omega)$ into a locally convex topological vector space.
    \end{enumerate}
\end{theorem}
\begin{proof}
\begin{enumerate}[label=(\alph*)]
    \item Let $V_1,V_2\in\tau$. We shall show that for all $\phi\in V_1\cap V_2$, there is some $W\in\beta$ such that $\phi + W\subseteq V_1\cap V_2$. Since each $V_i$ is open, there is some $W_i\in\beta$ such that $\phi\in\phi_i + W_i\subseteq$. Let $K\Subset\Omega$ such that $\phi,\phi_1,\phi_2\in\scrD_K$. Since each $\scrD_K\cap W_i$ is open in $\scrD_K$, $W_i$ is convex and balanced, and $\phi - \phi_i\in\scrD_K\cap W_i$. Since the Minkowski functional on $\scrD_K$ corresponding to $\scrD_K\cap W_i$ is continuous, there is a $0 < \delta_i < 1$ such that $\phi - \phi_i\in (1 - \delta_i) W_i$. Hence, 
    \begin{equation*}
        \phi - \phi_i + \delta_i W_i\subseteq(1 - \delta_i)\subseteq W_i\implies\phi + \delta_i W_i\subseteq\phi_i + W_i\subseteq V_i,
    \end{equation*}
    whence $\phi + (\delta_1 W_1)\cap(\delta_2 W_2)\subseteq V_1\cap V_2$. Since $\delta_1W_1\cap\delta_2W_2\in\beta$, the conclusion follows.
    \item Since $\beta$ consists of convex sets, it suffices to show that $\tau$ makes $\scrD(\Omega)$ a topological vector space. First, we must show that the space is $T_1$. Let $\phi_1\ne\phi_2\in\scrD(\Omega)$, and set 
    \begin{equation*}
        W = \{\phi\in\scrD(\Omega)\colon\|\phi\|_0 < \|\phi_1 - \phi_2\|_0\},
    \end{equation*}
    where $\|\cdot\|_0$ is precisely the $\sup$-norm on $\Omega$. By definition, it is easy to see that $W\in\beta$ and $\phi_1\notin\phi_2 + W$, consequently, $\{\phi_1\}$ is closed.

    To see that addition is continuous, let $(\phi_1,\phi_2)\mapsto\phi_1 + \phi_2$ and $V$ an open set containing $\phi_1 + \phi_2$. Since $\beta$ forms a local base for the topology, we can find some $W\in\beta$ such that $(\phi_1 + \phi_2) + W\subseteq V$, and 
    \begin{equation*}
        \left(\phi_1 + \frac{1}{2}W\right) + \left(\phi_2 + \frac{1}{2}W\right)\subseteq (\phi_1 + \phi_2) + W\subseteq V.
    \end{equation*}
    Thus, addition is continuous.

    Finally, we must show that scalar multiplication is continuous. Let $\alpha_0\in\K$ and $\phi_0\in\scrD(\Omega)$. Then, 
    \begin{equation*}
        \alpha\phi - \alpha_0\phi_0 = \alpha(\phi - \phi_0) + (\alpha - \alpha_0)\phi_0.
    \end{equation*}
    Let $V$ be an open set containing $\alpha_0\phi_0$, and choose a $W\in\beta$ such that $\alpha_0\phi_0 + W\subseteq V$. There is a $\delta > 0$ such that $\delta\phi_0\in\frac{1}{2}W$. Next, choose $c > 0$ such that $2c(|\alpha_0| + \delta) = 1$. For $|\alpha - \alpha_0| < \delta$ and $\phi - \phi_0\in cW$, we have 
    \begin{equation*}
        \alpha\phi - \alpha_0\phi_0\in |\alpha|cW + \frac{1}{2}W\subseteq c(|\alpha_0| + \delta) W + \frac{1}{2}W\subseteq W,
    \end{equation*}
    as desired. This completes the proof.\qedhere
\end{enumerate}
\end{proof}

\begin{theorem}\thlabel{thm:technical-theorem}
\begin{enumerate}[label=(\alph*)]
    \item A convex balanced subset $V$ of $\scrD(\Omega)$ is open if and only if $V\in\beta$. 
    \item The topology $\tau_K$ of any $\scrD_K\subseteq\scrD(\Omega)$ coincides with the subspace topology that $\scrD_K$ inherits from $\scrD(\Omega)$. 
    \item If $E$ is a bounded subset of $\scrD(\Omega)$, then $E\subseteq\scrD_K$ for some $K\subseteq\Omega$ and there are real numbers $0 < M_N < \infty$ such that every $\phi\in E$ satisfies the inequalities $\|\phi\|_N\le M_N$ for $N\ge 0$.
    \item $\scrD(\Omega)$ has the Heine-Borel property, that is, closed and bounded sets are compact. 
    \item If $\{\phi_i\}$ is a Cauchy sequence in $\scrD(\Omega)$, then $\{\phi_i\}\subseteq\scrD_K$ for some compact $K\subseteq\Omega$, and 
    \begin{equation*}
        \lim_{(i,j)\to\infty}\|\phi_i - \phi_j\|_N = 0
    \end{equation*}
    for all $N\ge 0$. 
    \item If $\phi_i\to 0$ in the topology of $\scrD(\Omega)$, then there is a compact set $K\subseteq\Omega$ which contains the support of every $\phi_i$ and $\partial^\alpha\phi_i\to 0$ uniformly as $i\to\infty$, for every multi-index $\alpha$.
    \item $\scrD(\Omega)$ is a Fr\'echet space.
\end{enumerate}
\end{theorem}
\begin{proof}
Let $V\in\tau$ and $\phi\in\scrD_K\cap V$. Since $\beta$ form a local base, there is a $W\in\beta$ such that $\phi + W\subseteq V$. Hence, 
\begin{equation*}
    \phi + (\scrD_K\cap W)\subseteq\scrD_K\cap V.
\end{equation*}
Since $\scrD_K\cap W$ is open in $\scrD_K$, we have that $\scrD_K\cap V\in\tau_K$.
\begin{enumerate}[label=(\alph*)]
\item Now, let $V$ be a convex balanced subset of $\scrD(\Omega)$. If $V$ is open, then due to our observation above, $V\in\beta$. The converse direction is trivial since $\beta\subseteq\tau$.

\item The above remark shows that the induced topology on $\scrD_K$ is coarser than $\tau_K$. Conversely, suppose $E\in\tau_K$. We have to show that $E = \scrD_K\cap V$ for some $V|in\tau$. By definition, for every $\phi\in E$, there is a positive integer $N$ and $\delta > 0$ such that 
\begin{equation*}
    \{\psi\in\scrD_K\colon\|\psi - \phi\|_N < \delta\}\subseteq E.
\end{equation*}
Set $W_\phi = \{\psi\in\scrD(\Omega)\colon \|\psi\|_N < \delta\}\in\beta$, so that
\begin{equation*}
    \scrD_K\cap(\phi + W_\phi) = \phi + \scrD_K\cap W_\phi\subseteq E.
\end{equation*}
Since $W_\phi\in\beta$ for every $\phi\in E$, we see that $V := \bigcup\limits_{\phi\in E} (\phi + W_\phi)$ is an element of $\tau$ and $V\cap E = E$, as desired. 

\item Suppose $E$ does not lies in any $\scrD_K$. Using an exhaustion of $\Omega$, we can find a sequence of functions $\phi_m\in E$ and distinct points $x_m\in\Omega$ with no limit point in $\Omega$ such that $\phi_m(x_m)\ne 0$. Let $W$ be the set of all $\phi\in\scrD(\Omega)$ which satisfy 
\begin{equation*}
    |\phi(x_m)| < \frac{1}{m}|\phi_m(x_m)| \qquad\forall~m\ge 1.
\end{equation*}
Note that 
\begin{equation*}
    W\cap\scrD_K = \bigcap_{x_m\in W\cap\scrD_K}\left\{\phi\in\scrD_K\colon|\phi(x_m)| < \frac{1}{m}|\phi_m(x_m)|\right\},
\end{equation*}
which is a finite intersection since only finitely many of the $x_m$'s can be contained in $K$ (as they do not admit a limit point in $\Omega$). Thus, $W\cap\scrD_K$ is open, owing to the continuity of the ``evaluation functionals'' on $\scrD_K$; hence $W\in\beta$. Since $\phi_m\notin mW$, no multiple of $W$ contains $E$, which shows that $E$ is not bounded. Hence, every bounded $E$ lies in some $\scrD_K$. Being a bounded subset of $\scrD_K$, every seminorm on $\scrD_K$ is bounded on $E$, whence the last assertion of (c) follows.

\item This follows immediately from the above parts, since every bounded set is contained in some $\scrD_K$, whose subspace topology is same as the canonical topology, in which it has the Heine-Borel property.

\item Every Cauchy sequence is bounded and hence, is contained in some $\scrD_K$, which has its canonical topology induced by the seminorms $\|\cdot\|_N$, whence the conclusion follows. 

\item This follows immediately from (e). 

\item Finally, we have shown that any Cauchy sequence in $\scrD(\Omega)$ lies in $\scrD_K$, which is Fr\'echet, whence it must converge. This completes the proof. \qedhere
\end{enumerate}
\end{proof}

\begin{theorem}
    Let $\Lambda$ be a linear map from $\scrD(\Omega)$ to a locally convex space $Y$. Then the following are equivalent:
    \begin{enumerate}[label=(\alph*)]
        \item $\Lambda$ is continuous. 
        \item $\Lambda$ is bounded. 
        \item If $\phi_i\to 0$ in $\scrD(\Omega)$, then $\Lambda\phi_i\to 0$ in $Y$. 
        \item The restriction of $\Lambda$ to every $\scrD_K\subseteq\scrD(\Omega)$ are continuous.
    \end{enumerate}
\end{theorem}
\begin{proof}
$(a)\implies(b)$ is well known. Next, if $\phi_i\to 0$ in $\scrD(\Omega)$, then it is contained in some $\scrD_K$ for a compact $K\Subset\Omega$. Since the restriction of $\Lambda$ to $\scrD_K$ is continuous and it is a metrizable topological vector space, $\Lambda\phi_i\to 0$ in $Y$, thereby proving $(b)\implies(c)$.

To see $(c)\implies(d)$, it suffices to show that the restriction of $\Lambda$ to each $\scrD_K$ is sequentially continuous. If $\phi_i\to 0$ in $\scrD_K$ and since the topology of $\scrD_K$ is the subspace topology, we see that $\phi_i\to 0$ in $\scrD(\Omega)$ and according to our assumption, $\Lambda\phi_i\to 0$ in $Y$, which proves sequential continuity.

Finally, let $U$ be a convex balanced neighborhood of $0$ in $Y$. It suffices to show that $V = \Lambda^{-1}U$ is open. Note that $V$ is a convex balanced subset of $\scrD(\Omega)$ containing $0$. Due to \thref{thm:technical-theorem} (a), $V$ is open in $\scrD(\Omega)$ if and only if $V\cap\scrD_K$ is open in $\scrD_K$ for every compact $K\Subset\Omega$. But this is precisely the content of (d), thereby completing the proof.
\end{proof}

\begin{definition}
    A linear functional on $\scrD(\Omega)$ continuous with respect to the topology $\tau$ is called a \define{distribution}.
\end{definition}

\begin{theorem}
    If $\Lambda$ is a linear functional on $\scrD(\Omega)$, the following are equivalent: 
    \begin{enumerate}[label=(\alph*)]
        \item $\Lambda\in\scrD'(\Omega)$. 
        \item To every compact $K\Subset\Omega$, corresponds a nonnegative integer $N$ and a constant $C < \infty$ such that
        \begin{equation*}
            |\Lambda\phi|\le C\|\phi\|_N\qquad\forall~\phi\in\scrD_K.
        \end{equation*}
    \end{enumerate}
\end{theorem}
\begin{proof}
    If $\Lambda\in\scrD'(\Omega)$, then the restriction of $\Lambda$ to every $\scrD_K$ is continuous and so is bounded on some neighborhood of the origin, containing an open neighborhood of the form 
    \begin{equation*}
        \{\phi\in\scrD_K\colon \|\phi\|_N < \frac{1}{N}\},
    \end{equation*}
    whence (b) follows.

    Conversely, suppose (b) holds. Then, as argued above, the restriction of $\Lambda$ to every $\scrD_K$ is continuous, and due to the preceding theorem, $\Lambda$ is continuous.
\end{proof}

\begin{example}
There are some canonical examples of distributions. If $f\in L^1_{loc}(\Omega)$, then the map $\Lambda_f: \scrD(\Omega)\to\R$ given by 
\begin{equation*}
    \Lambda_f(\phi) = \int_\Omega f(x)\phi(x)~dx\qquad\forall~\phi\in\scrD(\Omega).
\end{equation*}
If $K = \Supp\phi$, then 
\begin{equation*}
    |\Lambda_f(\phi)|\le\|f\|_{L^1(K)}\|\phi\|_0,
\end{equation*}
and hence $\Lambda_f$ is a distribution of order $0$.
\end{example}

\begin{example}[Dirac Distribution]
    The map $\Lambda:\scrD(\Omega)\to\R$ defined by $\phi\mapsto\phi(0)$ is a distribution of order $0$, since 
    \begin{equation*}
        |\Lambda(\phi)|\le\|\phi\|_0.
    \end{equation*}
    This is known as the \define{Dirac distribution}.
\end{example}

\begin{definition}
    Let $\Lambda\in\scrD'(\Omega)$ be a distribution and $\alpha$ a multi-index. Define $\partial^\alpha\Lambda\in\scrD'(\Omega)$ by 
    \begin{equation*}
        \left(\partial^\alpha\Lambda\right)(\phi) = (-1)^{|\alpha|}\Lambda\left(\partial^\alpha\phi\right).
    \end{equation*}
\end{definition}

We show that this is indeed a distribution. Let $K\Subset\Omega$ be a compact set. Then there is an integer $N$ and $C > 0$ such that 
\begin{equation*}
    |\Lambda(\phi)|\le C\|\phi\|_N\qquad\forall~\phi\in\scrD_K.
\end{equation*}
Then, 
\begin{equation*}
    \left|(\partial^\alpha\Lambda)(\phi)\right|\le C\|\partial^\alpha\phi\|_N\le C\|\phi\|_{N + |\alpha|}\qquad\forall~\phi\in\scrD_K,
\end{equation*}
as desired.

We further note that the formula 
\begin{equation*}
    \partial^\alpha\partial^\beta\Lambda = \partial^{\alpha + \beta}\Lambda = \partial^{\beta}\partial^\alpha\Lambda
\end{equation*}
holds for every distribution $\Lambda\in\scrD'(\Omega)$. This is quite straightforward, since 
\begin{equation*}
    \left(\partial^\alpha\partial^\beta\Lambda\right)(\phi) = (-1)^{|\alpha|}\partial^\beta\Lambda\left(\partial^\alpha\phi\right) = (-1)^{|\alpha| + |\beta|}\Lambda\left(\partial^{\alpha + \beta}\phi\right) = \left(\partial^{\alpha + \beta}\Lambda\right)(\phi),
\end{equation*}
as desired.

\begin{definition}
    Let $\Lambda\in\scrD'(\Omega)$ be a distribution and $f\in C^\infty(\Omega)$. Define $f\Lambda\in\scrD'(\Omega)$ as 
    \begin{equation*}
        (f\Lambda)(\phi) = \Lambda\left(f\phi\right)\qquad\forall~\phi\in\scrD_K.
    \end{equation*}
\end{definition}

We contend that this is indeed a distribution. For any multi-index $\alpha$, we have 
\begin{equation*}
    \partial^\alpha\left(f\phi\right) = \sum_{\substack{\beta + \gamma = \alpha\\\beta,\gamma\ge0}}\frac{\alpha!}{\beta!\gamma!}(\partial^\beta f)(\partial^\gamma\phi).
\end{equation*}

If $K\Subset\Omega$ is a compact set, then there is a positive integer $N$ and a constant $C > 0$ such that $|\Lambda(f)|\le C\|\phi\|_N$ for all $\phi\in\scrD_K$. Define 
\begin{equation*}
    C' = \sup\left\{|\partial^\beta f(x)|\colon x\in K,~|\beta|\le N\right\}.
\end{equation*}
Then, for any $x\in K$, and $|\alpha|\le N$, we have
\begin{equation*}
    \left|\partial^\alpha(f\phi)(x)\right|\le\sum_{\substack{\beta + \gamma = \alpha\\\beta,\gamma\ge 0}}\frac{\alpha!}{\beta!\gamma!}C'|\partial^\gamma\phi(x)|\le\left(\sum_{\substack{\beta + \gamma = \alpha\\\beta,\gamma\ge 0}}\frac{\alpha!}{\beta!\gamma!}C'\right)\|\phi\|_N
\end{equation*}
As $\alpha$ ranges over all multi-indices of absolute value at most $N$, we can take supremum of the left hand side to obtain 
\begin{equation*}
    \|f\phi\|_N\le C''\|\phi\|_N,
\end{equation*}
where $C''$ is a constant independent of $\phi$. Therefore, 
\begin{equation*}
    (f\Lambda)(\phi) = \Lambda(f\phi)\le C\|f\phi\|_N\le CC''\|\phi\|_N,
\end{equation*}
and hence, $f\Lambda$ is a distribution.

\begin{theorem}[Countable Partition of Unity]\thlabel{thm:countable-partition-unity}
    If $\Gamma$ is a c ollection of open sets in $\R^n$ whose union is $\Omega$, then there exists a sequence $\{\psi_i\}$ of elements in $\scrD(\Omega)$, with $\psi_i\ge 0$, such that 
    \begin{enumerate}[label=(\alph*)]
        \item each $\psi_i$ is supported in some member of $\Gamma$, 
        \item $\displaystyle\sum_i\psi_i(x) = 1$ for every $x\in\Omega$, 
        \item to every compact $K\Subset\Omega$, there is an integer $m$ and an open set $W\supseteq K$ such that 
        \begin{equation*}
            \psi_1(x) + \dots + \psi_m(x) = 1\qquad\forall~x\in W.
        \end{equation*}
    \end{enumerate}
    Such a collection $\{\psi_i\}$ is called a \define{locally finite partition of unity} in $\Omega$ \define{subordinate} to the open cover $\Gamma$.
\end{theorem}

\begin{definition}
    Suppose $\Lambda\in\scrD'(\Omega)$. If $\omega\subset\Omega$ is an open set and if $\Lambda\phi = 0$ for every $\phi\in\scrD(\Omega)$, we say that $\Lambda$ \define{vanishes in} $\omega$. Let $W$ be the union of all open $\omega\subseteq\Omega$ in which $\Lambda$ vanishes. The set $\Omega\setminus W$ is the \define{support} of $\Lambda$.
\end{definition}

\begin{theorem}
    If $W$ is as above, then $\Lambda$ vanishes in $W$.
\end{theorem}
\begin{proof}
    Let $\Gamma$ be the collection of all $\omega$ as in the above definition. Let $\{\psi_i\}$ be a locally finite partition of unity in $W$, subordinate to $\Gamma$. If $\phi\in\scrD(W)$, then since $\phi$ has compact support contained in $W$, all but finitely many $\psi_i$ vanish on the support of $\phi$, in particular, we can write 
    \begin{equation*}
        \phi = \sum_{i}\psi_i\phi,
    \end{equation*}
    where the sum on the right is essentially finite. Thus, we can write 
    \begin{equation*}
        \Lambda(\phi) = \sum_i \Lambda(\psi_i\phi),
    \end{equation*}
    but the support of each $\psi_i$ is contained in some $\omega$ on which $\Lambda$ vanishes. Consequently, the sum on the right is identically $0$, as desired.
\end{proof}

\begin{theorem}
    Let $\Lambda\in\scrD'(\Omega)$ and $S_\Lambda$ be the support of $\Lambda$. 
    \begin{enumerate}[label=(\alph*)]
        \item If the support of $\phi\in\scrD(\Omega)$ is disjoint from $S_\Lambda$, then $\Lambda\phi = 0$.
        \item If $S_\Lambda$ is empty, then $\Lambda = 0$. 
        \item If $\psi\in C^\infty(\Omega)$ and $\psi = 1$ on some open set $V$ containing $S_\Lambda$, then $\phi\Lambda = \Lambda$. 
        \item If $S_\Lambda$ is a compact subset of $\Omega$, then $\Lambda$ has finite order. In fact, there is a constant $C < \infty$ and a nonnegative integer $N$ such that $|\Lambda\phi|\le C\|\phi\|_N$ for every $\phi\in\scrD(\Omega)$. Further, $\Lambda$ extends in a unique way to a continuous linear functional on $C^\infty(\Omega)$.
    \end{enumerate}
\end{theorem}
\begin{proof}
\begin{enumerate}[label=(\alph*)]
    \item This is just a restatement of the preceding result. 
    \item Again, this is an immediate consequence of either the preceding result or (a). 
    \item Let $\psi\in\scrD(\Omega)$. Consider the function $\phi - \psi\phi$. This vanishes on $V$, an open set containing $S_\Lambda$, and hence, the support of $\phi - \psi\phi$ is disjoint from $S_\Lambda$. Due to (a), we must have 
    \begin{equation*}
        \Lambda\left(\phi - \psi\phi\right) = 0\implies\Lambda\phi = \Lambda(\psi\phi) = (\psi\Lambda)(\phi),
    \end{equation*}
    as desired. 
    \item In light of \thref{thm:countable-partition-unity} (d) with $\Gamma = \{\Omega\}$, there is a $\psi\in\scrD(\Omega)$ which is identically $1$ on an open set $V$ containing $S_\Lambda$. Due to (c), we have $\psi\Lambda = \Lambda$. Let $K$ denote the support of $\psi$. Then, there is a positive integer $N$ and a constant $C > 0$ such that $|\Lambda\phi|\le C\|\phi\|_N$ for all $\phi\in\scrD_K$. Consequently, for any $\phi\in\scrD(\Omega)$, we  can write 
    \begin{equation*}
        |\Lambda\phi| = |\Lambda(\psi\phi)|\le C\|\psi\phi\|_N\le CC'\|\phi\|_N,
    \end{equation*}
    where the last inequality has been argued earlier while showing that the differentiation of a distribution gives a distribution. It follows that $\Lambda$ is a distribution of finite order and that the constant $CC'$ is independent of the choice of compact set containing the support of $\phi$.

    Finally, we must show that there is a unique extension of $\Lambda$ to $C^\infty(\Omega)$. For each $f\in C^\infty(\Omega)$, define 
    \begin{equation*}
        \Lambda f = \Lambda(\psi f).
    \end{equation*}
    This is obviously an extension of $\Lambda$ defined on $\scrD(\Omega)$. We must show that this is continuous. Indeed, recall that $K$ is the support of $\psi$ and choose an exhaustion $K_0\subset K_1\subset\cdots$ of $\Omega$. Choose a positive integer $M$ sufficiently large so that $M\ge N$ and $K\subseteq K_M$. Further, using an analogous argument as before, there is a constant $\wt C$ such that $\rho_M(\psi f)\le\wt C\rho_M(f)$. Indeed, for $|\alpha|\le M$ and $x\in K_M$, we have 
    \begin{align*}
        \left|\partial^\alpha(f\psi)(x)\right| &= \left|\sum_{\substack{\beta + \gamma = \alpha\\\beta,\gamma\ge 0}}\frac{\alpha!}{\beta!\gamma!}\partial^\beta \psi(x)\partial^\gamma f(x)\right|\le\wt C\rho_M(f).
    \end{align*}
    It follows that $\Lambda$ is a continuous linear functional on $C^\infty(\Omega)$. It remains to show that $\scrD(\Omega)$ is dense in $C^\infty(\Omega)$, whence it would follow that the extension is unique. Indeed, for any $f\in C^\infty(\Omega)$ and positive integer $n$, we can find $\psi_n\in\scrD(\Omega)$ such that $\psi_n = 1$ on $K_n$. Then, $f - \psi_nf = 0$ on $K_m$ for all $m\le n$. In particular, this means that $p_m(f - \psi_nf) \to 0$ as $n\to\infty$, consequently, $\psi_nf\to f$ in $C^\infty(\Omega)$, as desired. This completes the proof.\qedhere
\end{enumerate}
\end{proof}