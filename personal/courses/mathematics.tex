\begin{rubric}{Mathematics}
    \entry*[AA] MA403: Real Analysis\hfill\emph{Instructor: Prof. Santanu Dey}

    \emph{Textbook:} Mathematical Analysis by Tom Apostol

    Metric spaces and their topological properties. Sequences and series, convergence theorems. Derivatives, integration. Fourier series, types of convergence.


    \entry*[AA] MA406: General Topology\hfill\emph{Instructor: Prof. Sandip Singh}

    \emph{Textbook:} Topology by Munkres

    Chapters I through V of the aforementioned book. Ending with a proof of Tychonoff's Theorem.

    \entry*[AA] MA408: Measure Theory\hfill\emph{Instructor: Prof. Santanu Dey}

    \emph{Textbook:} Real Analysis by Royden and Fitzpatrick

    Outer measures, Lebesgue measure on $\mathbb R$. Measurable functions. Integration of measurable functions on $\mathbb R$. Abstract measure spaces, Fubini and Tonelli's theorems.

    \entry*[AP] MA410: Multivariable Calculus\hfill\emph{Instructor: Prof. Preeti Raman}

    \emph{Textbook:} Calculus on Manifolds by Spivak

    Chapters I through IV of the above book. 

    \entry*[AP] MA412: Complex Analysis\hfill\emph{Instructor: Prof. Shripad Garge}

    \emph{Textbook:} Functions of One Complex Variable by Conway

    Chapters I through IV of the above book.

    \entry*[AP] MA417: Ordinary Differential Equations\hfill\emph{Instructor: Prof. Saikat Mazumdar}

    \emph{Textbook:} Ordinary Differential Equations and Dynamical Systems by Teschl 

    Chapters I, II, III, and V of the above book.

    \entry*[AA] MA419: Basic Algebra\hfill\emph{Instructor: Prof. Saurav Bhaumik}
    
    \emph{Textbook:} Abstract Algebra by Dummit and Foote; Algebra by Lang

    Groups, Sylow subgroups, solvable groups, nilpotent groups. Rings, domains, UFDs, PIDs, irreducibility, Gauss' Lemma.

    \entry*[AA] MA503: Functional Analysis\hfill\emph{Instructor: Prof. Chandan Biswas}
    
    \emph{Textbook:} Functional Analysis by Rudin

    Topological Vector Spaces, Completeness Arguments such as the Uniform Boundedness Principle, Open Mapping Theorem, and Closed Graph Theorem. Weak and Weak* Topologies, Banach-Alaoglu Theorem. Compact Operators.

    \entry*[AP] MA515: Partial Differential Equations\hfill\emph{Instructor: Prof. Harsha Hutridurga}


    \emph{Textbook:} No official textbook. A suggested reference was ``Partial Differential Equations: Classical Theory with a Modern Touch'' by Nandakumaran and Datti

    Laplace equation, harmonic functions, various theorems for harmonic functions such as Liouville and Harnack. Heat equation, Li-Yau inequality, Harnack's inequality. Wave equation, Poisson-Kirchhoff representation formula. First order PDEs, characteristic curves, Duhamel's principle. Functional inequalities.



    \entry*[AA] MA521: Theory of Analytic Functions\hfill\emph{Instructor: Prof. Shripad Garge}

    \emph{Textbook:} Functions of One Complex Variable by Conway

    Chapters V through VII of the above book. Classification of singularities, Carorati-Weierstrass. Compact open topologies, spaces of harmonic and meromorphic functions. Marty's theorem, Weierstrass factorization theorem. Gamma and Zeta function.

    \entry*[AA] MA523: Basic Number Theory\hfill\emph{Instructor: Prof. Ronnie Sebastian}

    \emph{Textbook:} A concise introduction to the theory of numbers by Alan Baker

    Modular arithmetic, quadratic residues, quadratic reciprocity. Binary quadratic forms. Continued fractions.

    \entry*[AA] MA526: Commutative Algebra\hfill\emph{Instructor: Prof. Jugal Verma}

    \emph{Textbook:} Commutative Ring Theory by Matsumura; Cohen-Macaulay Rings by Bruns and Herzog

    Noetherian and Artinian rings. Associated primes and length of modules. Primary decomposition in modules. Dimension Theory of modules. Depth and Cohen-Macaulay rings and modules.

    \entry*[AA] MA5106: Introduction to Fourier Analysis\hfill\emph{Instructor: Prof. Saikat Mazumdar}

    \emph{Textbook:} Fourier Analysis: An Introduction by Shakarchi and Stein

    Fourier series, forms of convergence. Fourier transforms, inversion thereof, Plancherel's theorem. Distributions and applications to PDEs.

    \entry*[AA] MA5110: Non-commutative Algebra\hfill\emph{Instructor: Prof. Shripad Garge}

    \emph{Textbook:} Associative Algebras by Pierce 

    Basic theory of non-commutative rings. Central simple algebras and the Brauer group. Valuations on division algebras. Local fields and the Brauer group of a local field.

    \entry*[AA] MA811: Algebra I\hfill\emph{Instructor: Prof. Jugal Verma}

    \emph{Textbook:} Algebra by Serge Lang; Field and Galois Theory by Patrick Morandi

    Normal, separable, Galois extensions. Purely inseparable extensions. Abelian, cyclic extensions. Hilbert theorem $90$ (additive and multiplicative) and its interpretation as Galois cohomology. Solvable extensions and the insolvability of the quintic. Transcendental extensions, separably generated, linearly disjoint extensions. Basic algebraic geometry using affine varieties and their dimension theory.

    \entry*[AA] MA812: Algebra II\hfill\emph{Instructor: Prof. Ronnie Sebastian}

    \emph{Textbook:} Introduction to Commutative Algebra by Atiyah and MacDonald; Algebra by Serge Lang

    Chapters IV through X of Atiyah and MacDonald. Non-commutative rings, semisimple rings, the Artin-Wedderburn Theorem. Representation theory of finite groups. Basic homological algebra.

    \entry*[AA] MA813: Measure Theory\hfill\emph{Instructor: Prof. Dipendra Prasad}

    \emph{Textbook:} Real and Complex Analysis by Rudin 

    Chapters I through IX of the above book.

    \entry*[AA] MA815: Differential Topology\hfill\emph{Instructor: Prof. Manoj Keshari}

    \emph{Textbook:} Differential Forms in Algebraic Topology by Bott and Tu

    Chapter I of Bott and Tu. 

    \entry*[AA] MA841: Topics in Algebra I\hfill\emph{Instructor: Prof. Shripad Garge}

    \emph{Textbook:} Introduction to Lie Algebras and Representation Theory by Humphreys

    Chapters I, II, III, parts of IV, V, and VI of the above book. In particular, solvable, nilpotent Lie algebras. Semisimple Lie algebras, root space decomposition. Root systems, Dynkin diagrams and their classification. Universal Enveloping Algebras and PBW. Abstract theory of weights.
\end{rubric}