\begin{rubric}{Reading Projects}
	
	\entry*[2024\phantom{}] \textbf{Representations of Reductive $p$-adic groups} \hfill \emph{\href{mailto:dprasad@iitb.ac.in}{Prof. Dipendra Prasad, IIT Bombay}}

	$\bullet$ Read through the first $10$ sections of Stephen DeBacker's notes on \emph{Representations of Reductive $p$-adic groups}, and as a by-product, also learnt some theory of algebraic groups on the fly.

	$\bullet$ Had weekly meetings with my supervisor and summarized the material read in that week.


	\entry*[2024\phantom{}] \textbf{Representations of Compact Lie Groups} \hfill \emph{\href{mailto:anand@math.iitb.ac.in}{Prof. U. K. Anandavardhanan, IIT Bombay}}

	$\bullet$ Read through the first four chapters of \emph{Representations of Compact Lie Groups} by Theodore Br\"ocker and Tammo tom Dieck. Supplemented this reading with Daniel Bump's \emph{Lie Groups}.

	$\bullet$ Gave presentations on the Peter-Weyl Theorem and Cartan's Theorem on conjugacy of Maximal Tori to my supervisor.


	\entry*[2024\phantom{}] \textbf{Tate's Thesis} \hfill \emph{\href{mailto:sandeepv@math.tifr.res.in}{Prof. Sandeep Varma, TIFR Mumbai}}

	$\bullet$ Read through Ramakrishnan and Valenza's \emph{Fourier Analysis on Number Fields}. 

	$\bullet$ Supplemented the chapters on harmonic analysis with Folland's \emph{A First Course in Harmonic Analysis}, especially for Pontryagin Duality.

	$\bullet$ Gave a final \href{https://swayamchube.github.io/math/vsrp/main.pdf}{presentation} on the analytic class number using adelic measures as derived by Tate in his thesis.


	\entry*[2024\phantom{}] \textbf{Algebraic Geometry} \hfill \emph{\href{mailto:saurav@math.iitb.ac.in}{Prof. Saurav Bhaumik, IIT Bombay}}

	$\bullet$ Read through Chapter I and the first two sections of Chapter II of Hartshorne's \emph{Algebraic Geometry}.

	$\bullet$ Weekly meetings with supervisor would generally be about drawing analogies between the theory of smooth manifolds and affine varieties. Also had brief discussions on vector bundles towards the end.

	$\bullet$ Prepared an (in-progress) set of \href{https://swayamchube.github.io/math/hartshorne-solutions/main.pdf}{solutions} to Hartshorne's exercises (mainly Chapter II).


	\entry*[2024\phantom{}] \textbf{Class Field Theory} \hfill \emph{\href{mailto:dprasad@iitb.ac.in}{Prof. Dipendra Prasad, IIT Bombay}}

	$\bullet$ Read Chapters II through V of Milne's notes on \emph{Class Field Theory}. Supplemented this reading with Cassels and Fr\"ohlich's \emph{Algebraic Number Theory} for Group Cohomology.

	$\bullet$ Read through the proofs of Local Class Field Theory in detail and learnt about the adelic and ideal theoretic formulation of the main results of Global Class Field Theory (no proofs).


	\entry*[2023\phantom{}] \textbf{Algebraic Topology} \hfill \emph{\href{mailto:saurav@math.iitb.ac.in}{Prof. Saurav Bhaumik, IIT Bombay}}

	$\bullet$ Mainly followed Hatcher's \emph{Algebraic Topology} and read the chapters on Fundamental Groups, Homology and part of the chapter on Cohomology (up to cup products). Supplemented this reading with Rotman's \emph{An Introduction to Algebraic Topology}.


	\entry*[2023\phantom{}] \textbf{Set Theory and Forcing} \hfill \emph{\href{mailto:adsul@cse.iitb.ac.in}{Prof. Bharat Adsul, IIT Bombay}}

	$\bullet$ Followed Kunen's \emph{Set Theory: An Introduction to Independence Proofs} while covering just enough to get to Paul Cohen's method of ``forcing''.

	$\bullet$ Gave weekly presentations to supervisor in the material learnt in that week.
\end{rubric}