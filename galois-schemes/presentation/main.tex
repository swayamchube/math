\documentclass{beamer}
\usetheme{Warsaw}

\title{Galois Categories and the \'Etale Fundamental Group}
\subtitle{or, what should be taught in MA 811}
\author{Swayam Chube}
\institute{Indian Institute of Technology, Bombay}
\date{\today}

\usepackage[utf8]{inputenc} % allow utf-8 input
\usepackage[T1]{fontenc}    % use 8-bit T1 fonts
\usepackage{hyperref}       % hyperlinks
\usepackage{url}            % simple URL typesetting
\usepackage{booktabs}       % professional-quality tables
\usepackage{amsfonts}       % blackboard math symbols
\usepackage{nicefrac}       % compact symbols for 1/2, etc.
\usepackage{microtype}      % microtypography
\usepackage{graphicx}
\usepackage{natbib}
\usepackage{doi}
\usepackage{amssymb}
\usepackage{bbm}
\usepackage{amsthm}
\usepackage{amsmath}
\usepackage{xcolor}
\usepackage{theoremref}
% \usepackage{enumitem}
% \usepackage{mathpazo}
% \usepackage{euler}
\usepackage{lmodern}
\usepackage{sansmath, sfmath}
\usepackage{mathrsfs}
\setlength{\marginparwidth}{2cm}
\usepackage{todonotes}
\usepackage{stmaryrd}
\usepackage[all,cmtip]{xy} % For diagrams, praise the Freyd-Mitchell theorem 
\usepackage{marvosym}
\usepackage{geometry}
\usepackage{titlesec}
\usepackage{mathtools}
% \usepackage{fontspec}
\usepackage{tikz}
\usetikzlibrary{cd}

\renewcommand{\qedsymbol}{$\blacksquare$}
\renewcommand{\familydefault}{\sfdefault}

% Uncomment to override  the `A preprint' in the header
% \renewcommand{\headeright}{}
% \renewcommand{\undertitle}{}
% \renewcommand{\shorttitle}{}

\hypersetup{
    pdfauthor={Lots of People},
    colorlinks=true,
}

\newtheoremstyle{thmstyle}%               % Name
  {}%                                     % Space above
  {}%                                     % Space below
  {}%                             % Body font
  {}%                                     % Indent amount
  {\bfseries\scshape}%                            % Theorem head font
  {.}%                                    % Punctuation after theorem head
  { }%                                    % Space after theorem head, ' ', or \newline
  {\thmname{#1}\thmnumber{ #2}\thmnote{ (#3)}}%                                     % Theorem head spec (can be left empty, meaning `normal')

\newtheoremstyle{defstyle}%               % Name
  {}%                                     % Space above
  {}%                                     % Space below
  {}%                                     % Body font
  {}%                                     % Indent amount
  {\bfseries\scshape}%                            % Theorem head font
  {.}%                                    % Punctuation after theorem head
  { }%                                    % Space after theorem head, ' ', or \newline
  {\thmname{#1}\thmnumber{ #2}\thmnote{ (#3)}}%                                     % Theorem head spec (can be left empty, meaning `normal')

% \theoremstyle{thmstyle}
% \newtheorem{theorem}{Theorem}[section]
% \newtheorem{lemma}[theorem]{Lemma}
\newtheorem{proposition}[theorem]{Proposition}

% \theoremstyle{defstyle}
% \newtheorem{definition}[theorem]{Definition}
% \newtheorem{corollary}[theorem]{Corollary}
% \newtheorem{porism}[theorem]{Porism}
% \newtheorem{remark}[theorem]{Remark}
% \newtheorem{interlude}[theorem]{Interlude}
% \newtheorem{example}[theorem]{Example}
% \newtheorem*{notation}{Notation}
% \newtheorem*{claim}{Claim}

% Common Algebraic Structures
\newcommand{\R}{\mathbb{R}}
\newcommand{\Q}{\mathbb{Q}}
\newcommand{\Z}{\mathbb{Z}}
\newcommand{\N}{\mathbb{N}}
\newcommand{\bbC}{\mathbb{C}} 
\newcommand{\K}{\mathbb{K}} % Base field which is either \R or \bbC
\newcommand{\calA}{\mathcal{A}} % Banach Algebras
\newcommand{\calB}{\mathcal{B}} % Banach Algebras
\newcommand{\calI}{\mathcal{I}} % ideal in a Banach algebra
\newcommand{\calJ}{\mathcal{J}} % ideal in a Banach algebra
\newcommand{\frakM}{\mathfrak{M}} % sigma-algebra
\newcommand{\frakS}{\mathfrak{S}} % symmetric group
\newcommand{\calO}{\mathcal{O}} % Ring of integers
\newcommand{\bbA}{\mathbb{A}} % Adele (or ring thereof)
\newcommand{\bbI}{\mathbb{I}} % Idele (or group thereof)

% Categories
\newcommand{\catTopp}{\mathbf{Top}_*}
\newcommand{\catGrp}{\mathbf{Grp}}
\newcommand{\catTopGrp}{\mathbf{TopGrp}}
\newcommand{\catSets}{\mathbf{Sets}}
\newcommand{\sets}[1]{#1\text{-}\mathbf{sets}}
\newcommand{\catFinSets}{\mathbf{FinSets}}
\newcommand{\catTop}{\mathbf{Top}}
\newcommand{\catRing}{\mathbf{Ring}}
\newcommand{\catCRing}{\mathbf{CRing}} % comm. rings
\newcommand{\catMod}{\mathbf{Mod}}
\newcommand{\catMon}{\mathbf{Mon}}
\newcommand{\catMan}{\mathbf{Man}} % manifolds
\newcommand{\catDiff}{\mathbf{Diff}} % smooth manifolds
\newcommand{\catAlg}{\mathbf{Alg}}
\newcommand{\catRep}{\mathbf{Rep}} % representations 
\newcommand{\catVec}{\mathbf{Vec}}

% Group and Representation Theory
\newcommand{\chr}{\operatorname{char}}
\newcommand{\Aut}{\operatorname{Aut}}
\newcommand{\GL}{\operatorname{GL}}
\newcommand{\im}{\operatorname{im}}
\newcommand{\tr}{\operatorname{tr}}
\newcommand{\id}{\mathbf{id}}
\newcommand{\cl}{\mathbf{cl}}
\newcommand{\Gal}{\operatorname{Gal}}
\newcommand{\Tr}{\operatorname{Tr}}
\newcommand{\sgn}{\operatorname{sgn}}
\newcommand{\Sym}{\operatorname{Sym}}
\newcommand{\Alt}{\operatorname{Alt}}

% Commutative and Homological Algebra
\newcommand{\spec}{\operatorname{spec}}
\newcommand{\mspec}{\operatorname{m-spec}}
\newcommand{\Spec}{\operatorname{Spec}}
\newcommand{\MaxSpec}{\operatorname{MaxSpec}}
\newcommand{\Tor}{\operatorname{Tor}}
\newcommand{\tor}{\operatorname{tor}}
\newcommand{\Ann}{\operatorname{Ann}}
\newcommand{\Supp}{\operatorname{Supp}}
\newcommand{\Hom}{\operatorname{Hom}}
\newcommand{\End}{\operatorname{End}}
\newcommand{\coker}{\operatorname{coker}}
\newcommand{\limit}{\varprojlim}
\newcommand{\colimit}{%
  \mathop{\mathpalette\colimit@{\rightarrowfill@\textstyle}}\nmlimits@
}
\makeatother


\newcommand{\fraka}{\mathfrak{a}} % ideal
\newcommand{\frakb}{\mathfrak{b}} % ideal
\newcommand{\frakc}{\mathfrak{c}} % ideal
\newcommand{\frakf}{\mathfrak{f}} % face map
\newcommand{\frakg}{\mathfrak{g}}
\newcommand{\frakh}{\mathfrak{h}}
\newcommand{\frakm}{\mathfrak{m}} % maximal ideal
\newcommand{\frakn}{\mathfrak{n}} % naximal ideal
\newcommand{\frakp}{\mathfrak{p}} % prime ideal
\newcommand{\frakq}{\mathfrak{q}} % qrime ideal
\newcommand{\fraks}{\mathfrak{s}}
\newcommand{\frakt}{\mathfrak{t}}
\newcommand{\frakz}{\mathfrak{z}}
\newcommand{\frakA}{\mathfrak{A}}
\newcommand{\frakI}{\mathfrak{I}}
\newcommand{\frakJ}{\mathfrak{J}}
\newcommand{\frakK}{\mathfrak{K}}
\newcommand{\frakL}{\mathfrak{L}}
\newcommand{\frakN}{\mathfrak{N}} % nilradical 
\newcommand{\frakO}{\mathfrak{O}} % dedekind domain
\newcommand{\frakP}{\mathfrak{P}} % Prime ideal above
\newcommand{\frakQ}{\mathfrak{Q}} % Qrime ideal above 
\newcommand{\frakR}{\mathfrak{R}} % jacobson radical
\newcommand{\frakU}{\mathfrak{U}}
\newcommand{\frakV}{\mathfrak{V}}
\newcommand{\frakW}{\mathfrak{W}}
\newcommand{\frakX}{\mathfrak{X}}

% General/Differential/Algebraic Topology 
\newcommand{\scrA}{\mathscr{A}} % abelian category
\newcommand{\scrB}{\mathscr{B}} % bbelian category
\newcommand{\scrC}{\mathscr{C}} % category
\newcommand{\scrD}{\mathscr{D}} % dategory
\newcommand{\scrF}{\mathscr{F}}
\newcommand{\scrM}{\mathscr{M}}
\newcommand{\scrN}{\mathscr{N}}
\newcommand{\scrP}{\mathscr{P}}
\newcommand{\scrO}{\mathscr{O}} % sheaf
\newcommand{\scrR}{\mathscr{R}}
\newcommand{\scrS}{\mathscr{S}}
\newcommand{\bbH}{\mathbb H}
\newcommand{\Int}{\operatorname{Int}}
\newcommand{\psimeq}{\simeq_p}
\newcommand{\wt}[1]{\widetilde{#1}}
\newcommand{\RP}{\mathbb{R}\text{P}}
\newcommand{\CP}{\mathbb{C}\text{P}}

% Miscellaneous
\newcommand{\wh}[1]{\widehat{#1}}
\newcommand{\calM}{\mathcal{M}}
\newcommand{\calP}{\mathcal{P}}
\newcommand{\onto}{\twoheadrightarrow}
\newcommand{\into}{\hookrightarrow}
\newcommand{\Gr}{\operatorname{Gr}}
\newcommand{\Span}{\operatorname{Span}}
\newcommand{\ev}{\operatorname{ev}}
\newcommand{\weakto}{\stackrel{w}{\longrightarrow}}

\newcommand{\define}[1]{\textcolor{blue}{\textit{#1}}}
\newcommand{\caution}[1]{\textcolor{red}{\textit{#1}}}
\newcommand{\important}[1]{\textcolor{red}{\textit{#1}}}
\renewcommand{\mod}{~\mathrm{mod}~}
\renewcommand{\le}{\leqslant}
\renewcommand{\leq}{\leqslant}
\renewcommand{\ge}{\geqslant}
\renewcommand{\geq}{\geqslant}
\newcommand{\Res}{\operatorname{Res}}
\newcommand{\floor}[1]{\left\lfloor #1\right\rfloor}
\newcommand{\ceil}[1]{\left\lceil #1\right\rceil}
\newcommand{\gl}{\mathfrak{gl}}
\newcommand{\ad}{\operatorname{ad}}
\newcommand{\Stab}{\operatorname{Stab}}
\newcommand{\bfX}{\mathbf{X}}
\newcommand{\Ind}{\operatorname{Ind}}
\newcommand{\bfG}{\mathbf{G}}
\newcommand{\rank}{\operatorname{rank}}
\newcommand{\calo}{\mathcal{o}}
\newcommand{\frako}{\mathfrak{o}}
\newcommand{\Cl}{\operatorname{Cl}}

\newcommand{\idim}{\operatorname{idim}}
\newcommand{\pdim}{\operatorname{pdim}}
\newcommand{\Ext}{\operatorname{Ext}}
\newcommand{\co}{\operatorname{co}}
\newcommand{\bfO}{\mathbf{O}}
\newcommand{\bfF}{\mathbf{F}} % Fitting Subgroup
\newcommand{\Syl}{\operatorname{Syl}}
\newcommand{\nor}{\vartriangleleft}
\newcommand{\noreq}{\trianglelefteqslant}
\newcommand{\subnor}{\nor\!\nor}
\newcommand{\Soc}{\operatorname{Soc}}
\newcommand{\core}{\operatorname{core}}
\newcommand{\Sd}{\operatorname{Sd}}
\newcommand{\mesh}{\operatorname{mesh}}
\newcommand{\sminus}{\setminus}
\newcommand{\diam}{\operatorname{diam}}
\newcommand{\Ass}{\operatorname{Ass}}
\newcommand{\projdim}{\operatorname{proj~dim}}
\newcommand{\injdim}{\operatorname{inj~dim}}
\newcommand{\gldim}{\operatorname{gl~dim}}
\newcommand{\embdim}{\operatorname{emb~dim}}
\newcommand{\hght}{\operatorname{ht}}
\newcommand{\depth}{\operatorname{depth}}
\newcommand{\ul}[1]{\underline{#1}}
\newcommand{\type}{\operatorname{type}}

\begin{document}

\maketitle

\begin{frame}{Galois Theory \`{a} la Grothendieck}
	Let $k$ be a field and fix its separable and algebraic closures $k_s\subseteq\overline k$. Let $G_k\coloneq\Gal(k_s\mid k)$, which is a profinite group through the isomorphism:
	\begin{equation*}
		\Gal(k_s\mid k)\cong\varprojlim_{\substack{k\subseteq K\subseteq k_s\\ [L : k] < \infty}}\Gal(L\mid k).
	\end{equation*}\pause

	If $L$ is a finite separable extension of $k$, then there is a natural action of $G_k$ on $\Hom_{k}(L, k_s)$ given by 
	\begin{equation*}
		G_k\times\Hom_{k}(L, k_s)\to\Hom_{k}(L, k_s)\qquad (g,\varphi)\mapsto g\circ\varphi.
	\end{equation*}\pause
	The stabilizer of $\varphi\in\Hom_k(L, k_s)$ is $\Gal(k_s\mid\varphi(L))$, i.e., an open subgroup of $G_k$, whence $\Hom_k(L, k_s)$ is a continuous $G_k$-set. \pause
	
	Clearly, this action is also transitive.
\end{frame}

\begin{frame}{\'Etale Algebras and the Fundamental Theorem}
	\begin{definition}
		A finite-dimensional $k$-algebra $A$ is said to be \define{\'etale} over $k$ if it is isomorphic to a finite direct product of separable extensions of $k$.
	\end{definition}\pause

	As in the preceding slide, there is a natural continuous action:
	\begin{equation*}
		G_k\times\Hom_{k}(A, k_s)\to\Hom_{k}(A, k_s)\qquad (g,\varphi)\mapsto g\circ\varphi.
	\end{equation*}\pause

	\begin{theorem}[Fundamental Theorem of Galois Theory]
		The functor mapping a finite \'etale $k$-algebra $A$ to the finite $G_k$-set $\Hom_k(A, k_s)$ gives an anti-equivalence between the category of finite \'etale $k$-algebras and the category of finte sets with a continuous $G_k$-action. Here separable extensions correspond to transitive $G_k$-sets.
	\end{theorem}
\end{frame}

\begin{frame}{Galois Theory of Covering Spaces}
	Let $X$ be a path connected, locally path connected, semilocally simply connected topological space and fix a basepoint $x_0\in X$.  \pause 

	If $p: Y\to X$ is a finite-sheeted covering space, there is a natural action of $\pi_1(X, x_0)$ on the (finite) fibre $p^{-1}(x_0)$, known as the \emph{monodromy action}.  \pause 

	Since this action factors through a finite quotient of $\pi_1(X, x_0)$, it induces a natural continuous action of the profinite completion $\wh{\pi_1(X, x_0)}$ on $p^{-1}(x_0)$.  \pause 

	\begin{theorem}[Classification of Covering Spaces]
		The aforementioned fibre functor induces an equivalence between the category of finite-sheeted covers of $X$ and the category of continuous finite sets with a continuous $\wh{\pi_1(X, x_0)}$-action. Here connected covers correspond to transitive $\wh{\pi_1(X, x_0)}$-sets.
	\end{theorem}
\end{frame}

\begin{frame}{The need for an ``algebraic'' fundamental group}
	The Zariski topology on an algebraic variety is way too coarse. In fact: 
	\begin{proposition}
		Every irreducible algebraic variety over an uncountable algebraically closed field is contractible in the Zariski Topology.
	\end{proposition}\pause 

	It is therefore clear that the na\"ive definition of a fundamental group using homotopy classes of paths is bound to fail for algebraic varieties. \pause 

	Motivated by the classification theorem for finite-sheeted covering spaces, Grothendieck envisioned a definition of an algebraic fundamental group. \pause 

	To do this, he needed the ``right'' notion of a finite covering space in algebraic geometry. This purpose is served by the \define{finite \'etale morphisms}.  \pause 
	
	All that remains is to establish an equivalence of some suitable categories.
\end{frame}

\begin{frame}{Galois Categories}
	\fontsize{10pt}{7.2}\selectfont
	A \define{Galois category} is a pair $(\scrC, F)$ where $\scrC$ is a category and $F:\scrC\to\catSets$ is a functor with $F(a)$ a finite set for each object $a\in\scrC$, satisfying the following axioms: \pause 

	\begin{enumerate}
		\item[G1] There is a terminal object in $\scrC$ and all fibred products exist in $\scrC$. \pause 

		\item[G2] An initial object exists in $\scrC$, finite coproducts exist in $\scrC$, and for any object in $\scrC$ the \caution{quotient} by a finite group of automorphisms exists. \pause 

		\item[G3] Every morphism $u$ in $\scrC$ factors as $u = u'u''$ where $u'$ is a monomorphism and $u''$ is an epimorphism. Every monomorphism $f: X\to Y$ in $\scrC$ is an isomorphism of $X$ with a direct summand of $Y$. \pause 

		\item[G4] The functor $F$ sends terminal objects to terminal objects and commutes with fibred products. \pause 

		\item[G5] The functor F transforms initial objects into initial objects, commutes with finite sums, sends epimorphisms to epimorphisms and commutes with passage to the quotient by a finite group of automorphisms. \pause 

		\item[G6] If $u$ is a morphism in $\scrC$ such that $F(u)$ is an isomorphism, then  $u$ is an isomorphism. \pause 
	\end{enumerate}
	In this case, the functor $F$ is called a \define{fundamental functor}.
\end{frame}

\begin{frame}{Examples of Galois Categories}
	\begin{itemize}
		\item Clearly, the category of finite sets $\catFinSets$ with the forgetful functor $U: \catFinSets\to\catSets$ is a Galois category. \pause 

		\item Let $\pi$ be a profinite group. The pair $(\sets{\pi}, U)$, where $\sets{\pi}$ is the category of finite $\pi$-sets with a continuous action and $U: \sets{\pi}\to\catSets$ is the forgetful functor is a Galois category. \pause 

		\item Let $X$ be a connected scheme. Let $\mathbf{FEt}_X$ denote the category of finite \'etale maps $Y\to X$ with morphisms $f: Y\to Z$ making 
		\begin{equation*}
			\xymatrix {
				Y\ar[rr]^f\ar[rd] & & Z\ar[ld]\\
				& X &
			}
		\end{equation*}
		commute. Fix a geometric point $x_0: \Spec\Omega\to X$. This defines a fibre functor $\operatorname{Fib}_{x_0}: \mathbf{FEt}_X\to\catSets$. The pair $\left(\mathbf{FEt}_X, \operatorname{Fib}_{x_0}\right)$ forms a Galois category.
	\end{itemize}
\end{frame}

\begin{frame}{The automorphism group of $F$}
	Let $(\scrC, F)$ be a \caution{small} Galois category. \pause 
	Note that $\Aut_{\scrC}(F)$ is a group of natural isomorphisms $\eta: F\to F$. \pause 

	Each such natural isomorphism can be identified with a tuple $(\eta_X)_{X\in\scrC}$, where $\eta_X: F(X)\to F(X)$ is an isomorphism in $\catSets$, i.e., is a bijection of sets. \pause 
	
	This tuple is an element in the profinite group
	\begin{equation*}
		\Gamma \coloneq \prod_{X\in\scrC}\frakS_{F(X)}.
	\end{equation*}\pause 
	Further the tuple must be such that for each morphism $f: X\to Y$ in $\scrC$, the diagram 
	\begin{equation*}
		\xymatrix {
			F(X)\ar[r]^{F(f)}\ar[d]_{\eta_X} & F(Y)\ar[d]^{\eta_Y}\\
			F(X)\ar[r]_{F(f)} & F(Y)
		}
	\end{equation*}
	commutes.
\end{frame}

\begin{frame}{The automorphism group of $F$, contd.}
	We show that $\Aut_{\scrC}(F)$ is closed in $\Gamma$. \pause 

	Fix $Y,Z\in\scrC$ and a morphism $f: Y\to Z$. Consider the set 
	\begin{equation*}
		C_f\coloneq\left\{\left(\eta_X\right)_{X\in\scrC}\in\Gamma\colon \eta_Y\circ F(f) = F(f)\circ\eta_X\right\}.
	\end{equation*} \pause 
	Note that 
	\begin{equation*}
		C_f = \bigcup\prod_{\substack{X\in\scrC\\ X\ne Y, Z}}\frakS_{F(X)}\times\{\eta_Y\}\times\{\eta_Z\},
	\end{equation*} 
	where the union ranges over all pairs $(\eta_Y,\eta_Z)$ satisfying $\eta_Y\circ F(f) = F(f)\circ\eta_Z$.  \pause 
	Being a finite union of closed sets, $C_f$ is closed in $\Gamma$. \pause 
	Finally, since
	\begin{equation*}
		\Aut_{\scrC}(F) = \bigcap_{\substack{Y\xrightarrow{f}Z\\\text{in }\scrC}} C_f,
	\end{equation*}
	the conclusion follows. \pause 
	In particular, $\Aut_{\scrC}(F)$ is a profinite group. 
\end{frame}

\begin{frame}{The equivalence functor}
	Next, let $Y\in\scrC$. There is a natural action of $\Aut_{\scrC}(F)$ on $F(Y)$ given by 
	\begin{equation*}
		\left(\eta_X\right)_{X\in\scrC}\cdot a = \eta_Y(a)\qquad \forall~a\in F(Y).
	\end{equation*} \pause 
	The stabilizer of $a\in F(Y)$ is 
	\begin{equation*}
		\Aut_{\scrC}(F)\cap\left(\prod_{\substack{X\in\scrC\\ X\ne Y}}\times\Stab_{\frakS_{F(X)}}(a)\right),
	\end{equation*}
	which is an open subgroup of $\Aut_{\scrC}(F)$. \pause 
	Thus, each $F(Y)$ is a finite $\Aut_{\scrC}(F)$-set with a continuous action. \pause 
	
	This gives a natural functor $H: \scrC\to\sets{\Aut_{\scrC}(F)}$.
\end{frame}

\begin{frame}{The Main Theorem}
	\begin{theorem}[Fundamental Theorem of Galois Categories]
		Let $(\scrC, F)$ be an essentially small Galois category. Then 
		\begin{enumerate}
			\item The functor $H:\scrC\to\sets{\Aut_{\scrC}(F)}$ is an equivalence of categories. \pause 
			\item If $\pi$ is a profinite group such that the categories $\scrC$ and $\sets{\pi}$ are equivalent through an equivalence such that the composition $\scrC\xrightarrow{\sim}\sets{\pi}\to\catSets$ yields the functor $F$, then $\pi$ is canonically isomorphic to $\Aut_{\scrC}(F)$. \pause 
			\item If $F'$ is another fundamental functor on $\scrC$, then $F$ and $F'$ are naturally isomorphic. \pause 
			\item If $\pi$ is a profinite group such that the categories $\scrC$ and $\pi$-sets are equivalent, then $\pi\cong\Aut_{\scrC}(F)$ as topological groups. \pause 
		\end{enumerate}
	\end{theorem}
\end{frame}

\begin{frame}{The \'Etale Fundamental Group}
	\begin{definition}
		Let $X$ be a connected scheme and $x_0: \Spec\Omega\to X$ be a geometric point. The \define{\'etale fundamental group} $\pi_1^{\acute{e}t}(X, x_0)$ to be $\Aut(\operatorname{Fib}_{x_0})$.
	\end{definition}

	Recall that there is an anti-equivalence between the category of commutative rings and the category of affine schemes. 

	Therefore, there is an anti-equivalence between the category $\mathbf{FEt}_{\Spec k}$ and the category of \'etale $k$-algebras.

	Choosing a geometric point in $\Spec k$ is tantamount to fixing a separable closure $k_s$ of $k$.

	As we have seen earlier, $\mathbf{FEt}_{\Spec k}$ is equivalent to the category $\sets{G_k}$. 
	In particular, $\pi_1^{\acute{e}t}(\Spec k, x_0)\cong\Gal(k_s\mid k)$.
\end{frame}

\begin{frame}{La fin}
	\Large 
	\begin{center}
		\textbf{Thank you for your attention!}
	\end{center}
\end{frame}

\end{document}