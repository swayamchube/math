\documentclass[10pt]{article}
% \usepackage{./arxiv}

\title{Galois Categories and the \'Etale Fundamental Group}
\author{Swayam Chube}
\date{\today}

\usepackage[utf8]{inputenc} % allow utf-8 input
\usepackage[T1]{fontenc}    % use 8-bit T1 fonts
\usepackage{hyperref}       % hyperlinks
\usepackage{url}            % simple URL typesetting
\usepackage{booktabs}       % professional-quality tables
\usepackage{amsfonts}       % blackboard math symbols
\usepackage{nicefrac}       % compact symbols for 1/2, etc.
\usepackage{microtype}      % microtypography
\usepackage{graphicx}
\usepackage{natbib}
\usepackage{doi}
\usepackage{amssymb}
\usepackage{bbm}
\usepackage{amsthm}
\usepackage{amsmath}
\usepackage{xcolor}
\usepackage{theoremref}
\usepackage{enumitem}
% \usepackage{mathpazo}
% \usepackage{euler}
\usepackage{lmodern}
\usepackage{sansmath, sfmath}
\usepackage{mathrsfs}
\setlength{\marginparwidth}{2cm}
\usepackage{todonotes}
\usepackage{stmaryrd}
\usepackage[all,cmtip]{xy} % For diagrams, praise the Freyd-Mitchell theorem 
\usepackage{marvosym}
\usepackage{geometry}
\usepackage{titlesec}
\usepackage{mathtools}
% \usepackage{fontspec}
\usepackage{tikz}
\usetikzlibrary{cd}

\renewcommand{\qedsymbol}{$\blacksquare$}
\renewcommand{\familydefault}{\sfdefault}

% Uncomment to override  the `A preprint' in the header
% \renewcommand{\headeright}{}
% \renewcommand{\undertitle}{}
% \renewcommand{\shorttitle}{}

\hypersetup{
    pdfauthor={Lots of People},
    colorlinks=true,
}

\newtheoremstyle{thmstyle}%               % Name
  {}%                                     % Space above
  {}%                                     % Space below
  {}%                             % Body font
  {}%                                     % Indent amount
  {\bfseries\scshape}%                            % Theorem head font
  {.}%                                    % Punctuation after theorem head
  { }%                                    % Space after theorem head, ' ', or \newline
  {\thmname{#1}\thmnumber{ #2}\thmnote{ (#3)}}%                                     % Theorem head spec (can be left empty, meaning `normal')

\newtheoremstyle{defstyle}%               % Name
  {}%                                     % Space above
  {}%                                     % Space below
  {}%                                     % Body font
  {}%                                     % Indent amount
  {\bfseries\scshape}%                            % Theorem head font
  {.}%                                    % Punctuation after theorem head
  { }%                                    % Space after theorem head, ' ', or \newline
  {\thmname{#1}\thmnumber{ #2}\thmnote{ (#3)}}%                                     % Theorem head spec (can be left empty, meaning `normal')

\theoremstyle{thmstyle}
\newtheorem{theorem}{Theorem}[section]
\newtheorem{lemma}[theorem]{Lemma}
\newtheorem{proposition}[theorem]{Proposition}

\theoremstyle{defstyle}
\newtheorem{definition}[theorem]{Definition}
\newtheorem{corollary}[theorem]{Corollary}
\newtheorem{porism}[theorem]{Porism}
\newtheorem{remark}[theorem]{Remark}
\newtheorem{interlude}[theorem]{Interlude}
\newtheorem{example}[theorem]{Example}
\newtheorem*{notation}{Notation}
\newtheorem*{claim}{Claim}

% Common Algebraic Structures
\newcommand{\R}{\mathbb{R}}
\newcommand{\Q}{\mathbb{Q}}
\newcommand{\Z}{\mathbb{Z}}
\newcommand{\N}{\mathbb{N}}
\newcommand{\bbC}{\mathbb{C}} 
\newcommand{\K}{\mathbb{K}} % Base field which is either \R or \bbC
\newcommand{\calA}{\mathcal{A}} % Banach Algebras
\newcommand{\calB}{\mathcal{B}} % Banach Algebras
\newcommand{\calI}{\mathcal{I}} % ideal in a Banach algebra
\newcommand{\calJ}{\mathcal{J}} % ideal in a Banach algebra
\newcommand{\frakM}{\mathfrak{M}} % sigma-algebra
\newcommand{\frakS}{\mathfrak{S}} % symmetric group
\newcommand{\calO}{\mathcal{O}} % Ring of integers
\newcommand{\bbA}{\mathbb{A}} % Adele (or ring thereof)
\newcommand{\bbI}{\mathbb{I}} % Idele (or group thereof)

% Categories
\newcommand{\catTopp}{\mathbf{Top}_*}
\newcommand{\catGrp}{\mathbf{Grp}}
\newcommand{\catTopGrp}{\mathbf{TopGrp}}
\newcommand{\catSets}{\mathbf{Sets}}
\newcommand{\sets}[1]{#1\text{-}\mathbf{sets}}
\newcommand{\catFinSets}{\mathbf{FinSets}}
\newcommand{\catTop}{\mathbf{Top}}
\newcommand{\catRing}{\mathbf{Ring}}
\newcommand{\catCRing}{\mathbf{CRing}} % comm. rings
\newcommand{\catMod}{\mathbf{Mod}}
\newcommand{\catMon}{\mathbf{Mon}}
\newcommand{\catMan}{\mathbf{Man}} % manifolds
\newcommand{\catDiff}{\mathbf{Diff}} % smooth manifolds
\newcommand{\catAlg}{\mathbf{Alg}}
\newcommand{\catRep}{\mathbf{Rep}} % representations 
\newcommand{\catVec}{\mathbf{Vec}}

% Group and Representation Theory
\newcommand{\chr}{\operatorname{char}}
\newcommand{\Aut}{\operatorname{Aut}}
\newcommand{\GL}{\operatorname{GL}}
\newcommand{\im}{\operatorname{im}}
\newcommand{\tr}{\operatorname{tr}}
\newcommand{\id}{\mathbf{id}}
\newcommand{\cl}{\mathbf{cl}}
\newcommand{\Gal}{\operatorname{Gal}}
\newcommand{\Tr}{\operatorname{Tr}}
\newcommand{\sgn}{\operatorname{sgn}}
\newcommand{\Sym}{\operatorname{Sym}}
\newcommand{\Alt}{\operatorname{Alt}}

% Commutative and Homological Algebra
\newcommand{\spec}{\operatorname{spec}}
\newcommand{\mspec}{\operatorname{m-spec}}
\newcommand{\Spec}{\operatorname{Spec}}
\newcommand{\MaxSpec}{\operatorname{MaxSpec}}
\newcommand{\Tor}{\operatorname{Tor}}
\newcommand{\tor}{\operatorname{tor}}
\newcommand{\Ann}{\operatorname{Ann}}
\newcommand{\Supp}{\operatorname{Supp}}
\newcommand{\Hom}{\operatorname{Hom}}
\newcommand{\End}{\operatorname{End}}
\newcommand{\coker}{\operatorname{coker}}
\newcommand{\limit}{\varprojlim}
\newcommand{\colimit}{%
  \mathop{\mathpalette\colimit@{\rightarrowfill@\textstyle}}\nmlimits@
}
\makeatother


\newcommand{\fraka}{\mathfrak{a}} % ideal
\newcommand{\frakb}{\mathfrak{b}} % ideal
\newcommand{\frakc}{\mathfrak{c}} % ideal
\newcommand{\frakf}{\mathfrak{f}} % face map
\newcommand{\frakg}{\mathfrak{g}}
\newcommand{\frakh}{\mathfrak{h}}
\newcommand{\frakm}{\mathfrak{m}} % maximal ideal
\newcommand{\frakn}{\mathfrak{n}} % naximal ideal
\newcommand{\frakp}{\mathfrak{p}} % prime ideal
\newcommand{\frakq}{\mathfrak{q}} % qrime ideal
\newcommand{\fraks}{\mathfrak{s}}
\newcommand{\frakt}{\mathfrak{t}}
\newcommand{\frakz}{\mathfrak{z}}
\newcommand{\frakA}{\mathfrak{A}}
\newcommand{\frakI}{\mathfrak{I}}
\newcommand{\frakJ}{\mathfrak{J}}
\newcommand{\frakK}{\mathfrak{K}}
\newcommand{\frakL}{\mathfrak{L}}
\newcommand{\frakN}{\mathfrak{N}} % nilradical 
\newcommand{\frakO}{\mathfrak{O}} % dedekind domain
\newcommand{\frakP}{\mathfrak{P}} % Prime ideal above
\newcommand{\frakQ}{\mathfrak{Q}} % Qrime ideal above 
\newcommand{\frakR}{\mathfrak{R}} % jacobson radical
\newcommand{\frakU}{\mathfrak{U}}
\newcommand{\frakV}{\mathfrak{V}}
\newcommand{\frakW}{\mathfrak{W}}
\newcommand{\frakX}{\mathfrak{X}}

% General/Differential/Algebraic Topology 
\newcommand{\scrA}{\mathscr{A}} % abelian category
\newcommand{\scrB}{\mathscr{B}} % bbelian category
\newcommand{\scrC}{\mathscr{C}} % category
\newcommand{\scrD}{\mathscr{D}} % dategory
\newcommand{\scrF}{\mathscr{F}}
\newcommand{\scrM}{\mathscr{M}}
\newcommand{\scrN}{\mathscr{N}}
\newcommand{\scrP}{\mathscr{P}}
\newcommand{\scrO}{\mathscr{O}} % sheaf
\newcommand{\scrR}{\mathscr{R}}
\newcommand{\scrS}{\mathscr{S}}
\newcommand{\bbH}{\mathbb H}
\newcommand{\Int}{\operatorname{Int}}
\newcommand{\psimeq}{\simeq_p}
\newcommand{\wt}[1]{\widetilde{#1}}
\newcommand{\RP}{\mathbb{R}\text{P}}
\newcommand{\CP}{\mathbb{C}\text{P}}

% Miscellaneous
\newcommand{\wh}[1]{\widehat{#1}}
\newcommand{\calM}{\mathcal{M}}
\newcommand{\calP}{\mathcal{P}}
\newcommand{\onto}{\twoheadrightarrow}
\newcommand{\into}{\hookrightarrow}
\newcommand{\Gr}{\operatorname{Gr}}
\newcommand{\Span}{\operatorname{Span}}
\newcommand{\ev}{\operatorname{ev}}
\newcommand{\weakto}{\stackrel{w}{\longrightarrow}}

\newcommand{\define}[1]{\textcolor{blue}{\textit{#1}}}
\newcommand{\caution}[1]{\textcolor{red}{\textit{#1}}}
\newcommand{\important}[1]{\textcolor{red}{\textit{#1}}}
\renewcommand{\mod}{~\mathrm{mod}~}
\renewcommand{\le}{\leqslant}
\renewcommand{\leq}{\leqslant}
\renewcommand{\ge}{\geqslant}
\renewcommand{\geq}{\geqslant}
\newcommand{\Res}{\operatorname{Res}}
\newcommand{\floor}[1]{\left\lfloor #1\right\rfloor}
\newcommand{\ceil}[1]{\left\lceil #1\right\rceil}
\newcommand{\gl}{\mathfrak{gl}}
\newcommand{\ad}{\operatorname{ad}}
\newcommand{\Stab}{\operatorname{Stab}}
\newcommand{\bfX}{\mathbf{X}}
\newcommand{\Ind}{\operatorname{Ind}}
\newcommand{\bfG}{\mathbf{G}}
\newcommand{\rank}{\operatorname{rank}}
\newcommand{\calo}{\mathcal{o}}
\newcommand{\frako}{\mathfrak{o}}
\newcommand{\Cl}{\operatorname{Cl}}

\newcommand{\idim}{\operatorname{idim}}
\newcommand{\pdim}{\operatorname{pdim}}
\newcommand{\Ext}{\operatorname{Ext}}
\newcommand{\co}{\operatorname{co}}
\newcommand{\bfO}{\mathbf{O}}
\newcommand{\bfF}{\mathbf{F}} % Fitting Subgroup
\newcommand{\Syl}{\operatorname{Syl}}
\newcommand{\nor}{\vartriangleleft}
\newcommand{\noreq}{\trianglelefteqslant}
\newcommand{\subnor}{\nor\!\nor}
\newcommand{\Soc}{\operatorname{Soc}}
\newcommand{\core}{\operatorname{core}}
\newcommand{\Sd}{\operatorname{Sd}}
\newcommand{\mesh}{\operatorname{mesh}}
\newcommand{\sminus}{\setminus}
\newcommand{\diam}{\operatorname{diam}}
\newcommand{\Ass}{\operatorname{Ass}}
\newcommand{\projdim}{\operatorname{proj~dim}}
\newcommand{\injdim}{\operatorname{inj~dim}}
\newcommand{\gldim}{\operatorname{gl~dim}}
\newcommand{\embdim}{\operatorname{emb~dim}}
\newcommand{\hght}{\operatorname{ht}}
\newcommand{\depth}{\operatorname{depth}}
\newcommand{\ul}[1]{\underline{#1}}
\newcommand{\type}{\operatorname{type}}


\geometry {
    margin = 1in
}

\titleformat
{\section}
[block]
{\Large\bfseries\sffamily}
{\S\thesection}
{0.5em}
{\centering}
[]


\titleformat
{\subsection}
[block]
{\normalfont\bfseries\sffamily}
{\S\S}
{0.5em}
{\centering}
[]


\begin{document}
\maketitle

\section{Preliminaries on Profinite Groups}

\begin{proposition}\thlabel{prop:profinite-action-properties}
    Let $\pi$ be a profinite group acting on a set $E$. Then 
    \begin{enumerate}[label=(\arabic*)]
        \item The action is continuous if and only if for each $e\in E$, $\Stab_{\pi}(e)$ is open in $\pi$.
        \item If $E$ is finite, the action is continuous if and only if its kernel $\left\{\sigma\in\pi\colon \sigma e = e~\forall~e\in E\right\}$ is open in $\pi$.
        \item Any finite transitive $\pi$-set is isomorphic to $\pi/\pi'$ for a certain open subgroup $\pi'$ of $\pi$.
    \end{enumerate}
\end{proposition}
\begin{proof}
\begin{enumerate}[label=(\arabic*)]
    \item If the action is continuous, then the function $\pi\to E$ given by $\sigma\mapsto \sigma e$ is continuous and the preimage of $e$, which is precisely the stabilizer of $e$ in $\pi$, is open. 
    
    Conversely, suppose every stabilizer is open. Let $A: \pi\times E\to E$ denote the action. Since $E$ is discrete, it suffices to show that $A^{-1}(e)$ is open for each $e\in E$. Let $e'\in\pi\cdot e$ and suppose $\tau_{e'}\in\pi$ is such that $\tau_{e'} e = e'$. Then 
    \begin{equation*}
        \left\{\sigma\colon \sigma e' = e\right\} = \tau_{e'}^{-1}\Stab_{\pi}(e'),
    \end{equation*}
    which is an open subset of $\pi$. Consequently, 
    \begin{equation*}
        A^{-1}(e) = \bigcup_{e'\in\pi\cdot e} \left\{(\sigma, e')\colon \sigma e' = e\right\} = \bigcup_{e'\in\pi\cdot e}\tau_{e'}^{-1}\Stab_\pi(e')\times\{e'\}
    \end{equation*}
    is an open subset of $\pi\times E$, as desired.
    \item % TODO: Add in the rest of the proof
\end{enumerate}
\end{proof}

\newpage
\section{Galois Categories}

\subsection{Statement of the Main Theorem}

\begin{definition}
    Let $\scrC$ be a category, $X$ an object of $\scrC$, and $G$ a subgroup of $\Aut_{\scrC}(X)$. The \define{quotient} of $X$ by $G$ is an object $X/G$ of $\scrC$ together with a morphism $p: X\to X/G$ satisfying
    \begin{enumerate}[label=(\roman*)]
        \item $p = p\circ\sigma$ for all $\sigma\in G$,
        \item if $X\xrightarrow{f} Y$ is a morphism in $\scrC$ such that $f = f\circ\sigma$ for all $\sigma\in G$, then there is a unique morphism $X/G\xrightarrow{g} Y$ making 
        \begin{equation*}
            \xymatrix {
                X\ar[r]^f\ar[d]_p & Y\\
                X/G\ar[ru]_g
            }
        \end{equation*}
        commute.
    \end{enumerate}
\end{definition}

The quotient of an object by a group need not exist in a category, but when it does, it must be unique up to a unique isomorphism.

\begin{definition}\thlabel{def:galois-categories}
    Let $\scrC$ be a category and $F:\scrC\to\catFinSets$ a (covariant) functor from $\scrC$ to the category of finite sets. We say that the pair $(\scrC, F)$ is a \define{Galois category}, or that $\scrC$ is a Galois category with \define{fundamental functor} $F$, if the following axioms are satisfied: 
    \begin{enumerate}[label=\textbf{(G\arabic*)}]
        \item There is a terminal object and $\scrC$ admits all fibred products. \label{G1}
        \item An initial object exists in $\scrC$, finite coproducts exist in $\scrC$, and for any object in $\scrC$, the quotient by a finite group of automorphisms exists. \label{G2}
        \item Any morphism $u$ in $\scrC$ factors as $u = u'\circ u''$ where $u'$ is a monomorphism and $u''$ is an epimorphism. 
        Every monomorphism $X\xrightarrow{f} Y$ in $\scrC$ is an isomorphism of $X$ with a direct summand of $Y$; i.e., there is an object $Z\xrightarrow{g} Y$ such that 
        \begin{equation*}
            \xymatrix {
                & X\ar[d]^f\\
                Z\ar[r]_g & Y
            }
        \end{equation*}
        is a coproduct diagram. \label{G3}
        \item The functor $F$ sends terminal objects to terminal objects and commutes with fibred products. \label{G4}
        \item The functor $F$ sends initial objects to initial objects, commutes with finite coproducts, sends epimorphisms to epimorphisms, and commutes with passage to the quotient by a finite group of automorphisms. \label{G5}
        \item If $u$ is a morphism in $\scrC$ such that $F(u)$ is an isomorphism, then $u$ is an isomorphism. \label{G6}
    \end{enumerate}
\end{definition}

\begin{proposition}\thlabel{prop:aut-F-exists}
    Let $(\scrC, F)$ be a small Galois category and set $\scrD = \left[\scrC,\catFinSets\right]$, the functor category between $\scrC$ and the category of finite sets. Then $\Aut_{\scrD}(F)$ is a profinite group acting continuously on $F(X)$ for every $X\in\scrC$.
\end{proposition}
\begin{proof}
    An element of $\Aut_{\scrD}(F)$ is a natural isomorphism $\eta: F\Rightarrow F$, i.e, each $\eta_X: F(X)\to F(X)$ is an isomorphism. Hence, we can identify $\Aut_{\scrD}(F)$ with a subgroup of $\displaystyle\prod_{X\in\scrC}\frakS_{F(X)}$, where $\frakS_{F(X)}$ is the group of permutations of $F(X)$. In particular, 
    \begin{equation*}
        \Aut_{\scrD}(F) = \left\{(\eta_X)_{X}\in\prod_{X\in\scrC}\frakS_{F(X)}\colon\text{ for each }Y\xrightarrow{f}Z\text{ in }\scrC,~\eta_Z\circ F(f) = F(f)\circ\eta_Y\right\}.
    \end{equation*}
    Let $Y\xrightarrow{f} Z$ be a morphism in $\scrC$. Then the set 
    \begin{equation*}
        \frakA_f = \left\{(\eta_X)_X \in\prod_{X\in\scrC} \frakS_{F(X)}\colon \eta_Z\circ F(f) = F(f)\circ\eta_Y\right\}.
    \end{equation*}
    is closed, as it is the finite union of the closed sets 
    \begin{equation*}
        \prod_{\substack{X\in\scrC\\ X\ne Y, Z}}\frakS_{F(X)}\times\{\eta_Y\}\times\{\eta_Z\},
    \end{equation*}
    where $\eta_Y\in\frakS_{F(Y)}$ and $\eta_Z\in\frakS_{F(Z)}$ satisfy $\eta_Z\circ F(f) = F(f)\circ\eta_Y$. Now, since 
    \begin{equation*}
        \Aut_{\scrD}(F) = \bigcap_{\substack{Y\xrightarrow{f} Z\\\text{in }\scrC}}\frakA_f,
    \end{equation*}
    it is a closed subgroup of $\displaystyle\prod_{X\in\scrC}\frakS_{F(X)}$, so that it is a profinite group.

    Finally, the map $\Aut_{\scrD}(F)\times F(X)\to F(X)$ given by $\left((\eta_X)_{X\in\scrC}, a\right)\longmapsto\eta_X(a)$ defines an action of $\Aut_{\scrD}(F)$ on $F(X)$. The stabilizer of each $a\in F(X)$ is precisely 
    \begin{equation*}
        \Aut_{\scrD}(F)\times\left(\prod_{\substack{Y\in\scrC\\ Y\ne X}\frakS_{F(Y)}}\times\Stab_{\frakS_{F(X)}}(a)\right),
    \end{equation*}
    which is an open subgroup of $\Aut_{\scrD}(F)$. Due to \thref{prop:profinite-action-properties}, this action is continuous.
\end{proof}

\begin{interlude}[Construction of the Main Functor]\thlabel{inter:main-functor}
    Let $(\scrC, F)$ be a small Galois category. Define the functor $H: \scrC\to\sets{\Aut(F)}$ sending each $X\in\scrC$ to $F(X)$ with the $\Aut(F)$-action as defined in the proof of \thref{prop:aut-F-exists}. If $Y\xrightarrow{f} Z$ is a morphism in $\scrC$, then the induced morphism $F(f): F(Y)\to F(Z)$ is $\Aut(F)$-linear: indeed, if $\eta = (\eta_X)_{X}\in\Aut(F)$, then for $y\in Y$, 
    \begin{equation*}
        F(f)\left(\eta y\right) = F(f)\left(\eta_Y y\right) = \eta_Z\left(F(f)(z)\right) = \eta F(f)(z).
    \end{equation*}
\end{interlude}

\begin{theorem}[Fundamental Theorem of Galois Categories]\thlabel{thm:fundamental-theorem}
    Let $(\scrC, F)$ be an essentially small Galois category. Then 
    \begin{enumerate}[label=(\arabic*)]
        \item The functor $H:\scrC\to\sets{\Aut(F)}$ is an equivalence of categories. 
        \item If $\pi$ is a profinite group such that the categories $\scrC$ and $\sets{\pi}$ are equivalent by an equivalence, that when composed with the forgetful functor $\sets{\pi}\to\catFinSets$ yields the funtor $F$, then $\pi$ is canonically isomorphic to $\Aut(F)$.
        \item If $F'$ is a second fundamental functor on $\scrC$, then $F$ and $F'$ are naturally isomorphic. 
        \item If $\pi$ is a profinite group such that the categories $\scrC$ and $\sets{\pi}$ are equivalent, then there is an isomorphism of profinite groups $\pi\cong\Aut(F)$ that is canonically determined up to an inner automorophism of $\Aut(F)$.
    \end{enumerate}
\end{theorem}

\begin{center}
    \boxed{\text{Henceforth, let $(\scrC, F)$ be a small Galois category}.}
\end{center}

\subsection{Subobjects and connected objects}

\begin{definition}
    Lt $X\in\scrC$. Consider the set $\left\{Y\to X\text{ a monomorphism}\right\}/\sim$ where 
    \begin{equation*}
        Y\xrightarrow{f} X\sim Y'\xrightarrow{f'} X
    \end{equation*}
    if and only if there is an isomorphism $Y\xrightarrow{\cong} Y'$ making 
    \begin{equation*}
        \xymatrix {
            Y\ar[d]_f\ar[r]^\cong & Y'\ar[ld]^{f'}\\
            X
        }
    \end{equation*}
    commute. Every equivalence class in the above is called a \define{subobject} of $X$.
\end{definition}

\begin{lemma}\thlabel{lem:monomorphism-iff-image-is-injective}
    $f$ is a monomorphism if and only if $F(f)$ is injective.
\end{lemma}
\begin{proof}
    Let $Y\xrightarrow{f} X$. We first show that $f$ is a monomorphism if and only if the canonical map $p_1: Y\times_X Y\to Y$ is an isomorphism. If $f$ is a monomorphism, then it is clear that $\xymatrix{
        Y\ar@{=}[r]\ar@{=}[d] & Y\ar[d]^f\\
        Y\ar[r]_f & X
    }$
    is a coproduct diagram, so that $p_1: Y\times_X Y\to Y$ is an isomorphism.

    Conversely, suppose $p_1: Y\times_X Y\to Y$ is an isomorphism and consider the commutative diagram 
    \begin{equation*}
        \xymatrix {
            Y\ar[rrd]^{\id_Y}\ar[rdd]_{\id_Y}\ar@{.>}[rd]|-{\theta} & & \\
            & Y\times_X Y\ar[r]^{p_1}\ar[d]_{p_2} & Y\ar[d]^f\\
            & Y\ar[r]_f & X
        }
    \end{equation*}
    Since $p_1$ is an isomorphism, it follows that $\theta = p_1^{-1}$ is an isomorphism. Further, since $p_2\circ\theta = \id_Y$, we must have that $p_1 = p_2$.

    Now, suppose $h_1, h_2: Z\to Y$ are morphisms in $\scrC$ satisfying $f\circ h_1 = f\circ h_2$, then there is a morphism $\varphi: Z\to Y\times_X Y$ making the required diagram commute. But then 
    \begin{equation*}
        h_1 = p_1\circ\varphi = p_2\circ\varphi = h_2,
    \end{equation*}
    so that $f$ is a monomorphism.  

    Coming back to the proof of the Lemma, we have 
    \begin{align*}
        F(f)\text{ is injective} &\iff F(f) \text{ is a monomorphism}\\
        &\iff F(p_1) \text{ is an isomorphism}\\
        &\iff  p_1\text{ is an isomorphism}\\
        &\iff f\text{ is a monomorphism},
    \end{align*}
    where the first equivalence follows from the classification of monomorphisms in $\catFinSets$, the second and last equivalences follow from what we just proved and \ref{G4}, and the third isomorphism follows from \ref{G6}.
\end{proof}

\begin{lemma}
    Two monomorphisms $Y\xrightarrow{f} X$ and $Y'\xrightarrow{f'} X$ are representative of the same subobject of $X$ if and only if $F(f)\left(F(Y)\right) = F(f')\left(F(Y')\right)$ as subsets of $F(X)$.
\end{lemma}
\begin{proof}
    Suppose the two objects represent the same subobject of $X$. Then there is an isomorphism $\theta: Y\xrightarrow{\sim} Y'$ such that $f = f'\circ\theta$. Then, $F(f)\left(F(Y)\right) = F(f')\circ F(\theta)\left(F(Y)\right)$ but $F(\theta)$ is an isomorphism, so is surjectiv,e and hence $F(f)\left(F(Y)\right) = F(f')\left(F(Y')\right)$.

    Conversely, suppose $F(f)\left(F(Y)\right) = F(f')\left(F(Y')\right)$. As $F$ commutes with fibred products, we have the following pullback squares 
    \begin{equation*}
        \xymatrix {
            Y\times_X Y'\ar[r]^-{p_1}\ar[d]_{p_2} & Y\ar[d]^f\\
            Y'\ar[r]_{f'} & X
        }
        \qquad 
        \xymatrix {
            F\left(Y\times_X Y'\right)\ar[r]^-{F(p_1)}\ar[d]_{F(p_2)} & Y\ar[d]^{F(f)}\\
            Y'\ar[r]_{F(f')} & X
        }
    \end{equation*}
    Since the latter is a pullback square, we have 
    \begin{equation*}
        F\left(Y\times_X Y'\right) = \left\{(y, y')\in F(Y)\times F(Y')\colon F(f)(y) = F(f')(y')\right\}.
    \end{equation*}
    As $F(f)$ and $F(f')$ are injective with the same image in $X$, it is clear that both $F(p_1)$ and $F(p_2)$ must be bijections, consequently, due to \ref{G6}, both $p_1$ and $p_2$ must be isomorphisms isomorphisms in $\scrC$. Finally, this gives $f = f'\circ(p_2\circ p_1^{-1})$, as desired.
\end{proof}

\begin{definition}
    An object $X\in\scrC$ is said to be \define{connected} if it has exactly two subobjects, $0\to X$ and $\id_X: X\to X$.
\end{definition}

\begin{proposition}
    Every object in $\scrC\ne 0$ is the coproduct of its connected subobjects.
\end{proposition}
\begin{proof}
    Let $X$ be a non-initial object in $\scrC$. We shall argue by induction on $\# F(X)$. If $\# F(X) = 1$, then $X$ is connected, for if $Y\xrightarrow{f} X$ is a subobject, then $F(Y)\xrightarrow{F(f)} F(X)$ is injective, so that $F(Y) = \emptyset$ or $F(Y) = F(X)$. In the latter case, $F(f)$ is an isomorphism and hence, so is $f$; on the other hand, if $F(Y) = \emptyset$, then $Y$ must be the initial object of $\scrC$\footnote{Indeed, if $0$ is ``the'' initial object of $\scrC$, then there is a unique morphism $0\xrightarrow{u} Y$ in $\scrC$. But since $F(u)$ is an isomorphism in $\catFinSets$, it follows from \ref{G6} that $u$ is an isomorphism.}. Suppose now that $\#F(X)\ge 2$; since there is nothing to prove when $X$ is connected, we may suppose that $X$ is not connected. Then there is a subobject $Y\xrightarrow{q_1} X$ of $X$ which is neither initial, nor an isomorphism. Due to \ref{G3}, there is a morphism $Z\xrightarrow{q_2} X$ such that $X = Y\coprod Z$. This coproduct diagram transforms into a coproduct diagram in $\catFinSets$, so that $F(q_2)$ is injective, consequently due to \thref{lem:monomorphism-iff-image-is-injective}, $q_2$ is a monomorphism. It follows that $Z\xrightarrow{q_2} X$ is another subobject of $X$. The inductive hypothesis applies and we can write $X$ coproduct of \emph{some} of its connected components. Since $\# F(X)$ is finite, it is clear that this is a finite coproduct.

    It remains to show that $X$ is the disjoint union of \emph{each} of its connected subobjects. Suppose $X = \coprod_{i = 1}^n X_i$ and $Y$ a connected subobject of $X$. I shall treat $F(Y)$ and $F(X_i)$ as subsets of $F(X)$ for ease of notation. Since $F(X) = \coprod_i F(X_i)$, there is some index $j$ such that $F(Y)\times_{F(X)} F(X_j) = F(Y)\cap F(X_j)\ne\emptyset$. As a result, $Y\times_X X_j$ is not the initial object of $\scrC$. Since $F(Y\times_X X_j)\to F(X_j)$ and $F(Y\times_X X_j)\to F(Y)$  are injective, due to \thref{lem:monomorphism-iff-image-is-injective}, the maps $Y\times_X X_j\to X_j$ and $Y\times_X X_j\to Y$ must be monomorphisms, and hence, must be isomorphisms. It follows that $X_j$ and $Y$ are the same subobject of $X$.
\end{proof}

\begin{lemma}\thlabel{lem:equalizers-exist}
    $\scrC$ admits all equalizers.
\end{lemma}
\begin{proof}
    Let $f, g: X\to Y$ be morphisms in $\scrC$. There are two fibred product diagrams 
    \begin{equation*}
        \xymatrix {
            X\times_Y X\ar[r]^-{p_1}\ar[d]_{p_2} & X\ar[d]^f\\
            X\ar[r]_g & Y
        }
        \qquad 
        \xymatrix {
            \left(X\times_Y X\right)\times_{X\times X} X\ar[r]\ar[d] & X\ar[d]^{\id_X\times\id_X}\\
            X\times_Y X\ar[r]_{p_1\times p_2} & X\times X
        }
        \qquad 
        \xymatrix {
            X\times X\ar[r]^{\pi_1}\ar[d]_{\pi_2} & X\\
            X & \ar[l]^{\id_X}X\ar[u]_{\id_X}\ar@{.>}[lu]
        }
    \end{equation*}
    We claim that $W = \left(X\times_Y X\right)\times_{X\times X} X\to X$ is the equalizer of $f$ and $g$. Clearly, we have the following equality of compositions: 
    \begin{align*}
        W\to X\xrightarrow{f} Y &= W\to X\xrightarrow{\id_X} X\xrightarrow{f} Y\\
        &= W\to X\to X\times X\xrightarrow{\pi_1} X\xrightarrow{f} Y\\
        &= W\to X\times_Y X\to X\times X\xrightarrow{\pi_1} X\xrightarrow{f} Y\\
        &= W\to X\times_Y X\xrightarrow{p_1} X\xrightarrow{f} Y\\
        &= W\to X\times_Y X\xrightarrow{p_2} X\xrightarrow{g} Y\\
        &= W\to X\times_Y X\to X\times X\xrightarrow{\pi_2} X\xrightarrow{g} Y\\
        &= W\to X\to X\times X\xrightarrow{\pi_2} X\xrightarrow{g} Y\\
        &= W\to X\xrightarrow{\id_X} X\xrightarrow{g} Y\\
        &= W\to X\xrightarrow{g} Y.
    \end{align*}
    If $h: Z\to X$ is such that $f\circ h = g\circ h$, then there is a unique map $\theta: Z\to X\times_Y X$ induced by $Z\xrightarrow{h} X$, which then induces a unique map $\phi: Z\to W$, as desired.
\end{proof}

\begin{proposition}\thlabel{prop:injectivity-of-some-random-map}
    Let $A$ be a connected object in $\scrC$ and $a\in F(A)$. Then for every $X\in\scrC$, the map 
    \begin{equation*}
        \scrC(A, X)\longrightarrow F(X)\qquad f\longmapsto F(f)(a)
    \end{equation*}
    is injective.
\end{proposition}
\begin{proof}
    Let $f,g\in\scrC(A, X)$ be such that $F(f)(a) = F(g)(a)$, and let $(C,\theta)$ be the equalizer of $f,g$, which is known to exist due to \thref{lem:equalizers-exist}. Since $F$ commutes with fibred products, it must commute with equalizers too, hence $(F(C), F(\theta))$ is an equalizer of $F(f), F(g): F(A)\to F(X)$. In particular, $F(\theta)$ is injective, so that $\theta$ is a monomorphism due to \thref{lem:monomorphism-iff-image-is-injective}. Moreover, 
    \begin{equation*}
        a\in F(C) = \left\{b\in F(A)\colon F(f)(b) = F(g)(b)\right\}\ne\emptyset,
    \end{equation*}
    and hence $C$ is not the initial object of $\scrC$, whence $\theta: C\to A$ is an isomorphism, which implies $f = g$.
\end{proof}

\begin{interlude}
    Consider the set $I = \left\{(A,a)\colon A\text{ connected}, ~a\in F(A)\right\}/\sim$ where $\sim$ is the equivalence relation: 
    \begin{equation*}
        (A, a)\sim (B, b)\iff\exists~f: A\to B\text{ an isomorphism such that } F(f)(a) = b.
    \end{equation*}
    We can define a partial order on $I$ by 
    \begin{equation*}
        (A, a)\geqq (B, b)\iff\exists~f: A\to B\text{ a morphism such that } F(f)(a) = b.
    \end{equation*}
    Note that due to \thref{prop:injectivity-of-some-random-map} the above map $f$, if it exists, is unique. We claim that $(I,\geqq)$ is a directed set under this order relation: 
    \begin{description}
        \item[Reflexivity:] Taking $f = \id_A$, we have $F(\id_A)(a) = a$, so $(A, a)\geqq(A, a)$. 
        \item[Anti-symmetry:]  If $(A, a)\geqq(B, b)$ and $(B, b)\geqq(A, a)$, then there are morphisms $A\xrightarrow{f} B$ and $B\xrightarrow{g} A$ such that $F(f)(a) = b$ and $F(g)(b) = a$. Consequently, $F(g\circ f)(a) = a$ and $F(f\circ g)(b) = b$. Using \thref{prop:injectivity-of-some-random-map}, it follows that $g\circ f = \id_A$ and $f\circ g = \id_B$, that is, $(A, a) = (B, b)$.
        \item[Transitivity:] If $(A, a)\geqq(B, b)\geqq(C, c)$ and $A\xrightarrow{f} B$ and $B\xrightarrow{g} C$ are the corresponding maps, then $g\circ f: A\to C$ is such that 
        \begin{equation*}
            F(g\circ f)(a) = F(g)\circ F(f)(a) = F(g)(b) = c.
        \end{equation*}
        \item[Directedness:] Let $(A, a), (B, b)\in I$. Choose a connected subobject $C\to A\times B$ such that the image of $F(C)$ in $F(A\times B) = F(A)\times F(B)$ contains $a\times b$; further, let $c\in C$ be the unique element in $F(C)$ mapping to $a\times b$. Compose the monomorphism $C\to A\times B$ with the canonical projections $A\times B\xrightarrow{p_1}A$ and $A\times B\xrightarrow{p_2}B$ to obtain maps $f_1$ and $f_2$. Then it is clear that $F(f_1)(c) = a$ and $F(f_2)(c) = b$, so that $(C, c)\geqq (A, a), (B, b)$.
    \end{description}
    We shall write $(A, a)\geqq_f (B, b)$ if we want to specify the morphism $A\xrightarrow{f} B$ satisfying $F(f)(a) = b$.
\end{interlude}

If $(A, a)\geqq_f (B, b)$, then the morphism $f: A\to B$ induces a natural transformation of functors $\scrC(B, -)\xrightarrow{-\circ f}\scrC(A, -)$. This gives us a projective system of functors in the functor category $[\scrC,\catFinSets]$. 

\begin{theorem}
    There is an isomorphism of functors 
    \begin{equation*}
        \varinjlim_{(A, a)\in I}\scrC(A, -)\longrightarrow F(-)\qquad f\longmapsto F(f)(a)
    \end{equation*}
\end{theorem}
\begin{proof} % TODO: May be add some more detail regarding the direct limit of set-valued functors
    Consider the maps $\phi_{(A, a)}: \scrC(A, X)\to F(X)$ given by $f\mapsto F(f)(a)$. If $(A, a)\geqq_\psi (B, b)$, then it is clear that the diagram 
    \begin{equation*}
        \xymatrix {
            \scrC(A, X)\ar[rd]_-{\phi_{(A, a)}} & & \scrC(B, X)\ar[ld]^-{\phi_{(B, b)}}\ar[ll]_{-\circ\psi}\\
            & X &
        }
    \end{equation*}
    commutes. This clearly induces a map $\phi: \varinjlim_{(A, a)\in I}\scrC(A, X)\to F(X)$ given by 
    \begin{equation*}
        \phi(f) = \phi_{(A, a)}(f) \quad\text{ if } f\in\scrC(A, X).
    \end{equation*}
    It suffices to show that this map is a bijection of sets, since then it would follow that $\phi$ is an isomorphism of functors.

    First, we show injectivity. Suppose $F(f)(a) = F(g)(b)$ for some $(A, a), (B, b)\in I$ and $f\in\scrC(A, X)$ and $g\in\scrC(B, X)$. Let $C\to A\times B$ be a connected subobject such that $(a, b)\in f(C)$, and let $p_1', p_2'$ be the compositions of the projection maps $p_1: A\times B\to A$ and $p_2: A\times B\to B$ with the monomorphism $C\to A\times B$. It is then clear that $\left(C, c\right)\geqq(A, a)$ and $\left(C, c\right)\geqq(B, b)$. 

    Under the map $\scrC(A, X)\to\scrC(C, X)$, the morphism $f$ maps to $f\circ p_1'$ and under the map $\scrC(B, X)\to\scrC(C, X)$, the morphism $g$ maps to $g\circ p_2'$. We contend that these two maps are the same. Indeed, since $F(fp_1')(c) = F(gp_2')(c)$, due to \thref{prop:injectivity-of-some-random-map}, $fp_1' = gp_2'$. This shows that $f$ and $g$ are equal in $\varinjlim_{(A, a)\in I}\scrC(A, X)$.

    Finally, to see surjectivity, take $x\in F(X)$ and consider $f: A\to X$, the connected component of $X$ such that $x\in F(A)$. Then $(A, x)\in I$ and $F(f)(x) = x$. This completes the proof.
\end{proof}

\subsection{Galois Objects}

If $A$ is a connected object, then we have the inequalities: 
\begin{equation*}
    \#\Aut_{\scrC}(A)\le\#\scrC(A, A)\le\# F(A),
\end{equation*}
where the second inequality follows from \thref{prop:injectivity-of-some-random-map}. In particular, the set of automorphisms of $A$ is finite, and therefore, it makes sense to talk about the quotient of a connected object by its group of automorphisms.

\begin{definition}
    An object $A\in\scrC$ is called a \define{Galois object} if $A/\Aut_{\scrC}(A)$ is a terminal object.
\end{definition}

% In particular, if $A\in\scrC$ is a connected Galois object, then $F(A)/G$ is a terminal object, where $G = F\left(\Aut_{\scrC}(A)\right)$. Thus, $\Aut_{\scrC}(A)$ acts transitively on $F(A)$. Therefore, $\#\Aut_{\scrC}(A) = \#\scrC(A, A) = \# F(A)$.

\begin{proposition}
    Let $X\in\scrC$. There exists $(A, a)\in I$ with $A$ Galois such that the map $\scrC(A, X)\to F(X)$ given by $f\mapsto F(f)(a)$ is bijective.
\end{proposition}
\begin{proof}
    
\end{proof}
\end{document}