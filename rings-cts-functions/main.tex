\documentclass[12pt]{article}

% \usepackage{./arxiv}

\title{Rings of Continuous Functions}
\author{Swayam Chube}
\date{\today}

\usepackage[utf8]{inputenc} % allow utf-8 input
\usepackage[T1]{fontenc}    % use 8-bit T1 fonts
\usepackage{hyperref}       % hyperlinks
\usepackage{url}            % simple URL typesetting
\usepackage{booktabs}       % professional-quality tables
\usepackage{amsfonts}       % blackboard math symbols
\usepackage{nicefrac}       % compact symbols for 1/2, etc.
\usepackage{microtype}      % microtypography
\usepackage{graphicx}
\usepackage{natbib}
\usepackage{doi}
\usepackage{amssymb}
\usepackage{bbm}
\usepackage{amsthm}
\usepackage{amsmath}
\usepackage{xcolor}
\usepackage{theoremref}
\usepackage{enumitem}
\usepackage{mathpazo}
% \usepackage{euler}
\usepackage{mathrsfs}
\setlength{\marginparwidth}{2cm}
\usepackage{todonotes}
\usepackage{stmaryrd}
\usepackage[all,cmtip]{xy} % For diagrams, praise the Freyd–Mitchell theorem 
\usepackage{marvosym}
\usepackage{geometry}
\usepackage{titlesec}

\renewcommand{\qedsymbol}{$\blacksquare$}

% Uncomment to override  the `A preprint' in the header
% \renewcommand{\headeright}{}
% \renewcommand{\undertitle}{}
% \renewcommand{\shorttitle}{}

\hypersetup{
    pdfauthor={Lots of People},
    colorlinks=true,
}

\newtheoremstyle{thmstyle}%               % Name
  {}%                                     % Space above
  {}%                                     % Space below
  {}%                             % Body font
  {}%                                     % Indent amount
  {\bfseries\scshape}%                            % Theorem head font
  {.}%                                    % Punctuation after theorem head
  { }%                                    % Space after theorem head, ' ', or \newline
  {\thmname{#1}\thmnumber{ #2}\thmnote{ (#3)}}%                                     % Theorem head spec (can be left empty, meaning `normal')

\newtheoremstyle{defstyle}%               % Name
  {}%                                     % Space above
  {}%                                     % Space below
  {}%                                     % Body font
  {}%                                     % Indent amount
  {\bfseries\scshape}%                            % Theorem head font
  {.}%                                    % Punctuation after theorem head
  { }%                                    % Space after theorem head, ' ', or \newline
  {\thmname{#1}\thmnumber{ #2}\thmnote{ (#3)}}%                                     % Theorem head spec (can be left empty, meaning `normal')

\theoremstyle{thmstyle}
\newtheorem{theorem}{Theorem}[section]
\newtheorem{lemma}[theorem]{Lemma}
\newtheorem{proposition}[theorem]{Proposition}

\theoremstyle{defstyle}
\newtheorem{definition}[theorem]{Definition}
\newtheorem*{corollary}{Corollary}
\newtheorem{remark}[theorem]{Remark}
\newtheorem{example}[theorem]{Example}
\newtheorem*{notation}{Notation}

% Common Algebraic Structures
\newcommand{\R}{\mathbb{R}}
\newcommand{\Q}{\mathbb{Q}}
\newcommand{\Z}{\mathbb{Z}}
\newcommand{\N}{\mathbb{N}}
\newcommand{\bbC}{\mathbb{C}} 
\newcommand{\K}{\mathbb{K}} % Base field which is either \R or \bbC
\newcommand{\calA}{\mathcal{A}} % Banach Algebras
\newcommand{\calB}{\mathcal{B}} % Banach Algebras
\newcommand{\calI}{\mathcal{I}} % ideal in a Banach algebra
\newcommand{\calJ}{\mathcal{J}} % ideal in a Banach algebra
\newcommand{\frakM}{\mathfrak{M}} % sigma-algebra
\newcommand{\calO}{\mathcal{O}} % Ring of integers
\newcommand{\bbA}{\mathbb{A}} % Adele (or ring thereof)
\newcommand{\bbI}{\mathbb{I}} % Idele (or group thereof)

% Categories
\newcommand{\catTopp}{\mathbf{Top}_*}
\newcommand{\catGrp}{\mathbf{Grp}}
\newcommand{\catTopGrp}{\mathbf{TopGrp}}
\newcommand{\catSet}{\mathbf{Set}}
\newcommand{\catTop}{\mathbf{Top}}
\newcommand{\catRing}{\mathbf{Ring}}
\newcommand{\catCRing}{\mathbf{CRing}} % comm. rings
\newcommand{\catMod}{\mathbf{Mod}}
\newcommand{\catMon}{\mathbf{Mon}}
\newcommand{\catMan}{\mathbf{Man}} % manifolds
\newcommand{\catDiff}{\mathbf{Diff}} % smooth manifolds
\newcommand{\catAlg}{\mathbf{Alg}}
\newcommand{\catRep}{\mathbf{Rep}} % representations 
\newcommand{\catVec}{\mathbf{Vec}}
\newcommand{\ob}{\operatorname{ob}}


% Group and Representation Theory
\newcommand{\chr}{\operatorname{char}}
\newcommand{\Aut}{\operatorname{Aut}}
\newcommand{\GL}{\operatorname{GL}}
\newcommand{\im}{\operatorname{im}}
\newcommand{\tr}{\operatorname{tr}}
\newcommand{\id}{\mathbf{id}}
\newcommand{\cl}{\mathbf{cl}}
\newcommand{\Gal}{\operatorname{Gal}}
\newcommand{\Tr}{\operatorname{Tr}}
\newcommand{\sgn}{\operatorname{sgn}}
\newcommand{\Sym}{\operatorname{Sym}}
\newcommand{\Alt}{\operatorname{Alt}}

% Commutative and Homological Algebra
\newcommand{\spec}{\operatorname{spec}} % Refrain from using lowercase
\newcommand{\mspec}{\operatorname{m-spec}} % Refrain from using lowercase
\newcommand{\Tor}{\operatorname{Tor}}
\newcommand{\tor}{\operatorname{tor}} % subscript
\newcommand{\Ann}{\operatorname{Ann}}
\newcommand{\Supp}{\operatorname{Supp}} % Support
\newcommand{\Ass}{\operatorname{Ass}} % Associated Primes
\newcommand{\Hom}{\operatorname{Hom}}
\newcommand{\End}{\operatorname{End}}
\newcommand{\coker}{\operatorname{coker}}
\newcommand{\limit}{\varprojlim}
\newcommand{\colimit}{%
  \mathop{\mathpalette\colimit@{\rightarrowfill@\textstyle}}\nmlimits@
}
\makeatother


\newcommand{\fraka}{\mathfrak{a}} % ideal
\newcommand{\frakb}{\mathfrak{b}} % ideal
\newcommand{\frakc}{\mathfrak{c}} % ideal
\newcommand{\frakf}{\mathfrak{f}} % face map
\newcommand{\frakg}{\mathfrak{g}}
\newcommand{\frakh}{\mathfrak{h}}
\newcommand{\frakm}{\mathfrak{m}} % maximal ideal
\newcommand{\frakn}{\mathfrak{n}} % naximal ideal
\newcommand{\frakp}{\mathfrak{p}} % prime ideal
\newcommand{\frakq}{\mathfrak{q}} % qrime ideal
\newcommand{\fraks}{\mathfrak{s}}
\newcommand{\frakt}{\mathfrak{t}}
\newcommand{\frakz}{\mathfrak{z}}
\newcommand{\frakA}{\mathfrak{A}}
\newcommand{\frakI}{\mathfrak{I}}
\newcommand{\frakJ}{\mathfrak{J}}
\newcommand{\frakK}{\mathfrak{K}}
\newcommand{\frakL}{\mathfrak{L}}
\newcommand{\frakN}{\mathfrak{N}} % nilradical 
\newcommand{\frakO}{\mathfrak{O}} % dedekind domain
\newcommand{\frakP}{\mathfrak{P}} % Prime ideal above
\newcommand{\frakQ}{\mathfrak{Q}} % Qrime ideal above 
\newcommand{\frakR}{\mathfrak{R}} % jacobson radical
\newcommand{\frakU}{\mathfrak{U}}
\newcommand{\frakX}{\mathfrak{X}}

% General/Differential/Algebraic Topology 
\newcommand{\scrA}{\mathscr A}
\newcommand{\scrB}{\mathscr B}
\newcommand{\scrF}{\mathscr F}
\newcommand{\scrN}{\mathscr N}
\newcommand{\scrP}{\mathscr P}
\newcommand{\scrR}{\mathscr R}
\newcommand{\scrS}{\mathscr S}
\newcommand{\bbH}{\mathbb H}
\newcommand{\Int}{\operatorname{Int}}
\newcommand{\psimeq}{\simeq_p}
\newcommand{\wt}[1]{\widetilde{#1}}
\newcommand{\RP}{\mathbb{R}\text{P}}
\newcommand{\CP}{\mathbb{C}\text{P}}

% Miscellaneous
\newcommand{\wh}[1]{\widehat{#1}}
\newcommand{\calM}{\mathcal{M}}
\newcommand{\calP}{\mathcal{P}}
\newcommand{\onto}{\twoheadrightarrow}
\newcommand{\into}{\hookrightarrow}
\newcommand{\Gr}{\operatorname{Gr}}
\newcommand{\Span}{\operatorname{Span}}
\newcommand{\ev}{\operatorname{ev}}
\newcommand{\weakto}{\stackrel{w}{\longrightarrow}}

\newcommand{\define}[1]{\textcolor{blue}{\textit{#1}}}
\newcommand{\caution}[1]{\textcolor{red}{\textit{#1}}}
\renewcommand{\mod}{~\mathrm{mod}~}
\renewcommand{\le}{\leqslant}
\renewcommand{\leq}{\leqslant}
\renewcommand{\ge}{\geqslant}
\renewcommand{\geq}{\geqslant}
\newcommand{\Res}{\operatorname{Res}}
\newcommand{\floor}[1]{\left\lfloor #1\right\rfloor}
\newcommand{\ceil}[1]{\left\lceil #1\right\rceil}
\newcommand{\gl}{\mathfrak{gl}}
\newcommand{\ad}{\operatorname{ad}}
\newcommand{\Stab}{\operatorname{Stab}}
\newcommand{\bfX}{\mathbf{X}}
\newcommand{\Ind}{\operatorname{Ind}}
\newcommand{\bfG}{\mathbf{G}}
\newcommand{\rank}{\operatorname{rank}}
\newcommand{\calo}{\mathcal{o}}
\newcommand{\frako}{\mathfrak{o}}
\newcommand{\Cl}{\operatorname{Cl}}

\newcommand{\idim}{\operatorname{idim}}
\newcommand{\pdim}{\operatorname{pdim}}
\newcommand{\Ext}{\operatorname{Ext}}
\newcommand{\co}{\operatorname{co}}
\newcommand{\Spec}{\operatorname{Spec}}
\newcommand{\MaxSpec}{\operatorname{MaxSpec}}
\newcommand{\scrG}{\mathscr{G}}
\newcommand{\bfO}{\mathbf{O}}
\newcommand{\bfF}{\mathbf{F}} % Fitting Subgroup
\newcommand{\Syl}{\operatorname{Syl}}
\newcommand{\nor}{\vartriangleleft}
\newcommand{\noreq}{\trianglelefteqslant}
\newcommand{\subnor}{\nor\!\nor}
\newcommand{\Soc}{\operatorname{Soc}}
\newcommand{\pr}{\operatorname{pr}}

\geometry {
    margin = 1in
}

\titleformat
{\section}
[block]
{\Large\bfseries\scshape}
{\S\thesection}
{0.5em}
{\centering}
[]


\titleformat
{\subsection}
[block]
{\normalfont\bfseries\sffamily}
{\S\S}
{0.5em}
{\centering}
[]


\begin{document}
\maketitle

\section{\texorpdfstring{$C(X)$}{C(X)}}

\subsection{Maximal Ideals}

\begin{theorem}
    Let $X$ be a compact Hausdorff space. Every maximal ideal in $C(X)$ is of the form 
    \begin{equation*}
        \frakm_x = \left\{f\in C(X)\colon f(x) = 0\right\}.
    \end{equation*}
\end{theorem}

\begin{proposition}
    Let $X$ be a compact Hausdorff space. Every prime ideal in $C(X)$ is contained in a unique maximal idea.
\end{proposition}

\begin{theorem}[Sury, ??]
    Every maximal ideal in $C[0, 1]$ is uncountably generated.
\end{theorem}

\subsection{Krull Dimension}

Throughout this (sub)section, $X$ denotes a compact Hausdorff space. For every $x\in X$, there is a maximal ideal 
\begin{equation*}
    \frakm_x = \left\{f\in C(X)\colon f(x) = 0\right\}.
\end{equation*}
These are the only maximal ideals in $C(X)$. The goal of this (sub)section is to prove the following 

\begin{theorem}\thlabel{thm:inf-krull-dim}
    If there is a point $p\in X$ and an $f\in C(X)$ such that $f(p) = 0$ and there is no neighborhood of $p$ on which $f$ vanishes, then $C(X)$ has infinite Krull dimension.
\end{theorem}

\begin{definition}
    A \define{partially ordered ring} is a pair $(A,\leqq)$ where $\leqq$ is a partial order on $A$ such that
    \begin{itemize}
        \item $x\leqq y$ implies $x + z \leqq y + z$, and 
        \item $0\leqq x$ and $0\leqq y$ implies $0\leqq xy$,
    \end{itemize}
    for all $x,y,z\in A$.
    A \define{totally ordered ring} is a partially ordered ring $(A,\leqq)$ such that $\leqq$ is a total order.
\end{definition}

The ring $C(X)$ has a canonical partial order, given by $f\leqq g$ if and only if $f(x)\le g(x)$ for all $x\in X$.

\begin{definition}
    An ideal $\fraka$ of a partially ordered ring $A$ is said to be \define{convex} if whenever $a,b\in A$ such that $0\leqq a\leqq b$ and $b\in\fraka$, then $a\in\fraka$.
\end{definition}

\begin{proposition}
    If $(A,\leqq)$ is a partially ordered ring, and $\fraka\noreq A$ is a convex ideal, then $A/\fraka$ has a natural partial order given by: 
    \begin{equation*}
        (a + \fraka)\leqq(b + \fraka) \quad\text{if}\quad a\leqq b.
    \end{equation*}
\end{proposition}
\begin{proof}
    Standard.
\end{proof}

\begin{proposition}
    Let $\frakP$ be a prime ideal in $C(X)$. Then, $\frakP$ is convex.
\end{proposition}
\begin{proof}
    Suppsoe $0\leqq f\leqq g$ in $C(X)$ and $g\in\frakP$. The function $h: X\to\R$ given by 
    \begin{equation*}
        h(x) = 
        \begin{cases}
            \frac{f(x)^2}{g(x)} & g(x)\ne 0\\
            0 & g(x) = 0
        \end{cases}
    \end{equation*}
    is a continuous function such that $f^2 = gh\in\frakP$, whence $f\in\frakP$.
\end{proof}

\begin{proposition}
    The ring $A = C(X)/\frakP$ is a totally ordered local domain. Further, the primes of $A$ are totally ordered by inclusion.
\end{proposition}
\begin{proof}
    That it is a local domain follows from the fact that there is a unique maximal ideal containing $\frakP$. For any $f\in C(X)$, $f^2\equiv |f|^2\pmod\frakP$, and hence, $f\equiv |f|\pmod\frakP$ or $f\equiv -|f|\pmod\frakP$. Consequently, $f + \frakP$ is comparable with $0$ in $A$, whence $A$ is totally ordered.

    Recall that the primes in $A$ are of the form $\frakp = \frakQ/\frakP$ for some prime $\frakQ\supseteq\frakP$. Since $\frakQ$ is convex, so is $\frakp$.

    Finally, let $\frakp$ and $\frakq$ be two primes in $A$ and suppose $a\in\frakq\setminus\frakp$. Then, for every $b\in\frakq$, $b < a$, else $a\in\frakp$. Hence, $b\in\frakp$. This shows that $\frakp\subseteq\frakq$, whence the primes are totally ordered by inclusion.
\end{proof}


Let $p\in X$ be a point such that there is an $f\in C(X)$ such that $f(p) = 0$ but there is no neighborhood of $p$ on which $f$ is identically $0$. Upon multiplying by a suitable real scalar, we may suppose that $0\le f(x) < e^{-2} < 1$ on $X$.

\begin{proposition}
    The maximal ideal $\frakm_p$ properly contains a prime ideal, say $\frakP$.
\end{proposition}
\begin{proof}
    Consider the local ring $C(X)_{\frakm_p}$. If $\frakm_p$ does not properly contain a prime ideal, then $C(X)_{\frakm_p}$ is a local ring of dimension $0$, whence the maximal ideal is the nilradical. But this ring is isomorphic to the ring of germs at $p$ and the germ of $f$ at $p$ is not nilpotent since it does not vanish on any neighborhood of $p$.
\end{proof}

Let 
\begin{equation*}
    I_p = \left\{g\in C(X)\colon g\text{ vanishes on a neighborhood of }p\right\}\subseteq\frakm_p.
\end{equation*}

\begin{proposition}
    $I_p\subseteq\frakP$.
\end{proposition}
\begin{proof}
    The localization map $C(X)\to C(X)_{\frakm_p}$ is a surjective ring homomorphism whose kernel is $I_p$. Note that $\frakP^{ec} = \frakP$, since $\frakP$ is prime. But upon contracting, we see that $\frakP$ must contain the kernel.
\end{proof}

\begin{proof}[Proof of \thref{thm:inf-krull-dim}]
Let $A = C(X)/\frakP$. We shall show that $A$ has infinite Krull dimension. To this end, it suffices to show that we can find a prime ideal $\frakQ$ such that $\frakP\subsetneq\frakQ\subsetneq\frakm_p$, since this process can then be repeated ad infinitum.

Define the function $g: X\to\R$ by 
\begin{equation*}
    g(x) = 
    \begin{cases}
        \frac{1}{|\log f(x)|} & f(x)\ne 0\\
        0 & f(x) = 0.
    \end{cases}
\end{equation*}
This is a continuous function on $X$. Further, from basic calculus, it is evident that for every positive integer $k$, there is a neighborhood of $0$ in $[0,\infty)$ on which $t|\log t|^k < 1$. Hence, for every positive integer $k$, there is a neighborhood $U$ of $p$ on which $g(x)^k\ge f(x)$.

Since $C(X)/I_p$ is ordered, we see that $g^k + I_p\geqq f + I_p$ for all positive integers $k$ in $C(X)/I_p$. Since $A$ is a quotient of $C(X)/I_p$, we have that $g^k + \frakP\geqq f + \frakP$ for all positive integers $k$ in $A$. 

Let $a,b\in A$ denote the images of $f$ and $g$ respectively. Then $a\le b^k$ for all positive integers $k$. Note that by construction, $0\le g(x) < \frac{1}{2}$ on all of $X$. Suppose $A$ has no prime ideals other than the maximal ideal and $(0)$, then the radical of $(a)\noreq A$, which is the intersection of all primes containing $(a)$ must be equal to the maximal ideal.

In particiular, there is a positive integer $n$ such that $b^n\in (a)$, whence there is some $c\in A$ such that $b^n = ac$. Since $0 < a, b$, we have that $0 < c$. Therefore, we can find some $0\leqq h\in C(X)$ such that $c$ is the image of $h$ in $A$. Since the supremum of $g$ on $X$ is smaller than $\frac{1}{2}$, and $h$ is bounded on $X$ (since $X$ is compact), we have that for sufficiently large positive integers $k$, $0 < g^kh < 1$. That is, for sufficiently large $k$, $0\leqq b^kc\leqq 1$. 

Hence, for all sufficiently large $k$, we have 
\begin{equation*}
    a\leqq b^{n + k} = a(b^kc)\leqq a\implies b^{n + k} = a.
\end{equation*}
Consequently, $b^N = b^{N + 1}$ for sufficiently large $N$. Since $A$ is a domain, this is possible if and only if $b\in\{0, 1\}$, neither of which is the case. This completes the proof.
\end{proof}

\begin{proposition}
    Let $X$ be a compact Hausdorff space such that $C(X)$ consists of only the locally constant functions on $X$. Then $X$ is a finite set.
\end{proposition}
\begin{proof}
    We first show that every $G_\delta$-set in $X$ is open. To this end, suppose $U_1, U_2,\dots$ is a collection of open subsets of $X$ containing a point $p\in X$. Urysohn's lemma furnishes continuous functions $f_n: X\to[0, 1]$ such that $f_n(p) = 1$ and $f_n$ vanishes on $X\setminus U_n$. Define $f: X\to[0, 1]$ by 
    \begin{equation*}
        f(x) = \sum_{n = 1}^\infty\frac{1}{2^n}f_n(x)\qquad x\in X.
    \end{equation*}
    This series converges uniformly due to the Weierstrass $M$-test, whence $f$ is continuous, i.e., locally constant. Thus, there is a neighborhood $V$ of $p$ in $X$ on which $f$ is identically $1$. Note that if $f(q) = 1$ then $q\in U_n$ for all $n\ge 1$. Thus, 
    \begin{equation*}
        V\subseteq\bigcap_{n = 1}^\infty U_n\implies\bigcap_{n = 1}^\infty U_n\text{ is open}.
    \end{equation*}

    Next, given disjoint points $a,b\in X$, there is a continuous function $f:X\to [0,1]$ such that $f(a) = 0$ and $f(1) = b$. Note that the zero set of $f$ is open because $f$ is locally constant. Whence, we have a disjoint union of clopen sets $U_{a,b}\sqcup U_{b, a}$ such that $a\in U_{a,b}$ and $b\in U_{b, a}$.

    Suppose now that $X$ is not finite and choose a countably infinite set $A\subseteq X$. Consider the collection $\scrS$ of sets $S$ such that
    \begin{itemize}
        \item $S$ is a collection of pairs $(a, b)$ with $a, b\in A$ and $a\ne b$. 
        \item For all $a\ne b$ in $A$, exactly one of $(a, b)$ and $(b, a)$ is in $S$.
    \end{itemize}
    Note that every $S\in\mathscr S$ is a countable set. Next, define 
    \begin{equation*}
        U_S = \bigcap_{(a, b)\in S} U_{a, b}.
    \end{equation*}
    Since every $G_\delta$ in $X$ is open, every $U_S$ is clopen. Further, 
    \begin{equation*}
        X = \bigsqcup_{S\in\mathscr S} U_S.
    \end{equation*}
    Finally, note that the elements of $a$ lie in disjoint $U_S$'s by construction. As a result, at least countably many of the $U_S$'s are non-empty. Hence, we have expressed $X$ as a disjoint union of at least countably many disjoint open sets, a contradiction to the compactness of $X$. This completes the proof.
\end{proof}

To summarize, we have: 
\begin{theorem}
    Let $X$ be a compact Hausdorff space. Then, 
    \begin{itemize}
        \item $\dim C(X) = 0$ if $X$ is a finite set. 
        \item $\dim C(X) = \infty$ in all other cases.
    \end{itemize}
\end{theorem}
\begin{proof}
    If $X$ is finite, then $C(X) = \R^n$ as a ring, whence $\dim C(X) = 0$. The other cases have been handled above.
\end{proof}
\end{document}