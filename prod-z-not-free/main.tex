\documentclass[12pt]{article}

% \usepackage{./arxiv}

\title{$\displaystyle\prod_{i = 1}^\infty\Z$}
\author{Swayam Chube}
\date{\today}

\usepackage[utf8]{inputenc} % allow utf-8 input
\usepackage[T1]{fontenc}    % use 8-bit T1 fonts
\usepackage{hyperref}       % hyperlinks
\usepackage{url}            % simple URL typesetting
\usepackage{booktabs}       % professional-quality tables
\usepackage{amsfonts}       % blackboard math symbols
\usepackage{nicefrac}       % compact symbols for 1/2, etc.
\usepackage{microtype}      % microtypography
\usepackage{graphicx}
\usepackage{natbib}
\usepackage{doi}
\usepackage{amssymb}
\usepackage{bbm}
\usepackage{amsthm}
\usepackage{amsmath}
\usepackage{xcolor}
\usepackage{theoremref}
\usepackage{enumitem}
\usepackage{mathpazo}
% \usepackage{euler}
\usepackage{mathrsfs}
\setlength{\marginparwidth}{2cm}
\usepackage{todonotes}
\usepackage{stmaryrd}
\usepackage[all,cmtip]{xy} % For diagrams, praise the Freyd–Mitchell theorem 
\usepackage{marvosym}
\usepackage{geometry}
\usepackage{titlesec}

\renewcommand{\qedsymbol}{$\blacksquare$}

% Uncomment to override  the `A preprint' in the header
% \renewcommand{\headeright}{}
% \renewcommand{\undertitle}{}
% \renewcommand{\shorttitle}{}

\hypersetup{
    pdfauthor={Lots of People},
    colorlinks=true,
}

\newtheoremstyle{thmstyle}%               % Name
  {}%                                     % Space above
  {}%                                     % Space below
  {}%                             % Body font
  {}%                                     % Indent amount
  {\bfseries\scshape}%                            % Theorem head font
  {.}%                                    % Punctuation after theorem head
  { }%                                    % Space after theorem head, ' ', or \newline
  {\thmname{#1}\thmnumber{ #2}\thmnote{ (#3)}}%                                     % Theorem head spec (can be left empty, meaning `normal')

\newtheoremstyle{defstyle}%               % Name
  {}%                                     % Space above
  {}%                                     % Space below
  {}%                                     % Body font
  {}%                                     % Indent amount
  {\bfseries\scshape}%                            % Theorem head font
  {.}%                                    % Punctuation after theorem head
  { }%                                    % Space after theorem head, ' ', or \newline
  {\thmname{#1}\thmnumber{ #2}\thmnote{ (#3)}}%                                     % Theorem head spec (can be left empty, meaning `normal')

\theoremstyle{thmstyle}
\newtheorem{theorem}{Theorem}
\newtheorem{lemma}[theorem]{Lemma}
\newtheorem{proposition}[theorem]{Proposition}

\theoremstyle{defstyle}
\newtheorem{definition}[theorem]{Definition}
\newtheorem*{corollary}{Corollary}
\newtheorem{remark}[theorem]{Remark}
\newtheorem{example}[theorem]{Example}
\newtheorem*{notation}{Notation}

% Common Algebraic Structures
\newcommand{\R}{\mathbb{R}}
\newcommand{\Q}{\mathbb{Q}}
\newcommand{\Z}{\mathbb{Z}}
\newcommand{\N}{\mathbb{N}}
\newcommand{\bbC}{\mathbb{C}} 
\newcommand{\K}{\mathbb{K}} % Base field which is either \R or \bbC
\newcommand{\calA}{\mathcal{A}} % Banach Algebras
\newcommand{\calB}{\mathcal{B}} % Banach Algebras
\newcommand{\calI}{\mathcal{I}} % ideal in a Banach algebra
\newcommand{\calJ}{\mathcal{J}} % ideal in a Banach algebra
\newcommand{\frakM}{\mathfrak{M}} % sigma-algebra
\newcommand{\calO}{\mathcal{O}} % Ring of integers
\newcommand{\bbA}{\mathbb{A}} % Adele (or ring thereof)
\newcommand{\bbI}{\mathbb{I}} % Idele (or group thereof)

% Categories
\newcommand{\catTopp}{\mathbf{Top}_*}
\newcommand{\catGrp}{\mathbf{Grp}}
\newcommand{\catTopGrp}{\mathbf{TopGrp}}
\newcommand{\catSet}{\mathbf{Set}}
\newcommand{\catTop}{\mathbf{Top}}
\newcommand{\catRing}{\mathbf{Ring}}
\newcommand{\catCRing}{\mathbf{CRing}} % comm. rings
\newcommand{\catMod}{\mathbf{Mod}}
\newcommand{\catMon}{\mathbf{Mon}}
\newcommand{\catMan}{\mathbf{Man}} % manifolds
\newcommand{\catDiff}{\mathbf{Diff}} % smooth manifolds
\newcommand{\catAlg}{\mathbf{Alg}}
\newcommand{\catRep}{\mathbf{Rep}} % representations 
\newcommand{\catVec}{\mathbf{Vec}}

% Group and Representation Theory
\newcommand{\chr}{\operatorname{char}}
\newcommand{\Aut}{\operatorname{Aut}}
\newcommand{\GL}{\operatorname{GL}}
\newcommand{\im}{\operatorname{im}}
\newcommand{\tr}{\operatorname{tr}}
\newcommand{\id}{\mathbf{id}}
\newcommand{\cl}{\mathbf{cl}}
\newcommand{\Gal}{\operatorname{Gal}}
\newcommand{\Tr}{\operatorname{Tr}}
\newcommand{\sgn}{\operatorname{sgn}}
\newcommand{\Sym}{\operatorname{Sym}}
\newcommand{\Alt}{\operatorname{Alt}}

% Commutative and Homological Algebra
\newcommand{\spec}{\operatorname{spec}}
\newcommand{\mspec}{\operatorname{m-spec}}
\newcommand{\Tor}{\operatorname{Tor}}
\newcommand{\tor}{\operatorname{tor}}
\newcommand{\Ann}{\operatorname{Ann}}
\newcommand{\Supp}{\operatorname{Supp}}
\newcommand{\Hom}{\operatorname{Hom}}
\newcommand{\End}{\operatorname{End}}
\newcommand{\coker}{\operatorname{coker}}
\newcommand{\limit}{\varprojlim}
\newcommand{\colimit}{%
  \mathop{\mathpalette\colimit@{\rightarrowfill@\textstyle}}\nmlimits@
}
\makeatother


\newcommand{\fraka}{\mathfrak{a}} % ideal
\newcommand{\frakb}{\mathfrak{b}} % ideal
\newcommand{\frakc}{\mathfrak{c}} % ideal
\newcommand{\frakf}{\mathfrak{f}} % face map
\newcommand{\frakg}{\mathfrak{g}}
\newcommand{\frakh}{\mathfrak{h}}
\newcommand{\frakm}{\mathfrak{m}} % maximal ideal
\newcommand{\frakn}{\mathfrak{n}} % naximal ideal
\newcommand{\frakp}{\mathfrak{p}} % prime ideal
\newcommand{\frakq}{\mathfrak{q}} % qrime ideal
\newcommand{\fraks}{\mathfrak{s}}
\newcommand{\frakt}{\mathfrak{t}}
\newcommand{\frakz}{\mathfrak{z}}
\newcommand{\frakA}{\mathfrak{A}}
\newcommand{\frakI}{\mathfrak{I}}
\newcommand{\frakJ}{\mathfrak{J}}
\newcommand{\frakK}{\mathfrak{K}}
\newcommand{\frakL}{\mathfrak{L}}
\newcommand{\frakN}{\mathfrak{N}} % nilradical 
\newcommand{\frakO}{\mathfrak{O}} % dedekind domain
\newcommand{\frakP}{\mathfrak{P}} % Prime ideal above
\newcommand{\frakQ}{\mathfrak{Q}} % Qrime ideal above 
\newcommand{\frakR}{\mathfrak{R}} % jacobson radical
\newcommand{\frakU}{\mathfrak{U}}
\newcommand{\frakX}{\mathfrak{X}}

% General/Differential/Algebraic Topology 
\newcommand{\scrA}{\mathscr A}
\newcommand{\scrB}{\mathscr B}
\newcommand{\scrF}{\mathscr F}
\newcommand{\scrN}{\mathscr N}
\newcommand{\scrP}{\mathscr P}
\newcommand{\scrR}{\mathscr R}
\newcommand{\scrS}{\mathscr S}
\newcommand{\bbH}{\mathbb H}
\newcommand{\Int}{\operatorname{Int}}
\newcommand{\psimeq}{\simeq_p}
\newcommand{\wt}[1]{\widetilde{#1}}
\newcommand{\RP}{\mathbb{R}\text{P}}
\newcommand{\CP}{\mathbb{C}\text{P}}

% Miscellaneous
\newcommand{\wh}[1]{\widehat{#1}}
\newcommand{\calM}{\mathcal{M}}
\newcommand{\calP}{\mathcal{P}}
\newcommand{\onto}{\twoheadrightarrow}
\newcommand{\into}{\hookrightarrow}
\newcommand{\Gr}{\operatorname{Gr}}
\newcommand{\Span}{\operatorname{Span}}
\newcommand{\ev}{\operatorname{ev}}
\newcommand{\weakto}{\stackrel{w}{\longrightarrow}}

\newcommand{\define}[1]{\textcolor{blue}{\textit{#1}}}
\newcommand{\caution}[1]{\textcolor{red}{\textit{#1}}}
\renewcommand{\mod}{~\mathrm{mod}~}
\renewcommand{\le}{\leqslant}
\renewcommand{\leq}{\leqslant}
\renewcommand{\ge}{\geqslant}
\renewcommand{\geq}{\geqslant}
\newcommand{\Res}{\operatorname{Res}}
\newcommand{\floor}[1]{\left\lfloor #1\right\rfloor}
\newcommand{\ceil}[1]{\left\lceil #1\right\rceil}
\newcommand{\gl}{\mathfrak{gl}}
\newcommand{\ad}{\operatorname{ad}}
\newcommand{\Stab}{\operatorname{Stab}}
\newcommand{\bfX}{\mathbf{X}}
\newcommand{\Ind}{\operatorname{Ind}}
\newcommand{\bfG}{\mathbf{G}}
\newcommand{\rank}{\operatorname{rank}}
\newcommand{\calo}{\mathcal{o}}
\newcommand{\frako}{\mathfrak{o}}
\newcommand{\Cl}{\operatorname{Cl}}

\newcommand{\idim}{\operatorname{idim}}
\newcommand{\pdim}{\operatorname{pdim}}
\newcommand{\Ext}{\operatorname{Ext}}
\newcommand{\co}{\operatorname{co}}

\geometry {
    margin = 1in
}

\titleformat
{\section}
[block]
{\Large\bfseries\scshape}
{\S\thesection}
{0.5em}
{\centering}
[]


\titleformat
{\subsection}
[block]
{\normalfont\bfseries\sffamily}
{\S\S}
{0.5em}
{\centering}
[]


\begin{document}
\maketitle

In $G = \displaystyle\prod_{i = 1}^\infty\Z$, there are the ``standard basis vectors'' $\{e_i\colon i\ge 1\}$ given by 
\begin{equation*}
    e_i(j) = 
    \begin{cases}
        1 & i = j\\
        0 & \text{otherwise}.
    \end{cases}
\end{equation*}
These are linearly independent over $\Z$ and hence generate a free abelian subgroup of $G$, which we denote by $H$.

\begin{lemma}\thlabel{lem:vanish-on-H}
    Let $f: G\to\Z$ be a homomorphism such that $f|_H = 0$. Then $f = 0$.
\end{lemma}
\begin{proof}
    As a consequence of the hypothesis, it is easy to see that $f(a_1,a_2,\dots) = f(b_1,b_2,\dots)$ if $a_i = b_i$ for all but finitely many $i\ge 1$. 

    Let $p\ge 2$ be any prime and $(a_n)_{n\ge 1}$ be any sequence of integers. Our observation above yields that for all $n\ge 1$, 
    \begin{equation*}
        f(a_1p, a_2p^2,\dots) = f(\underbrace{0,\dots,0}_{n - 1\text{ times}}, a_np^n, a_{n + 1}p^{n + 1},\dots).
    \end{equation*}
    The right hand side is divisible by $p^n$ and hence, $p^n\mid f(a_1p,a_2p^2,\dots)$ for all $n\ge 1$. This is possible if and only if 
    \begin{equation*}
        f(a_1p, a_2p^2,\dots) = 0.
    \end{equation*}

    Finally, let $(a_n)_{n\ge 1}\in G$. Since $2^n$ and $3^n$ are coprime for all $n\ge 1$, there are $b_n, c_n\in\Z$ such that $a_n = b_n2^n + c_n3^n$. Thus, 
    \begin{equation*}
        f(a_1,a_2,\dots) = f(2b_1, 4b_2, \dots) + f(3c_1, 9c_2,\dots) = 0.
    \end{equation*}
    This completes the proof.
\end{proof}

\begin{theorem}
    $G$ is not a free abelian group.
\end{theorem}
\begin{proof}
    Suppose $G$ were free, then there is a set $S$ and an isomorphism $\displaystyle\varphi: G\to F = \bigoplus_{S}\Z$. Since $G$ is uncountable, so is $S$. Let $\pi_s: F\to\Z$ denote the projection onto the $s$-th coordinate.

    For each $i\ge 1$, $\varphi(e_i)$ has only finitely many nonzero coordinates, say $S_i\subseteq S$. Then, $\bigcup_i S_i$ is a countable subset of $S$, whence $T = S\setminus\bigcup_{i} S_i$ is still uncountable, in particular, nonempty.

    For each $t\in T$, $\pi_t\circ\varphi(e_i) = 0$ for all $i\ge 1$ and hence, $\pi_t\circ\varphi|_H = 0$. Due to the preceding lemma, $\pi_t\circ\varphi = 0$ for all $t\in T$, but this is absurd, since $\varphi$ is surjective.
\end{proof}

\begin{lemma}\thlabel{lem:vanish-on-ei}
    If $f: G\to\Z$ is a homomorphism, then $f(e_i) = 0$ for all but finitely many $i\ge 1$.
\end{lemma}
\begin{proof}
    Suppose not, then there is a sequence $1\le i_1 < i_2 < \cdots$ such that $f(e_{i_j})\ne 0$. By composing $f$ with a suitable endomorphism of $G$, we may suppose that $i_j = j$ and $f(e_{i_j}) > 0$ for all $j\ge 1$.

    Let $a_i = f(e_i)$ and $p$ be a prime not dividing $a_1$. Define two sequences $(x_n)_{n\ge 1}$ and $(y_n)_{n\ge 1}$ by setting $x_1 = 1$,
    \begin{equation*}
        x_{n + 1} = pf(x_1,\dots,x_n,0,0,\dots)\quad\text{for }n\ge 1,
    \end{equation*}
    and 
    $y_1 = 1$, 
    \begin{equation*}
        y_{n} = f(x_1,\dots,x_{n - 1},0,0,\dots)\quad\text{for }n\ge 2.
    \end{equation*}
    Hence, $x_n = py_n$ for $n\ge 2$.
    Note that for $n\ge 2$
    \begin{equation*}
        y_{n + 1} = y_n + a_nx_n = y_n + pa_ny_n = y_n(1 + pa_n)\ge (p + 1)y_n.
    \end{equation*}
    In particular, $y_n\to\infty$ as $n\to\infty$. Also, multiplying both sides of the above equation by $p$, we have 
    \begin{equation*}
        x_{n + 1} = x_n(1 + pa_n)\quad\text{for }n\ge 2.
    \end{equation*}
    Therefore, for $n\ge 2$, $y_n\mid x_m$ whenever $m\ge n$. We can write 
    \begin{equation*}
        f(x_1,x_2,\dots) = y_n + f(0,\dots,0,x_n,x_{n + 1},\dots).
    \end{equation*}
    Since $y_n$ divides the right hand side, $y_n\mid f(x_1,x_2,\dots)$ for all $n\ge 1$. But $y_n$ grows without bound, and hence, $f(x_1,x_2,\dots) = 0$. Finally, note that 
    \begin{equation*}
        0 = f(x_1,x_2,\dots) = a_1 + f(0,x_2,x_3,\dots)\equiv a_1\pmod p,
    \end{equation*}
    since $p\mid x_i$ for all $i\ge 2$. This is absurd, since $p\nmid a_1$.
\end{proof}

\begin{theorem}
    $\Hom_{\Z}(G,\Z)$ is a free abelian group with basis $\pi_i: G\to\Z$, the canonical projections.
\end{theorem}
\begin{proof}
    Follows from \thref{lem:vanish-on-H} and \thref{lem:vanish-on-ei}
\end{proof}
\end{document}