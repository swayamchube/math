\begin{definition}
    Let $A$ be a ring, $B$ an $A$-algebra, and $I$ an ideal of $B$. Endow $B$ with the $I$-adic topology. We say that $B$ is \define{$I$-smooth} over $A$ if given an $A$-algebra $C$, an ideal $N$ of $C$ satisfying $N^2 = 0$, and an $A$-algebra homomorphism $u\colon B\to C/N$ which is continuous when $C/N$ is given the discrete topology, then there exists an $A$-algebra homomorphism $v\colon B\to C$ making 
    \begin{equation*}
        \xymatrix {
            A\ar[d]\ar[r] & C\ar[d]\\
            B\ar[r]_-u\ar@{.>}[ru]^-v & C/N
        }
    \end{equation*}
    commute. Similarly, we say that $B$ is \define{$I$-unramified} over $A$ if there is at most one such lift. Finally, $B$ is said to be \define{$I$-\'etale} over $A$ if it is both $I$-smooth and $I$-unramified.
\end{definition}

\begin{remark}
    Recall that the $I$-adic topology on $B$ is given by the neighborhood base $\{I^n\colon n\ge 0\}$ at $0\in B$. The continuity of $u$ is therefore equivalent to the existence of an integer $\nu > 0$ such that $I^\nu\subseteq\ker u$.
\end{remark}

\begin{theorem}[Transitivity]\thlabel{transitivity}
    Let $A\xrightarrow{g} B\xrightarrow{g'} B'$ be ring homomorphisms, and suppose that $g'$ is continuous for the $I$-adic topology on $B$ and the $I'$-adic topology on $B'$. If $B$ is $I$-smooth over $A$, and $B'$ is $I'$-smooth over $B$, then $B'$ is $I'$-smooth over $A$. 

    The same statement holds with ``unramified'' replacing ``smooth'' everywhere.
\end{theorem}
\begin{proof}
    Consider the diagram 
    \begin{equation*}
        \xymatrix {
            A\ar[r]\ar[d]_g & C\ar[dd]\\
            B\ar[d]_{g'} & \\
            B'\ar[r]_-u & C/N
        }
    \end{equation*}
    where we begin with a map $u\colon B'\to C/N$, which is continuous with respect to the $I'$-adic topology on $B'$. Thus the composition $u\circ g'\colon B\to C/N$ is continuous with respect to the $I$-adic topology on $B$. It follows that there exists a lift $v\colon B\to C$. Finally, using that $B'$ is $I'$-smooth over $B$, there is a lift $w\colon B'\to C$, so that $B'$ is $I'$-smooth over $A$.

    Now, suppose we are working with unramified extensions and there are two lifts $w, w'\colon B'\to C$. Then setting $(v, v') = (w\circ g', w'\circ g')$, it follows that $v$ and $v'$ are lifts of $u\circ g'$, but since $B$ is $I$-unramified over $A$, $w\circ g' = w'\circ g'$. Finally, since $B'$ is $I'$-smooth over $B$, we have that $w = w'$, thereby completing the proof.
\end{proof}

\begin{theorem}[Base Change]
    Let $A$ be a ring, $B$ and $A'$ two $A$-algebras, and set $B' = B\otimes_A A'$. If $B$ is $I$-smooth over $A$, then $B'$ is $IB'$-smooth over $A'$.

    The same statement holds with ``unramified'' replacing ``smooth'' everywhere.
\end{theorem}
\begin{proof}
    Suppose there is a commutative diagram of ring homomorphisms: 
    \begin{equation*}
        \xymatrix{
            A\ar[r]\ar[d] & A'\ar[d]\ar[r] & C\ar[d]^\pi\\
            B\ar[r] & B'\ar[r]_-{u} & C/N
        }
    \end{equation*}
    where $u$ is an $A'$-algebra homomorphism continuous with respect to that $IB'$-adic topology on $B'$. The composition $B\to C/N$ is therefore continuous with respect to the $I$-adic topology on $B$. This gives an $A$-algebra lifting $v\colon B\to C$. The universal property of a pushout therefore furnishes a map $w\colon B'\to C$. We wish to show that $\pi\circ w = u$. Indeed, note that the composition 
    \begin{equation*}
        A'\to B'\xrightarrow{w} C\xrightarrow{\pi} C/N = A'\to C\xrightarrow{\pi} C/N = A'\to B'\xrightarrow{u} C/N
    \end{equation*}
    and 
    \begin{equation*}
        B\to B'\xrightarrow{w} C\xrightarrow{\pi} C/N = B\xrightarrow{v} C\xrightarrow{\pi} C/N = B\to B'\xrightarrow{u} C/N.
    \end{equation*}
    It follows from the universal property of a pushout that $\pi\circ w = u$, as desired. 

    Next, suppose the extensions are unramified instead of smooth and that there are two lifts $w, w'\colon B'\to C$ of $u$. In this case, we see that the composition 
    \begin{equation*}
        B\to B'\xrightarrow{w} C = B\to B'\xrightarrow{w'} C.
    \end{equation*}
    But we also have 
    \begin{equation*}
        A'\to B'\xrightarrow{w} C = A'\to C = A'\to B'\xrightarrow{w'} C,
    \end{equation*}
    and hence, from the universal property of a pushout, we see that $w = w'$, thereby completing the proof.
\end{proof}

\begin{mdframed}
Let $(A,\frakm, K)$ be a local ring. Recall that $\chr A$ is either $0$ or a prime power. Indeed, if $\chr A = n > 0$ and $n = ab$ with $a, b > 1$ coprime integers, then setting $\fraka = \Ann_A(a)$ and $\frakb = \Ann_A(b)$, we see that $\fraka\cap\frakb = (0)$ but $\fraka + \frakb = A$, a contradiction, since $A$ is connected. 

If $\chr A = p > 0$, a rational prime, then we must have that $\chr K = p$. On the other hand, if $\chr K = 0$, then $\chr A = 0$, and there is an inclusion $\Z\subseteq A$ such that $\Z\cap\frakm = (0)$, whence every element of $\Z$ is a unit in $A$, i.e., $\Q\into A$.
\end{mdframed}

\begin{definition}
    Let $(A,\frakm,K)$ be a local ring. We say that $A$ is \define{equicharacteristic} if $\chr A = p > 0$ a rational prime, or $\chr K = 0$.

    If $A$ is not equicharacteristic, then it is said to be of \define{mixed characteristic}, that is, either $\chr A = 0$ and $\chr K > 0$, or $\chr A = p^n$ for some $n > 1$ and rational prime $p > 0$.
\end{definition}

\begin{definition}
    Let $(A,\frakm, K)$ be an equicharacteristic local ring and let $K'$ be a subfield of $A$. We say that $K'$ is a \define{coefficient field} of $A$ if $K'$ maps onto $K$ under the natural map $A\onto A/\frakm = K$, or equivalently, if $A = K' + \frakm$ as abelian groups. 

    We say that $K'$ is a \define{quasi-coefficient field} of $A$ if $K$ is $0$-\'etale over (the image of) $K'$.
\end{definition}

\begin{theorem}\thlabel{existence-of-quasi-coefficient-field}
    Let $(A,\frakm, K)$ be an equicharacteristic local ring. 
    \begin{enumerate}[label=(\arabic*)]
        \item If $K$ is separable over (the image of) a subfield $k\subseteq A$, then $A$ has a quasi-coefficient field $K'$ containing $k$.
        \item $A$ has a quasi-coefficient field. 
        \item If $K'$ is a quasi-coefficient field of $A$, then there exists a unique coefficient field $K''$ of the completion $\wh A$ containing (the image of) $K'$.
        \item If $A$ is complete, then it has a coefficient field. 
    \end{enumerate}
\end{theorem}
\begin{proof}
\begin{enumerate}[label=(\arabic*)]
    \item Choose a differential basis $B = \{\xi_i\}$ for the extension $K/k$, and choose preimages $x_i\in A$ for each $\xi_i$. As we have seen in \thref{algebraic-independence-of-differential-basis}, $B$ is algebraically independent, and hence, $k[\{x_i\}]\cap\frakm = (0)$, so that each element in this subring is invertible. Set $K' = \operatorname{Frac}\left(k[\{x_i\}]\right)$. The image of $K'$ under $A\to K$ is precisely $k(B)$ and due to \thref{0-etale-over-k(B)}, $K$ is $0$-\'etale over $k(B)$, as desired.

    \item Let $\Pi$ denote the prime subfield of $A$, which exists since $A$ is equicharacteristic. Since $\Pi$ is perfect, we can apply (1) to the inclusion $\Pi\subseteq A$ to obtain the desired conclusion.
    
    \item Consider the diagram 
    \begin{equation*}
        \xymatrix {
            K'\ar[r]\ar[dddd] & \wh{A}\\
            & \vdots\ar[d]\\
            & \wh A/\wh\frakm^3\ar[d]\\
            & \wh A/\wh\frakm^2\ar[d]\\
            K\ar[r]_-{\sim}\ar[ru]\ar[ruu]\ar[ruuuu] & \wh{A}/\wh{\frakm}
        }
    \end{equation*}
    Using the fact that $K/K'$ is $0$-\'etale, one can lift the isomorphism $K\xrightarrow{\sim}\wh A/\wh\frakm$ successively to the quotients $A/\frakm^n$. Taking the inverse limit over these quotients, one obtains a lift $K\to\wh A$. In particular, this means that the surjection $\wh A\to K$ splits, i.e., $\wh A$ admits a coefficient field. 

    \item Immediate from (2) and (3). \qedhere
\end{enumerate}
\end{proof}

\begin{proposition}\thlabel{replacing-by-completion}
    Let $k$ be a ring, $(A,\frakm)$ a local ring, $(\wh A,\wh\frakm)$ its completion, and $k\to A$ a ring homomorphism. Then 
    \begin{enumerate}[label=(\arabic*)]
        \item $\wh A$ is $\wh\frakm$-\'etale over $A$. 
        \item $A$ is $\frakm$-smooth (resp. $\frakm$-unramified) over $k$ if and only if $\wh A$ is $\wh\frakm$-smooth (resp. $\wh\frakm$-unramified) over $k$.
    \end{enumerate}
\end{proposition}
\begin{proof}
\begin{enumerate}[label=(\arabic*)]
    \item Consider a commutative diagram: 
    \begin{equation*}
        \xymatrix {
            A\ar[r]^-f\ar[d] & C\ar[d]^\pi\\
            \wh A\ar[r]_-u & C/N
        }
    \end{equation*}
    where $C$ is an $A$-algebra with an ideal $N$ such that $N^2 = 0$. Since $u(\wh\frakm^\nu) = 0$ for some $\nu > 0$, the map $u$ factors through $\wh A\to \wh A/\wh\frakm^\nu\cong A/\frakm^\nu$ as $A$-algebras. In particular, this means that $f$ sends $\frakm^\nu$ into $N$, therefore, $\frakm^{2\nu}\subseteq\ker f$. Factoring $u$ through $\wh A/\frakm^{2\nu}\cong A/\frakm^{2\nu}$, the construction of the lift is clear. The uniqueness of such a lift follows from the fact that $A/\frakm^{2\nu}$ is $0$-unramified over $A$, since the map is surjective\footnote{Recall that in \thref{equivalent-0-unramified}, we showed that being $0$-unramified is equivalent to the relative K\"ahler differentials being zero and later showed that the relative K\"ahler differentials $\Omega_{B/A}$ is zero whenever the map $A\to B$ is surjective}.

    \item Suppose $A$ is $\frakm$-smooth (resp. $\frakm$-unramified) over $k$, then due to \thref{transitivity}, $\wh A$ is $\wh\frakm$-smooth (resp. $\wh\frakm$-unramified) over $k$. Conversely, suppose $\wh A$ is $\wh\frakm$-smooth over $k$ and consider a commutative diagram of $k$-algebra homomorphisms 
    \begin{equation*}
        \xymatrix {
            k\ar[r]\ar[d] & C\ar[d]\\
            A\ar[r] & C/N
        }
    \end{equation*}
    satisfying the requirements. Then there is a positive integer $\nu > 0$ such that $u$ factors through $A/\frakm^\nu\cong\wh A/\wh\frakm^\nu$, resulting in a commutative diagram 
    \begin{equation*}
        \xymatrix {
            k\ar[rr]\ar[d]\ar[rd] & & C\ar[d]\\
            A\ar[r]\ar[d] & \wh A\ar[r]\ar[d]\ar@{.>}[ru] & C/N\\
            A/\frakm^\nu\ar[r]_-{\sim} & \wh A/\wh\frakm^\nu\ar[ru] & 
        }
    \end{equation*}
    Using the fact that $\wh A$ is $\wh\frakm$-smooth over $k$, there is a lift $\wh A\to C$, which gives a lift of $u$ after compositing with the morphism $A\to\wh A$. Argue similarly for ``unramified'' replacing ``smooth''. \qedhere
\end{enumerate}
\end{proof}

Before proceeding, we recall a useful result from the theory of completions: 
\begin{lemma}\thlabel{nak-type-completions}
    Let $A$ be a ring, $I$ an ideal in $A$, and $M$ an $A$-module. Suppose $A$ is $I$-adically complete, and $M$ is separated for the $I$-adic topology. If $M/IM$ is generated over $A/I$ by $\overline\omega_1,\dots,\overline\omega_n$, and $\omega_i\in M$ are arbitrary preimages of $\overline\omega_i$, then $M$ is generated over $A$ by $\omega_1,\dots,\omega_n$.
\end{lemma}

\begin{lemma}
    Let $(A,\frakm, K)$ be a Noetherian local ring containing a field $k$. If $A$ is $\frakm$-smooth over $k$, then $A$ is regular. 
    Conversely, if $K$ is separable over (the image of) $k$ and $A$ is regular, then $A$ is $\frakm$-smooth over $k$.
\end{lemma}
\begin{proof}
    Let $k_0$ denote the prime subfield of $k$, which is perfect, so that $k$ is $0$-smooth over $k_0$ in view of \thref{0-smooth-iff-separable}. Since $A$ is $\frakm$-smooth over $k$, by \thref{transitivity}, it is $\frakm$-smooth over $k_0$. Therefore, we may assume without loss of generality that $k$ is a perfect field. Further, in view of \thref{replacing-by-completion}, we may replace $A$ by $\wh A$ and assume that $A$ is a complete local ring. Due to \thref{existence-of-quasi-coefficient-field}, $A$ has a quasi-coefficient field containing $k$, which we identify with $K$. Let $\{x_1,\dots,x_n\}$ be a minimal $K$-basis of $\frakm/\frakm^2$. There is a natural $K$-algebra homomorphism 
    \begin{equation*}
        K[X_1,\dots,X_n]/(X_1,\dots,X_n)^2\to A/\frakm^2,
    \end{equation*}
    which is the identity on $K$ and sends $X_i\mapsto x_i$. Clearly, this is an isomorphism. Consider the composite 
    \begin{equation*}
        A\to A/\frakm^2\xrightarrow{\sim} K[X_1,\dots,X_n]/(X_1,\dots,X_n)^2\xrightarrow{\sim} K\llbracket X_1,\dots, X_n\rrbracket/(X_1,\dots,X_n)^2.
    \end{equation*}
    Since $A$ is $\frakm$-smooth over $k$, this lifts to a ring homomorphism $\varphi\colon A\to K\llbracket X_1,\dots, X_n\rrbracket$, which follows by lifting successively to the quotients $K\llbracket X_1,\dots,X_n\rrbracket/(X_1,\dots,X_n)^i$. We claim that this map is surjective. Indeed, the lift is identity on $K$, and $(X_1,\dots,X_n)/(X_1,\dots,X_n)^2$ is generated by the images of $x_1,\dots, x_n$ as a $K = A/\frakm$-module. Thus, due to \thref{nak-type-completions}, the map $\varphi$ is surjective.
    Therefore, $\dim A\ge\dim K\llbracket X_1,\dots, X_n\rrbracket = n = \embdim A$, consequently $A$ is regular.

    Conversely, suppose $K$ is separable over (the image of) $k$ and that $A$ is regular. Then $\wh A$ has a coefficient field $K$ containing $k$. Let $\{x_1,\dots,  x_n\}$ be a regular system of parameters of $\wh A$, and define a homomorphism of $K$-algebras $\psi\colon K\llbracket X_1,\dots, X_n\rrbracket \to \wh A$ sending $\psi(X_i) = x_i$. Then $\wh\frakm/\wh\frakm^2$ is generated as a $K = K\llbracket X_1,\dots, X_n\rrbracket/(X_1,\dots, X_n)$-module by $x_1,\dots,x_n$, whence by \thref{nak-type-completions}, $\wh\frakm$ is generated by $\{x_1,\dots,x_n\}$ as a $K\llbracket X_1,\dots,X_n\rrbracket$-module. Thus $\psi$ is surjective. Using the fact that $K\llbracket X_1,\dots,X_n\rrbracket$ is catenary and comparing dimensions, it follows that the map $\psi$ must be an isomorphism. Hence $\wh A$ is $\wh\frakm$-smooth over $K$, and since $K$ is $0$-smooth over $k$, we see that $\wh A$ is $\wh\frakm$-smooth over $k$, so that $A$ is $\frakm$-smooth over $k$, thereby completing the proof.
\end{proof}

% \begin{definition}
%     Let $k\to A\to B$ be ring homomorphisms. We say that $B$ is \define{$I$-smooth over $A$ relative to $k$} if for any $A$-algebra $C$, and an ideal $N$ of $C$ with $N^2 = 0$, given an $A$-algebra homomorphism $u\colon B\to C/N$ continuous with respect to the $I$-adic topology on $B$, if $u$ has a lifting $B\to C$ as a $k$-algebra, then it also has a lifting as an $A$-algebra. 
%     \begin{equation*}
%         \xymatrix {
%             k\ar[r] & A\ar[r]\ar[d] & C\ar[d]\\
%             & B\ar[r]_-u\ar@{.>}[ru] & C/N.
%         }
%     \end{equation*}
% \end{definition}

% \begin{theorem}
%     Let $k\xrightarrow{f} A\xrightarrow{g} B$ and $I$ an ideal of $B$. Then the following conditions are equivalent: 
%     \begin{enumerate}[label=(\arabic*)]
%         \item $B$ is $I$-smooth over $A$ relative to $k$. 
%         \item if $N$ is a $B$-module such that $I^\nu N = 0$ for some $\nu\gg 0$, then the natural map $\Der_k(B, N)\to\Der_k(A, N)$ is surjective.
%         \item for every $\nu > 0$, the map 
%         \begin{equation*}
%             \Omega_{A/k}\otimes_A\left(B/I^\nu\right)\to\Omega_{B/k}\otimes_B\left(B/I^\nu\right)
%         \end{equation*}
%         has a left inverse, i.e., it maps injectively onto a direct summand.
%     \end{enumerate}
% \end{theorem}

\begin{definition}
    Let $B$ be an $A$-algebra, $I$ an ideal of $B$, and equip $B$ with the $I$-adic topology. Let $N$ be a $B$-module such that $I^\nu N = 0$ for some $\nu > 0$. Such a $B$-module is said to be \define{discrete}, since it is discrete in the $I$-adic topology.

    An $A$-bilinear map $f\colon B\times B\to N$ is called a \define{continuous symmetric $2$-cocycle} if 
    \begin{enumerate}[label=(\roman*)]
        \item $xf(y, z) - f(xy, z) + f(x, yz) - f(x, y)z = 0$ for all $x, y, z\in B$, \label{cocycle-condition}
        \item $f(x, y) = f(y, x)$, \label{symmetry}
        \item there exists $\mu\ge\nu$ such that $f(x, y) = 0$ if either $x\in I^\mu$ or $y\in I^\mu$.\label{continuity}
    \end{enumerate}
\end{definition}

\begin{equation*}
    \boxed{\text{\textcolor{red}{What we do henceforth is extremely technical and the reader is advised to skip to the next section.}}}
\end{equation*}

Note that if $f$ is a continuous symmetric $2$-cocycle, then setting $\tau = f(1, 1)$, and substituting $y = z = 1$ in \ref{cocycle-condition}, we obtain $f(x, 1) = x\tau$. Define a product on the $A$-module $C = (B/I^\mu)\oplus N$ by 
\begin{equation*}
    (\overline x, \xi)\cdot(\overline y, \eta) = \left(\overline{xy}, - f(x, y) + x\eta + y\xi\right)\quad\text{ for all } x, y\in B.
\end{equation*}
Then $C$ is a commutative ring with multiplicative identity $(1, \tau)$, and $N$ is an ideal of $C$ such that $N^2 = 0$. Next, define a map $A\to C$ sending $a\mapsto (\overline a, a\tau)$. Then this is a ring homomorphism, and the diagram 
\begin{equation*}
    \xymatrix {
        A\ar[rr]\ar[d] & & C\ar[d]\\
        B\ar[r] & B/I^\mu\ar@{=}[r] & C/N.
    }
\end{equation*}
It can then be checked that a necessary and sufficient condition for $B\to C/N$ to have a lift is that there should exist an $A$-linear map $g\colon B\to N$ such that % TODO: check this
\begin{equation*}
    f(x, y) = xg(y) - g(xy) + g(x)y.\label{split}\tag{$\alpha'$}
\end{equation*}
We say that the $2$-cocycle $f$ \define{splits} if there is such an $A$-linear map $g$ satisfying \eqref{split}.  For any $A$-linear map $g\colon B\to N$, define $\delta g\colon B\times B\to N$ given by the right hand side of \eqref{split}. It is easy to check that $g$ satisfies the conditions \ref{cocycle-condition} and \ref{symmetry}. Further, if $g$ is continuous, i.e., there exists $\mu > 0$ such that $g(I^\mu) = 0$, then it also satisfies \ref{continuity}. % TODO: check this

\begin{theorem}\thlabel{cocycle-conditions-for-smoothness}
    Let $A$ be a ring, and $B$ an $A$-algebra equipped with an $I$-adic topology for some ideal $I$ of $B$. 
    \begin{enumerate}[label=(\arabic*)]
        \item If $B$ is $I$-smooth over $A$ then every continuous $2$-cocycle $f\colon B\times B\to N$ with values in a discrete $B$-module $N$ splits. 
        \item If $B/I^n$ is projective for infinitely many positive integers $n$, and if every continuous symmetric $2$-cocycle with values in a discrete $B$-module splits, then $B$ is $I$-smooth over $A$.
    \end{enumerate}
\end{theorem}
\begin{proof}
\begin{enumerate}[label=(\arabic*)]
    \item This is immediate from the discussion preceeding the statement of the Theorem. 
    \item Consider a commutative diagram of $A$-algebra homomorphisms 
    \begin{equation*}
        \xymatrix {
            A\ar[r]\ar[d] & C\ar[d]\\
            B\ar[r]_-u & C/N
        }
    \end{equation*}
    satisfying the usual requirements, i.e., $N^2 = 0$ and $u(I^\nu) = 0$ for some $\nu > 0$. One can view the $C$-module $N$ naturally as a $C/N$-module, and therefore, as a $B$-module through the map $u\colon B\to C/N$. As a result, $N$ is a discrete $B$-module. Take an integr $n > \nu$ such that $B/I^n$ is a projective $A$-module. Considering the following diagram of $A$-modules, 
    \begin{equation*}
        \xymatrix {
            & B/I^n\ar[d]\ar@{.>}[ld]\\
            C\ar@{->>}[r] & C/N
        }
    \end{equation*}
    there is a lift $B/I^n\to C$, which gives an $A$-map $\lambda\colon B\to C$ after composing with the natural projection $B\to B/I^n$. Note that $\lambda(I^n) = 0$. For $x, y\in B$, set 
    \begin{equation*}
        f(x, y) = \lambda(xy) - \lambda(x)\lambda(y).
    \end{equation*}
    Since modulo $N$, $\lambda$ is a ring homomorphism, it follows that $f(x, y)\in N$ for all $x,y\in B$. Next, for some $\xi\in N$ and $x\in B$, we havve $\lambda(x)\cdot\xi = x\cdot\xi$, where the right hand side is the $B$-module structure on $N$ described above. This immediately implies that $f$ satisfies \ref{cocycle-condition} and \ref{symmetry}. Also, since $\lambda(I^n) = 0$, we obtain \ref{continuity} too. Thus $f$ is a continuous symmetric $2$-cocycle. Hence, by assumption, there is an $A$-linear map $g\colon B\to N$ satisfying 
    \begin{equation*}
        f(x, y) = xg(y) - g(xy) + g(x)y.
    \end{equation*}
    Set $v = \lambda + g\colon B\to C$. Then 
    \begin{align*}
        v(xy) &= \lambda(xy) + g(xy)\\
        &= \lambda(x)\lambda(y) + f(x, y) + g(xy)\\
        &= \lambda(x)\lambda(y) + \lambda(x)g(y) + g(x)\lambda(y)\\
        &= v(x)v(y),
    \end{align*}
    since $g(x)g(y) = 0$. It follows that $v$ is an $A$-algebra homomorphism lifting $u$, thereby completing the proof.
\end{enumerate}
\end{proof}

\begin{theorem}\thlabel{easier-version-of-ega-theorem}
    Let $(A,\frakm, k)$ be a local ring, and $B$ a flat $A$-algebra such that $\overline B = B/\frakm B = B\otimes_A k$ is $0$-smooth over $k$. Then $B$ is $\frakm B$-smooth over $A$.
\end{theorem}
\begin{proof}
    First note that it suffices to show that $B/\frakm^\nu B$ is $0$-smooth over $A/\frakm^\nu$, indeed, given the standard diagram 
    \begin{equation*}
        \xymatrix {
            A\ar[r]\ar[d] & C\ar[d]\\
            B\ar[r] & C/N,
        }
    \end{equation*}
    the map $u$ factors through $B/\frakm^\nu B$, whence the map $A\to C$ factors through $A/\frakm^{2\nu}$. Thus, if $B/\frakm^{2\nu} B$ were $0$-smooth over $A/\frakm^{2\nu}$, then we are done. 

    Now, $B/\frakm^\nu B$ is flat over $A/\frakm^\nu$, so we can assume that $\frakm$ is nilpotent in $A$. Then every flat $A$-module is free due to \cite[Theorem 7.10]{matsumura-crt}\footnote{Note that this does NOT require the module to be finitely generated, as opposed to when $\frakm$ may not be nilpotent.}. Thus, $B/\frakm^n B$ is a projective $A$-module for infinitely many positive integers $n$. Therefore, in view of \thref{cocycle-conditions-for-smoothness}, it suffices to show that every continuous symmetric $2$-cocycle with values in a discrete $B$-module splits. But since $\frakm$ is nilpotent, all the topological considerations can be dropped. 

    Let $f\colon B\times B\to N$ be a symmetric $2$-cocycle where $N$ is a $B$-module. Suppose first that $\frakm N = 0$. Since $f$ is $A$-bilinear, it descends to a $2$-cocycle $\overline f\colon\overline B\times\overline B\to N$ with $f(x, y) = \overline f(\overline x, \overline y)$. Now $\overline B$ is $0$-smooth over $k$, so that by \thref{cocycle-conditions-for-smoothness}, $\overline f$ splits, that is, there is a $k$-linear map $\overline g\colon\overline B\to N$ such that $\overline f = \delta\overline g$. Setting $g\colon B\to N$ by $g(x) = \overline g(\overline x)$, we have 
    \begin{equation*}
        f(x, y) = \overline f(\overline x, \overline y) = \overline x\overline g(\overline y) - \overline g(\overline x\overline y) + \overline g(\overline x)\overline y = xg(y) - g(xy) + g(x)y,
    \end{equation*}
    so that $f = \delta g$, i.e., the cocycle splits.

    Next, in the general case, let $\varphi\colon N\to N/\frakm N$ be the natural projection, and consider $\varphi\circ f\colon B\times B\to N/\frakm N$, which splits as seen above, that is, there is an $A$-linear map $\overline g\colon B\to N/\frakm N$ such that $\varphi\circ f = \delta\overline g$. Next, since $B$ is a projective $A$-module, there is a lift $g\colon B\to N$ of the map $\overline g$. 
    \begin{equation*}
        \xymatrix {
            & B\ar[d]^{\overline g}\ar@{.>}[ld]_{g}\\
            N\ar@{->>}[r] & N/\frakm N.
        }
    \end{equation*}
    Then $f - \delta g$ is a $2$-cocycle with values in $\frakm N$. Repeating the above argument, we find $h\colon B\to\frakm N$ such that $f - \delta(g + h)$ is a $2$-cocycle with values in $\frakm^2 N$. Continuing in this fashion, this process msut terminate since $\frakm$ is nilpotent. This completes the proof.
\end{proof}

\begin{remark}
    We note here that the statement of \thref{easier-version-of-ega-theorem} is true if ``smooth'' is replaced by eihter ``unramified'' or ``\'etale''. To see this, note that it suffices to prove the ``unramified'' case since the ``\'etale'' case would then follow by putting the two together.

    Consider the standard commutative diagram setup: 
    \begin{equation*}
        \xymatrix {
            A\ar[r]\ar[d] & C\ar[d]\\
            B\ar[r]_-u & C/N
        }
    \end{equation*}
    where $u(\frakm^\nu B) = 0$. Note that $N$ is naturally a $C/N$-module and hence a $B$-module through $u$, and an $A$-module through $A\to B$. As an $A$-module, it is clear that $\frakm^\nu N = 0$. If there were two lifts of $u$, then their difference would give an $A$-derivation $D\colon B\to N$. We shall show that $D = 0$. 
    
    Note that $D$ induces a $k$-derivation $\overline D\colon \overline{B} = B/\frakm B\to N/\frakm N$. But since $\overline B$ is $0$-unramified over $k$, $\Omega_{\overline B/k} = 0$, so that $\overline D = 0$, i.e, $D$ takes values in $\frakm N$. Repeating this procedure with $\frakm N/\frakm^2 N$, it would follow that $D$ takes values in $\frakm^2 N$. Since $\frakm^\nu N = 0$, successive repetition shows that $D = 0$. Thus $B$ is $\frakm B$-unramified over $A$.
\end{remark}

\section{The Cohen Structure Theorems}

\epigraph{``\emph{Polynomials and power series\\May they forever rule the world.}''}{Shreeram S. Abhyankar}


% \begin{interlude}[Surjectivity of Inverse Limit]
%     Consider a morphism between two inverse systems of abelian groups
%     \begin{equation*}
%         \xymatrix {
%             \cdots\ar[r] & A_{n + 1}\ar[r]^-{\lambda_{n + 1}}\ar[d]_{f_{n + 1}} & A_n\ar[r]^-{\lambda_n}\ar[d]_{f_n} & A_{n - 1}\ar[r]^-{\lambda_{n - 1}}\ar[d]_{f_{n - 1}} & \cdots\ar[r] & A_1\ar[r]\ar[d]^{f_1} & 0\\
%             \cdots\ar[r] & B_{n + 1}\ar[r]_{\mu_{n + 1}} & B_n\ar[r]_{\mu_n} & B_{n - 1}\ar[r]_{\mu_{n - 1}} & \cdots\ar[r] & B_1\ar[r] & 0
%         }
%     \end{equation*}
%     such that each $\lambda_i$, $\mu_i$, and $f_i$ is surjective. We shall show that the induced map 
%     \begin{equation*}
%         f\colon\varprojlim A_n\to\varprojlim B_n
%     \end{equation*}
%     is surjective. Indeed, consider an element $\xi\in\varprojlim B_n$. This is represented by a ``coherent sequence'' $(b_i)_{i\in\N}$, where $b_i\in B_i$ for all $i$, and $\mu_i(b_i) = b_{i - 1}$ for all $i\ge 2$. Set $C_i = f_i^{-1}(b_i)\subseteq A_i$. Note that for any $i\ge 2$ and $c_i\in C_i$, we have $f_{i - 1}\circ\mu_i(c_i) = \mu_i\circ f_i(c_i) = b_{i - 1}$, so that $\mu_i(C_i)\subseteq C_{i - 1}$.

%     $\mu^{-1}\circ f^{-1}$
% \end{interlude}

\begin{lemma}\thlabel{express-every-element-as-power-series}
    Let $(A,\frakm, k)$ be a local ring with $\frakm$ a finitely generated ideal, say $\frakm = (x_1,\dots,x_n)$. If $R$ is a subring of $A$ such that $A = R + \frakm$, then there is a surjective $R$-algebra homomorphism 
    \begin{equation*}
        \varphi\colon R\llbracket X_1,\dots,X_n\rrbracket\to A \qquad X_i\mapsto x_i\quad\text{ for } 1\le i\le n.
    \end{equation*}
\end{lemma}
\begin{proof}
    Let us first establish the existence of such a map. Indeed, set $\varphi_j\colon R[X_1,\dots,X_n]/(X_1,\dots,X_n)^j\to A/\frakm^j$ sending $X_i\mapsto x_i$ for $1\le i\le n$. Clearly these maps form morphisms between two inverse systems, thereby inducing a map 
    \begin{equation*}
        \varphi\colon R\llbracket X_1,\dots,X_n\rrbracket = \varprojlim R[X_1,\dots,X_n]/(X_1,\dots,X_n)^j\to\varprojlim A/\frakm^j = A.
    \end{equation*}
    Let $a\in A$. Since $A = \varprojlim A/\frakm^n$, we can write $a = (a_n)_{n\ge 1}$ such that $a_{n + 1} - a_n\in\frakm^n$. We shall explicitly find an element in $R\llbracket X_1,\dots,X_n\rrbracket$ mapping to $a\in A$. First note that $a_1 = r_0 + m_1$ for some $m_1\in\frakm$. Any element of $\frakm$ can be written as a linear combination $p_1 x_1 + \dots + p_nx_n$ for some $p_1,\dots,p_n\in A$. Again, using the fact that $A = R + \frakm$, we  can write each $p_i = q_i + \wt m_i$. Substituting this in the expression for $m_1$, we can write 
    \begin{equation*}
        m_1 = (\text{$R$-linear combination of }x_1,\dots,x_n) + m_2,
    \end{equation*}
    where $m_2\in\frakm^2$. Next, any element in $\frakm^2$ can be written as an $A$-linear combination of the degree two monomials in $x_1,\dots,x_n$. Continuing in this fashion, we obtain power series in $R\llbracket X_1,\dots,X_n\rrbracket$ mapping onto $a$, as desired.
\end{proof}

As an immediate application of this observation, we have the following version of the Cohen Structure Theorem for equicharacteristic local rings: 

\begin{theorem}
    An equicharacteristic complete Noetherian local ring $(A,\frakm, K)$ is a quotient of some power series ring $K\llbracket X_1,\dots, X_n\rrbracket$.
\end{theorem}
\begin{proof}
    Let $x_1,\dots,x_n$ be a set of minimal generators for the maximal ideal. Due to \thref{existence-of-quasi-coefficient-field}, $A$ admits a quasi-coefficient field $K\into A$, so that we can write $A = K\oplus\frakm$ as $K$-modules. In view of \thref{express-every-element-as-power-series}, there is a surjective $K$-algebra homomorphism $\varphi\colon K\llbracket X_1,\dots, X_n\rrbracket\to A$, as desired.
\end{proof}

Our next goal will be to extend the above result to local rings of mixed characteristic.

\begin{definition}
    A DVR of characteristic $0$ is said to be a \define{$p$-ring} if its (unique) maximal ideal is generated by the rational prime $p$.
\end{definition}

\begin{theorem}\thlabel{DVR-with-given-residue-field}
    Let $(A, tA, k)$ be a DVR and $K$ an extension field of $k$, then there exists a DVR $(B, tB, K)$ containing $A$.
\end{theorem}
\begin{proof}
    % TODO: Add in later
\end{proof}

\begin{theorem}\thlabel{lifting-map-on-residue-fields}
    Let $(A,\frakm, K)$ be a complete local ring, $(R, pR, k)$ a $p$-ring, and $\varphi_0\colon k\to K$ a field homomorphism. Then there exists a local homomorphism $\varphi\colon R\to A$ which induces $\varphi_0$ on the residue fields.
\end{theorem}
\begin{proof}
    Set $S = \Z_{p\Z}$, and let $\Pi\subseteq k$ denote the prime subfield, which in this case is just a copy of $\F_p$. Since $\varphi_0(\Pi)$ puts another copy of $\F_p$ in $K$, it follows that $p\in\frakm$ when viewed as the image of $p$ under the maop $\Z\to A$. Therefore, the unique homomorphism $\Z\to A$ sends every element not in $p\Z$ to an element not in $\frakm$, therefore it naturally extends to a ring homomorphism $S\to A$.

    Next, we claim that $R$ is $pR$-smooth over $S$. Note that $R\otimes_S \Pi = R/pR = k$ is a separable extension of $\Pi$, and hence is $0$-smooth over $\Pi$ in view of \thref{0-smooth-iff-separable}. Further, since $R$ is a torsion free $S$-module, it is a flat $S$-module\footnote{This follows since $R$ is a direct limit of finitely generated $S$-submodules, and a finite torsion-free $S$-module is free, in particular, flat. More generally, it is true that a torsion-free module over a valuation ring is flat.}. Thus due to \thref{easier-version-of-ega-theorem}, $R$ is $pR$-smooth over $S$. We shall now lift the map $R\to k\xrightarrow{\varphi_0} K$ to an $S$-algebra homomorphism $R\to A$.
    \begin{equation*}
        \xymatrix {
            S\ar[r]\ar[dddd] & {A}\\
            & \vdots\ar[d]\\
            &  A/\frakm^3\ar[d]\\
            &  A/\frakm^2\ar[d]\\
            R\ar[r]\ar[ru]\ar[ruu]\ar[ruuuu] & K
        }
    \end{equation*}
    This is achieved by lifting successively to quotients $A/\frakm^i$ as before, and then taking the inverse limit of all such lifts. This completes the proof.
\end{proof}

\begin{corollary}
    A complete $p$-ring is determined up to isomorphism by its residue field.
\end{corollary}
\begin{proof}
    Suppose $(R, pR, k)$ and $(R', pR', k)$ are complete $p$-rings with the same residue field $k$. Then the identity map $k\to k$ induces a local homomorphism $\varphi\colon R\to R'$ as in \thref{lifting-map-on-residue-fields}. In view of \thref{nak-type-completions}, since $R'/pR'$ is generated by $\{1\}$ as an $R/pR$-module, we must have that $R'$ is generated by $\{1\}$ as an $R$-module, so that $\varphi$ is surjective.

    Next, note that the map must also be injective, since $ker\varphi$ is a prime ideal but not equal to the maximal ideal of $R$, therefore must be zero. This shows that $\varphi$ is an isomorphism.
\end{proof}

\begin{definition}
    Let $(A,\frakm, k)$ be a complete local ring of mixed characteristic and let $\chr k = p > 0$. A subring $A_0\subseteq A$ is said to be a \define{coefficient ring} if $A_0$ is a complete Noetherian local ring with maximal ideal $pA_0$ and 
    \begin{equation*}
        A = A_0 + \frakm\quad\text{ that is }\quad k = A/\frakm = A_0/pA_0.
    \end{equation*}
\end{definition}

\begin{remark}
    We note that the above definition subsumes the notion of a coefficient field. Indeed, if $\chr k = 0$, then we have seen that $A$ admits a coefficient field, which is precisely a complete Noetherian subring $A_0\subseteq A$ whose maximal ideal is $(0)$ and $A = A_0 + \frakm$.
\end{remark}

\begin{theorem}\thlabel{coefficient-ring-exists}
    If $(A,\frakm, k)$ is a complete local ring and $\chr k = p > 0$, then $A$ has a coefficient ring $A_0$. If $\chr A = 0$, then $A_0$ is a complete DVR, and if $\chr A > 0$, then $A_0$ is the quotient of a complete $p$-ring.
\end{theorem}
\begin{proof}
    Due to \thref{DVR-with-given-residue-field}, there exists a $p$-ring $S$ such that $S/pS = k$. Write $R$ for the completion of $S$, so that $R$ is a complete $p$-ring with residue field $k$. By \thref{lifting-map-on-residue-fields}, there exists a local homomorphism $\varphi\colon R\to A$ inducing the identity map on the residue fields. Set $A_0 = \varphi(R)\subseteq A$. This is clearly a coefficient ring of $A$. If $A$ has characteristic $0$, then $\ker\varphi$ must be zero, for if it were non-zero, then it would be of the form $p^n R$ but the image of $p^n$ under $\varphi$ is a non-zero element of $A$ for all $n\ge 1$. Thus if $\chr A = 0$, then $A_0$ is a complete DVR.
\end{proof}

\begin{theorem}\thlabel{step-towards-cohen-structure-theorem}
\begin{enumerate}[label=(\arabic*)]
    \item If $(A,\frakm)$ is a complete local ring and $\frakm$ is finitely generated, then $A$ is Noetherian. 
    \item A Noetherian complete local ring is a quotient of a regular local ring, and in particular, is universally catenary.
    \item If $A$ is a Noetherian complete local ring (and in the case of mixed characteristic, $A$ is an integral domain), then there exists a subring $A'\subseteq A$ such that: 
    \begin{enumerate}[label=(\roman*)]
        \item $A'$ is a complete regular local ring with the same residue field as $A$, and 
        \item $A$ is a finite $A'$-module, in particular, the extension $A'\subseteq A$ is integral.
    \end{enumerate}
\end{enumerate}
\end{theorem}
\begin{proof}
\begin{enumerate}[label=(\arabic*)]
    \item Due to \thref{existence-of-quasi-coefficient-field} and \thref{coefficient-ring-exists}, $A$ admits a coefficient ring $A_0\subseteq A$, which is Noetherian in either case. By \thref{express-every-element-as-power-series}, $A$ is a quotient of some power-series ring $A\llbracket X_1,\dots,X_n\rrbracket$, so that $A$ is Noetherian. 
    
    \item Note that $A_0$ is either a field, a complete DVR, or the quotient of a complete $p$-ring. In either case, we can write $A$ as the quotient of a power-series ring $R\llbracket X_1,\dots, X_n\rrbracket$, where $R$ is either a field, or a complete DVR, in particular, is regular. Thus $A$ is the quotient of a regular local ring, so that it is universally catenary. 
    
    \item Let $n = \dim A$. If $A$ were equicharacteristic, then choose $\{y_1,\dots,y_n\}\subseteq A$ to be a system of parameters. On the other hand, if $A$ were of mixed characteristic, then $A$ is a domain so that $\chr A = 0$, so that one can choose a system of parameters $\{y_1 = p, y_2,\dots, y_n\}\subseteq A$. Let $A_0\subseteq A$ denote the coefficient ring $R$. In the equicharacteristic case, this is a subfield, while in the mixed characteristic case, this is a complete $p$-ring. In the former case, consider the homomorphism $\varphi\colon R\llbracket Y_1,\dots, Y_n\rrbracket\to A$ and in the latter case $\varphi\colon R\llbracket Y_2,\dots, Y_n\rrbracket\to A$ sending $Y_i\mapsto y_i$. Let $A' = \im\varphi$, and let $\frakm'$ denote the image of the maximal ideal of $R\llbracket \vec Y\rrbracket$. Clearly this is the (unique) maximal ideal of $A'$, and further $A/\frakm = A'/\frakm'$, so that every $A$-module of finite length has the same length as an $A'$-module. In particular, since $\frakm' A$ is $\frakm$-primary, it follows that $A/\frakm' A$ has finite length as an $A'$-module. Also, since $A$ is $\frakm'$-adically separated (again because $\frakm'A$ is $\frakm$-primary), it follows from \thref{nak-type-completions} that $A$ is a finite $A'$-module, and hence $A'\subseteq A$ is an integral extension which in turn implies $\dim A' = \dim A = n$. Finally, note that $R\llbracket\vec Y\rrbracket$ is an $n$-dimensional integral domain, and if $\ker\varphi\ne 0$, then we would have $\dim A' < n$, a contradiction. Hence $\varphi$ is injective, and $A'\cong R\llbracket\vec Y\rrbracket$. This completes the proof. \qedhere
\end{enumerate}
\end{proof}

\begin{remark}
    In some applications, it is important to know exactly how the subring $A'\subseteq A$ is constructed, i.e., that we are free to choose any system of parameters of our choice. 
\end{remark}

\begin{definition}
    Let $(A,\frakm, K)$ be a local ring of mixed characteristic, and suppose $\chr K = p > 0$. The ring $A$ is said to be \define{ramified} if $p\in\frakm^2$ and said to be \define{unramified} otherwise. We shall also include equicharacteristic rings in the class of unramified local rings.
\end{definition}

\begin{theorem}
    An unramified complete regular local ring is a formal power series ring over a field or over a complete $p$-ring.
\end{theorem}
\begin{proof}
    Let $R$ be a coefficient ring of $A$. If $A$ is equicharacteristic, then $R$ is a field. Let $\{x_1,\dots,x_n\}$ be a regular system of parameters of $A$. Then just as argued in the proof of \thref{step-towards-cohen-structure-theorem}, one can show that the map $R\llbracket X_1,\dots, X_n\rrbracket\to A$ sending $X_i\mapsto x_i$ is an isomorphism. 

    Next, if $A$ is of mixed characteristic, then $\chr A = 0$ since it is an integral domain and $R$ is a complete $p$-ring. Further since $p\in\frakm\setminus\frakm^2$, one can choose a regular system of parameters $\{x_1 = p, x_2,\dots,x_n\}$ of $A$. Again, arguing as in the proof of \thref{step-towards-cohen-structure-theorem}, the map $R\llbracket X_2,\dots,X_n\rrbracket\to A$ sending $X_i\mapsto x_i$ is an isomorphism, thereby completing the proof.
\end{proof}

It remains to take care of the ramified case. 

\begin{lemma}[Eisenstein]
    Let $A$ be a ring and $f(X)\in X^n + a_{n - 1}X^{n - 1} + \dots + a_0$ with $a_i\in A$. If there exists a prime ideal $\frakp$ of $A$ such that $a_1,\dots,a_n\in\frakp$ but $a_n\notin\frakp^2$, then $f(X)$ is irreducible in $A[X]$. Moreover, if $A$ is an integrally closed domain, then the principal ideal $(f)$ is a prime ideal of $A[X]$.
\end{lemma}
\begin{proof}
    The first assretion is well-known and follows from taking a factorization of $f(X)$ and viewing it in $A/\frakp[X]$, which is an integral domain. Next, suppose $A$ is an integrally closed domain, and $K$ the fraction field of $A$. Then due to Gauss' lemma for normal domains, $f(X)$ remains irreducible in $K[X]$. Further since $f$ is monic, we have $f\cdot A[X] = f\cdot K[X]\cap A[X]$, which is the contraction of a prime ideal, and hence is prime.
\end{proof}

\begin{definition}
    Let $(A,\frakm, k)$ be an integrally closed domain. Then an extension ring $B = A[X]/(f) = A[x]$ defined by an Eisenstein polynomial 
    \begin{equation*}
        f(X) = X^n + a_{n - 1}X^{n - 1} + \dots + a_0\quad\text{ with }a_i\in\frakm\text{ for all }0\le i\le n - 1\text{ and } a_n\notin\frakm^2
    \end{equation*}
    is called an \define{Eisenstein extension} of $A$.
\end{definition}

From Eisenstein's lemma, $B$ is an integral domain that is an integral extension of $A$. Further, note that 
\begin{equation*}
    B/\frakm B = A[x]/\frakm[x] = A[X]/(\frakm, f(X)) = k[X]/X^n,
\end{equation*}
which is a local ring. Thus there is exactly one maximal ideal of $B$ lying over $\frakm$, so that $B$ is a local domain with the same residue field as $A$.

\begin{theorem}
\begin{enumerate}[label=(\arabic*)]
    \item If $(A,\frakm)$ is a regular local ring, then an Eisenstein extension of $A$ is again a regular local ring. 
    \item If $A$ is a ramified complete regular local ring and $R$ is a coefficient ring of $A$, then there exists a subring $A_0\subseteq A$ such that: 
    \begin{enumerate}[label=(\roman*)]
        \item $A_0$ is an unramified complete regular local ring containing $R$.
        \item $A$ is an Eisenstein extension of $A_0$.
    \end{enumerate}
\end{enumerate}
\end{theorem}
\begin{proof}
\begin{enumerate}[label=(\arabic*)]
    \item Let $B = A[x]$ and $x^m + a_{m - 1}x^{m - 1} + \dots + a_0 = 0$ with $a_i\in\frakm$ for all $i$ but $a_0\notin\frakm^2$. There exists a regular system of parameters $\{y_1,\dots, y_n = a_0\}$ of $A$. The maximal ideal of $B$ is $\frakm B + xB$, but $a_0\in xB$ from the above relation, so that $\{y_1,\dots, y_{n - 1}, x\}$ is also a regular system of parameters of $B$. Indeed, these elements generate the maximal ideal of $B$ and $\{y_1,\dots, y_{n - 1}\}$ is clearly a regular sequence. Finally, $x$ is clearly a regular element on $A/(y_1,\dots, y_{n - 1})[x]$. Thus $B$ is also a regular local ring.

    \item Our first goal will be to choose a regular system of parameters $\{x_1,\dots,x_n\}$ of $A$ such that $\{p, x_2,\dots, x_n\}$ is a system of parameters for $A$. This is achieved by induction on $n = \dim A$. To this end, we shall first choose a regular system of parameters $\{x_1,\dots,x_n\}$ of $A$ such that $p\notin (x_1,\dots,x_{n - 1})$ by induction on $n$. The base case $n = 1$ is trivial. Suppose now $n > 1$ and take any regular system of parameters $\{y_1,\dots,y_n\}$ of $A$. Let $\frakp_1,\dots,\frakp_r$ be those prime ideals in $A$ of height $1$ that contain $pA$. Note that none of the $\frakp_i$ can contain all the $y_j$'s since $\dim A > 1$. We can assume without loss of generality that $y_n\notin\frakp_1,\dots,\frakp_s$ but $y_n\in\frakp_{s + 1},\dots,\frakp_r$. Define 
    \begin{equation*}
        x_n = y_n + \sum_{j = s + 1}^r a_j z_j,
    \end{equation*}
    where $z_j$ is some element in $\{y_1,\dots,y_n\}\setminus\frakp_j$ and $a_i\in\bigcap\limits_{j\ne i}\frakp_j\setminus\frakp_i$. Then $x_n\notin\bigcup\limits_{j = 1}^r\frakp_j$. Further, the elements $y_1,\dots,y_{n - 1}, x_n$ generate $\frakm$ and form a regular sequence. Finally, note that $p\notin(x_n)$, lest $pA\subseteq (x_n)$, and hence $(x_n)$ would be a prime ideal of height $1$, so that $(x_n) = \frakp_i$\footnote{Recall that in a regular local ring, every element in a regular system of parameters must generate a prime ideal.} for some $1\le i\le r$, which is absurd. Moving to the ring $A/x_n A$, and arguing inductively, we have our desired regular sequence. 

    Now that we have a regular system of parameters $\{x_1,\dots,x_n\}$ such that $p\notin (x_1,\dots,x_{n - 1})$, set $\overline A = A/(x_1,\dots,x_{n - 1})$ so that $\overline p\ne 0$ in $\overline A$ and $\overline p\in\overline\frakm^2$. Further note that $\overline A$ is a regular local ring of dimension $1$, i.e. a DVR, so that there is an invertible elemente $\overline a\in\overline A$ and an integer $t\ge 2$ such that $\overline{p} = \overline a\overline x_n^t$. This shows that $x_n^t\in (p,x_1,\dots,x_{n - 1})$, whence $\frakm^t\subseteq(p,x_1,\dots,x_{n - 1})$, therefore $\{p,x_1,\dots,x_{n - 1}\}$ is a system of parameters, as desired.

    Coming back to the assertion at hand, we have that $A$ is a ramified complete regular local ring, in particular, an integral domain, so that $\chr A = 0$, and hence $R$ is a complete $p$-ring. 
    Set $A_0 = R\llbracket x_2,\dots,x_n\rrbracket$; formally this is the image of the ring homomorphism $R\llbracket X_2,\dots, X_n\rrbracket\to A$ sending $X_i\mapsto x_i$ for $2\le i\le n$. Arguing as in \thref{step-towards-cohen-structure-theorem}, $A$ is a finite $A_0$-module, and $A_0$ is a regular local ring isomorphic to $R\llbracket X_2,\dots, X_n\rrbracket$. The latter is clearly an unramified complete regular local ring. Let $\frakm_0$ denote the (unique) maximal ideal of $A_0$ and note that $A = \frakm_0 A + A_0[x_1]$ whence due to Nakayama's lemma, $A = A_0[x_1]$. Let 
    \begin{equation*}
        f(X) = X^m + a_{m - 1}X^{m - 1} + \dots + a_0
    \end{equation*}
    with $a_i\in A_0$ be the minimal polynomial of $x_1$ over $A_0$. Then $a_0\in x_1 A\subseteq\frakm$ so that $a_0\in\frakm_0 = \frakm\cap A_0$. Using Hensel's lemma, we can conclude that all the $a_i$'s must lie in $\frakm_0$, for if not, then moving to $A_0/\frakm_0[X]$, one could factor $\overline f$ as the product of two coprime monic polynomials and lift the factorization to $A_0[X]$, which would contradict the minimality of $f(X)$. 
    
    With this established, it only remains to show that $a_0\notin\frakm^2$. Write 
    \begin{equation*}
        p = \sum_{i = 1}^n b_ix_i
    \end{equation*}
    with $b_i\in A$ and we can write $b_i = \varphi_i(x_1)$ with $\varphi_i(X)\in A_0[X]$. Then $x_1$ is a root of the polynomial 
    \begin{equation*}
        F(X) = \varphi_1(X)X + \sum_{i = 2}^d \varphi_i(X)x_i - p,
    \end{equation*}
    so that $F(X)$ is divisible by $f(X)$ in $A_0[X]$. Hence the constant term $F(0)$ of $F$ is divisible by $a_0$ in $A_0$. But note that 
    \begin{equation*}
        F(0) = \sum_{i = 2}^n\varphi_i(0)x_i - p,
    \end{equation*}
    and $p, x_2,\dots, x_n$ is a regular system of parameters for $A_0$, so that $F(0)\notin\frakm_0^2$, and hence we must have that $a_0\notin\frakm_0^2$, as desired. This completes the proof. \qedhere
\end{enumerate}
\end{proof}