\documentclass[10pt]{article}

\title{Local Cohomology}
\author{Swayam Chube}
\date{Last Updated: \today}

\usepackage[utf8]{inputenc} % allow utf-8 input
\usepackage[T1]{fontenc}    % use 8-bit T1 fonts
\usepackage{hyperref}       % hyperlinks
\usepackage{url}            % simple URL typesetting
\usepackage{booktabs}       % professional-quality tables
\usepackage{amsfonts}       % blackboard math symbols
\usepackage{nicefrac}       % compact symbols for 1/2, etc.
\usepackage{microtype}      % microtypography
\usepackage{graphicx}
\usepackage{natbib}
\usepackage{doi}
\usepackage{amssymb}
\usepackage{bbm}
\usepackage{amsthm}
\usepackage{amsmath}
\usepackage{xcolor}
\usepackage{theoremref}
\usepackage{enumitem}
\usepackage{fouriernc}
\usepackage{mathrsfs}
\setlength{\marginparwidth}{2cm}
\usepackage{todonotes}
\usepackage{stmaryrd}
\usepackage[all,cmtip]{xy} % For diagrams, praise the Freyd-Mitchell theorem 
\usepackage{marvosym}
\usepackage{geometry}
\usepackage{titlesec}
\usepackage{mathtools}
\usepackage{tikz}
\usetikzlibrary{cd}

\renewcommand{\qedsymbol}{$\blacksquare$}
% \renewcommand{\familydefault}{\sfdefault} % Do you want this font? 

% Uncomment to override  the `A preprint' in the header
% \renewcommand{\headeright}{}
% \renewcommand{\undertitle}{}
% \renewcommand{\shorttitle}{}

\hypersetup{
    pdfauthor={Swayam Chube},
    colorlinks=true,
	citecolor=blue,
}

\newtheoremstyle{thmstyle}%               % Name
  {}%                                     % Space above
  {}%                                     % Space below
  {}%                             % Body font
  {}%                                     % Indent amount
  {\bfseries\scshape}%                            % Theorem head font
  {.}%                                    % Punctuation after theorem head
  { }%                                    % Space after theorem head, ' ', or \newline
  {\thmname{#1}\thmnumber{ #2}\thmnote{ (#3)}}%                                     % Theorem head spec (can be left empty, meaning `normal')

\newtheoremstyle{defstyle}%               % Name
  {}%                                     % Space above
  {}%                                     % Space below
  {}%                                     % Body font
  {}%                                     % Indent amount
  {\bfseries\scshape}%                            % Theorem head font
  {.}%                                    % Punctuation after theorem head
  { }%                                    % Space after theorem head, ' ', or \newline
  {\thmname{#1}\thmnumber{ #2}\thmnote{ (#3)}}%                                     % Theorem head spec (can be left empty, meaning `normal')

\theoremstyle{thmstyle}
\newtheorem{theorem}{Theorem}[section]
\newtheorem{lemma}[theorem]{Lemma}
\newtheorem{proposition}[theorem]{Proposition}

\theoremstyle{defstyle}
\newtheorem{definition}[theorem]{Definition}
\newtheorem{corollary}[theorem]{Corollary}
\newtheorem{porism}[theorem]{Porism}
\newtheorem{remark}[theorem]{Remark}
\newtheorem{interlude}[theorem]{Interlude}
\newtheorem{example}[theorem]{Example}
\newtheorem*{notation}{Notation}
\newtheorem*{claim}{Claim}

% Common Algebraic Structures
\newcommand{\R}{\mathbb{R}}
\newcommand{\Q}{\mathbb{Q}}
\newcommand{\Z}{\mathbb{Z}}
\newcommand{\N}{\mathbb{N}}
\newcommand{\bbC}{\mathbb{C}} 
\newcommand{\K}{\mathbb{K}} % Base field which is either \R or \bbC
\newcommand{\calA}{\mathcal{A}} % Banach Algebras
\newcommand{\calB}{\mathcal{B}} % Banach Algebras
\newcommand{\calI}{\mathcal{I}} % ideal in a Banach algebra
\newcommand{\calJ}{\mathcal{J}} % ideal in a Banach algebra
\newcommand{\frakM}{\mathfrak{M}} % sigma-algebra
\newcommand{\calO}{\mathcal{O}} % Ring of integers
\newcommand{\bbA}{\mathbb{A}} % Adele (or ring thereof)
\newcommand{\bbI}{\mathbb{I}} % Idele (or group thereof)

% Categories
\newcommand{\catTopp}{\mathbf{Top}_*}
\newcommand{\catGrp}{\mathbf{Grp}}
\newcommand{\catTopGrp}{\mathbf{TopGrp}}
\newcommand{\catSet}{\mathbf{Set}}
\newcommand{\catTop}{\mathbf{Top}}
\newcommand{\catRing}{\mathbf{Ring}}
\newcommand{\catCRing}{\mathbf{CRing}} % comm. rings
\newcommand{\catMod}{\mathbf{Mod}}
\newcommand{\catMon}{\mathbf{Mon}}
\newcommand{\catMan}{\mathbf{Man}} % manifolds
\newcommand{\catDiff}{\mathbf{Diff}} % smooth manifolds
\newcommand{\catAlg}{\mathbf{Alg}}
\newcommand{\catRep}{\mathbf{Rep}} % representations 
\newcommand{\catVec}{\mathbf{Vec}}

% Group and Representation Theory
\newcommand{\chr}{\operatorname{char}}
\newcommand{\Aut}{\operatorname{Aut}}
\newcommand{\GL}{\operatorname{GL}}
\newcommand{\im}{\operatorname{im}}
\newcommand{\tr}{\operatorname{tr}}
\newcommand{\id}{\mathbf{id}}
\newcommand{\cl}{\mathbf{cl}}
\newcommand{\Gal}{\operatorname{Gal}}
\newcommand{\Tr}{\operatorname{Tr}}
\newcommand{\sgn}{\operatorname{sgn}}
\newcommand{\Sym}{\operatorname{Sym}}
\newcommand{\Alt}{\operatorname{Alt}}

% Commutative and Homological Algebra
\newcommand{\spec}{\operatorname{spec}}
\newcommand{\mspec}{\operatorname{m-spec}}
\newcommand{\Spec}{\operatorname{Spec}}
\newcommand{\MaxSpec}{\operatorname{MaxSpec}}
\newcommand{\Tor}{\operatorname{Tor}}
\newcommand{\tor}{\operatorname{tor}}
\newcommand{\Ann}{\operatorname{Ann}}
\newcommand{\Supp}{\operatorname{Supp}}
\newcommand{\Hom}{\operatorname{Hom}}
\newcommand{\End}{\operatorname{End}}
\newcommand{\coker}{\operatorname{coker}}
\newcommand{\limit}{\varprojlim}
\newcommand{\colimit}{%
  \mathop{\mathpalette\colimit@{\rightarrowfill@\textstyle}}\nmlimits@
}
\makeatother


\newcommand{\fraka}{\mathfrak{a}} % ideal
\newcommand{\frakb}{\mathfrak{b}} % ideal
\newcommand{\frakc}{\mathfrak{c}} % ideal
\newcommand{\frakf}{\mathfrak{f}} % face map
\newcommand{\frakg}{\mathfrak{g}}
\newcommand{\frakh}{\mathfrak{h}}
\newcommand{\frakm}{\mathfrak{m}} % maximal ideal
\newcommand{\frakn}{\mathfrak{n}} % naximal ideal
\newcommand{\frakp}{\mathfrak{p}} % prime ideal
\newcommand{\frakq}{\mathfrak{q}} % qrime ideal
\newcommand{\fraks}{\mathfrak{s}}
\newcommand{\frakt}{\mathfrak{t}}
\newcommand{\frakz}{\mathfrak{z}}
\newcommand{\frakA}{\mathfrak{A}}
\newcommand{\frakI}{\mathfrak{I}}
\newcommand{\frakJ}{\mathfrak{J}}
\newcommand{\frakK}{\mathfrak{K}}
\newcommand{\frakL}{\mathfrak{L}}
\newcommand{\frakN}{\mathfrak{N}} % nilradical 
\newcommand{\frakO}{\mathfrak{O}} % dedekind domain
\newcommand{\frakP}{\mathfrak{P}} % Prime ideal above
\newcommand{\frakQ}{\mathfrak{Q}} % Qrime ideal above 
\newcommand{\frakR}{\mathfrak{R}} % jacobson radical
\newcommand{\frakU}{\mathfrak{U}}
\newcommand{\frakV}{\mathfrak{V}}
\newcommand{\frakW}{\mathfrak{W}}
\newcommand{\frakX}{\mathfrak{X}}

% General/Differential/Algebraic Topology 
\newcommand{\scrA}{\mathscr{A}}
\newcommand{\scrB}{\mathscr{B}}
\newcommand{\scrF}{\mathscr{F}}
\newcommand{\scrM}{\mathscr{M}}
\newcommand{\scrN}{\mathscr{N}}
\newcommand{\scrP}{\mathscr{P}}
\newcommand{\scrO}{\mathscr{O}} % sheaf
\newcommand{\scrR}{\mathscr{R}}
\newcommand{\scrS}{\mathscr{S}}
\newcommand{\bbH}{\mathbb H}
\newcommand{\Int}{\operatorname{Int}}
\newcommand{\psimeq}{\simeq_p}
\newcommand{\wt}[1]{\widetilde{#1}}
\newcommand{\RP}{\mathbb{R}\text{P}}
\newcommand{\CP}{\mathbb{C}\text{P}}

% Miscellaneous
\newcommand{\wh}[1]{\widehat{#1}}
\newcommand{\calM}{\mathcal{M}}
\newcommand{\calP}{\mathcal{P}}
\newcommand{\onto}{\twoheadrightarrow}
\newcommand{\into}{\hookrightarrow}
\newcommand{\Gr}{\operatorname{Gr}}
\newcommand{\Span}{\operatorname{Span}}
\newcommand{\ev}{\operatorname{ev}}
\newcommand{\weakto}{\stackrel{w}{\longrightarrow}}

\newcommand{\define}[1]{\textcolor{blue}{\textit{#1}}}
% \newcommand{\caution}[1]{\textcolor{red}{\textit{#1}}}
\newcommand{\important}[1]{\textcolor{red}{\textit{#1}}}
\renewcommand{\mod}{~\mathrm{mod}~}
\renewcommand{\le}{\leqslant}
\renewcommand{\leq}{\leqslant}
\renewcommand{\ge}{\geqslant}
\renewcommand{\geq}{\geqslant}
\newcommand{\Res}{\operatorname{Res}}
\newcommand{\floor}[1]{\left\lfloor #1\right\rfloor}
\newcommand{\ceil}[1]{\left\lceil #1\right\rceil}
\newcommand{\gl}{\mathfrak{gl}}
\newcommand{\ad}{\operatorname{ad}}
\newcommand{\Stab}{\operatorname{Stab}}
\newcommand{\bfX}{\mathbf{X}}
\newcommand{\Ind}{\operatorname{Ind}}
\newcommand{\bfG}{\mathbf{G}}
\newcommand{\rank}{\operatorname{rank}}
\newcommand{\calo}{\mathcal{o}}
\newcommand{\frako}{\mathfrak{o}}
\newcommand{\Cl}{\operatorname{Cl}}

\newcommand{\idim}{\operatorname{idim}}
\newcommand{\pdim}{\operatorname{pdim}}
\newcommand{\Ext}{\operatorname{Ext}}
\newcommand{\co}{\operatorname{co}}
\newcommand{\bfO}{\mathbf{O}}
\newcommand{\bfF}{\mathbf{F}} % Fitting Subgroup
\newcommand{\Syl}{\operatorname{Syl}}
\newcommand{\nor}{\vartriangleleft}
\newcommand{\noreq}{\trianglelefteqslant}
\newcommand{\subnor}{\nor\!\nor}
\newcommand{\Soc}{\operatorname{Soc}}
\newcommand{\core}{\operatorname{core}}
\newcommand{\Sd}{\operatorname{Sd}}
\newcommand{\mesh}{\operatorname{mesh}}
\newcommand{\sminus}{\setminus}
\newcommand{\diam}{\operatorname{diam}}
\newcommand{\Ass}{\operatorname{Ass}}
\newcommand{\projdim}{\operatorname{proj~dim}}
\newcommand{\injdim}{\operatorname{inj~dim}}
\newcommand{\gldim}{\operatorname{gl~dim}}
\newcommand{\embdim}{\operatorname{emb~dim}}
\newcommand{\hght}{\operatorname{ht}}
\newcommand{\depth}{\operatorname{depth}}
\newcommand{\ul}[1]{\underline{#1}}
\newcommand{\type}{\operatorname{type}}
\newcommand{\CM}{\operatorname{CM}}
\newcommand{\cech}[1]{\mathbin{\check{#1}}}
\newcommand{\cdim}{\operatorname{cdim}}
\newcommand{\ara}{\operatorname{ara}}

\geometry {
    margin = 1in
}

\titleformat
{\section}
[block]
{\Large\bfseries\sffamily}
{\S\thesection}
{0.5em}
{\centering}
[]


\titleformat
{\subsection}
[block]
{\normalfont\bfseries\sffamily}
{\S\S}
{0.5em}
{\centering}
[]


\begin{document}
\maketitle

\section{The \texorpdfstring{$I$}{I}-torsion functor}

\begin{definition}
    Let $R$ be a ring, $I\noreq R$ an ideal, and $M$ an $R$-module. Define 
    \begin{equation*}
        \Gamma_I(M) \coloneq\left\{x\in M\colon \text{there is a positive integer $n\in\N$ such that }I^nx = 0\right\} = \bigcup_{n\ge 1}(0 :_M I^n).
    \end{equation*}
    This is known as the \define{$I$-torsion functor}.
\end{definition}

It is clear that $\Gamma_I(M)$ is a submodule of $M$ and any $R$-linear map $\varphi: M\to N$ restricts to an $R$-linear map $\Gamma_I(\varphi): \Gamma_I(M)\to\Gamma_I(N)$. Thus, $\Gamma_I: {}_R\mathfrak{Mod}\to{}_R\mathfrak{Mod}$ is a functor.

\begin{lemma}
	The functor $\Gamma_I$ is left-exact.
\end{lemma}
\begin{proof}
	Let $0\to M'\xrightarrow{\alpha} M\xrightarrow{\beta} M''\to 0$ be a short exact sequence of $R$-modules.
\end{proof}

\begin{definition}
	The right derived functors of $\Gamma_I: {}_R\mathfrak{Mod}\to{}_R\mathfrak{Mod}$ are called the \define{local cohomology functors with support in $I$}.
\end{definition}

\begin{center}
	\boxed{\text{Henceforth $R$ is a Noetherian ring unless specified otherwise.}}
\end{center}

There are some properties of $\Gamma_I$ which are trivial to verify: 
\begin{itemize}
	\item $\Gamma_I(M) = \Gamma_{\sqrt I}(M)$ as submodules of $M$. 
	\item If $\Gamma_I(M) = 0$, then $I$ cannot be contained in any associated prime of $M$. In particular, if $M$ is a finite $R$-module, then $\Ass_R(M)$ is finite, and hence, using Prime Avoidance, there is an $M$-regular element in $I$, that is, $\depth(I, M)\ge 1$.
	\item Given a family of $R$-modules $\{M_\alpha\}_{\alpha\in\Lambda}$, $\Gamma_I\left(\bigoplus\limits_{\alpha\in\Lambda} M_\alpha\right) = \bigoplus\limits_{\alpha\in\Lambda}\Gamma_I(M_\alpha)$ as submodules of $\bigoplus\limits_{\alpha\in\Lambda} M_\alpha$.
	\item If $S\subseteq R$ is a multiplicative subset, then $\Gamma_{S^{-1}I}(S^{-1}M) = S^{-1}\Gamma_I(M)$ as submodules of $S^{-1}M$.
	\item For $\frakp\in\Spec(R)$, 
	\begin{equation*}
		\Gamma_I\left(E_R(R/\frakp)\right) = 
		\begin{cases}
			E_R(R/\frakp) & I\subseteq\frakp\\
			0 & \text{otherwise}.
		\end{cases}
	\end{equation*}
	In particular if $E$ is an injective $R$-module, then $\Gamma_I(E)$ is an injective $R$-module, and is a direct summand of $E$.
\end{itemize}
Since all the above isomorphisms are natural, these extend to isomorphisms on local cohomology, that is, for $i\ge 0$:
\begin{itemize}
	\item $H^i_I(M) = \Gamma_{\sqrt I}(M)$ as submodules of $M$. 
	\item Given a family of $R$-modules $\{M_\alpha\}_{\alpha\in\Lambda}$, $H^i_I\left(\bigoplus\limits_{\alpha\in\Lambda} M_\alpha\right) = \bigoplus\limits_{\alpha\in\Lambda}\Gamma_I(M_\alpha)$ as submodules of $\bigoplus\limits_{\alpha\in\Lambda} M_\alpha$.
	\item If $S\subseteq R$ is a multiplicative subset, then $H^i_{S^{-1}I}(S^{-1}M) = S^{-1}\Gamma_I(M)$ as submodules of $S^{-1}M$.
\end{itemize}

\begin{lemma}\thlabel{lem:I-torsion-of-injective-hull}
	Let $M$ be an $R$-module. If $\Gamma_I(M) = M$, then $\Gamma_I(E_R(M)) = E_R(M)$.
\end{lemma}
\begin{proof}
	Suppose not and choose some $x\in E_R(M)\setminus\Gamma_I(E_R(M))$. Since $R$ is Noetherian, there is an associated prime $\frakp$ of $E_R(M)$ containing $\Ann_R(x)$. But since there is no power of $I$ annihilating $x$, we must have $I\not\subseteq\frakp$.

	On the other hand, since $\Ass_R(E_R(M)) = \Ass_R(M)$, it follows that there is some $y\in M$ with $\frakp = \Ann_R(y)$. Further, since $\Gamma_I(M) = M$, there is a positive integer $n > 0$ such that $I^ny = 0$, i.e., $I^n\subseteq\frakp$, and hence, $I\subseteq\frakp$, a contradiction.
\end{proof}

\begin{corollary}\thlabel{cor:cohomology-of-I-torsion-module}
	Let $M$ be an $R$-module. If $\Gamma_I(M) = M$, then $H^i(M) = 0$ for $i > 0$.
\end{corollary}
\begin{proof}
	Let $0\to M\to E^\bullet$ be a minimal injective resolution of $M$. Due to \thref{lem:I-torsion-of-injective-hull}, it follows that $\Gamma_I(E^i) = E^i$ for $i\ge 0$, and the conclusion follows, since the resolution remains unchanged after applying $\Gamma_I$.
\end{proof}

\begin{proposition}\thlabel{prop:killing-I-torsion}
	Let $M$ be an $R$-module, and set $N\coloneq M/\Gamma_I(M)$. Then $\Gamma_I(N) = 0$ and $H^i_I(N)\cong H^i_I(M)$ for $i > 0$.
\end{proposition}
\begin{proof}
	Set $L\coloneq\Gamma_I(M)$. Then there is a short exact sequence $0\to L\to M\to N\to 0$, and $L$ is $\Gamma_I$-acyclic. It is then clear from the long exact sequence that $H^i_I(N)\cong H^i_I(M)$ for $i > 0$. Finally, since the induced map $\Gamma_I(L)\to\Gamma_I(M)$ is an isomorphism and $H^1_I(L) = 0$, it follows that $\Gamma_I(N) = 0$.
\end{proof}

\begin{theorem}[Grothendieck Vanishing Theorem]\thlabel{lem:grothendieck-vanishing}
	Let $(R,\frakm,k)$ be a Noetherian local ring, $I\subseteq R$ an ideal, and $M$ a finite $R$-module. Then $H^j_I(M) = 0$ for $j > \dim_R M$.
\end{theorem}
\begin{proof}
	We argue by induction on $d\coloneq\dim_R(M)$. If $d = 0$, then $M$ is Artinian, so that $\Ass_R(M) = \{\frakm\}$. It follows that every element of $M$ is annihilated by a power of $\frakm$, and thus by a power of $I$. Consequently, $\Gamma_I(M) = M$. Due to \thref{cor:cohomology-of-I-torsion-module}, $H^i_I(M) = 0$ for $i > 0$, and this establishes the base case.

	Suppose now that $d > 0$. Set $N\coloneq M/\Gamma_I(M)$. As we have seen in \thref{prop:killing-I-torsion}, $\Gamma_I(N) = 0$ and $H^i_I(N)\cong H^i_I(M)$ for $i > 0$. Further, $\dim_R N\le d$. If this inequality is strict, then we are done due to the induction hypothesis. Hence we may assume that $\dim_R N = d$. As we remarked earlier, since $\Gamma_I(N) = 0$, and $N$ is a finite $R$-module, $\depth(I, N)\ge 1$. Choose an $N$-regular element $a\in I$. Let $x\mapsto\overline x$ denote the natural surjection $M\onto M/aM\eqcolon\overline M$ and $\mu_a: M\to M$ be multiplication by $a$. The short exact sequence $0\to M\xrightarrow{\mu_a} M\to\overline M\to 0$ induces a long exact sequence: 
	\begin{equation*}
		\cdots\to H^{i - 1}_I(\overline M)\to H^i_I(M)\xrightarrow{\mu_a} H^i_I(M)\to H^i_I(\overline M)\to\cdots.
	\end{equation*}
	For $i > d$, note that $i - 1 > d - 1 = \dim_R\overline M$, so that $H^{i - 1}_I(\overline M) = H^i_I(\overline M) = 0$. This shows that $\mu_a: H^i_I(M)\to H^i_I(M)$ is an isomorphism of $R$-modules. Recall that $H^i_I(M)$ is $I$-torsion and $a\in I$. If $H^i_I(M)\ne 0$, then for $n\gg 0$, the composition $\mu_a^n$ would have non-trivial kernel, which is absurd since it is an isomorphism. This shows that $H^i_I(M) = 0$ for $i > d$, as desired.
\end{proof}

\begin{proposition}
	Let $(R,\frakm,k)$ be a Gorenstein local ring with $d = \dim R$. Then 
	\begin{equation*}
		H^d_{\frakm}(R)\cong E_R(k).
	\end{equation*}
\end{proposition}
\begin{proof}
	It is well-known that the minimal injective resolution of a Gorenstein local ring looks like:
	\begin{equation*}
		0\to R\to\bigoplus_{\hght\frakp = 0} E_R(R/\frakp)\to\bigoplus_{\hght\frakp = 1} E_R(R/\frakp)\to\cdots\to E_R(k)\to 0.
	\end{equation*}
	Further, it is clear that 
	\begin{equation*}
		\Gamma_{\frakm}\left(E_R(R/\frakp)\right) = 
		\begin{cases}
			E_R(k) & \frakp = \frakm\\
			0 & \text{otherwise},
		\end{cases}
	\end{equation*}
	whence the conclusion follows.
\end{proof}

It is also possible to characterize the depth of an ideal using the local cohomology modules: 
\begin{proposition}\thlabel{depth-in-terms-of-local-cohomology}
	Let $R$ be a Noetherian ring and $I\noreq R$ an ideal. If $M$ is a finite $R$-module such that $IM\ne M$, then 
	\begin{equation*}
		\depth(I, M) = \inf\left\{i\colon H^i_I(M)\ne 0\right\}.
	\end{equation*}
\end{proposition}
\begin{proof}
	We induct on $d = \depth(I, M)$. If $d = 0$, then $I\subseteq\frakp$ for some associated prime $\frakp$ of $M$. % TODO: Complete this later
\end{proof}

\subsection{The Mayer-Vietoris Sequences}

Let $M$ be an $R$-module and $I, J\noreq R$ be two ideals. It is easy to see that
\begin{equation*}
	0\to\Gamma_{I + J}(M)\xrightarrow{x\mapsto(x, x)}\Gamma_I(M)\oplus\Gamma_J(M)\xrightarrow{(x, y)\mapsto x - y}\Gamma_{I\cap J}(M)\to 0
\end{equation*}
is exact. 

\begin{theorem}[Mayer-Vietoris, Version 1]\thlabel{mayer-vietoris-version-1}
	Let $R$ be a Noetherian ring, $I, J\noreq R$ be ideals, and $M$ an $R$-module. Then there is a long exact sequence 
	\begin{equation*}
		0\to\Gamma_{I + J}(M)\to\Gamma_I(M)\oplus\Gamma_J(M)\to\Gamma_{I\cap J}(M)\to H^1_{I + J}(M)\to H^1_I(M)\oplus H^1_J(M)\to H^1_{I\cap J}(M)\to H^2_{I + J}(M)\to\cdots.
	\end{equation*}
\end{theorem}
\begin{proof}
	Let $0\to M\to E^\bullet$ be an $R$-injective resolution of $M$. In view of the above remark, there is a short exact sequence of complexes 
	\begin{equation*}
		0\to\Gamma_{I + J}\left(E^\bullet\right)\to\Gamma_I\left(E^\bullet\right)\oplus\Gamma_J\left(E^\bullet\right)\to\Gamma_{I\cap J}\left(E^\bullet\right)\to 0.
	\end{equation*}
	Taking cohomologies, the conclusion follows.
\end{proof}

\begin{theorem}[Mayer-Vietoris, Version 2]\thlabel{mayer-vietoris-version-2}
	Let $R$ be a Noetherian ring, $x\in R$, $I\noreq R$ an ideal, and $M$ an $R$-module. Then there is a long exact sequence 
	\begin{equation*}
		0\to\Gamma_{(I, x)}(M)\to\Gamma_I(M)\to\Gamma_{IR_x}(M_x)\to H^1_{(I, x)}(M)\to H^1_I(M)\to H^1_{IR_x}(M_x)\to H^2_{(I, x)}(M)\to\cdots.
	\end{equation*}
\end{theorem}
\begin{proof}
	% TODO: Add in later
\end{proof}

\subsection{Set-theoretic Complete Intersections}

\begin{lemma}
	Let $R$ be a Noetherian ring, $I\noreq R$ an ideal, and $M$ an $R$-module. If $I$ is generated by $n$ elements, then $H^j_I(M) = 0$ for $j > n$.
\end{lemma}
\begin{proof}
	We shall argue by induction on $n$. If $n = 0$, then $I = 0$, where it is clear that $H^j_I(M) = 0$ for $j > 0$. Suppose now that $n > 0$. Then there exists an ideal $J\subseteq I$ and $x\in I$ such that $J$ is generated by $n - 1$ elements and $I = (J, x)$. Using \thref{mayer-vietoris-version-2}, for $j > n$, we have an exact sequence
	\begin{equation*}
		H^{j - 1}_{JR_x}(M_x)\to H^j_I(M)\to H^j_J(M).
	\end{equation*}
	Since $j - 1 > n - 1$, $H^j_J(M) = 0$ and since $JR_x$ is generated by $n - 1$ elements, $H^{j - 1}_{JR_x}(M_x) = 0$. It follows that $H^j_I(M) = 0$ too.
\end{proof}

\begin{definition}
	Let $R$ be a Noetherian ring and $I\noreq R$ an ideal. We define the \define{arithmetic rank} of $I$ to be 
	\begin{equation*}
		\ara(I) = \min\left\{n\in\Z_{\ge 0}\colon\text{there exist }a_1,\dots,a_n\in R\text{ such that }\sqrt I = \sqrt{(a_1,\dots,a_n)}\right\}.
	\end{equation*}
\end{definition}

\subsection{Connectedness of the Punctured Spectrum}

\begin{theorem}[Hartshorne]
	Let $(R,\frakm, k)$ be a Noetherian local ring such that $\depth R\ge 2$. Then $\Spec^\circ (R) \coloneq\Spec(R)\setminus\{\frakm\}$ is connected in the Zariski topology.
\end{theorem}
\begin{proof}
	Suppose $\Spec^\circ (R)$ is not connected. Then there exist ideals $I$ and $J$ of $R$ such that
	\begin{equation*}
		\Spec^\circ(R) = \left(V(I)\setminus\{\frakm\}\right)\sqcup\left(V(J)\setminus\{\frakm\}\right)
	\end{equation*}
	and $V(I)\setminus\{\frakm\}$ is not empty, nor the entire $\Spec^\circ(\frakm)$. The latter condition is equivalent to the fact that $I$ and $J$ are neither $\frakm$-primary or nilpotent. Further, the first condition is equivalent to $\sqrt{I + J} = \frakm$ and $I\cap J$ being nilpotent.

	With this setup, using \thref{mayer-vietoris-version-1}, there is a long exact sequence 
	\begin{equation*}
		0\to\Gamma_{I + J}(R)\to\Gamma_I(R)\oplus\Gamma_J(R)\to\Gamma_{I\cap J}(R)\to H^1_{I + J}(R)\to H^1_I(R)\oplus H^1_J(R)\to H^1_{I\cap J}(R)\to\cdots.
	\end{equation*}
	Since $I\cap J$ is nilpotent, it follows that $\Gamma_{I\cap J}(R) = R$ and $H^j_{I\cap J}(R) = 0$ for $j > 0$. Since $\sqrt{I + J} = \frakm$, $H^j_{I + J}(R) = H^j_\frakm(R)$ for $j\ge 0$. But due to \thref{depth-in-terms-of-local-cohomology}, $H^j_\frakm(R) = 0$ for $j = 0, 1$. Hence, the above exact sequence gives 
	\begin{equation*}
		R\cong \Gamma_I(R)\oplus\Gamma_J(R),
	\end{equation*}
	But $R$ being a direct sum of $R$-modules is equivalent to $R$ being a product of rings, which is absurd since $R$ is local.
\end{proof}

\section{\v Cech Cohomology}

\begin{definition}
	Let $R$ be a Noetherian ring, and $\ul a = a_1,\dots,a_n\in R$. Let $\cech{C}^\bullet(a_i)$ denote the cochain complex:
	\begin{equation*}
		\cdots 0\to R\to R_{a_i}\to 0\to\cdots,
	\end{equation*}
	and define $\cech{C}^\bullet(\ul a)$ to be the cochain complex:
	\begin{equation*}
		\cech{C}^\bullet(a_1)\otimes\cdots\otimes \cech{C}^\bullet(a_n).
	\end{equation*}
	Further, if $M$ is an $R$-module, then define $\cech{C}^\bullet(\ul a, M)\coloneq \cech{C}^\bullet(\ul a)\otimes M$. This is known as the \define{\v{C}ech complex}. The cohomology modules of this cochain complex are known as the \define{\v{C}ech cohomology modules} and are denoted by $\cech{H}^i_{\ul a}(M)$.
\end{definition}

\begin{remark}
	Since the tensor product of chain complexes is associative and commutative, the order of tensoring above doesn't matter. 
\end{remark}

\begin{theorem}
	Let $R$ be a Noetherian ring, $I\noreq R$ an ideal, and $\ul a\coloneq a_1,\dots, a_n\in R$ such that $\sqrt{(a_1,\dots ,a_n)} = \sqrt I$. Then 
	\begin{equation*}
		\cech{H}^j_{\ul a}(M)\cong  H^j_I(M)
	\end{equation*}
	for every $R$-module $M$.
\end{theorem}
\begin{proof}
	
\end{proof}

\begin{proposition}
	If $R\to S$ is a flat morphism of Noetherian rings, $M$ an $R$-module, and $I$ an ideal in $R$, then 
	\begin{equation*}
		H^j_I(M)\otimes_R S\cong H^j_{IS}(M\otimes_R S)
	\end{equation*}
	as $S$-modules.
\end{proposition}

\begin{proposition}
	Let $R\to S$ be a homomorphism of Noetherian rings, $I\noreq R$ an ideal, and $M$ an $S$-module. Then 
	\begin{equation*}
		H^j_{I}(M)\cong H^j_{IS}(M)
	\end{equation*}
	as $S$-modules.
\end{proposition}

\begin{definition}
	Let $R$ be a Noetherian ring and $I\noreq R$ an ideal. Define the \define{cohomological dimension} of $I$ in $R$ to be 
	\begin{equation*}
		\cdim(I, R)\coloneq\inf\left\{i\colon H^j_I(M) = 0\text{ for all $R$-modules $M$ and all $j > i$}\right\}.
	\end{equation*}
\end{definition}

\begin{proposition}\thlabel{prop:better-defn-of-cohomological-dimension}
	\begin{equation*}
		\cdim(I, R) = \inf\left\{i\colon H^j_I(R) = 0 \text{ for all }j > i\right\}.
	\end{equation*}
\end{proposition}

\begin{corollary}
	Let $(R,\frakm, k)$ be a Noetherian local ring. Then $\cdim(I, R) = \cdim(\wh I,\wh R)$.
\end{corollary}
\begin{proof}
	This is immediate from \thref{prop:better-defn-of-cohomological-dimension} and the fact that $R\to\wh R$ is faithfully flat.
\end{proof}

%%%%%%%%%%%%%%%%%%%%%BIBLIOGRAPHY%%%%%%%%%%%%%%%%%%%%%%%%
\bibliographystyle{alpha}
\bibliography{references}
\end{document}
