\documentclass{article}
% \usepackage{./arxiv}

\title{Buildings}
\author{Swayam Chube}
\date{Last Updated: \today}

\usepackage[utf8]{inputenc} % allow utf-8 input
% \usepackage[T1]{fontenc}    % use 8-bit T1 fonts
% \usepackage{lmodern}
\usepackage{hyperref}       % hyperlinks
\usepackage{url}            % simple URL typesetting
\usepackage{booktabs}       % professional-quality tables
\usepackage{amsfonts}       % blackboard math symbols
\usepackage{nicefrac}       % compact symbols for 1/2, etc.
\usepackage{microtype}      % microtypography
\usepackage{graphicx}
\usepackage{natbib}
\usepackage{doi}
\usepackage{amssymb}
\usepackage{bbm}
\usepackage{amsthm}
\usepackage{amsmath}
\usepackage{xcolor}
\usepackage{theoremref}
\usepackage{enumitem}
% \usepackage{mathpazo}
% \usepackage{euler}
\usepackage{mathrsfs}
\setlength{\marginparwidth}{2cm}
\usepackage{todonotes}
\usepackage{stmaryrd}
\usepackage[all,cmtip]{xy} % For diagrams, praise the Freyd-Mitchell theorem 
\usepackage{marvosym}
\usepackage{geometry}
\usepackage{titlesec}
\usepackage{mathtools}
\usepackage{tikz}
\usetikzlibrary{cd}
\usepackage{sansmath}
\usepackage{sfmath}
\usepackage[scr=boondox, cal=esstix]{mathalpha}
\usepackage{mdframed}

\renewcommand{\qedsymbol}{$\blacksquare$}
\renewcommand{\familydefault}{\sfdefault}

% Uncomment to override  the `A preprint' in the header
% \renewcommand{\headeright}{}
% \renewcommand{\undertitle}{}
% \renewcommand{\shorttitle}{}

\hypersetup{
    pdfauthor={Lots of People},
    colorlinks=true,
}

\newtheoremstyle{thmstyle}%               % Name
  {}%                                     % Space above
  {}%                                     % Space below
  {}%                             % Body font
  {}%                                     % Indent amount
  {\bfseries}%                            % Theorem head font
  {.}%                                    % Punctuation after theorem head
  { }%                                    % Space after theorem head, ' ', or \newline
  {\thmname{#1}~2.\thmnumber{#2}\thmnote{ (#3)}}%                                     % Theorem head spec (can be left empty, meaning `normal')

\newtheoremstyle{defstyle}%               % Name
  {}%                                     % Space above
  {}%                                     % Space below
  {}%                                     % Body font
  {}%                                     % Indent amount
  {\bfseries}%                            % Theorem head font
  {.}%                                    % Punctuation after theorem head
  { }%                                    % Space after theorem head, ' ', or \newline
  {\thmname{#1}\thmnumber{ #2}\thmnote{ (#3)}}%                                     % Theorem head spec (can be left empty, meaning `normal')

\theoremstyle{thmstyle}
\newtheorem{theorem}{Theorem}[section]
\newtheorem{lemma}[theorem]{Lemma}
\newtheorem{proposition}[theorem]{Proposition}

\theoremstyle{defstyle}
\newtheorem{definition}[theorem]{Definition}
\newtheorem{corollary}[theorem]{Corollary}
\newtheorem{porism}[theorem]{Porism}
\newtheorem{remark}[theorem]{Remark}
\newtheorem{interlude}[theorem]{Interlude}
\newtheorem{example}[theorem]{Example}
\newtheorem*{notation}{Notation}
\newtheorem*{claim}{Claim}

% Common Algebraic Structures
\newcommand{\R}{\mathbb{R}}
\newcommand{\Q}{\mathbb{Q}}
\newcommand{\Z}{\mathbb{Z}}
\newcommand{\N}{\mathbb{N}}
\newcommand{\bbC}{\mathbb{C}} 
\newcommand{\K}{\mathbb{K}} % Base field which is either \R or \bbC
\newcommand{\calA}{\mathcal{A}} % Banach Algebras
\newcommand{\calB}{\mathcal{B}} % Banach Algebras
\newcommand{\calI}{\mathcal{I}} % ideal in a Banach algebra
\newcommand{\calJ}{\mathcal{J}} % ideal in a Banach algebra
\newcommand{\frakM}{\mathfrak{M}} % sigma-algebra
\newcommand{\calO}{\mathcal{O}} % Ring of integers
\newcommand{\bbA}{\mathbb{A}} % Adele (or ring thereof)
\newcommand{\bbI}{\mathbb{I}} % Idele (or group thereof)

% Categories
\newcommand{\catTopp}{\mathbf{Top}_*}
\newcommand{\catGrp}{\mathbf{Grp}}
\newcommand{\catTopGrp}{\mathbf{TopGrp}}
\newcommand{\catSet}{\mathbf{Set}}
\newcommand{\catTop}{\mathbf{Top}}
\newcommand{\catRing}{\mathbf{Ring}}
\newcommand{\catCRing}{\mathbf{CRing}} % comm. rings
\newcommand{\catMod}{\mathbf{Mod}}
\newcommand{\catMon}{\mathbf{Mon}}
\newcommand{\catMan}{\mathbf{Man}} % manifolds
\newcommand{\catDiff}{\mathbf{Diff}} % smooth manifolds
\newcommand{\catAlg}{\mathbf{Alg}}
\newcommand{\catRep}{\mathbf{Rep}} % representations 
\newcommand{\catVec}{\mathbf{Vec}}

% Group and Representation Theory
\newcommand{\chr}{\operatorname{char}}
\newcommand{\Aut}{\operatorname{Aut}}
\newcommand{\GL}{\operatorname{GL}}
\newcommand{\im}{\operatorname{im}}
\newcommand{\tr}{\operatorname{tr}}
\newcommand{\id}{\mathbf{id}}
\newcommand{\cl}{\mathbf{cl}}
\newcommand{\Gal}{\operatorname{Gal}}
\newcommand{\Tr}{\operatorname{Tr}}
\newcommand{\sgn}{\operatorname{sgn}}
\newcommand{\Sym}{\operatorname{Sym}}
\newcommand{\Alt}{\operatorname{Alt}}

% Commutative and Homological Algebra
\newcommand{\spec}{\operatorname{spec}}
\newcommand{\mspec}{\operatorname{m-spec}}
\newcommand{\Spec}{\operatorname{Spec}}
\newcommand{\MaxSpec}{\operatorname{MaxSpec}}
\newcommand{\Tor}{\operatorname{Tor}}
\newcommand{\tor}{\operatorname{tor}}
\newcommand{\Ann}{\operatorname{Ann}}
\newcommand{\Supp}{\operatorname{Supp}}
\newcommand{\Hom}{\operatorname{Hom}}
\newcommand{\End}{\operatorname{End}}
\newcommand{\coker}{\operatorname{coker}}
\newcommand{\limit}{\varprojlim}
\newcommand{\colimit}{%
  \mathop{\mathpalette\colimit@{\rightarrowfill@\textstyle}}\nmlimits@
}
\makeatother


\newcommand{\fraka}{\mathfrak{a}} % ideal
\newcommand{\frakb}{\mathfrak{b}} % ideal
\newcommand{\frakc}{\mathfrak{c}} % ideal
\newcommand{\frakf}{\mathfrak{f}} % face map
\newcommand{\frakg}{\mathfrak{g}}
\newcommand{\frakh}{\mathfrak{h}}
\newcommand{\frakm}{\mathfrak{m}} % maximal ideal
\newcommand{\frakn}{\mathfrak{n}} % naximal ideal
\newcommand{\frakp}{\mathfrak{p}} % prime ideal
\newcommand{\frakq}{\mathfrak{q}} % qrime ideal
\newcommand{\fraks}{\mathfrak{s}}
\newcommand{\frakt}{\mathfrak{t}}
\newcommand{\frakz}{\mathfrak{z}}
\newcommand{\frakA}{\mathfrak{A}}
\newcommand{\frakI}{\mathfrak{I}}
\newcommand{\frakJ}{\mathfrak{J}}
\newcommand{\frakK}{\mathfrak{K}}
\newcommand{\frakL}{\mathfrak{L}}
\newcommand{\frakN}{\mathfrak{N}} % nilradical 
\newcommand{\frakO}{\mathfrak{O}} % dedekind domain
\newcommand{\frakP}{\mathfrak{P}} % Prime ideal above
\newcommand{\frakQ}{\mathfrak{Q}} % Qrime ideal above 
\newcommand{\frakR}{\mathfrak{R}} % jacobson radical
\newcommand{\frakU}{\mathfrak{U}}
\newcommand{\frakV}{\mathfrak{V}}
\newcommand{\frakW}{\mathfrak{W}}
\newcommand{\frakX}{\mathfrak{X}}

% General/Differential/Algebraic Topology 
\newcommand{\scrA}{\mathscr{A}}
\newcommand{\scrB}{\mathscr{B}}
\newcommand{\scrb}{\mathscr{b}}
\newcommand{\scrF}{\mathscr{F}}
\newcommand{\scrI}{\mathscr{I}}
\newcommand{\scrM}{\mathscr{M}}
\newcommand{\scrN}{\mathscr{N}}
\newcommand{\scrP}{\mathscr{P}}
\newcommand{\scrO}{\mathscr{O}} % sheaf
\newcommand{\scrR}{\mathscr{R}}
\newcommand{\scrS}{\mathscr{S}}
\newcommand{\bbH}{\mathbb H}
\newcommand{\Int}{\operatorname{Int}}
\newcommand{\psimeq}{\simeq_p}
\newcommand{\wt}[1]{\widetilde{#1}}
\newcommand{\RP}{\mathbb{R}\text{P}}
\newcommand{\CP}{\mathbb{C}\text{P}}

% Miscellaneous
\newcommand{\wh}[1]{\widehat{#1}}
\newcommand{\calM}{\mathcal{M}}
\newcommand{\calP}{\mathcal{P}}
\newcommand{\onto}{\twoheadrightarrow}
\newcommand{\into}{\hookrightarrow}
\newcommand{\Gr}{\operatorname{Gr}}
\newcommand{\Span}{\operatorname{Span}}
\newcommand{\ev}{\operatorname{ev}}
\newcommand{\weakto}{\stackrel{w}{\longrightarrow}}

\newcommand{\define}[1]{\textcolor{blue}{\textit{#1}}}
\newcommand{\caution}[1]{\textcolor{red}{\textit{#1}}}
\newcommand{\important}[1]{\textcolor{red}{\textit{#1}}}
\renewcommand{\mod}{~\mathrm{mod}~}
\renewcommand{\le}{\leqslant}
\renewcommand{\leq}{\leqslant}
\renewcommand{\ge}{\geqslant}
\renewcommand{\geq}{\geqslant}
\newcommand{\Res}{\operatorname{Res}}
\newcommand{\floor}[1]{\left\lfloor #1\right\rfloor}
\newcommand{\ceil}[1]{\left\lceil #1\right\rceil}
\newcommand{\gl}{\mathfrak{gl}}
\newcommand{\ad}{\operatorname{ad}}
\newcommand{\Stab}{\operatorname{Stab}}
\newcommand{\bfX}{\mathbf{X}}
\newcommand{\Ind}{\operatorname{Ind}}
\newcommand{\bfG}{\mathbf{G}}
\newcommand{\rank}{\operatorname{rank}}
\newcommand{\calo}{\mathcal{o}}
\newcommand{\frako}{\mathfrak{o}}
\newcommand{\Cl}{\operatorname{Cl}}

\newcommand{\idim}{\operatorname{idim}}
\newcommand{\pdim}{\operatorname{pdim}}
\newcommand{\Ext}{\operatorname{Ext}}
\newcommand{\co}{\operatorname{co}}
\newcommand{\bfO}{\mathbf{O}}
\newcommand{\bfF}{\mathbf{F}} % Fitting Subgroup
\newcommand{\Syl}{\operatorname{Syl}}
\newcommand{\nor}{\vartriangleleft}
\newcommand{\noreq}{\trianglelefteqslant}
\newcommand{\subnor}{\nor\!\nor}
\newcommand{\Soc}{\operatorname{Soc}}
\newcommand{\core}{\operatorname{core}}
\newcommand{\Sd}{\operatorname{Sd}}
\newcommand{\mesh}{\operatorname{mesh}}
\newcommand{\sminus}{\setminus}
\newcommand{\diam}{\operatorname{diam}}
\newcommand{\Ass}{\operatorname{Ass}}
\newcommand{\projdim}{\operatorname{proj~dim}}
\newcommand{\injdim}{\operatorname{inj~dim}}
\newcommand{\gldim}{\operatorname{gl~dim}}
\newcommand{\embdim}{\operatorname{emb~dim}}
\newcommand{\hght}{\operatorname{ht}}
\newcommand{\depth}{\operatorname{depth}}
\newcommand{\ul}[1]{\underline{#1}}
\newcommand{\type}{\operatorname{type}}



\geometry {
    margin = 0.8in
}

\titleformat
{\section}
[block]
{\Large\bfseries\sffamily}
{\S 2.\thesection}
{0.5em}
{\centering}
[]


\titleformat
{\subsection}
[block]
{\normalfont\bfseries\sffamily}
{\S\S}
{0.5em}
{\centering}
[]

\begin{document}
\maketitle

\setcounter{section}{3}
\section{Buildings}

Let $(G, B, N, R)$ be a Tits system with $H = B\cap N$. Suppose there is a reduced and irreducible root system $\Sigma_0$ on a Euclidean space $A$, a chamber $C$ of the associated root system $\Sigma$, and a surjective homomorphism $\nu: N\onto W$ such that 
\begin{enumerate}[label=(\roman*)]
    \item $\ker\nu = H$, so that we may identify the Weyl group $N/H$ of the Tits system with the affine Weyl group $W$ of $\Sigma$. We shall implicitly make this identification henceforth.
    \item under this identification, the distinguished generators of $N/H$ are the reflections in the walls of the chamber $C$, i.e., 
    \begin{equation*}
        R = \left\{w_\alpha\colon\alpha\in\Pi\right\}.
    \end{equation*}
\end{enumerate}
Following the notation of \cite{macdonald-spherical-functions}, the conjugates of $B$ in $G$ are called the \define{Iwahori subgroups} of $G$ and a \define{parahoric} subgroup of $G$ is a \emph{proper} subgroup containing an Iwahori subgroup. We have seen last time that that every Iwahori subgroup of $G$ is conjugate to a unique $P_S \coloneq BW_SW$, where $S\subseteq R$. In particular, each parahoric subgroup of $G$ uniquely determines a subset $S$ of $R$.

This sets up a bijective correspondence $S\longleftrightarrow F$ between the subsets $S$ of $R$ and the facets $F$ of the chamber $C$: to a facet $F$ corresponds the set of all $w_\alpha\in R$ which fix $F$. Under this correspondence, $\emptyset\longleftrightarrow C$, and $R\longleftrightarrow\emptyset$. If $S\longleftrightarrow F$, then we write $P_F$ for $P_S$. Clearly, each parahoric subgroup $P$ uniquely determines a facet $F(P)$ of $C$: namely $F(P) = F$ if and only if $P$ is conjugate to $P_F$.

The \define{building} associated with the Tits system structure on $G$ is the set 
\begin{equation*}
    \scrI = \left\{(P, x)\colon x\in F(P)\right\}.
\end{equation*}
With each parahoric subgroup $P$ associate 
\begin{equation*}
    \scrF(P) = \left\{(P, x)\colon x\in F(P)\right\}\subseteq\scrI.
\end{equation*}
The set $\scrF(P)$ is called a \define{facet} of $\scrI$ of \define{type} $F(P)$. In particular, if $P$ is an Iwahori subgroup, $\scrF(P)$ is called a \define{chamber} of $\scrI$. We define the \define{closure of a facet} as 
\begin{equation*}
    \overline{\scrF(P)} = \bigcup_{\substack{Q\supseteq P\\ Q\le G}} \scrF(Q).
\end{equation*}
The group $G$ acts on $\scrI$ as 
\begin{equation*}
    g\cdot(P, x) = \left(gPg^{-1}, x\right).
\end{equation*}

\subsection{Apartments}

Set 
\begin{equation*}
    \scrA_0 \coloneq \bigcup_{w\in W}\overline{\scrF(wBw^{-1})}\subseteq\scrI.
\end{equation*}
Since $\overline{\scrF(w Bw^{-1})}$ is the union of $\scrF(P)$'s for all parahorics containing $wBw^{-1}$ and conjugation by $w$ is the same as conjugation by some $n\in N$ for which $\nu(n) = w$, it follows that 
\begin{equation*}
    \scrA_0 = \bigcup_{n\in N}n\scrF(P).
\end{equation*}

\begin{proposition}
    There exists a \emph{unique} bijection $j: A\to\scrA_0$ such that 
    \begin{enumerate}[label=(\arabic*)]
        \item for each facet $F$ of $C$ and each $x\in F$, 
        \begin{equation*}
            j(x) = (P_F, x),
        \end{equation*}
        \item $j\circ w = w\circ j$ for all $w\in W$.
    \end{enumerate}
\end{proposition}
\begin{proof}
    Let $y\in A$. Then there is a unique $x\in\overline C$ such that there exists a $w\in W$ such that $y = wx$. Let $F$ be the facet of $C$ containing $x$. Define $j(y) = (wP_Fw^{-1}, x)\in\scrA_0$. We must check that $j$ is well-defined. Suppose $w'\in W$ is such that $y = w'x$. Then $w^{-1}w'$ fixes $x$ and hence, belongs to the subgroup of $W$ generated by $\{w_\alpha\colon\alpha\in\Pi,~w_\alpha\text{ fixes }x\}$ (\cite[last line on pg. 16]{macdonald-spherical-functions}). That is, $w^{-1}w'\in W_S$, where $S\longleftrightarrow F$. In particular, $w^{-1}w'\in P_F = P_S$, therefore, $wP_Fw^{-1} = w'P_Fw'^{-1}$. Hence, $j$ is well-defined and clearly satisfies (1). As for (2), let $w''\in W$ and $y\in Y$ as before. Then $w''y$ is conjugate to $x$ under $w''w$, therefore, $j(w''y) = \left(w''wP_Fw^{-1}w''^{-1}, x\right) = w''\left(P_F, x\right) = w''j(y)$. The uniqueness is clear since the conjugates of $\overline C$ cover $A$.
\end{proof} 

\begin{lemma}\thlabel{lem:bijection-lies-in-W}
    If $g\scrA_0 = \scrA_0$, then $j^{-1}\circ\left(g|_{\scrA_0}\right)\circ j\in W$.
\end{lemma}
\begin{proof}
    Let $\scrb_0\subseteq\scrA_0$ denote the chamber $\scrF(B) = j(C)$ of $\scrI$. Note that $g\scrb_0$ is another chamber of $\scrI$ and is contained in $\displaystyle\scrA_0 = \bigcup_{n\in N}\bigcup_{P\supseteq B} n\scrF(P)$, therefore there exists $n_0\in N$ such that $g\scrb_0 = n_0\scrb_0$. Hence, $g_0 = n_0^{-1}g$ normalizes $B$, and hence, lies in $B$ as we have seen last time. Notice that $g_0\scrA_0 = n_0^{-1}g\scrA_0 = n_0^{-1}\scrA_0 = \scrA_0$, since $\nu(n_0)\in W$. It is also clear that $g_0$ fixes $\scrb_0$ and each of its facets. It is clear that the map $j^{-1}\circ\left(g|_{\scrA_0}\right)\circ j$ is a bijection from $A$ to $A$ which fixes the chamber $C$ and each of its facets. Now, since $w\in W$, and $j$ commutes with the action fo the affine Weyl group on $A$, we have 
    \begin{equation*}
        \left(j^{-1}\circ\left(g_0|_{\scrA_0}\right)\circ j\right)(wx) = w \left(j^{-1}\circ\left(g_0|_{\scrA_0}\right)\circ j\right)(x) = wx.
    \end{equation*}
    In particular, $j^{-1}\circ\left(g_0|_{\scrA_0}\right)\circ j$ is the identity map. Hence, 
    \begin{equation*}
        j^{-1}\circ\left(g|_{\scrA_0}\right)\circ j = j^{-1}\circ\left(n_0|_{\scrA_0}\right)\circ j = \nu(n_0)\in W, 
    \end{equation*}
    as desired.
\end{proof}

\begin{mdframed}
The subsets $g\scrA_0$ of $\scrI$ for $g\in G$ are called the \define{apartments} of the building $\scrI$. If $\scrA = g\scrA_0$ is an apartment, transport the Euclidean structure of $A$ onto $\scrA$ via the bijection $\left(g|_{\scrA_0}\right)\circ j: A\to\scrA$. We must check that this structure is well-defined. Indeed, if $\scrA = g'\scrA_0$, then 
\begin{equation*}
    \left[\left(g'|_{\scrA_0}\right)\circ j\right]^{-1}\circ\left[\left(g|_{\scrA_0}\circ j\right)\right] = j^{-1}\circ\left(g'^{-1}g|_{\scrA_0}\right)\circ j 
\end{equation*}
is an element of the affine Weyl group, in particular, it is an affine transformation that preserves lengths. Therefore, there is a well-defined Euclidean structure on $\scrA$.
\end{mdframed}

\begin{lemma}\thlabel{lem:facets-contained-in-apartment}
    Any two facets of $\scrI$ are contained in a single apartment.
\end{lemma}
\begin{proof}
    Consider two facets $\scrF(P_1)$ and $\scrF(P_2)$ where $P_1, P_2$ are parahoric subgroups of $G$, say $P_i - g_iP_{F_i}g_i^{-1}$ for $i\in\{1,2\}$, where $F_1, F_2$ are facets of the chamber $C$ in $A$. Since $G = BWB$, we can write $g_1^{-1}g_2 = b_1nb_2$ for some $b_1,b_2\in B$ and $n\in N$. Setting $g= g_1b_1$, then 
    \begin{equation*}
        P_1 = gP_{F_1}g^{-1}\quad\text{and}\quad P_2 = g\left(nP_{F_2}n^{-1}\right)g^{-1},
    \end{equation*}
    whence $\scrF(P_1)$ and $\scrF(P_2)$ are both contained in $g\scrA_0$.
\end{proof}

\begin{lemma}\thlabel{lem:G-acts-transitively}
    $G$ acts transitively on the set 
    \begin{equation*}
        \left\{(\scrb, \scrA)\colon\scrb\text{ is a chamber in }\scrA\right\}.
    \end{equation*}
\end{lemma}
\begin{proof}
    Since $\scrb = g\scrb_0$ where $\scrb_0 = \scrF(B)$ for some $g\in G$, we may suppose without loss of generality that $\scrb = \scrb_0$. If $\scrA = g\scrA_0$ contains $\scrb_0$, then $g^{-1}\scrb_0 = n\scrb_0$ for some $n\in N$. Setting $g_1 = gn$, we see that $A = g_1\scrA_0$ and $\scrb_0 = g_1\scrb_0$.
\end{proof}

\begin{proposition}\thlabel{prop:bijection-preserving-stuff}
    Let $\scrA$, $\scrA'$ be two apartments and let $\scrb$ be a chamber contained in $\scrA\cap\scrA'$. Then there exists a unique bijection $\rho:\scrA'\to\scrA$ such that 
    \begin{enumerate}[label=(\arabic*)]
        \item There exists $g\in G$ such that $\rho x = gx$ for all $x\in\scrA'$, and 
        \item $\rho x = x$ for all $x\in\scrb$.
    \end{enumerate}
    Moreover, $\rho x = x$ for all $x\in\scrA\cap\scrA'$, and $d_{\scrA'}(x, y) = d_{\scrA}(\rho x, \rho y)$ for all $x,y\in\scrA'$.
\end{proposition}
\begin{proof}
    Due to \thref{lem:G-acts-transitively}, there exists $g\in G$ which sends the pair $(\scrb, \scrA')$ to the pair $(\scrb, \scrA)$. Note that $g\scrb = \scrb$ and $\scrb = \scrF(B')$ for some Iwahori subgroup $B'$ of $G$. This means that $g$ normalizes $B'$, and hence, $g\in B'$. Thus, this map fixes every element of $\scrb$, and hence, satisfies the desired conditions.

    Next, we argue uniqueness. If $\rho_1, \rho_2:\scrA'\to\scrA$ are two such maps, then $\rho_1\circ\rho_2^{-1}$ is a bijection from $\scrA$ to $\scrA$ which fixes $\scrb$. There exists $h\in G$ such that $h$ maps $(\scrb_0,\scrA_0)$ to the pair $(\scrb,\scrA)$. Therefore, $h^{-1}gh\scrA_0 = \scrA_0$ and fixes $\scrb_0$. Due to \thref{lem:bijection-lies-in-W}, it follows that $h^{-1}gh$ is the identity on $\scrA_0$, whence $g$ is the identity on $\scrA$. The assertion $d_{\scrA}(\rho x ,\rho y) = d_{\scrA'}(x, y)$ is clear from the definition of the metric.

    It remains to show that $\rho x = x$ for all $x\in\scrA\cap\scrA'$. Due to \thref{lem:G-acts-transitively}, we may assume $\scrA' = \scrA_0$, $\scrA = g\scrA_0$ and $\scrb = \scrb_0 = \scrF(B)$. Since $g\scrb_0 = \scrb_0$, it follows that $b\in B$ as before. Now let $\scrF = \scrF(P)$ be a facet contained in $\scrA\cap\scrA'$. 

    Since $\scrF(P)\subseteq\scrA_0\cap g\scrA_0$, we have 
    \begin{equation*}
        P = n_1Pn_1^{-1} = g\left(n_2P_Fn_2^{-1}\right)g^{-1}
    \end{equation*}
    for some facet $F$ of $C$ and $n_1, n_2\in N$. The above equality implies $n_1^{-1}gn_2$ normalizes $P_F$ and hence lies in $P_F$, therefore, $Bn_1P_F = Bn_2P_F$. But due to \cite[2.3.5]{macdonald-spherical-functions},
    \begin{equation*}
        Bn_1P_F = Bn_1W_FB = Bn_2W_FB = Bn_2P_F,
    \end{equation*}
    where $W_F$ is the subgroup of $W$ fixing $F$. Recall again (\cite[2.3.1]{macdonald-spherical-functions}) that there is a bijection between $N/H$ and $B\backslash G/B$. Hence $n_1W_F = n_2W_F$, in other words, $n_1P_Fn_1^{-1} = n_2P_Fn_2^{-1}$, consequently, $\scrF(P) = g\scrF(P) = \rho\scrF$, as desired.
\end{proof}

\subsection{Retraction of the building onto an apartment}
\begin{theorem}\thlabel{thm:retraction-of-building} 
    Let $\scrA$ be an apartment and $\scrb$ a chamber in $\scrA$. Then there exists a unique mapping $\rho:\scrI\to\scrA$ such that for all apartments $\scrA'$ containing $\scrb$, $\rho|_{\scrA'}$ is the bijection $\scrA'\to\scrA$ of \thref{prop:bijection-preserving-stuff}.
\end{theorem}
\begin{proof}
    Let $x\in\scrI$. By \thref{lem:facets-contained-in-apartment}, there exists an apartment $\scrA_1$ containing $x$ and $\scrb$. Let $\rho_1: \scrA_1\to\scrA$ be the bijection of \thref{prop:bijection-preserving-stuff} and define $\rho(x)\coloneq\rho_1(x)$. We must show that this map is well-defined first. Indeed, suppose $\scrA_2$ is another apartment of $\scrI$ containing $x$ and $\scrb$ and $\rho_2: \scrA_2\to\scrA$ be the bijectio nof \thref{prop:bijection-preserving-stuff}, then $\rho_1^{-1}\circ\rho_2: \scrA_2\to\scrA_1$ is again the bijection of \thref{prop:bijection-preserving-stuff} for the apartments $\scrA_2$, $\scrA_1$, and the chamber $\scrb$. Thus, $\rho_1^{-1}\circ\rho_2$ fixes $x\in\scrA_1\cap\scrA_2$, i.e., $\rho_1(x) = \rho_2(x)$. This shows the existence of a desired retraction. 

    To see uniqueness, again use the fact that for any $x\in\scrI$, there exists an apartment containing $x$ and $\scrb$. This completes the proof.
\end{proof}

The mapping $\rho$ defined above is called the \define{retraction of $\scrI$ onto $\scrA$ with centre $\scrb$}.

\begin{proposition}
Let $\rho$ be the retraction of \thref{thm:retraction-of-building}. Then 
\begin{enumerate}[label=(\arabic*)]
    \item $\rho x = x$ for all $x\in\scrA$. 
    \item For each facet $\scrF$ in $\scrI$, $\rho|_{\overline{\scrF}}$ is a surjective affine isometry of $\overline{\scrF} \onto \overline{\rho\scrF}$.
    \item If $x\in\overline\scrb$, then $\rho^{-1}(x) = \{x\}$.
\end{enumerate}
\end{proposition}
\begin{proof}
\begin{enumerate}[label=(\arabic*)]
    \item According to \thref{thm:retraction-of-building}, $\rho|_{\scrA}$ is the unique bijection of \thref{prop:bijection-preserving-stuff}, which is just the identity map, and hence $\rho x = x$ for all $x\in\scrA$.
    \item Let $\scrA'$ be an apartment containing $\scrF$ and $\scrb$, which exists due to \thref{lem:facets-contained-in-apartment}. Note that $\overline\scrF\subseteq\scrA'$. Since $\rho:\scrA'\to\scrA$ is an isometry due to \thref{prop:bijection-preserving-stuff}, the assertion follows.
    \item Let $\scrF'$ be a facet of $\scrI$ mapping to $\scrF$ under $\rho$. Note that $\rho:\scrA'\to\scrA$ is multiplication by some $g\in G$ which leaves $\scrb$ fixed, therefore, must leave all its facets fixed too, after all the facets are those corresponding to the parahorics containing the Iwahori corresponding to $\scrb$. \qedhere
\end{enumerate}
\end{proof}

\begin{proposition}
\begin{enumerate}[label=(\arabic*)]
    \item There exists a unique function $d:\scrI\times\scrI\to\R_{+}$ such that $d|_{\scrA\times\scrA}$ is the metric $d_{\scrA}$ for each apartment $\scrA$ of $\scrI$.
    \item If $\rho$ is a retraction of $\scrI$ onto an apartment $\scrA$ as in \thref{thm:retraction-of-building}, then $d(\rho(x), \rho(y))\le d(x, y)$ for all $x,y\in\scrI$.
    \item $d$ is a $G$-invariant metric on $\scrI$.
\end{enumerate}
\end{proposition}
\begin{proof}
\begin{enumerate}[label=(\arabic*)]
\item Let $x,y\in\scrI$, then due to \thref{lem:facets-contained-in-apartment}, there is an apartment $\scrA$ containing $x$ and $y$. We define $d(x, y)\coloneq d_{\scrA}(x, y)$. Suppose $\scrA'$ is another apartment containing $x$ and $y$. We must show that $d_{\scrA}(x, y) = d_{\scrA'}(x, y)$.

Let $\scrb$ be a chamber in $\scrA$ such that $x\in\overline\scrb$, this can be done, since every facet corresponds to a parahoric, which contains an Iwahori. Similarly, let $\scrb'$ be a chamber in $\scrA'$ such that $y\in\overline\scrb'$. Again by \thref{lem:facets-contained-in-apartment}, there is an apartment $\scrA''$ containing $\scrb$ and $\scrb'$. From \thref{prop:bijection-preserving-stuff}, we have that $d_{\scrA}(x, y) = d_{\scrA''}(x, y)$ because $\scrA$ and $\scrA''$ share the chamber $\scrb$. Analogously, $d_{\scrA'}(x, y) = d_{\scrA''}(x, y)$. Thus, the distance $d$ is well-defined. That it is $G$-invariant follows from the definition of $d_{\scrA}$ as $\left(g|_{\scrA_0}\right)\circ j: A\to\scrA$.

\item This is cumbersome to write out formally but here's the main idea: Choose an apartment $\scrA'$ in $\scrI$ containing $x$ and $y$. This apartment is in bijection with $A$, through which its metric is defined. The affine line joining $x$ to $y$ in $A$ will intersect finitely many facets in the tessellation of $A$. Thus, this line segment can be broken into a union of smaller closed line segments, each lying in the closure of a facet. Under $\rho$, the image of each such line segment is a line segment of the same length. In particular, the image of $[xy]$ under $\rho$ is a polygonal line, whose ``total length'' is $d_{\scrA'}(x, y)$. The triangle inequality implies the desired conclusion.

\item Let $x, y, z\in\scrI$ and let $\scrA$ be an apartment containing $x$ and $y$. Let $\rho$ be a retraction of $\scrI$ onto $\scrA$ as in \thref{thm:retraction-of-building}. Then keeping in mind that $\rho(x) = x$ and $\rho(y) = y$, we have 
\begin{equation*}
    d(x, y) = d_{\scrA}(\rho(x), \rho(y))\le d(\rho(x), \rho(z)) + d(\rho(z), \rho(y))\le d(x, z) + d(z, y),
\end{equation*}
where the last equality follows from (2). \qedhere
\end{enumerate}
\end{proof}

% \begin{proposition}
%     Let $x, y\in\scrI$. Then there is a unique geodesic joining $x$ to $y$.
% \end{proposition}
% \begin{proof}
%     % TODO: Add in this proof. I have no idea what a geodesic is.
% \end{proof}

\begin{proposition}
    $\scrI$ is complete with respect to the metric $d$.
\end{proposition}
\begin{proof}
    Let $(x_n)_{n\ge 1}$ be a Cauchy sequence in $\scrI$ with respect to the metric $d$. Let $\rho$ be a retraction of $\scrI$ onto an apartment $\scrA_0$ as in \thref{thm:retraction-of-building}. Then $\left(\rho x_n\right)_{n\ge 1}$ is a Cauchy sequence in $\scrA_0$, and as such, converges to some $x\in\scrA_0$. Let $x = (P, a)\in\scrA_0$ where $a\in A$. Then there is a $\mu > 0$ such that $d(x, wx)\ge\mu$ for all $w\in W$, the affine Weyl group. Let $g\in G$ be such that $x\ne gx$. We claim that $d(x, gx)\ge\mu$. Indeed, there is an apartment $\scrA' = h\scrA_0$ containing both $x$ and $gx$ for some $h\in G$. Then, from the $G$-invariance of $d$, 
    \begin{equation*}
        d(x, gx) = d(h^{-1}x, h^{-1}gx)\ge\mu,
    \end{equation*}
    which is clear from the bijection $A\leftrightarrow\scrA_0$. Again, since $d$ is $G$-invariant, it follows that $d(gx, g'x)\ge\mu$ for all $g,g'\in G$ such that $gx\ne g'x$.

    Now, let $N > 0$ be a positive integer such that for all $m,n\ge N$, 
    \begin{equation*}
        d(\rho x_n, x) < \frac{1}{3}\mu\quad\text{ and }\quad d(x_m, x_n) < \frac{1}{3}\mu.
    \end{equation*}
    By definition, each $\rho x_n$ is of the form $g_nx_n$ for some $g_n\in G$. Set $y_n = g_n^{-1}x$. Then for $n\ge N$, using the $G$-invariance of $d$, we have 
    \begin{align*}
        d(y_n, y_{n + 1}) &\le d(y_n, x_n) + d(x_n, x_{n + 1}) + d(x_{n + 1}, y_{n + 1})\\
        &= d(x, \rho x_n) + d(x_n, x_{n + 1}) + d(x_{n + 1}, y_{n + 1})\\
        &< \frac{1}{3}\mu + \frac{1}{3}\mu + \frac{1}{3}\mu < \mu.
    \end{align*}
    Hence, $y_N = y_{N + 1} = \cdots \eqcolon y$. Finally, for $n\ge N$, we have 
    \begin{equation*}
        d(x_n, y) = d(x_n, y_n) = d(g_nx_n, g_ny_n) = d(\rho x_n, x)\to 0,
    \end{equation*}
    as $n\to\infty$.
\end{proof}

\subsection*{Fixed point theorem}

A subset $X\subseteq\scrI$ is said to be \define{convex} if whenever $x, y\in X$, $[xy]\subseteq X$.

\begin{lemma}\thlabel{lem:apollonius}
    Let $x, y, z\in\scrI$ and let $m$ be the midpoint of $[xy]$ Then 
    \begin{equation*}
        d(z, x)^2 + d(z, y)^2\ge 2d(z, m)^2 + \frac{1}{2}d(x, y)^2.
    \end{equation*}
\end{lemma}
\begin{proof}
    If $x, y, z$ lie in the same apartment, then upon moving to the Euclidean space $A$, this is just a restatement of the well-known Apollonius' theorem. In the general case, let $\scrA$ be an apartment containing $x$ and $y$ and choose a chamber $\scrb$ in $\scrA$ such that $m\in\overline\scrb$. Let $\rho: \scrI\to\scrA$ be the retraction with centre $\scrb$ as in \thref{thm:retraction-of-building}. Note that due to \thref{lem:facets-contained-in-apartment}, we can choose an apartment $\scrA'$ containing $\scrb$ and $z$. Then, using \thref{prop:bijection-preserving-stuff}, it is clear that $d(\rho(z), m) = d(z, m)$. Hence, we have 
    \begin{align*}
        d(z, x)^2 + d(z, y)^2 &\ge d(\rho(z), x)^2 + d(\rho(z), y)^2\\
        &= 2d(\rho(z), m)^2 + \frac{1}{2}d(x, y)^2\\
        &= 2d(z, m)^2 + \frac{1}{2}d(x, y)^2,
    \end{align*}
    as desired.
\end{proof}

\begin{theorem}\thlabel{thm:fixed-point-theorem}
    Let $X$ be a bounded non-empty subset of $\scrI$. Then the group of \textcolor{red}{(affine?)} isometries $\gamma$ of $\scrI$ such that $\gamma(X)\subseteq X$ has a fixed point in the closure of the convex hull of $X$.
\end{theorem}
\begin{proof}
    Let $\delta(X)$ denote the diameter of the set $X$. Fix a real number $k\in (0, 1)$ and let 
    \begin{equation*}
    fX\coloneq\left\{m\in\scrI\colon m\text{ is the midpoint of }[xy]\text{ with }x,y\in X\text{ and }d(x, y)\ge k\delta(X)\right\}.
    \end{equation*}
    If $m\in fX$ and $z\in X$, then $m$ is the midpoint of $[xy]$ for some $x,y\in X$ with $d(x, y)\ge k\delta(X)$. Using \thref{lem:apollonius}, 
    \begin{equation*}
    d(z, m)^2\le\frac{1}{2}d(z, x)^2 + d(z, y)^2 - \frac{1}{4}d(x, y)^2\le\underbrace{\left(1 - \frac{1}{4}k^2\right)}_{k_1}\delta(X)^2.
    \end{equation*}
    Next, if $m, z\in fX$, then $m$ is the midpoint of $[xy]$ for some $x,y\in X$ with $d(x, y)\ge k\delta(X)$.  Again, using \thref{lem:apollonius}, 
    \begin{equation*}
        d(z, m)^2\le\frac{1}{2}d(z, x)^2 + \frac{1}{2}d(z, y)^2 - \frac{1}{4}d(x, y)^2\le\underbrace{\left(1 - \frac{1}{2}k^2\right)}_{k_2}\delta(X)^2.
    \end{equation*}
    That is, $\delta(fX)\le k_2\delta(X)$. Hence, $\delta(f^nX)\to 0$ as $n\to\infty$. For each positive integer $n$, pick $x_n\in f^nX$. Then, it is clear that 
    \begin{equation*}
        d(x_n, x_{n + 1})\le k_2\delta\left(f^n X\right)\le k_1k_2^n\delta(X).
    \end{equation*}
    Hence $(x_n)_{n\ge 1}$ is a Cauchy sequence in $\scrI$, so that it converges to some $x\in\scrI$. Clearly $x$ lies in the closure of the convex hull of $X$, since each $f^nX$ is contained in the convex hull of $X$. We claim that $x$ is the desired fixed point.

    Finally, let $\gamma$ be an isometry of $\scrI$ such that $\gamma X\subseteq X$. Then $\gamma f^nX\subseteq f^nX$. Let $x_n' = \gamma(x_n)$. Then $(x_n')_{n\ge 1}$ is a Cauchy sequence with $x_n'\in f^nX$ for all $n\ge 1$ and converges to $\gamma x$. But since $\delta(f^n X)\to 0$, it follows that $d(x_n, x_n')\to 0$ as $n\to\infty$. Hence $\gamma x = x$, as desired.
\end{proof}

A subset $M\subseteq G$ is said to be \define{bounded} if $MX$ is bounded for all bounded subsets $X\subseteq\scrI$.

\begin{lemma}\thlabel{lem:bounded-iff-finite-double-cosets}
    $M$ is bounded if and only if $M$ (non-trivially) intersects only finitely any double cosets in $B\backslash G/B$.
\end{lemma}
\begin{proof}
    Let $\rho$ be the retraction of $\scrI$ onto the apartment $\scrA_0$ with centre $\scrb_0 = \scrF(B)$. Then $X\subseteq\scrI$ is bounded if and only if $\rho X$ is bounded. Indeed, it is clear that if $X$ is bounded, then so is $\rho X$; conversely, suppose $\rho X$ is bounded and pick some $x\in X$ and fix a $b_0\in\scrb_0$. Then there is an apartment containing $x$ and $\scrb_0$ on which $\rho$ acts by some element $g\in G$. But since $d$ is a $G$-invariant metric, it follows that 
    \begin{equation*}
        d(x, b_0) = d(gx, gb_0)\le d(gx, b_0) + d(b_0, gb_0)\le d(\rho x, b_0) + \diam\scrb_0.
    \end{equation*}
    Note that $\rho X$ is bounded if and only if it is contained in a finite union o closed chambers $\overline\scrb$ of $\scrA_0$. Hence $M$ is bounded if and only if $M\scrb$ is bounded for each chamber $\scrb$ of $\scrA_0$ which is possible if and only if $M\scrb_0$ is bounded.

    For each $m\in M$, let $w_m\in W$ denote the unique element such that $m\in Bw_mB$. Then $M\scrb_0$ is bounded if and only if $\bigcup_{m\in M}w_m\scrb_0$ is bounded if and only if the set $\{w_m\colon m\in M\}$ is finite, that is, $M$ intersects only finitely many double cosets in $B\setminus G/B$.
\end{proof}

\begin{theorem}\thlabel{thm:bounded-iff-contained-in-parahoric}
    A subgroup $\Gamma$ of $G$ is bounded if and only if $\Gamma$ is contained in a parahoric subgroup.
\end{theorem}
\begin{proof}
    Suppose $\Gamma$ is bounded and let $x\in\scrI$. Then $X = \Gamma x$ is bounded and is stable under the action of $\Gamma$ which acts through affine isometries. Thus, using \thref{thm:fixed-point-theorem}, there exists a fixed point $y\in\scrI$, i.e., $\Gamma y = y$. If $y$ lies in the facet $\scrF(P)$, then $\Gamma$ must normalize the parahoric subgroup $P$, whene $\Gamma\subseteq P$ due to \cite[2.3.6]{macdonald-spherical-functions}.

    Conversely suppose $\Gamma$ is contained in a parahoric subgroup, which by conjugating can be assumed to be of the form $P_S$, where $S$ is a \emph{proper} subset of $R$. Now note that $P_S = BW_SB$ where $W_S$ is finite because $S\ne R$. Hence $\Gamma$ is bounded due to \thref{lem:bounded-iff-finite-double-cosets}.
\end{proof}

\bibliographystyle{alpha}
\bibliography{references}
\end{document}